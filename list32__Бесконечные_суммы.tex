% !TeX encoding = windows-1251
\documentclass[a4paper, 12pt]{article}
\usepackage{newlistok}
%\documentstyle[11pt, russcorr, listok]{article}
\newcommand{\0}[1]{\overline{#1}}
\def\C{\mbox{$\Bbb C$}}

\УвеличитьШирину{1.5truecm}
\УвеличитьВысоту{3.5truecm}

%\documentstyle[11pt, russcorr, ll]{article}
%\def\dfrac{\displaystyle\frac}

\Подзаголовок{}
\НомерЛистка{19}
\ДатаЛистка{02.2009}
\Заголовок{Бесконечные суммы}

\begin{document}

%\scalebox{1}{\vbox{
%\ncopy{1}{

\СоздатьЗаголовок

\опр
Пусть дана последовательность чисел $a_1,a_2,a_3\dots$. {\it Бесконечный числовой ряд}
(или просто {\it ряд}) --- это формальное выражение $a_1+a_2+a_3+\dots$
(краткое обозначение: $\sum\limits_{k=1}^{\infty} a_k$).
\копр

\задача
Дайте определение суммы ряда. Всегда ли она существует?
\кзадача

\опр
Если у ряда есть сумма, то его называют {\it сходящимся}, а если нет --- то {\it расходящимся}.
\копр

\задача
Восстановите ряд по последовательности его {\it частичных сумм}
$a_1+a_2+\dots+a_n=:\sum\limits_{k=1}^n a_k$.
\кзадача

\задача
Верно ли, что:
\\
\пункт
если ряд $\sum\limits_{n=1}^{\infty} a_n$ сходится, то $\lim\limits_{n\to\infty} a_n=0$;
\пункт
если $\lim\limits_{n\to\infty} a_n=0$, то ряд $\sum\limits_{n=1}^{\infty} a_n$ сходится?
\кзадача

\задача
Найдите сумму ряда $\sum\limits_{n=0}^{\infty} x^n$.
\кзадача

\задача Приведем примеры нахождения суммы ряда.
\\
1) Пусть $S=1-1+1-1+\dots$. Тогда $S=1-(1-1+1-1+\dots)=1-S$, откуда $S=1/2$.
\\
2) Пусть $S=1+1+1+\dots$. Тогда $S=1+(1+1+1+\dots)=1+S$, откуда $0=1$.
\\
3) Пусть $S=1+1/2+1/4+1/8+\dots$. Тогда $S=1+1/2(1+1/2+1/4+\dots)=1+S/2$,
откуда $S=2$.
\\
В каких случаях рассуждение корректно, а в каких --- допущена ошибка?
Почему один и тот же способ вычисления иногда дает верный ответ,
а иногда приводит к ошибке?
\кзадача

\задача
Выясните, какие из следующих рядов сходятся, и найдите их суммы:\\
\вСтрочку
\пункт
$\sum\limits_{n=1}^\infty \frac{n}{2^n}$;
\спункт
$\sum\limits_{n=1}^\infty \frac{n^2}{2^n}$;
\пункт
$\sum\limits_{n=1}^{\infty} \frac1{n(n+1)}$;
\пункт
$\sum\limits_{n=1}^{\infty} \frac1{n(n+2)}$;
\спункт
$\sum\limits_{n=1}^{\infty} \frac1{n(n+1)(n+2)\dots(n+k)}$;

\опр Обозначим за $C_x^n$ число $\frac{x(x-1)\dots(x-n+1)}{n!}$, где число $x$ --- произвольное действительное чилсло.
\копр

\спункт
$\sum\limits_{n=1}^{\infty}\frac1{C_{n+x}^n}$
\кзадача

{\bf Теорема Кантора.} Любая неубывающая ограниченная последовательность имеет предел
\footnote{при решении задач этого листка
теоремой Кантора можно пользоваться без доказательства}.

%\задача
%Докажите, что тогда ряд из положительных членов сходится если и только если %последовательность его частичных сумм ограничена.
%\кзадача
%
\задача
При каких натуральных $k$ ряд $\sum\limits_{n=1}^\infty \frac{1}{n^k}$ сходится?
\кзадача

\задача
\пункт
Пусть ряд, составленный из положительных членов, сходится. Докажите, что ряд, полученнный из него перестановкой членов, тоже сходится, и его сумма совпадает с суммой исходного ряда.
\\
\пункт
Верно ли утверждение пункта а) для произвольного сходящегося ряда?
\кзадача

\задача Дана бесконечная вправо и вниз таблица из положительных чисел.
Пусть ряды, составленные из чисел каждой строчки этой таблицы, сходятся.
Кроме того, пусть ряд, составленный из сумм этих рядов, тоже сходится и имеет сумму $S$.
Докажите, что ряды, составленные из чисел каждого столбца, сходятся,
и ряд, составленный из сумм этих рядов, также имеет сумму~$S$.
\кзадача

\задача Найдите суммы рядов:
\пункт
$\sum\limits_{n=0}^{\infty}nx^n$;
\пункт
$\sum\limits_{n=0}^{\infty}C_{n+k}^kx^n$;
\пункт
$\sum\limits_{n=0}^{\infty} n^3 x^n$.
\кзадача

\задача Даны два сходящихся ряда из положительных чисел:
$\sum\limits_{n=0}^{\infty}a_n$ и $\sum\limits_{n=0}^{\infty}b_n$.
Рассмотрим таблицу чисел, у которой на месте с координатами $(n,m)$
стоит число $a_n\cdot b_m$.
Перенумеруем все числа в таблице произвольным образом и составим из них ряд.
Докажите, что он сходится и его сумма равна произведению сумм двух исходных рядов.
\кзадача

\задача ({\it задача про экспоненту}) Докажите, что:
\пункт
ряд $\sum\limits_{n=0}^{\infty}\frac{x^n}{n!}$ сходится при положительных $x$;
\пункт
$\sum\limits_{n=0}^{\infty}\frac{x^n}{n!}=\lim\limits_{n\to\infty}(1+\frac{x}{n})^n$ при положительных $x$;
\пункт
$\sum\limits_{n=0}^{\infty}\frac{x^n}{n!}\cdot\sum\limits_{n=0}^{\infty}\frac{y^n}{n!}=\sum\limits_{n=0}^{\infty}\frac{(x+y)^n}{n!}$ при положительных $x$ и $y$;
\пункт
$\sum\limits_{n=0}^{\infty}\frac{x^n}{n!}=e^x$ при всех положительных рациональных $x$, где $e:=\sum\limits_{n=0}^{\infty}\frac{1}{n!}$.
\кзадача

\сзадача Пусть $p_k$ --- $k$-ое по счёту простое число. Докажите, что $\lim\limits_{n\to\infty} (\frac1{1-\frac1{p_1^m}}\frac1{1-\frac1{p_2^m}}\dots\frac1{1-\frac1{p_n^m}}) = \sum\limits_{n=1}^{\infty}\frac1{n^m}$, где $m$ --- натуральное число больше одного.
\кзадача

\сзадача В задаче 1 определена сумма счетного числа слагаемых.
А что такое сумма несчетного числа слагаемых? Когда она может существовать?
Чему равна сумма несчетного числа слагаемых, если она существует?
\кзадача

\опр Назовём ряд $\sum\limits_{n=0}^{\infty}a_n$ абсолютно сходящимся, если ряд $\sum\limits_{n=0}^{\infty}|a_n|$ сходится.
\копр

\задача Докажите, что абсолютно сходящийся ряд сходится.
\кзадача

\задача Докажите, что если в абсолютно сходящемся ряду переставить его члены некоторым образом, то его сумма не изменится.
\кзадача

\задача Пусть даны два абсолютно сходящихся ряда $\sum\limits_{n=0}^{\infty}a_n$ и $\sum\limits_{n=0}^{\infty}b_n.$ Рассмотрим таблицу чисел, у которой на месте с координатами $(n,m)$ стоит число $a_n\cdot b_m$. Пронумеруем числа в таблице произвольным образом и составим ряд из них. Докажите, что его сумма равна произведению сумм двух данных рядов.
\кзадача

\задача {\bf Обобщённый бином Ньютона.}
\пункт Докажите формулу $\sum\limits_{k=0}^{n} C_{\alpha}^k C_{\beta}^{n-k} = C_{\alpha+\beta}^n$ для натуральных $\alpha$ и $\beta$.

\пункт Докажите вышеуказанную формулу для любых $\alpha$ и $\beta$.

\пункт Докажите, что ряд $\sum\limits_{n=0}^{\infty} C_{\alpha}^nx^n $ абсолютно сходится при $|x|<1.$

\пункт По свойству абсолютно сходящихся рядов ряд из прошлого пункта сходится, обозначим его сумму $S(\alpha).$ Докажите, что $S(\alpha)\cdot S(\beta)=S(\alpha+\beta).$

\пункт Покажите, что сумма ряда $\sum\limits_{n=0}^{\infty} C_{\frac12}^nx^n $ равна $\sqrt{1+x}$.

\пункт Найдите $S(-1)$.

\кзадача

\задача Докажите, что в задаче про экспоненту $x$ может быть отрицательным.
\кзадача



\ЛичныйКондуит{0mm}{8mm}


\end{document}
