% !TeX encoding = windows-1251
\documentclass[a4paper,11pt]{article}
\usepackage{newlistok}
%\documentstyle[11pt, russcorr, listok]{article}
\newcommand{\del}{\mathrel{\raisebox{-.3 ex}{${\vdots}$}}}

\УвеличитьШирину{1.3truecm}
\УвеличитьВысоту{3.1truecm}
\hoffset=-2.5truecm
\voffset=-27.2truemm

\Заголовок{Целые числа: алгоритм Евклида}
%и цепные дроби}
\НомерЛистка{9}
\ДатаЛистка{02.2013}

\renewcommand{\spacer}{\vfil}

\begin{document}

\СоздатьЗаголовок

%\задача
%Докажите, что Ваше 28-летие будет в такой же день недели,
%в какой Вы родились.
%\кзадача


\опр
Два отрезка называются \выд{соизмеримыми}, если они имеют
\выд{общую меру} --- третий отрезок, который укладывается
в каждом из них целое число раз.
\копр

\задача
Верно ли, что два отрезка соизмеримы тогда и только тогда, когда
найд\"ется третий отрезок, в котором каждый из двух укладывается
целое число раз?
\кзадача

\задача
Докажите, что
%отрезки
$a$ и $b$ соизмеримы в том и только том случае, когда
%отрезки
$a$ и $a+2b$ соизмеримы.
\кзадача

\задача
От прямоугольника %размерами
$a\times b$ отрезают квадраты со стороной, равной меньшей стороне
прямоугольника, пока это возможно. %(будем называть это
%\лк операцией Евклида\пк).
С~оставшимся прямоугольником делают тоже самое,
%снова применяют операцию Евклида,
и т.~д.
\сНовойСтроки
\пункт
Сколько и каких квадратов получится, если
$a=324$, $b=141$?
\пункт
Докажите: если $a$ и $b$ %стороны прямоугольника
соизмеримы, то
%в итоге %конце концов
%мы разрежем %его
прямоугольник разрежут на конечное число квадратов;
\пункт
Докажите, что если в итоге прямоугольник разрежут на
конечное число квадратов, то стороны прямоугольника соизмеримы, и
сторона %последнего
самого маленького
квадрата является их общей мерой;
\пункт
%Пусть числа $a$ и $b$ целые.
%Докажите, что длина стороны последнего квадрата равна $(a,b)$.
Докажите, что в пункте в)
сторона последнего квадрата является \выд{наибольшей}
общей мерой сторон прямоугольника, и любая другая их общая мера
укладывается в ней целое число раз.
\кзадача

\задача
От прямоугольника отрезали квадрат и получили прямоугольник,
подобный исходному.\\
\вСтрочку
\пункт
Соизмеримы ли его стороны?
\пункт
Найдите отношение сторон исходного прямоугольника.
\кзадача

\задача
Найдите наибольшую общую меру отрезков длиной
$15/28$ и $6/35$.
%$\displaystyle{{15\over28}}$ и $\displaystyle{{6\over35}}$%,
%то есть найдите такое наибольшее число $\alpha$, что числа
%$\displaystyle{{15\over28\alpha}}$ и
%$\displaystyle{{6\over35\alpha}}$ --- целые.
\кзадача

\опр {\it Наибольший общий делитель} $(a,b)$ целых чисел $a$ и $b$
--- это %называется
наибольшее целое число, делящее и $a$ и $b$.
%(предполагается, что $a$ и $b$ не равны одновременно нулю).
%Обозначение: $(a,b)$. %или ${\rm НОД}(a,b)$.\\
%Если $(a,b)=1$, то $a$ и $b$ называют \выд{взаимно простыми}.
Число $(a,b)$ существует и единственно,
если $a$ и $b$ не равны одновременно нулю (докажите!).
\копр

%\задача
%Докажите, что $(a,b)=|b|$ тогда и только тогда, когда $a$ делится на $b$.
%\кзадача

%\задача
%%Докажите: %, что
%Для каких пар целых чисел $a$, $b$ число
%$(a,b)$ существует (и единственно)?
%%для любых целых  $a$ и $b$,
%%кроме случая $a=b=0$. %(в этом случае считаем НОД неопредел\"енным).
%\кзадача


\задача
Докажите, что $(a,b)=(a-b,b)=(r,b)$, где $r$ --- остаток
от деления $a$ на $b$.
\кзадача

\задача
%Число $n$ целое.
Найдите возможные значения \вСтрочку
\пункт $(n,12)$;
\пункт $(n,n+1)$;
%\пункт $(n,n+6)$;
\пункт $(2n+3,7n+6)$;
\пункт $(n^2,n+1)$.
\кзадача

\задача На клетчатой бумаге нарисован прямоугольник
размерами $a\times b$
клеток (стороны лежат на линиях сетки).
%, в н\"ем проведена диагональ.
На сколько частей делят его диагональ
\вСтрочку
\пункт узлы сетки;
\пункт линии сетки?
\кзадача

%\задача Докажите, что $(a,b)$ --- общий делитель $a$ и $b$,
%делящийся на любой общий делитель~\hbox{$a$ и $b$.}
%\кзадача

%\задача Пусть $d>0$ делит $a$, $b$ и
%делится на любой общий делитель~\hbox{$a$ и $b$.}
%Докажите, что $d=(a,b)$.
%\кзадача


\задача %[Алгоритм Евклида]
Даны целые числа $a>b>0$.
\выд{Алгоритм Евклида} можно описать так: делим $a$ на $b$,
получаем остаток $r_1<b$,
затем делим $b$ на $r_1$,
получаем остаток $r_2<r_1$,
%затем
делим $r_1$ на $r_2$,
получаем остаток $r_3<r_2$, и т.~д.
Докажите, что %в конце концов
какой-то остаток $r_{n-1}$ разделится нацело на $r_n$,
и тогда $r_n=(a,b)$.
\кзадача

\задача Найдите %, не раскладывая
%на множители  %С помощью алгоритма Евклида найдите
\вСтрочку
\пункт
$(525,231)$;
\пункт
$(7\,777\,777,7\,777)$;
\пункт
$(10946,17711)$;
\спункт
$(2^m-1,2^n-1)$.
\кзадача

%\сзадача Докажите, что $(2^m-1,2^n-1)=2^{(m,n)}-1$.
%\кзадача

\задача
Для каких пар чисел алгоритм Евклида работает \лк дольше всего\пк\ --- каждый раз частное равно 1?
\кзадача

\задача
\пункт
В обозначениях задачи 9 докажите, что каждое из чисел $r_1$, $r_2$, $\dots$
представимо в виде $ax+by$ с целыми $x$ и $y$.
\пункт
Как с помощью алгоритма Евклида найти такие целые %числа
$x$ и $y$,
что $ax+by=(a,b)$?
\кзадача

\задача
%Докажите, что $(a,b)$ делится на любой
%общий делитель чисел $a$ и $b$.
Докажите, что все общие делители целых чисел $a$ и $b$ --- это все делители некоторого числа.
Какого?
\кзадача

\задача
%Имеются два шаблона: длины $a$ см и длины $b$ см, $(a,b)=d$.
Какие расстояния можно отложить от данной точки на прямой,
пользуясь двумя шаблонами (без делений)
\вСтрочку
\пункт длины $6$~см и $15$~см;
\пункт длины $a$~см и $b$~см (где $(a,b)=d$)?
%\пункт $a$~см и $b$~см, где $(a,b)=d$?
\кзадача

\задача
Пусть целые числа $a$ и $b$ \выд{взаимно просты} (то есть $(a,b)=1$).
Докажите, что\\
\вСтрочку
\пункт найдутся такие целые $x$ и $y$, что $ax+by=1$;
\пункт если число $c$ целое и $ac$ делится на $b$,  то $c$ делится на $b$.
%\вСтрочку
%\пункт $(ac,b)=(c,b)$;
\кзадача

%\задача Числа $a$, $b$ и $c$ целые, $(a,b)=1$. Докажите, что
%если $ac\del b$,  то $c\del b$.
%%\вСтрочку
%%\пункт $(ac,b)=(c,b)$;
%\кзадача




\задача
Решите в целых числах $x$, $y$ уравнения
\вСтрочку
\пункт
$12x=42y$;
\пункт
$ax+by=0$, где $(a,b)=d$.
\кзадача

%\задача
%\вСтрочку
%\пункт Докажите, что уравнение
%$ax+by=c$ имеет решение в целых числах~$x$,~$y$~тогда
%и только тогда, когда %$c$ делится на $(a,b)$.
%$c$ делится на $(a,b)$,
%и в этом случае
%найдется целое решение $x,\ y$, где $0\leq x<b$.
%\пункт Как найти одно из решений (укажите %какой-нибудь
%способ)?
%\пункт Как, зная одно решение,
%найти остальные? %решения?
%\кзадача

\задача
\вСтрочку
\пункт Докажите, что уравнение
$ax+by=c$ имеет решение в целых числах~$x$,~$y$~если
и только если %$c$ делится на $(a,b)$.
$c$ делится на $(a,b)$.
%и в этом случае найдется целое решение $x,\ y$, где $0\leq x<b$.
\пункт Как найти одно из решений?
%(укажите какой-нибудь способ)?
\пункт Зная одно решение,
найдите формулу для остальных. %решения?
\кзадача


\задача
Решите в целых  $x$, $y$: \вСтрочку
%\пункт $5x+7y=11$;
\пункт $17x+23y=36$; \пункт $nx+(2n-1)y=3$, \ $n$ --- целое; \пункт $525x-231y=42$.
\кзадача

%\задача Пусть $a$ и $b$ --- натуральные числа, $(a,b)=d$.
%По окружности длины $a$ см катится колесо длины $b$ см.
%В колесо вбит гвоздь, который, ударяясь об окружность, оставляет
%на ней отметки.\\
%\вСтрочку
%%\сНовойСтроки
%\пункт
%Сколько всего таких отметок оставит гвоздь на окружности?
%\пункт
%Сколько раз прокатится колесо по окружности, %прежде чем
%пока гвоздь не попад\"ет в уже отмеченную %ранее
%точку?
%\кзадача


\задача Синим на числовой оси отметили числа,
дающие при делении на 24 остаток 17, белым~--- дающие
при делении на 40 остаток 7.
Найдите наименьшее расстояние между белой и синей точками.
\кзадача



\задача На плоскости дана фигура, которая
при повороте вокруг точки $O$ на угол $48^\circ$
переходит в себя.
Обязательно ли эта фигура переходит в себя при повороте
вокруг $O$ на угол
\вСтрочку
\пункт $72^\circ$;
\пункт $90^\circ$?
\кзадача


\задача
По окружности длины $a$ см катится колесо длины $b$ см
($a$ и $b$ натуральные,~\hbox{$(a,b)=d$).}
В~колесо вбит гвоздь, он оставляет отметки на окружности.
%\вСтрочку
%\сНовойСтроки
%\пункт
Сколько отметок оставит гвоздь?
%\пункт
%Сколько раз прокатится колесо по окружности, %прежде чем
%пока гвоздь не попад\"ет в уже отмеченную %ранее
%точку?
\кзадача


%\задача Пусть натуральные числа $a$ и $b$ взаимно просты
%и не равны одновременно 1. Докажите, что существуют
%такие целые числа $x$ и $y$, что $|x|<b$, $|y|<a$, и $ax+by=1$.
%\кзадача

\сзадача
Решите в целых числах уравнение $2x+3y+5z=11$.
\кзадача

\сзадача
Даны $m$ целых чисел. За один ход разрешается прибавить
по единице к любым $n$ из них. При каких $m$ и $n$ всегда можно
за несколько таких ходов сделать числа одинаковыми?
\кзадача

\сзадача
Натуральные числа $m$ и $n$ взаимно просты.
Дробь $(m + 1000n)/(1000m+n)$
%$(m + 2003n)/(2003m+n)$
можно сократить на число $d$. Каково наибольшее  возможное значение $d$?
\кзадача

\сзадача
Натуральные числа $a$ и $b$ взаимно просты. %, $c$ --- целое.
Докажите, что уравнение $ax+by=c$
\сНовойСтроки
\пункт
при любом целом $c$
имеет такое решение в целых числах $x$ и $y$, что $0\leq x<b$;
\пункт
имеет решение в целых неотрицательных числах $x$ и $y$, если
$c$ целое, большее $ab-a-b$;
\пункт
при целых $c$ от 0 до $ab-a-b$ ровно в половине случаев
имеет целое неотрицательное решение,
прич\"ем если для %какого-то
$c=c_0$
такое решение есть, то для
$c=ab-a-b-c_0$ таких решений нет.
\кзадача


\ЛичныйКондуит{0mm}{6mm}

%\СделатьКондуит{4mm}{7mm}

%\сзадача
%На доске записаны два разных натуральных числа. За ход разрешается стереть
%минимальное из имеющихся чисел $a$, $b$ и записать вместо него число
%$ab/|a-b|$. Докажите, что после нескольких таких ходов на доске
%будут написаны два одинаковых натуральных числа.
%\кзадача

\end{document}

\сзадача
Даны два разных натуральных числа. За ход меньшее из имеющихся чисел
$a$, $b$ заменяют на $ab/|a-b|$. Докажите, что
через несколько ходов числа станут равными.
%после нескольких таких ходов на доске
%будут написаны два одинаковых натуральных числа.
\кзадача


\сзадача
Число $p$ простое. Докажите, что среди любых $p+1$ различных натуральных
чисел найдутся два таких числа $a>b$, что $a/(a,b)\geq p+1$.
\кзадача

%\vfill
%\newpage


%\задача
%Какое наименьшее расстояние можно отложить от данной точки на прямой,
%пользуясь двумя шаблонами (без делений) длины
%\вСтрочку
%\пункт $13$~см и $29$~см;
%\пункт $21$~см и $30$~см?
%\кзадача


\раздел{Для контрольных работ}


\задача Назов\"ем наибольшим общим делителем двух целых чисел $a$
и $b$ такой их общий положительный
делитель, который делится на любой общий делитель
чисел $a$ и $b$. Эквивалентно ли это определение данному в листочке 6?
\кзадача


\задача Фигура на плоскости переходит в себя
при повороте вокруг точки $O$ на угол $48^\circ$.
Обязательно ли эта фигура переходит в себя при повороте
вокруг $O$ на угол
\вСтрочку
\пункт $72^\circ$;
\пункт $90^\circ$?
\кзадача


\задача
Сколько у уравнения $kx+ly=kl$
решений в натуральных %числах
$x,y$
($k,l$ --- натуральные)?
%имеет уравнение $mx+ny=mn$,
%где $m$ и $n$ --- данные натуральные числа?
\кзадача

\задача
Натуральные числа $m$ и $n$ взаимно просты.
Дробь $\displaystyle\frac{m + 2003n}{2003m+n}$
%$(m + 2003n)/(2003m+n)$
можно сократить на число $d$. Каково наибольшее  возможное значение $d$?
\кзадача

\задача
%Какое наименьшее расстояние и какие вообще расстояния
Какие расстояния можно отложить от данной точки на прямой,
пользуясь двумя шаблонами (без делений) длины
\вСтрочку
\пункт $13$~см и $29$~см;
\пункт $21$~см и $30$~см;
\пункт $a$~см и $b$~см, где $(a,b)=d$?
\кзадача

\задача Существует ли такое целое число $a$, что число $a^2+2$
делится на 5?
\кзадача

\задача Верно ли, что из любых 100 целых чисел можно выбрать
\вСтрочку
\пункт 15;
\пункт 16 таких, у которых разность любых двух делится на 7?
\кзадача

\задача Найдите остаток от деления числа $65^{328}$ на 9.
\кзадача

\задача Пусть $a$ и $b$ --- натуральные числа.
Обязательно ли
\вСтрочку
\пункт числа $a$ и $b/(a,b)$ взаимно просты;
\пункт либо числа $a$ и $b/(a,b)$ взаимно просты,
либо числа $b$ и $a/(a,b)$ взаимно просты?
\кзадача

\задача Найдите $(a,b)$  и одну пару таких целых чисел $x$ и $y$,
что $ax+by=(a,b)$, если $a=44\,863$ и $b=70\,499$.
(Не забудьте привести вычисления --- одного ответа не достаточно).
\кзадача

\задача Пусть $a$ и $b$ --- натуральные числа, $p$ --- простое число,
прич\"ем $ab$ делится на $p$. Докажите, что хотя бы одно из чисел
$a$ и $b$ делится на $p$.
\кзадача

\задача Сумма квадратов двух целых чисел делится на 3. Докажите,
что каждое из этих чисел делится на 3.
\кзадача

\задача
Существует ли бесконечно много таких целых
чисел $a$, что числа $a+9$ и $a-9$ взаимно просты\/?
\кзадача

\задача Найдите остаток от деления числа
$1^{4}+2^{4}+3^{4}+\ldots+99^{4}+100^{4}$ на 7.
\кзадача

\задача
Найдите все целые решения уравнения $273\,x+185\,y=2$.
\кзадача

\задача В последовательности Фибоначчи (1, 1, 2, 3, 5, 8, 13, \ldots)
каждый член равен сумме двух предыдущих.
Заменим каждое число в этой последовательности остатком от деления
его на~11. Найдите 1000-ый член получившейся последовательности.
\кзадача

\задача Верно ли, что из любых 5 целых чисел можно выбрать
два числа, разность квадратов которых делится на~7?
\кзадача

\задача Пусть $n\in\N$. Сколько двоек в разложении числа
$(n+1)\cdot(n+2)\cdot\ldots\cdot(2n)$ на простые множители?
\кзадача

\задача
Найдите все целые числа, которые при делении на~3 дают остаток~2,
а при делении на~5 дают остаток~3 (и докажите, что других нет).
\кзадача

\задача
Число $n$ натуральное. Обозначим через $k$ наименьшее
натуральное число, которое больше 1 и
взаимно просто с каждым из чисел $1$, $2$, \dots, $n$.
Докажите, что $k$ существует и является простым.
\кзадача

\задача
Найдите все натуральные числа, которые делятся на 30 и имеют ровно
20 натуральных делителей.
\кзадача

\задача Пусть $m, n\in\N$ и $(m,n)=1$.
Найдите все решения в натуральных числах $x$, $y$ уравнения $x^m=y^n$.
\кзадача

\задача
Пусть $p_1^{\a_1}\cdot \dots \cdot p_k^{\a_k}$ --- каноническое
разложение числа $n$ на простые множители.
Найдите произведение всех натуральных делителей числа $n$.
\кзадача

\сзадача Пусть $n$ --- натуральное число. Рассмотрим пары натуральных
чисел $(u;v)$, наименьшее общее кратное которых равно $n$.
(Если $u\ne v$, то пара $(u;v)$ считается отличной от пары $(v;u)$.)
Докажите, что таких пар существует столько же, сколько
существует натуральных делителей у числа $n^2$.
\кзадача

\задача Число $n$ натуральное. Докажите, что
количество натуральных делителей у $n$ меньше $2\sqrt n$.
\кзадача

\задача При каких натуральных $k$ число $(k-1)!$ не делится на $k$?
\кзадача

\задача
Простые числа (и только они) имеют ровно два натуральных делителя.
А какие натуральные числа имеют ровно три натуральных делителя?
\кзадача

\задача
На какую максимальную степень двойки делится число $C_{50}^{25}$?
\кзадача

\задача Пусть $a, b\in\N$, $(a,b)=d$. Сколько членов арифметической
прогрессии $a,\ 2a,\ \ldots,\ ba$ делятся на~$b$?
\кзадача

\задача
Пусть $p_1^{\a_1}\cdot \dots \cdot p_k^{\a_k}$ --- каноническое
разложение числа $n$ на простые множители.
Найдите произведение всех натуральных делителей числа $n$.
\кзадача


\задача Пусть $a$, $b$, $c$, $d$ --- натуральные числа, прич\"ем $ab=cd$.
Докажите, что найдутся такие целые числа $k$, $l$, $m$, $n$,
что $a=kl$, $b=mn$, $c=km$, $d=ln$.
\кзадача

\задача
Найдите все такие пары целых чисел $a$ и $b$, что $a+b=45$
и $(a,b)+[a,b]=81.$
\кзадача


\раздел{***}

\задача
По окружности записаны $n$ цифр. Известно, что при повороте ...
\кзадача


\опр
Последовательность чисел $f_1, f_2, f_3, \dots$, где $f_1=f_2=1$
и $f_{n+2}=f_{n+1}+f_{n}$ при всех натуральных $n$, называется
последовательностью чисел Фибоначчи.
\копр

\end{document}

\сзадача
Натуральные числа $a$ и $b$ взаимно просты.
Докажите, что
\сНовойСтроки
\пункт
если $ab>1$, то существуют
такие целые числа $x$ и $y$, что $|x|<b$, $|y|<a$, и $ax+by=1$.
\пункт
уравнение $ax+by=c$ имеет решение в целых неотрицательных числах, если
$c\geq(a-1)(b-1)$;
\пункт
среди целых чисел от 0 до $ab-a-b$ ровно половина таких $c$,
для которых уравнение $ax+by=c$ имеет целое неотрицательное решение,
прич\"ем если для какого-то $c=c_0$ уравнение имеет
такое решение, то для
$c=ab-a-b-c_0$ таких решений нет.
\кзадача

\задача
\пункт Выпишите первые 10 чисел Фибоначчи.
\пункт
\кзадача





\задача
Натуральные числа $a$ и $b$ взаимно просты. Докажите, что наибольший
общий делитель чисел $a+b$ и $a^2+b^2$ равен 1 или 2.
\кзадача

\задача
Дан прямоугольник со сторонами $a\times b$, где $a$ и $b$ --- целые числа.

\кзадача

------------------
\опр
Пусть $a$ и $b$ --- целые числа, $b>0$.
\выд{Разделить} $a$ на $b$ \выд{с остатком} значит найти
такие целые числа $k$ (частное) и $r$ (остаток),
что $a = kb + r$ и $0\leq r < b$.
\копр


\задача
Числа $a$ и $b$ --- целые, $b>0$.
Отметим на числовой прямой все числа, кратные~$b$.
Они разобьют прямую на отрезки длины $b$.
Точка $a$ лежит на одном из них. %этих отрезков.
Пусть $kb$ --- левый конец этого отрезка.
Докажите, что $k$ --- частное, а %разность
$r = a - kb$ --- остаток от деления $a$ на $b$.
\кзадача

\задача
Докажите, что частное и остаток определены однозначно.
\кзадача


%\задача
%Пусть $a$ и $b$ --- целые числа, $b>0$. Докажите, что
%существуют и единственны такие целые числа $q$ и $r$,
%что $a=bq+r$ и $0 \leq r < b$.\hfill\break
%Число $q$ называется {\it частным}, а число $r$ --- {\it остатком}
%от деления $a$ на $b$.
%\кзадача

\задача
Найдите частные и остатки от деления $2003$ на $23$, $-17$ на $4$ и
$n^2-n+1$~на~$n$.
\кзадача

\задача
Найдите все возможные частные и остатки от деления числа $53$.
\кзадача

------------------


\задача
Докажите, что Ваше 28-летие будет в такой же день недели,
в какой Вы родились.
\кзадача

\опр
Пусть $n$ и $k$ --- целые числа, $k\ne0$.
Говорят, что число $n$ \выд{делится}
на число~$k$, если существует такое целое число $m$,
что $n=k\cdot m$. Обозначение: $n\del k$.
Говорят также, что $n$ \выд{кратно} $k$
или что $k$ \выд{делит} $n$.
\копр

\задача Верно ли, что
\вСтрочку
\пункт
если $n\del k$ и $k\del n$, то $n=k$;
\пункт
если $b\del a$ и $c\del a$, то $b+c\del a$;
\пункт
если $b\del a$ и $c\del b$, то $c\del a$;
\пункт
если $a\del b$ и $c\del d$, то $ac\del bd$;
\пункт
если $a$ и $b$ не делятся на $c$, то $ab$ не делится на $c^2$?
\кзадача

\задача
Докажите, что тр\"ехзначное число вида $\overline{aaa}$ делится на $37$.
\кзадача

\задача
\пункт
Пусть $a,b$ --- целые числа, прич\"ем $a+2\del7$ и $9-b\del7$.
Докажите, что $a+b\del7$.
\пункт
Пусть $5m+3n\del11$. Докажите, что тогда $9m+n\del11$ ($m$ и $n$ --- целые).
\кзадача

\задача
\пункт Докажите, что целое число делится на 4 тогда и только тогда,
когда две его последние цифры образуют число, делящееся на 4.
\пункт Сформулируйте и докажите признаки делимости на 2, 5, 8, 10.
\кзадача

\задача
\пункт
Из натурального числа $\overline{a_n\ldots a_1a_0}$ вычли сумму
его цифр $a_n+\ldots+a_1+a_0$. Докажите, что получилось число,
делящееся на 9.
\пункт
Выведите из предыдущего пункта признаки делимости на 3 и на 9.
\кзадача

\задача
Докажите, что число, составленное из 81 единицы, делится на 81.
\кзадача

%\задача
%Может ли число, составленное из 10 единиц, 10 нулей и 10 двоек,
%быть точным квадратом?
%\кзадача
%\задача
%Докажите, что если в тр\"ехзначном числе две последние цифры одинаковы, а
%сумма его цифр делится на $7$, то и само число делится на $7$.
%\кзадача

\задача
Целые числа $a$ и $b$ различны. Докажите, что $a^n-b^n\del a-b$
при любом $n\in\N$. Чему равно частное?
%(Сравните с задачей о сумме геометрической прогрессии).
\кзадача

\задача
Пусть $a,b\in\Z$  и $a+b\ne0$. Докажите, что
$a^n+b^n\del a+b$ при любом неч\"етном~$n\in\N$.
\кзадача

\задача
Найдите все целые $n$, при которых число $(n^3+3)/(n+3)$ целое.
\кзадача

%\задача
%Докажите, что
%$1^k+2^k+\dots+(n-1)^k\del n$
%при неч\"етных натуральных $n$ и $k$.
%\кзадача

%\задача
%Числа $a,b,c,d$ --- натуральные. Обязательно ли число
%$\displaystyle\frac{(a+b+c+d)!}{a!\ b!\ c!\ d!}$ целое?
%\кзадача

%\задача
%Докажите, что $m(m+1)(m+2)$ делится на 6 при любом целом $m$.
%\кзадача

%\задача
%Всегда ли делится на $n!$ произведение любых $n$ последовательных
%целых чисел?
%\кзадача

\задача
Решите в натуральных числах уравнения:
\вСтрочку
\пункт
$x^2-y^2=31$;
\пункт
$x^2-y^2=2002$.
\кзадача

%\задача
%Числа $a$ и $b$ не делятся на число $c$. Может ли число $ab$ делиться
%на $c^2$?
%\кзадача

\задача
Число $a$ ч\"етно, но не делится на 4. Докажите, что количество
ч\"етных делителей числа $a$ равно количеству неч\"етных делителей $a$.
\кзадача

\задача
Какие натуральные числа имеют неч\"етное количество натуральных делителей?
\кзадача

\задача
%\пункт
%На сколько нулей оканчивается число $28!$?
\вСтрочку
\пункт
Может ли $n!$ оканчиваться ровно на 4 нуля?
\пункт
А ровно на 5 нулей?
\кзадача

\опр
Натуральное число $p>1$ называется \выд{простым}, если оно имеет ровно два
натуральных делителя: 1 и $p$, в противном случае оно
называется \выд{составным}.
\копр

\задача Докажите, что любое натуральное число, большее 1,
либо само простое, либо раскладывается в произведение нескольких
простых множителей.
\кзадача


\задача %Докажите следующие утверждения:
%\сНовойСтроки
\вСтрочку
\пункт Даны натуральные числа $a_1$, \ldots, $a_n$.
Придумайте число, которое не делится ни на одно из чисел
$a_1$, \ldots, $a_n$.
%произведение первых $n$ простых чисел больше
%следующего простого числа {($n>1$);}
\пункт Докажите, что простых чисел бесконечно много;\\
%\вСтрочку
\пункт Докажите, что простых чисел вида $3k+2$ бесконечно много ($k\in\N$).
\кзадача


\задача
\пункт
%Существуют
Могут ли 100 последовательных натуральных чисел
все быть составными?
%ни одного простого числа?
%(например, $1001!+2$, $1001!+3$, \ldots, $1001!+1001$).
\пункт
%А существуют
Найдутся ли 100 последовательных натуральных чисел,
среди которых ровно 5 простых?
\кзадача

\сзадача
Из чисел 1, 2, 3, \dots , 100 выбрали произвольным образом 51 число.
Докажите, что среди выбранных чисел найдутся два, одно из которых
делится на другое.
\кзадача

\сзадача
При каких натуральных $k$ число вида $101010\dots101$, составленное
из $k$ нулей и $k+1$ единиц, простое?
\кзадача


%\задача
%Докажите, что $12^{2n+1}+11^{n+2}$ делится на $133$
%при любом натуральном $n$.
%\кзадача

\задача
Есть три кучки камней: в одной~--- 51 камень, в другой~--- 49 камней,
а в третьей~---~5 камней. Разрешается объединять любые кучки в одну,
а также разделять кучку из четного количества камней на две равные.
Можно ли получить 105~кучек по одному камню в каждой?
\кзадача

\задача
Вася написал пример на умножение двузначных чисел, а затем заменил
в нём все цифры на буквы, причём одинаковые цифры --- на одинаковые буквы,
а разные --- на разные.  В итоге у него получилось $АБ\cdot ВГ = ДДЕЕ$.
Докажите, что он где-то ошибся.
\кзадача

\задача Пусть $a$ и $b$ --- целые числа, $b\ne0$. Докажите, что
существуют и единственны такие целые числа $q$ и $r$,
что $a=bq+r$ и $0 \leq r < |b|$.\hfill\break
Число $q$ называется {\it частным}, а число $r$ --- {\it остатком}
от деления $a$ на $b$.
\кзадача

\задача
Найдите частные и остатки от деления $2000$ на $23$, $-17$ на $4$ и
$n^2-n+1$~на~$n$.
\кзадача

\задача
Найдите все возможные частные и остатки от деления числа $53$.
\кзадача

\задача
\пункт Найдите остатки от деления на 9 чисел 10, 100, 1000, \dots.
\пункт Пусть $a$ --- цифра.
Найдите остаток от деления
числа $\overline{a0\dots0}$
на 9.
\пункт Докажите,
что целое число делится на 9 тогда и только тогда, когда сумма его цифр
делится на~9.
\пункт
Сформулируйте и докажите признаки делимости на 3 и на 11.
\кзадача

\задача
Какой цифрой оканчивается число \вСтрочку
\пункт $14^{14}$?
\пункт $14^{14^{14}}$?
\кзадача

\задача
Найдите все $k\in\N$, при которых \вСтрочку
\пункт $2^k-1$ делится на $7$; \пункт $2^k+1$ делится на $7$.
\кзадача

\сзадача
Даны 20 чисел, ни одно из которых не делится на 5. Докажите, что
сумма 20-ых степеней этих чисел делится на 5.
\кзадача

\задача
Докажите, что из любых 100 целых чисел всегда можно выбрать 2 числа,
разность которых делится на 99.
\кзадача

\задача
Докажите, что из любых 52 целых чисел всегда можно выбрать 2 числа,
сумма или разность которых делится на 100.
\кзадача

\сзадача
Докажите, что из любых $n$ целых чисел всегда можно выбрать несколько,
сумма которых делится на $n$ (или одно число, делящееся на $n$).
\кзадача

%\сзадача
%\пункт Докажите, что для любого $n$ есть число вида $1\dots10\dots0$,
%делящееся~на~$n$.
%\пункт Докажите, что существует число вида $11\dots11$, которое делится на
%2001.  \кзадача

\end{document}

%\задача
%\пункт
%От прямоугольника со сторонами $324\times 141$ мм отрезают несколько
%квадратов со стороной $141$ мм до тех пор, пока не останется
%прямоугольник, у которого длина одной стороны меньше 141 мм.
%От полученного прямоугольника отрезают квадраты, стороны которого
%равны его меньшей стороне, до тех пор, пока это возможно, и т.~д.
%Какова длина стороны последнего квадрата?
%\пункт
%Ту же самую процедуру проделывают с произвольным прямоугольником
%со сторонами $a\times b$ мм (где $a$ и $b$ --- целые числа).
%Докажите, что в конце концов останется квадрат, длина стороны которого
%делит оба числа $a$ и $b$.
%\пункт
%Докажите, что у каждого из получающихся прямоугольников (включая
%последний квадрат) наибольший общий делитель длин сторон такой же,
%как и у длин сторон исходного прямоугольника.
%\кзадача 