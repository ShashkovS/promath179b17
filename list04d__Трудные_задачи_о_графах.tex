% !TeX encoding = windows-1251
\documentclass[a4paper,11pt]{article}
\usepackage{newlistok}
\usepackage{tikz}
%\documentstyle[11pt, russcorr, listok]{article}

%\УвеличитьШирину{1.2truecm}
%\УвеличитьВысоту{3.5truecm}
\УвеличитьШирину{1.7truecm}
\УвеличитьВысоту{3.2truecm}
\hoffset=-2.5truecm
\voffset=-24truemm


\Заголовок{Трудные задачи о графах}
%Графы 2}
%\Подзаголовок{Деревья. Формула Эйлера. Плоские графы}

\НомерЛистка{4д}
\ДатаЛистка{03.2014}

\begin{document}

\thispagestyle{empty}

\СоздатьЗаголовок

%\vspace*{-1truemm}


%\сзадача
%Связный граф с непересекающимися р\"ебра\-ми
%нарисован на торе (поверхность бублика).
%Как изменится формула Эйлера?
%\кзадача



%\раздел{Трудные задачи о графах}


%\задача
%Можно ли разбить какой-нибудь выпуклый шестиугольник на несколько
%меньших выпуклых
%шестиугольников так, чтобы любые два шестиугольника разбиения
%\кзадача

%\опр
%Разбиение многоугольника на треугольники называется \выд{триангуляцией},
%если любые два из этих треугольников либо не имеют общих точек,
%либо имеют только общую вершину или общую сторону.
%\копр

%\задача
%Пусть имеется триангуляция равносторонего треугольника на меньшие
%равносторонние треугольники. Докажите, что все треугольники в
%триангуляции одинаковы.
%%Равносторонний треугольник разбит на несколько меньших равносторонних
%%треугольников так, что у любых двух треугольников либо нет общих
%%точек, либо общая точка ровно одна, и это их общая вершина,
%%либо эти два треугольника имеют общую сторону.
%%Докажите, что все треугольники разбиения одинаковы.
%\кзадача


%\задача
%На плоскости отмечено несколько точек, никакие три из которых не лежат на
%одной прямой. Двое играют в такую игру: они
%по очереди соединяют какие-то две ещ\"е не соедин\"енные
%точки отрезком так, чтобы отрезки не пересекались нигде, кроме
%отмеченных точек. Проигрывает тот, кто не может сделать ход.
%Зависит ли исход этой игры от того, как играют соперники?
%\кзадача

\задача
Каких графов на $n$ данных вершинах больше: связных или несвязных?
Дайте ответ для всех~$n$.
\кзадача

%\сзадача
%Сколько остовов имеет граф
%с вершинами $V_0,\ldots,V_n$ и $2n-1$ ребром, где
%вершина~$V_0$ соединена р\"ебрами с
%остальными вершинами, и при $1\leq i<n$ соединены ребром
%вершины $V_i$~и~$V_{i+1}$?
%\кзадача



\задача [Теорема Холла] В некоторой компании $n$ юношей.
При каждом $k$ от 1 до $n$ верно утверждение:
для любых $k$ юношей в компании число девушек,
знакомых хотя бы с одним из этих $k$ юношей, не меньше $k$.
Можно ли женить всех юношей на знакомых девушках?
\кзадача

\задача
\выд{Латинским прямоугольником}
называется таблица $m\times n$, где $m\leq n$, в каждой клетке которой записано
целое число от 1 до $n$, причем в каждой строке все числа различны
и в каждом столбце все числа различны. Докажите, что
любой латинский прямоугольник $m\times n$, где $m<n$,
можно дополнить до латинского квадрата $n\times n$.
%эту таблицу
%можно дополнить до квадрата $n\times n$, записав в новые клетки
%целые числа от 1 до $n$ так, чтобы по-прежнему
%в каждой строке и в каждом столбце числа были различны.
\кзадача

\задача
\вСтрочку
\пункт
В гости ожидают $m$ или $n$ человек, где $(m,n)=1$.
На какое наименьшее число секторов надо разрезать круглый торт,
чтобы из них можно было сложить как $m$, так и $n$ одинаковых кусков?
\пункт А если $(m,n)=d$?
\кзадача



\задача
В графе $n>2$ вершин, причём степень каждой вершины не меньше $n/2$.
Докажите, что в графе есть \выд{гамильтонов цикл}, т.~е.~цикл,
содержащий все вершины графа по одному~разу.
\кзадача


\задача
%Гриша забыл тр\"ехзначный код своего замк\'а. Замок
%открывается, если три~цифры кода набраны подряд (даже если ранее
%были набраны другие цифры). Докажите, что Гриша сможет открыть замок
%не более чем за 1002 секунды, набирая по одной цифре в секунду.
\спункт Рассеянный математик, забыв трёхзначный код своего подъезда,
нажимает кнопки с цифрами 0, 1, 2, \dots , 8, 9 по одной в секунду.
Дверь откроется, если три
цифры кода в нужном порядке будут набраны подряд. Математик уверен, что даже
в случае крайнего невезениям (если нужная комбинация встретится последней)
он сможет войти в подъезд не позже чем через 1002 секунды. Прав ли он?
Как он должен действовать, чтобы попасть в дом за наименьшее время?\\
Ответьте на аналогичный вопрос, если\\
\пункт исправны только кнопки с цифрами 1, 2 и 3, а никакие другие цифры
в код не входят;\\
\спункт исправны все кнопки, но математик помнит, что все три цифры кода
различны.
\кзадача

\задача
Опишите простые
графы с $n$ вершинами $A_1$, \dots, $A_n$ и $n$ р\"ебрами
$b_1$, \dots, $b_n$, со свойст\-вом: любые две различные
вершины $A_i$ и $A_j$
соединены ребром если и только если р\"ебра $b_i$ и $b_j$ имеют
общую вершину.
%смежны.
\кзадача

\задача
Докажите, что среди любых 50 человек найдутся двое, у которых ч\"етное
число общих знакомых (быть может, 0) среди остальных 48 человек.
\кзадача

\задача
Оля и Максим оплатили путешествие по архипелагу из 2009 островов, где
некоторые острова связаны двусторонними маршрутами катера.
Они путешествуют, играя. Сначала Оля выбирает остров, на который они
прилетают. Затем они путешествуют вместе на катерах,
по очереди выбирая остров, на котором еще не были
(первый раз выбирает Максим). Кто не сможет выбрать остров,
проиграл. Докажите, что при любой схеме маршрутов Оля может выиграть,
как бы ни играл Максим.
\кзадача


\задача
В графе с $n$ вершинами нет треугольников (циклов длины 3).
Какое наибольшее число рёбер может быть в этом графе?
Дайте ответ для всех $n$.
\кзадача


\задача
В стране 1000 городов и 2008 дорог (каждая дорога соединяет
два города). Докажите, что можно указать кольцевой маршрут,
проходящий не более, чем через 18 городов.
\кзадача

\задача
На плоскости расположены $n$ непересекающихся отрезков и $n+2$ точки,
не лежащие на этих отрезках. Докажите, что какие-то две точки \лк видят
друг друга\пк\ (то есть если соединить эти две точки отрезком, он не пересечёт
ни одного из данных $n$ отрезков).
\кзадача

\задача [Теорема Кели]
Докажите, что полный граф с $n$ вершинами имеет $n^{n-2}$ остовов.
\кзадача

%\задача
%Задача про расстановку стрелок.
%\кзадача


\vspace*{-1truemm}
\раздел{Графы и раскраски.}

\vspace*{-1truemm}
\задача
В лесу $k\cdot l$ тропинок и несколько полянок.
Каждая тропинка соединяет две полянки.~Из\-вес\-т\-но, что
тропинки можно раскрасить в $l$ цветов так, чтобы к каждой
полянке сходились тропин\-ки разного цвета. Докажите, что это
можно сделать, покрасив каждым цветом ровно $k$ тропинок.
\кзадача

\задача
Найдите наименьшее значение $n$, для которого любой коллектив,
где каждый недолюбливает не более семи из остальных, можно разбить
на не более чем $n$ частей так, чтобы ни в какой части не нашлось
двух человек, хотя бы один из которых недолюбливает другого.
\кзадача


\опр Раскраска вершин графа называется \выд{правильной}, если
никакие две вершины~од\-но\-го цвета не соединены ребром.
Простой граф
называется \выд{$k$-дольным}, если правильная раскраска его вершин
возможна $k$ цветами, но не менее
(такая раскраска называется \выд{минимальной}).
\копр

\задача Равносильна ли двудольность графа
отсутствию в н\"ем циклов неч\"етной длины?
\кзадача

\задача
Вершины $k$-дольного графа правильно раскрашены в цвета
1, 2, \dots, $k$. Докажите, что этом в графе есть путь,
в котором первая вершина --- цвета 1, вторая --- цвета 2,
\dots, $k$-я~---~цвета~$k$.
%Есть ли в $k$-дольном графе с минимальной окраской
%путь из $k$ разноцветных~\hbox{вершин?}
\кзадача


\ЛичныйКондуит{0mm}{7mm}
%\СделатьКондуит{8mm}{7mm}




%\vspace*{-1truemm}

%\раздел{Разное}

%\vspace*{-1truemm}







%\сзадача
%В графе $n$ вершин $A_1$, \dots, $A_n$ и $n$ р\"ебер $b_1$, \dots, $b_n$.
%Известно, что любые две вершины $A_i$ и $A_j$ этого графа
%соединены ребром если и только если р\"ебра $b_i$ и $b_j$ выходят из одной
%вершины. Докажите, что степень каждой вершины равна двум.
%\кзадача


%\сзадача
%В графе $n$ вершин $A_1$, \dots, $A_n$ и $n$ р\"ебер $b_1$, \dots, $b_n$.
%Известно, что любые две вершины $A_i$ и $A_j$ этого графа
%соединены ребром если и только если р\"ебра $b_i$ и $b_j$ выходят из одной
%вершины. Докажите, что степень каждой вершины равна двум.
%\кзадача





\end{document}
