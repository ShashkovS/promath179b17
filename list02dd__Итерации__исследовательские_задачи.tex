% !TeX encoding = windows-1251
\documentclass[a4paper,12pt]{article}
\usepackage{newlistok}
\usepackage{tikz}
\usetikzlibrary{calc}
%\documentstyle[11pt, russcorr, listok]{article}



%%%%%%%%%%%%%%%%%%%%%%%%%%%%%%%%%%%%%%%%%%%%%%%%%%%%%%%%%%%%%%%%%%%%%%%%%%%%%%%%%%%%%%%%%%%%%%%%%%%
%%%%%%%%%%%%%%%%%%%%%%%%%%%%%%%%%%%%%%%%%%%%%%%%%%%%%%%%%%%%%%%%%%%%%%%%%%%%%%%%%%%%%%%%%%%%%%%%%%%
%%%%%%%%%%%%%%%%%%%%%%%%%%%%%%%%%%%%%%%%%%%%%%%%%%%%%%%%%%%%%%%%%%%%%%%%%%%%%%%%%%%%%%%%%%%%%%%%%%%
\newcount\jcnt
\newcount\correct
\correct=0
\def\dof{\doit\doit\doit\doit\doit\doit\doit\doit}
\def\doff{\doitn\doitn\doitn\doitn\doitn\doitn\doitn\doitn}
\def\dofff{\if\sth0\def\len{7}\fi\if\stg0\def\len{6}\fi\if\stf0\def\len{5}\fi\if\ste0\def\len{f}\fi%
\if\std0\def\len{3}\fi\if\stc0\def\len{2}\fi\if\stb0\def\len{1}\fi\if\sta0\def\len{0}\fi%
\doitnn\doitnn\doitnn\doitnn\doitnn\doitnn\doitnn\doitnn}
\def\doit{%
\advance\jcnt by 1%
\ifcase\jcnt\or\def\stx{\sta}\or\def\stx{\stb}\or\def\stx{\stc}\or\def\stx{\std}\or\def\stx{\ste}%
\or\def\stx{\stf}\or\def\stx{\stg}\or\def\stx{\sth}\fi%
\ifcase\stx%
\or{\draw ($ (0,-1)+(\jcnt,1)$)  rectangle ($(1,0)+(\jcnt,1)$);}%
\else{\foreach \i in {1,2,...,\stx}{\draw ($ (0,-1)+(\jcnt,\i)$)  rectangle ($(1,0)+(\jcnt,\i)$);}}%
\fi%
}
\def\doitn{%
\advance\jcnt by 1%
\ifcase\jcnt\or\def\stx{\sta}\or\def\stx{\stb}\or\def\stx{\stc}\or\def\stx{\std}\or\def\stx{\ste}%
\or\def\stx{\stf}\or\def\stx{\stg}\or\def\stx{\sth}\fi%
\ifcase\stx%
\or{\filldraw[fill=green!30!white] ($(0,-1)+(\jcnt,1)+(-.5,-.5)$)  rectangle ($(1,0)+(\jcnt,1)+(-.5,-.5)$);
    \draw[<-,thick] ($ (0,-1)+(\jcnt,2)+(0,.5)$)--($ (0,-1)+(\jcnt,2)+(0,.5)+(1,0)$);%
    }%
\or{\draw[fill=green!30!white] ($ (0,-1)+(\jcnt,1)+(-.5,-.5)$)  rectangle ($(1,0)+(\jcnt,1)+(-.5,-.5)$);
    \draw ($ (0,-1)+(\jcnt,2)$)  rectangle ($(1,0)+(\jcnt,2)$);}%
\else{\draw[fill=green!30!white] ($ (0,-1)+(\jcnt,1)+(-.5,-.5)$)  rectangle ($(1,0)+(\jcnt,1)+(-.5,-.5)$);
\foreach \i in {2,3,...,\stx}{\draw ($ (0,-1)+(\jcnt,\i)$)  rectangle ($(1,0)+(\jcnt,\i)$);}}%
\fi%
}
\def\doitnn{%
\ifcase\len%
\or{\draw[fill=green!30!white] (0,0) rectangle (1,1);}%
\else{\foreach\i in {1,2,...,\len}{\draw[fill=green!30!white] ($(1,-1)+(0,\i)$)  rectangle (2,\i);}}%
\fi%
\advance\jcnt by 1%
\ifcase\jcnt\or\def\stx{\sta}\or\def\stx{\stb}\or\def\stx{\stc}\or\def\stx{\std}\or\def\stx{\ste}%
\or\def\stx{\stf}\or\def\stx{\stg}\or\def\stx{\sth}\fi%
\ifcase\stx%
\advance\correct by -1%
\or\advance\correct by -1%
\or{\draw ($ (0,-1)+(\jcnt,1)$)  rectangle ($(1,0)+(\jcnt,1)+(\correct,0)$);}%
\else{\foreach \i in {2,3,...,\stx}{\draw ($ (0,-2)+(\jcnt,\i)+(\correct,0)$)  rectangle ($(1,-1)+(\jcnt,\i)+(\correct,0)$);}}%
\fi%
}
%%%%%%%%%%%%%%%%%%%%%%%%%%%%%%%%%%%%%%%%%%%%%%%%%%%%%%%%%%%%%%%%%%%%%%%%%%%%%%%%%%%%%%%%%%%%%%%%%%%
%%%%%%%%%%%%%%%%%%%%%%%%%%%%%%%%%%%%%%%%%%%%%%%%%%%%%%%%%%%%%%%%%%%%%%%%%%%%%%%%%%%%%%%%%%%%%%%%%%%
%%%%%%%%%%%%%%%%%%%%%%%%%%%%%%%%%%%%%%%%%%%%%%%%%%%%%%%%%%%%%%%%%%%%%%%%%%%%%%%%%%%%%%%%%%%%%%%%%%%






%\УвеличитьШирину{1.3truecm}
%\УвеличитьВысоту{3.5truecm}
\УвеличитьШирину{.2truecm}
\УвеличитьВысоту{.5truecm}
\hoffset=-2.35truecm
\voffset=-35truemm


\Заголовок{Итерации: исследовательские задачи}
\НомерЛистка{2д*}
\ДатаЛистка{11.2013}

\begin{document}

\СоздатьЗаголовок


\bigskip

\задача
На столе у чиновника Министерства Околичностей лежит $n$ томов
Британской энциклопедии, сложенных в несколько стопок.
Стопки лежат на столе в один ряд.
Каждый день, приходя на работу, чиновник берет по одному тому
из каждой стопки, образует из них новую стопку, которую кладет
в начало ряда,
%располагает стопки по количеству томов (в невозрастающем порядке)
и записывает в ведомость
количество томов в каждой стопке. Например, если в первый день в ведомости
записано $(8,3,1,1)$, то на следующий день запись будет
$(4,7,2)$, потом --- $(3,3,6,1)$, $(4,2,2,5)$ и т.~д.
\сНовойСтроки
\пункт Пусть $n=36$. Разложите книги %на стопки
так, чтобы чиновник %всегда %ежедневно
делал в ведомости одну и ту же запись.
\пункт Что будет записано в ведомости на 31-й день,
если в первый день там записано $(4,4,4)$?\
\пункт Докажите, что %если $n\ne1+2+3+\dots+k$ ни для какого $k$,
после какого-то момента записи в ведомости
будут циклически повторяться.
\пункт Что чиновник запишет через месяц, если $n=6$?
(Начальное разбиение на стопки~\hbox{неизвестно.)}  % произвольным.)
%\УстановитьГраницы{5mm}{5mm}
{\small \indent Чтобы проследить за путём конкретной книги,
будем считать, что чиновник берёт самую
нижнюю книгу из
первой стопки, на неё кладёт самую
нижнюю книгу из второй стопки,
и т.~д. Каждая книга имеет две координаты: текущий номер её стопки
и высоту внутри стопки. Всё это удобно изображать
на клетчатой бумаге в первой координатной четверти:
книге отвечает %соответствует
закрашенная клетка с теми же координатами.}
%\ВосстановитьГраницы
%\УстановитьГраницы{0cm}{9.5truecm}
\пункт Докажите, что действие чиновника можно описать так:
он отрезает нижнюю строчку от закрашенной фигуры, сдвигает то,
что осталось, на одну клетку вправо и вниз, а отрезанную строчку поворачивает
на $90^\circ$ (превращая её в первый столбик), см. рис. ниже;
затем он, возможно, сдвигает некоторые
столбики влево (чтобы не было пустых столбиков).\\
%\ВосстановитьГраницы
%Это последовательные высоты столбцов исходной таблицы. Если в некий момент указан нуль, то это конец
\def\sta{8}
\def\stb{3}
\def\stc{1}
\def\std{4}
\def\ste{1}
\def\stf{5}
\def\stg{0}
\def\sth{0}
\begin{tikzpicture}[scale=.3]
  \jcnt=1
  \dof
\end{tikzpicture}
\qquad
\begin{tikzpicture}[scale=.3]
  \jcnt=1
  \doff
\end{tikzpicture}
\qquad
\begin{tikzpicture}[scale=.3]
  \jcnt=1
  \dofff
\end{tikzpicture}
\пункт Какой путь проделала книга $(2,4)$ из пункта б) этой задачи?
\пункт Докажите, что при действиях чиновника сумма координат
каждой книги либо не изменяется, либо уменьшается.
\пункт Докажите, что, начиная с какого-то момента, стопки будут
располагаться по числу книг в невозрастающем порядке,
и каждая книга, начиная с этого момента, будет двигаться по циклу.
\пункт Докажите, что если $n$ --- треугольное число (то есть $n=1+2+3+\dots+k$ для некоторого $k$), то, начиная с какого-то момента,
чиновник ежедневно будет записывать в ведомость одно и то же.
Что именно?
%\пункт Пусть $n\ne1+2+3+\dots+k$ ни для какого $k$,
%то, начиная с какого-то момента, записи в ведомости
%будут циклически повторяться.
\пункт Пусть $n$ --- не треугольное число. Докажите, что период $t$, с которым
после какого-то момента будут повторяться записи в ведомости, удовлетворяет условию
$(t-1)t<2n<t(t+1).$
\кзадача


\bigskip

%\title М.\,Хованов."Решение --- в \номер6--1990>\ЗАДт{1204*}
\задача
На плоскости заданы точки $A$, $B$, $C$~--- центры трёх кругов.
Каждый круг равномерно раздувается (радиус увеличивается
с~одинаковой для всех кругов скоростью). Как только два круга
касаются друг друга, они \лк лопаются\пк~--- их радиусы
уменьшаются до~0~--- и начинают расти снова. Верно~ли, что если
длины $AB$, $BC$, $CA$~--- целые числа, то этот процесс
периодический?\\
Изучите, как может развиваться этот процесс, если треугольник
$ABC$\\
\пункт равносторонний;\\
\пункт равнобедренный;\\
\спункт прямоугольный
со~сторонами 3, 4 и~5.\\
Начальное состояние может быть произвольным (не~только \лк нулевым\пк).
\кзадача

\bigskip

\ЛичныйКондуит{0mm}{7mm}

%\СделатьКондуит{8mm}{7mm}

\end{document}

\vspace*{-3truemm}
\раздел{***}

\vspace*{-2truemm}


%\задача
%Гномы из Сказочной страны живут в белых и красных домиках. Ежегодно
%они одновременно красят свои домики, но меняют цвет
%домика только те гномы, у кого больше половины друзей
%жили последний год в домиках другого цвета.
%Докажите, что наступит год, начиная с которого цвет некоторых
%домиков %вовсе
%не будет меняться,
%а остальных --- будет меняться ежегодно. %каждый год.
%Вокруг поляны стоят 12 домиков, покрашенных в белый и красный цвета,
%в которых живут 12 гномов. В январе первый гном красит свой дом
%в тот цвет, в который окрашены дома большинства его друзей
%(или в прежний, если ).
%В феврале это же делает второй (по часовой стрелке) гном,
%в марте --- третий и т.д. Докажите, что наступит момент, после которого
%цвет дома у каждого гнома перестанет меняться.
%\кзадача


\задача
Дано число 1. За ход разрешается
умножить имеющееся число на 2 или прибавить к нему~1. За какое
наименьшее число ходов можно получить число
\вСтрочку
\пункт 10;
\пункт 1000?
\кзадача




\задача
Все натуральные числа выписали подряд без промежутков на бесконечную
ленту:\\ 1234567891011\dots.
Затем ленту разрезали на полоски по 7 цифр в каждой. Докажите, что
любое 7-значное число встретится
\вСтрочку
\пункт  хотя бы на одной полоске;
\пункт  на бесконечном числе полосок.
\кзадача



\end{document}
