% !TeX encoding = windows-1251
\documentclass[a4paper,12pt]{article}
\usepackage{newlistok}

\УвеличитьВысоту{1.7truecm}
\УвеличитьШирину{0.8truecm}

\Заголовок{Целые числа\т 2}
\НомерЛистка{15}
\ДатаЛистка{10.2014}

\begin{document}
\СоздатьЗаголовок

\опр
Если число~$d$ делит числа~$a$ и~$b$, то $d$~называется \выд{общим делителем} чисел $a$ и~$b$. Наибольший среди общих делителей чисел $a$ и~$b$ называется \выд{наибольшим общим делителем} $a$ и~$b$ (обозначение: $(a,b)$). В~том случае, когда $(a,b)=1$, говорят, что числа $a$ и~$b$ \выд{взаимно простые}.
\копр

\соглашение
Пусть $a$ и~$b$\т два фиксированных целых числа.
В данном листке через $I$ будем обозначать множество всех целых чисел, 
представимых в~виде $ax+by$ (где $x$ и~$y$\т целые числа).
\ксоглашение


\ввзадача[о сумме идеалов]
Пусть $d$\т наименьшее положительное число в~$I$. Докажите, что
\невСтрочку
\пункт
каждое число из $I$ делится на любой общий делитель чисел $a$ и~$b$ (а~значит, и~на $(a,b)$);
\пункт
каждое число из $I$ делится на~$d$;
\пункт
$d=(a,b)$;
\пункт
число $d=(a,b)$ является наименьшим натуральным числом, делящимся на любой общий делитель~$a$~и~$b$.
\кзадача



\ввзадача[Алгоритм Евклида]
Пусть $a$ и $b$\т два фиксированных натуральных числа. Будем последовательно заменять большее из этих чисел остаток от деления на меньшее. Докажите, что:
\невСтрочку
\пункт
все числа, которые мы будем получать, лежат в множестве $I$;
\пункт
в некоторый момент мы получим пару $(d, 0)$, $d\ne 0$;
\пункт
$(a,b)=d$;
\пункт
Как именно для данных чисел~$a$ и~$b$ при помощи алгоритма Евклида искать такие целые числа~$x$ и~$y$, что $ax+by=(a,b)$?
\кзадача

\задача
\пункт
Докажите, что для любого натурального~$k$ выполнено $(ka,kb)=k\cdot(a,b)$.\\
\пункт
Докажите, что если $m$\т общий натуральный делитель чисел~$a$ и~$b$, то $(a/m,b/m)=(a,b)/m$.
\кзадача

\задача
Докажите, что числа~$a$ и~$b$ взаимно просты тогда и~только тогда, когда существуют такие целые~$x$ и~$y$, что $ax+by=1$.
\кзадача

\задача
Числа $a$, $b$ и $c$ целые, $(a,b) = 1$. Докажите, что \\
\пункт
если $ac \dv b$,  то $c \dv b$;
\пункт
если $c \dv a$ и $c \dv b$,  то $c \dv ab$.
\кзадача

\ввзадача[Основная теорема арифметики] Докажите следующие утверждения:
\сНовойСтроки
\пункт для каждого целого $n>1$ найдутся такие простые числа
$p_1,\dots,p_k$, что $n=p_1\cdot \dots \cdot  p_k$; %($k$ --- натуральное);
\пункт[каноническое разложение] Для каждого целого $n>1$ найдутся такие
различные простые $p_1,\dots,p_k$ и натуральные $\al_1,\dots,\al_k$, что
$n=p_1^{\al_1}\cdot \dots \cdot p_k^{\al_k}$; %($k$ --- натуральное);
\пункт разложения из пунктов а) и б) единственны с точностью до порядка
сомножителей.
\кзадача

\задача
Числа $a$, $b$, $c$, $n$ натуральные, $(a,b) = 1$, $ab = c^n$. Найдётся ли такое целое $x$, что~$a = x^n$?
\кзадача

\задача
Решите в натуральных числах уравнение $x^{42} = y^{55}$.
\кзадача

\задача
Найдите каноническое разложение числа
\пункт
$2013$,
\пункт
$1002001$,
\пункт
$17!$,
\пункт
$C_{20}^{10}$.
\кзадача


\опр
\emph{Общим кратным} ненулевых целых чисел~$a$ и~$b$ называется целое число, которое делится как на~$a$, так и~на~$b$. Наименьшее среди положительных общих кратных называется \emph{наименьшим общим кратным} чисел~$a$ и~$b$. Обозначение:~$[a,b]$.
\копр

\взадача
Пусть $a=p_1^{\al_1}\cdot p_2^{\al_2}\cdot\ldots\cdot p_n^{\al_n},\,\,b=p_1^{\beta_1}\cdot p_2^{\beta_2}\cdot\ldots\cdot p_n^{\beta_n}$, причём $\al_i,\,\beta_i\geqslant0$.\\
\пункт
Найдите~$(a,b)$ и~$[a,b]$.
\пункт
Докажите, что $ab=(a,b)\cdot[a,b]$.
\кзадача

\взадача
Докажите, что любое общее кратное чисел~$a$ и~$b$ делится на~$[a,b]$.
\кзадача

\задача
Верно ли, что
\пункт
$[ca,cb]=c\cdot[a,b]$ при $c\in\N$;
\пункт
числа $[a,b]/a$ и~$[a,b]/b$ взаимно просты?
\кзадача

\задача
Про натуральные числа $a$ и $b$ известно, что $(a,b) = 15$, $[a,b] = 840$. Найдите $a$ и $b$.
\кзадача


\ЛичныйКондуит{0mm}{6mm}
%\GenXML

\end{document}




