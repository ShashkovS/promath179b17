% !TeX encoding = windows-1251
%\documentstyle[11pt, russcorr, ll]{article}

%\documentclass[11pt]{article}
%\usepackage[epsfigures,ALGsize,diagrams]{lll}

\documentclass[a4paper,12pt]{article}
\usepackage{newlistok}
%\documentstyle[11pt, russcorr, listok]{article}

\УвеличитьШирину{1truecm}
\УвеличитьВысоту{2.7truecm}
%\hoffset=-2.5truecm
%\voffset=-27.3truemm
%\documentstyle[11pt, russcorr, ll]{article}

\pagestyle{empty}

\Заголовок{Непрерывность в задачах}
\Подзаголовок{}
\НомерЛистка{26$\frac12$}
\ДатаЛистка{01.2016}

%\overfullrule=3pt

\begin{document}

\СоздатьЗаголовок


\задача Уравнение $x^3+ax+1=0$ имеет три действительных корня.
Докажите, что найд\"ется такое $\varepsilon>0$, что для всякого
$b\in(a-\varepsilon;a+\varepsilon)$ уравнение $x^3+bx+1=0$ имеет три
действительных корня.
\кзадача

\задача
Функции $f$ и $g$ непрерывны на $\R$.
Верно ли, что функция $\max(f(x),g(x))$
непрерывна на $\R$?
\кзадача

\задача
Пусть $f$ непрерывна на $\R$,  и пусть уравнение $f(x)=x$ не имеет корней.
Докажите, что уравнение $f(f(x))=x$ также не имеет корней.
\кзадача

\задача
Запишем каждое  $x\in(0;1)$ в троичной системе счисления:
$x=0,a_1a_2a_3\dots$
%(где среди цифр $a_1$, $a_2$, $a_3$, \dots
%нет бесконечного числа двоек подряд)
(без бесконечного числа двоек подряд),
и положим
$f(x)=a_1/10+a_2/10^2+a_3/10^3+\dots.$ Будет ли %так определенная функция
$f$  непрерывна на $(0;1)$?
\кзадача

\задача
Пусть $f:\R\rightarrow\R$ --- непрерывная функция, $(a;b)$ --- любая точка
на координатной плоскости. Докажите, что среди всех точек графика функции $f$
найд\"ется такая, расстояние от которой до точки $(a,b)$ минимально
(то есть не больше, чем расстояние от любой другой точки графика $f$ до
$(a;b)$).
\кзадача

%\задача
%Однажды утром (в 9 часов) отшельник, живущий на
%склоне горы, начал восхождение к е\"е~вершине и добрался туда в 8 часов
%вечера. В 9~часов утра
%следующего дня отшельник начал спуск с вершины (по той же дороге,
%что и поднимался) и в 8 часов вечера вернулся домой.
%Докажите, что на дороге есть точка, которую отшельник
%проходил~в~одно и то же время на пути к вершине и на пути к дому.
%%(Скорость отшельника при движении~непостоянна.)
%\кзадача

\задача
Однажды утром (в $9^{00}$) турист
%начал подъ\"ем на
вышел из лагеря к вершине горы и добрался туда в $20^{00}$.
В $9^{00}$ следующего дня он начал спуск с вершины (по той же тропе,
что и поднимался) и в $20^{00}$ вернулся в лагерь.
Найд\"ется ли на тропе точка, которую турист
проходил в одно и то же время %на пути к вершине и на пути к лагерю?
в день подъ\"ема и в день спуска?
%(Скорость отшельника при движении~непостоянна.)
\кзадача


\задача Верно ли, что каждая внутренняя точка любого выпуклого многогранника
принадлежит какому-то отрезку, концы которого находятся на р\"ебрах этого
многогранника?
\кзадача

%\задача
%Докажите, что $f\in C(\R)$ тогда и только тогда, когда для любого интервала
%из $\R$ множество точек, отображающихся в этот интервал, представимо
%в виде объединения интервалов.
%\кзадача


%\задача
%Функции $f$ и $g$ непрерывны на $\R$.
%Верно ли, что функция $\max(f(x),g(x))$
%непрерывна на $\R$?
%\кзадача

\задача Выпуклый многоугольник $M$, прямая $l$ и точка $A$ лежат в одной
плоскости. Докажите, что~най\-д\"ет\-ся прямая $l'$, которая
делит $M$ на две равновеликие части и
\вСтрочку
\пункт параллельна $l$;
\пункт проходит через $A$.
\кзадача


\задача Пусть $S^1$ --- окружность на плоскости.
\пункт
Дайте определение
непрерывной функции $f:S^1\rightarrow\R$.
\пункт
Пусть $f:S^1\rightarrow\R$ непрерывна.
Докажите, что %для любой функции $f$ из п.~а)
у $S^1$ найд\"ется такой диаметр $AB$, что
$f(A)=f(B)$.
\кзадача

\задача {\small\sc (Теоремы о блинах).} На сковороде лежат два блина (многоугольники). Докажите, что\\
%\вСтрочку
\пункт
есть прямая, делящая каждый блин на две равновеликие части
(часть может состоять из многих кусков);\\
\пункт
%На сковороде лежит блин (многоугольник).
найдутся две перпендикулярные прямые, делящие первый блин
на четыре равновеликие части.
\кзадача


\задача
Фигура ограничена и выпукла. Докажите: $\!\!\!\!$
\пункт вокруг неё можно описать квадрат;     $\!\!\!\!$
\пункт в неё~можно вписать квадрат, если она центрально-симметрична;
\пункт некая прямая делит пополам её площадь и периметр.
\кзадача

%\сзадача
%На сковороде лежат 2 блина (многоугольники). Докажите,
%что можно
%\вСтрочку
%\пункт
%одной прямой разделить каждый блин на 2 равновеликие части;
%%(часть может состоять из нескольких кусков).
%\пункт
%двумя перпендикулярными прямыми разделить первый блин
%на 4 равновеликие части.
%\кзадача



%\задача %{\small\sc (Теорема о монотонной функции.)}
%Функция $f$ непрерывна на промежутке $I\in \R$.
%Докажите, что $f$ обратима на $I$ тогда и только тогда,
%когда $f$ строго монотонна на $I$
%(прич\"ем обратная функция также строго
%монотонна и непрерывна).
%\кзадача


%\сзадача Функция $f$, определ\"енная на интервале $(a;b)$,
%называется \выд{выпуклой вниз} на $(a;b)$ если
%для любого отрезка $[x;y]\subset(a;b)$ график $f$ на отрезке $[x;y]$
%лежит не выше прямой, соединяющей  точки $(x;f(x))$ и
%$(y;f(y))$.  Докажите, что функция, выпуклая вниз на интервале $(a;b)$,
%непрерывна на н\"ем.
%\кзадача
%
%\сзадача Функция $f$ определ\"ена %и \выд{выпукла вниз}
%на $(a;b)$, прич\"ем для любого отрезка $[x;y]\subset(a;b)$
%график $f$ на $[x;y]$ лежит не выше прямой, соединяющей  точки $(x;f(x))$ и
%$(y;f(y))$).  Докажите, что $f$ непрерывна на $(a;b)$.
%\кзадача

%\раздел{В дополнительный листок}

\задача Функция $f:\R\rightarrow\R$ \выд{выпукла вниз}:
для любого отрезка $[a;b]$ график $f$ на $[a;b]$ лежит не выше прямой,
соединяющей  точки $(a;f(a))$ и $(b;f(b))$.
Докажите, что $f$ непрерывна на $\R$.
\кзадача

\задача
Пусть $f\in C(\R)$, причем $f(\frac{x+y}2) \leq \frac{f(x)+f(y)}2$ при $x,y\in\R$. Докажите, что $f$ выпукла вниз на $\R$.
\кзадача

\задача
%Существует ли непрерывная функция $f:\R\rightarrow\R$, значения которой
%в рациональных точках иррациональны, а в иррациональных --- рациональны?
%Могут ли все значения в точках из $\Q$
%непостоянной непрерывной на $\R$ функции быть иррациональными?
Пусть $f\in C(\R)$, $f$ не является константой.
Докажите, что найд\"ется такое $r\notin\Q$, что $f(r)\notin\Q$.
\кзадача


%\сзадача
%%Любое ли счетное подмножество $\R$
%%будет множеством точек разрыва некоторой функции?
%Пусть $M\subset\R$ --- сч\"етно. Найд\"ется ли %функция
%$f:\R\rightarrow\R$,
%%разрывная в точках из $M$ и непрерывная в остальных?
%множество точек разрыва которой есть $M$?
%\кзадача

\задача
Пусть $M\subset\R$ --- сч\"етно. Найд\"ется ли %функция
$f:\R\rightarrow\R$,
множество точек разрыва которой есть $M$?
\кзадача

\сзадача
$f$ монотонна и ограничена на отрезке $I$. Докажите: %, что %множество
%точек разрыва функции $f$ (на этом отрезке) не более чем сч\"етно.
у $f$ не более чем счетное число разрывов на $I$.
\кзадача

\сзадача
Найдётся ли функция, принимающая
на каждом отрезке все действительные значения?
\кзадача


%%\опр
%\сзадача
%%Говорят, что
%%график функции $f:M\to\R$ имеет \выд{горизонтальную хорду длиной} $\delta$,
%%если найдутся такие точки $x,y\in M$, что $|x-y|=\delta$ и
%%$f(x)=f(y).$
%%\копр
%Пусть $f\in C([0;1])$ и $f(0)=f(1)$.
%Для каких чисел $a>0$ у графика $f$ обязательно найд\"ется
%\выд{горизонтальная хорда длины} $a$
%(т.~е.~найдутся такие точки $x,y\in [0;1]$, что $|x-y|=a$ и
%$f(x)=f(y)$).
%\кзадача

\опр
Пусть функция $f$ определена в некоторой окрестности $\cal U$ точки $a$. Для каждого интервала $I\subset\cal U$ назовем \выд{колебанием функции $f$ на интервале $I$} число
$$\omega_I(f)=\sup\{f(x)\ |\ x\in I\}-\inf\{f(x)\ |\ x\in I\}.$$
\выд{Колебанием функции $f$ в точке $a$} назовем число
$\omega_a(f)=\inf\{\omega_I\ |\ I\subset\R,\ a\in I\}.$
\копр

\задача
Докажите, что $f$ непрерывна
в точке $a$ %тогда и только тогда, когда
если и только если $\omega_a(f)=0$.
\кзадача

\опр
Говорят, что
график функции $f:M\to\R$ имеет \выд{горизонтальную хорду длиной} $\delta$,
если найдутся такие точки $x,y\in M$, что $|x-y|=\delta$ и
$f(x)=f(y).$
\копр

\задача
Пусть $f$ --- непрерывная периодическая функция на прямой с периодом $T$. Докажите, что ее график имеет горизонтальную хорду длины %$l$, где
\вСтрочку
а) $T/2$;
в) $T/3$;
г) $Tq$ при любом $q\in\Q$;
д) $l$ при любом $l\in\R$.
\кзадача

\задача
Пусть $f\in C([0;1])$ и $f(0)=f(1)$.\\
%\вСтрочку
\пункт
Докажите, что для любого $n\in\N$ у графика $f$ найд\"ется
\выд{горизонтальная хорда длины} $1/n$.\\
%(т.~е.~найдутся такие точки $x,y\in [0;1]$, что $|x-y|=1/n$ и $f(x)=f(y)$).
\пункт
Есть ли ещ\"е  числа $a$, для которых у графика $f$ обязательно найд\"ется
горизонтальная хорда длины $a$?
\кзадача




\сзадача
%Пол представляет собой график непрерывной функции от двух переменных.
%От времени
Некогда плоский пол стал неровным (но отличается от плоскости не слишком сильно).
%Из-за этого квадратный табурет стоит на полу только тремя ножками --- четвертая висит в воздухе.
Докажите, что можно так поставить на пол %передвинуть
квадратный табурет, чтобы он стоял на полу всеми четырьмя ножками.
\кзадача

\сзадача
Среди ровной степи стоит гора. На вершину ведут две тропы
(считаем их графиками непрерывных функций),
не опускающиеся ниже уровня степи.
Два альпиниста одновременно начали подъ\"ем (по разным тропам),
соблюдая условие:
%в каждый момент времени
вс\"е время быть на одинаковой высоте.
Смогут ли они %альпинисты
достичь вершины, двигаясь непрерывно, если
\вСтрочку
\пункт
тропы состоят из конечного числа подъ\"емов и спусков;
\пункт в общем случае?
\кзадача

%\УстановитьГраницы{0cm}{2.3truecm}

%\end{document}

\сзадача %{\small\sc (Н.~Н.~Константинов)}
Из $A$ в $B$ ведут две дороги,
не пересекающие друг друга и сами себя.
Две машины,~\hbox{связанные}~ве\-р\"ев\-кой длины $15$ м,
%смогли
проехали из $A$ в $B$ по разным дорогам,
не порвав вер\"евки.
%смогли одновременно выехать из $А$ и
%(по разным дорогам) из $А$ в $Б$,
%одновременно прибыть в $Б$, не порвав вер\"евки.
Два круглых воза радиуса $8$~м
выезжают одновременно по разным дорогам, один из $A$ в $B$,
% по первой дороге,
другой из $B$ в $A$. %по второй. % дороге.
Могут ли они разминуться?
%, не коснувшись,
%два круглых воза радиуса $l$, центры которых движутся по этим дорогам
%навстречу друг другу?
\кзадача
%\ВосстановитьГраницы

\ЛичныйКондуит{0mm}{8mm}


%\СделатьКондуит{6.7mm}{7.5mm}


\end{document}





\УстановитьГраницы{0cm}{2truecm}
\сзадача {\small\sc (Н.~Н.~Константинов).}
Из $А$ в $Б$ ведут две дороги
(не пересекающие друг друга и самих себя).
%Известно, что
Две машины, связанные вер\"евкой длины, меньшей $2l$,
проехали (по разным дорогам) из $А$ в $Б$,
не порвав вер\"евки. Могут ли разминуться, не коснувшись,
два круглых воза радиуса $l$, центры которых движутся по этим дорогам
навстречу друг другу?
\кзадача
\ВосстановитьГраницы


\задача
Функции $f$ и $g$ непрерывны на $\R$, прич\"ем $f(x)=g(x)$ для
любого $x\in\Q$. Докажите, что тогда $f=g$
(иначе говоря,
непрерывная функция определяется своими значениями в рациональных точках.)
\кзадача

\задача
Найдите все непрерывные на $\R$ функции $f$, %удовлетворяющие условию:
такие что
$f(x+y)=f(x)+f(y)$ для любых $x,y\in\R$.
\кзадача


\раздел{Для контрольных работ.}

\задача Даны $f\in C(\R)$ и положительное число $k$.
%Определим \лк срезку\пк\ $g$ функции $f$
%следующим образом:
Назов\"ем \лк срезкой\пк\ функции
$f$ такую функцию $g$, что
$g(x)=k$, если $f(x)>k$;
$g(x)=f(x)$,  если $|f(x)|\le k$;
$g(x)=-k$, если $f(x)<-k$.
Докажите, что $g\in C(\R)$.
%(функция $g$ называется \лк срезкой\пк\ функции $f$).
\кзадача

\задача
Запишем каждое  $x\in(0;1)$ в троичной системе счисления
$x=0,a_1a_2a_3\dots$
%(где среди цифр $a_1$, $a_2$, $a_3$, \dots
%нет бесконечного числа двоек подряд)
(без бесконечного числа двоек подряд)
и положим
$f(x)=a_1/10+a_2/10^2+a_3/10^3+\dots$. Будет ли %так определенная функция
$f$  непрерывна на $(0;1)$?
\кзадача

\end{document}



\задача Какие из следующих функций непрерывны (на своей области
определения):
\вСтрочку
\пункт $\sin x$.
\пункт Как проще всего вывести  из пункта г) непрерывность
функции $\cos x$?
\пункт Докажите непрерывность функций $\tg x$ и $\ctg x$.
\кзадача




\сзадача \пункт Приведите пример
непостоянной периодической функции, не
имеющей наименьшего положительного периода.
\пункт Пусть периодическая функция имеет точку непрерывности.
Докажите, что она либо является постоянной функцией, либо среди е\"е
положительных периодов есть наименьший.
\кзадача

\задача
{\small\sc (Теорема о промежуточном значении.)}
Пусть $f\in C([a;b]),$ $f(a) < f(b)$.
Верно ли, что для любого числа $c \in [f(a), f(b)]$ существует такая
точка $x\in [a,b]$, что $f(x) = c$?
\кзадача 