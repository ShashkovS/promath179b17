% !TeX encoding = windows-1251
\documentclass[a4paper,11pt]{article}
\usepackage{newlistok}

\УвеличитьШирину{1.5truecm}
\УвеличитьВысоту{2.5truecm}
% \hoffset=-2.5truecm
% \voffset=-27.3truemm
%\def\hang{\hangindent\parindent}
%\documentstyle[11pt, russcorr, ll]{article}



\Заголовок{Производная. Экстремумы}
\Подзаголовок{}
\НомерЛистка{29}
\ДатаЛистка{03.2016}

%\overfullrule=3pt

\begin{document}

\СоздатьЗаголовок

%\раздел{Нули производной}


%\noindent {\Bf Соглашение.}
%Скажем, что функция $f$ удовлетворяет условию $(*)$, если $f$
%непрерывна на отрезке $[a,b]$  и дифференцируема на интервале $(a,b)$.

\задача
Пусть $f$
непрерывна на $[a,b]$, дифференцируема на $(a,b)$
и $f'(x_0)>0$ в некоторой точке~\hbox{$x_0\in (a,b)$.}
\пункт Найдется ли такая окрестность $U$ точки $x_0$, что
для всех \hbox{$x\in U$} если $x>x_0$, то
$f(x)>f(x_0)$, а если $x<x_0$, то $f(x)<f(x_0)$?
\спункт Верно ли, что $f$ монотонно
возрастает в некоторой окрестности $x_0$?
\кзадача


\опр Точку $c$ называют точкой \выд{локального максимума} $f$,
если $f(c)\ge f(x)$ для всех $x$ из некой окрестности $c$.
Если верно строгое неравенство, говорят о \выд{строгом} локальном максимуме.
Аналогично определяют точку\выд{(строгого) локального минимума}.
Такие точки называют точками \выд{(строгого) локального экстремума}.
\копр

%\задача Докажите, что у любой функции $f$,
%непрерывной на отрезке $[a,b]$  и дифференцируемой на интервале $(a,b)$,
%существует точка локального максимума и точка локального минимума.
%\кзадача

\задача \пункт \выд{(Теорема Ферма)} Пусть $f$
непрерывна на $[a,b]$  и дифференцируема на $(a,b)$.
Докажите, что если $x\in (a,b)$ --- точка локального
максимума (минимума) $f$, то $f'(x)=0$. \пункт Верно ли обратное?
%\пункт Верно ли, что если $f'(x)=0$ для $x\in (a,b)$, то $x$ --- точка
%локального максимума или минимума?
\кзадача

\задача
\пункт Пусть $f$ непрерывна на $[a,b]$ и дифференцируема на $(a,b)$. Тогда максимум (минимум) $f$ на $[a,b]$ достигается и может быть только в $a$, $b$ и в точках $x\in(a,b)$, где $f'(x)=0$. \пункт Если $f$ дифференцируема на $\R$ и $f(x)\rightarrow+\infty$ при $x\rightarrow\pm\infty$, то минимум $f$ на $\R$ достигается, и в точке минимума $x$ обязательно $f'(x)=0$.
\кзадача


\задача Докажите для всех $x$:
\label{ex}
\вСтрочку
\пункт $x^4+x^3\ge -\frac{3^3}{4^4}$;
\пункт $x^6-6x+5\ge 0$;
\пункт $x^4-4x^3+10x^2-12x+5\ge 0$.
\кзадача


%\задача Найдите все точки локальных экстремумов:
%функций на интервале $(-\infty ,\infty )$:
%\label{ex}
%\вСтрочку
%\пункт $x+1$;
%\пункт $x^2-1$;
%\пункт $x^3+x$
%\пункт $\sin x$.
%\кзадача

\задача Найдите наибольшее и наименьшее значение при $x\in [0,1]$ функций
из задачи~\ref{ex}.
\кзадача



\задача Найдите наименьшее значение % выражения
при $x>0$:
\вСтрочку
\пункт $x+\frac{1}{x}$;
\пункт $x+\frac{1}{x^2}$;
\пункт $x^2+2x+\frac{4}{x}$.
\кзадача

\задача \выд{(Теорема Ролля)} %Пусть функция $f$ удовлетворяет условию $(*)$
Пусть $f$ непрерывна на $[a,b]$  и дифференцируема на $(a,b)$,
и, кроме того, $f(a)=f(b)$. Докажите, что найд\"ется такая точка $x\in
(a,b)$, что $f'(x)=0$.
\кзадача

\задача  \выд{(Теорема Лагранжа)}
%\вСтрочку
Пусть $f$ непрерывна на $[a,b]$  и дифференцируема на $(a,b)$.
%Пусть функция $f$ удовлетворяет условию $(*)$.
Докажите, что найд\"ется такое $x\in (a,b)$, что
$f'(x)=\frac{f(b)-f(a)}{b-a}$ и
объясните геометрический смысл этой теоремы. % Лагранжа.
\кзадача

\задача %Пусть функция $f$ удовлетворяет условию $(*)$
Пусть $f$ непрерывна на $[a,b]$  и дифференцируема на $(a,b)$.
Докажите, что если для всех $x\in (a,b)$ выполнено:
\вСтрочку
\пункт $f'(x)=0$, то $f$ постоянна
на $[a,b]$.
\пункт $f'(x)>0$,
то $f$ возрастает на $[a,b]$.
\кзадача

\задача Докажите, что для для всех $x>0$ выполнены неравенства:\\
\вСтрочку
%\пункт $e^x>1+x$;
%\пункт $e^x>1+x+\frac{x^2}{2}$;
\пункт $\sin x>x-\frac{x^3}{6}$;
\пункт $1-\frac{x^2}{2}<\cos x<1-\frac{x^2}{2!}+\frac{x^4}{4!}$;
\спункт $e^x>1+x+\frac{x^2}{2}+\ldots +\frac{x^n}{n!}$, где $n\in \N$.
\кзадача

\сзадача Найдите все дифференцируемые на $\R$ функции $f$, такие что
$f'(x)=f(x)$ для всех $x\in \R$.
\кзадача


\раздел{***}


\задача
\пункт Какую наибольшую площадь может иметь трапеция,
три стороны которой равны~1?\\
\пункт Какова наибольшая возможная площадь %может иметь
четыр\"ехугольника, 3 стороны которого равны~1?\\
\пункт У какого равностороннего шестиугольника со стороной 1 %имеет
площадь наибольшая?
\кзадача

\задача
Из пункта $A$, находящегося в лесу в 5 км от прямой
дороги, пешеходу нужно попасть в пункт $B$, расположенный
на  этой дороге в 13 км от $A$.
Наибольшая скорость пешехода на дороге --- 5 км/ч, а в лесу --- 3~км/ч.
За какое наименьшее время пешеход сможет попасть из $A$ в $B$?
\кзадача

\задача
Даны две точки $A$ и $B$ по разные стороны от прямой $l$, разделяющей
две среды. Требуется найти такую точку $D$ на прямой $l$,
чтобы время преодоления
светом пути $ADB$ было минимальным при условии, что скорость распространения
света в верхней среде $v_1$, а в нижней --- $v_2$.
Докажите, что такая точка $D$ существует и определяется условием
$\sin\alpha_1/\sin\alpha_2=v_1/v_2$, где $\alpha_1$ и $\alpha_2$ ---
углы, образованные прямыми $AD$ и $BD$ с прямой, проходящей через точку
$D$ перпендикулярно $l$.
\кзадача

\задача
Найдите точку параболы $y=x^2$, ближайшую к точке $(-1;2)$.
\кзадача

\задача
Найдите на эллипсе $4\,x^2+9\,y^2=36$ точку, касательная в которой
образует вместе с осями координат треугольник минимально возможной площади.
\кзадача

\задача В круглый бокал, осевое сечение которого --- график
функции $y=x^4$, опускают вишенку --- шар радиусом $r$.
При каком наибольшем $r$ шар косн\"ется нижней точки дна?
(Другими словами, каков максимальный радиус $r$ круга, лежащего в
области $y\leq x^4$ и содержащего начало координат)?
\кзадача




\задача Пусть %функция
$f$ определена и дифференцируема
на $(a,b)$.
%\сНовойСтроки
\вСтрочку
\пункт Верно ли, что $f'$ непрерывна на $(a,b)$?\\
\пункт Пусть у $f'$ существуют пределы слева и справа в точке $x_0\in(a,b)$.
Верно ли, что %эти пределы
они совпадают?\\
%\кзадача
%
%\сзадача
\пункт [Теорема Дарбу]
%Пусть функция $f$ определена и дифференцируема на некотором интервале, и
Пусть $[c,d]\subset(a,b)$.
Докажите, что $f'$  принимает на $[c,d]$ все значения между
$f'(c)$ и $f'(d)$.
\кзадача

\сзадача
Пусть функция $f$ дифференцируема в некоторой окрестности $\cal U$
точки $a$, прич\"ем
%\сНовойСтроки
%\пункт
%Пусть для всех $x\in\cal U$ %из этой окрестности
%если $x<a$, то $f'(x)<0$, и если $x>a$, то $f'(x)>0$. % из этой окрестности.
%Докажите, что $a$ --- точка строгого локального минимума $f$.
%Есть ли аналогичная теорема о локальном максимуме?
%\пункт
%\кзадача
%\задача
%Пусть функция $f$ дифференцируема в некоторой окрестности точки $a$, прич\"ем
%Пусть $f'(x)<0$ для всех $x\ne a$ из $\cal U$. %этой окрестности
%Может ли $a$ быть точкой локального %экстремума (
%максимума или минимума $f$?
%\кзадача
%\спункт
%
%\сзадача
%Пусть функция $f$ дифференцируема в некоторой окрестности точки $a$, прич\"ем
$a$ --- точка строгого локального минимума $f$. Всегда ли
найд\"ется ли такая окрестность точки $a$, что $f'(x)<0$ для всех $x<a$
из этой окрестности, и $f'(x)>0$ для всех $x>a$ из этой окрестности?
\кзадача


\ЛичныйКондуит{0mm}{8mm}

%\СделатьКондуит{4.5mm}{7.5mm}

%\GenXMLW



\end{document}

\раздел{Нули производной}


%\noindent {\Bf Соглашение.}
%Скажем, что функция $f$ удовлетворяет условию $(*)$, если $f$
%непрерывна на отрезке $[a,b]$  и дифференцируема на интервале $(a,b)$.

\задача
Пусть $f$
непрерывна на $[a,b]$, дифференцируема на $(a,b)$
и $f'(x_0)>0$ в некоторой точке~\hbox{$x_0\in (a,b)$.}
\пункт Найдется ли такая окрестность $U$ точки $x_0$, что
для всех \hbox{$x\in U$} если $x>x_0$, то
$f(x)>f(x_0)$, а если $x<x_0$, то $f(x)<f(x_0)$?
\спункт Верно ли, что $f$ монотонно
возрастает в некоторой окрестности $x_0$?
\кзадача


\опр Точка $x_0$ называется точкой \выд{локального максимума} функции $f$,
если $f(x_0)\ge f(x)$ для всех $x$ из некоторой окрестности $x_0$.
Если выполнено строгое неравенство $f(x_0)> f(x)$, говорят о строгом локальном максимуме.
Аналогично определяется точка \выд{(строгого) локального минимума}.
Такие точки называют точками \выд{(строгого) локального экстремума}
\копр

%\задача Докажите, что у любой функции $f$,
%непрерывной на отрезке $[a,b]$  и дифференцируемой на интервале $(a,b)$,
%существует точка локального максимума и точка локального минимума.
%\кзадача

\задача \пункт \выд{(Теорема Ферма)} Пусть $f$
непрерывна на $[a,b]$  и дифференцируема на $(a,b)$.
Докажите, что если $x\in (a,b)$ --- точка локального
максимума (минимума) $f$, то $f'(x)=0$. \пункт Верно ли обратное?
%\пункт Верно ли, что если $f'(x)=0$ для $x\in (a,b)$, то $x$ --- точка
%локального максимума или минимума?
\кзадача

\задача
Пусть функция определена на отрезке $[a,b]$. Тогда точками её локального экстремума
\кзадача


\задача Докажите для всех $x$:
\label{ex}
\вСтрочку
\пункт $x^4+x^3\ge -\frac{3^3}{4^4}$;
\пункт $x^6-6x+5\ge 0$;
\пункт $x^4-4x^3+10x^2-12x+5\ge 0$.
\кзадача


%\задача Найдите все точки локальных экстремумов:
%функций на интервале $(-\infty ,\infty )$:
%\label{ex}
%\вСтрочку
%\пункт $x+1$;
%\пункт $x^2-1$;
%\пункт $x^3+x$
%\пункт $\sin x$.
%\кзадача

\задача Найдите наибольшее и наименьшее значение при $x\in [0,1]$ функций
из задачи~\ref{ex}.
\кзадача



\задача Найдите наименьшее значение % выражения
при $x>0$:
\вСтрочку
\пункт $x+\frac{1}{x}$;
\пункт $x+\frac{1}{x^2}$;
\пункт $x^2+2x+\frac{4}{x}$.
\кзадача

\задача
\пункт Какую наибольшую площадь может иметь трапеция,
три стороны которой равны~1?\\
\пункт Какова наибольшая возможная площадь %может иметь
четыр\"ехугольника, 3 стороны которого равны~1?\\
\пункт У какого равностороннего шестиугольника со стороной 1 %имеет
площадь наибольшая?
\кзадача

\задача \выд{(Теорема Ролля)} %Пусть функция $f$ удовлетворяет условию $(*)$
Пусть $f$ непрерывна на $[a,b]$  и дифференцируема на $(a,b)$,
и, кроме того, $f(a)=f(b)$. Докажите, что найд\"ется такая точка $x\in
(a,b)$, что $f'(x)=0$.
\кзадача

\задача  \выд{(Теорема Лагранжа)}
%\вСтрочку
Пусть $f$ непрерывна на $[a,b]$  и дифференцируема на $(a,b)$.
%Пусть функция $f$ удовлетворяет условию $(*)$.
Докажите, что найд\"ется такое $x\in (a,b)$, что
$f'(x)=\frac{f(b)-f(a)}{b-a}$ и
объясните геометрический смысл этой теоремы. % Лагранжа.
\кзадача

\задача %Пусть функция $f$ удовлетворяет условию $(*)$
Пусть $f$ непрерывна на $[a,b]$  и дифференцируема на $(a,b)$.
Докажите, что если для всех $x\in (a,b)$ выполнено:
\вСтрочку
\пункт $f'(x)=0$, то $f$ постоянна
на $[a,b]$.
\пункт $f'(x)>0$,
то $f$ возрастает на $[a,b]$.
\кзадача

\задача Докажите, что для для всех $x>0$ выполнены неравенства:\\
\вСтрочку
%\пункт $e^x>1+x$;
%\пункт $e^x>1+x+\frac{x^2}{2}$;
\пункт $\sin x>x-\frac{x^3}{6}$;
\пункт $1-\frac{x^2}{2}<\cos x<1-\frac{x^2}{2!}+\frac{x^4}{4!}$;
\спункт $e^x>1+x+\frac{x^2}{2}+\ldots +\frac{x^n}{n!}$, где $n\in \N$.
\кзадача

\сзадача Найдите все дифференцируемые на $\R$ функции $f$, такие что
$f'(x)=f(x)$ для всех $x\in \R$.
\кзадача

%\задача
%Пусть функция $f$ дифференцируема на $\R$,
%и для каждой точки $a\in\R$ либо $f(a)=0$, либо $f'(a)=0$.
%Докажите, что $f$ --- константа.
%\кзадача


----------

\раздел{Касательная}

%Рассмотрим график функции $y=f(x)$. Фиксируем такую точку $(x_0,f(x_0))$,
\опр
Пусть функция $f$ определена в некоторой окрестности $U$ точки $x_0$.
Для каждой точки $x\in U$, $x\ne x_0$, рассмотрим прямую $l(x)$, проходящую
через точки $(x_0,f(x_0))$ и $(x,f(x))$. % (она называется секущей).
Если существует предельная прямая для семейства прямых $l(x)$
при $x\to x_0$, то она называется {\em касательной} к графику
$f$ в точке $x_0$. %Уточните это определение и
\копр

\взадача
%\пункт
Напишите уравнение касательной к графику функции $f(x)$ в точке $x_0$.
%\пункт Совпадает ли оно для окружности %(график функции $y=\sqrt{R^2-x^2}$)
%с известным из~геометрии?
\кзадача


\задача
Под каким углом пересекаются кривые:
\вСтрочку
\пункт
$y=x^2$ и $x=y^2$;
\пункт
$y=\sin x$ и $y=\cos x$?
\кзадача

\задача
Найдите геометрическое место точек, из которых парабола $y=x^2$
видна под прямым углом.
\кзадача

\задача
Докажите, что отрезок любой касательной к графику функции $y=1/x$,
концы которого расположены на осях координат, делится точкой касания
пополам.
\кзадача

\сзадача
%\пункт
Параллельный пучок лучей, падающий на параболу $y=x^2$ по
вертикали сверху, отражается от не\"е по закону
\лк угол падения равен углу отражения\пк.
Докажите, что все лучи этого пучка после первого отражения
пройдут через одну и ту же точку, и найдите эту точку.
%\пункт
%Решите эту задачу для произвольной параболы $y=ax^2+bx+c$, где $a>0$.
\кзадача

\сзадача
%Гипербола --- это геометрическое место точек, разность расстояний
%от которых до двух данных точек $F_1$ и $F_2$ постоянна.
Дана гипербола с фокусами $F_1$ и $F_2$.
Докажите, что поток лучей от точечного источника света в $F_1$, отразившись
от гиперболы, предстанет стороннему наблюдателю как поток лучей от точечного
источника~в~$F_2$.
\кзадача

%\сзадача
%\пункт Из точки $A$ проведены касательные $AB$ и $AC$ к эллипсу с фокусами $F_1$ и $F_2$. Докажите, что $\angle F_1AB=\angle F_2AC.$
%\пункт Докажите, что луч, выпущенный из внутренней точки эллипса, отражаясь от зеркальных стенок эллипса, будет всегда касаться некоторого другого эллипса или гиперболы, если он не проходит через фокусы эллипса и не летает по одной прямой.
%\кзадача




%\noindent {\Bf Соглашение.}
%Скажем, что функция $f$ удовлетворяет условию $(*)$, если $f$
%непрерывна на отрезке $[a,b]$  и дифференцируема на интервале $(a,b)$.

\сзадача
%\вСтрочку
%\пункт
%Найдите все $r$, при которых на %координатной
%плоскости $Oxy$
Существует ли окружность, % радиуса $r$,
пересекающая параболу $y=x^2$ ровно в двух точках,
прич\"ем в одной из этих точек у параболы и окружности есть общая
касательная, а в другой --- нет?
%\пункт
%Приведите пример такой окружности (хотя бы для одного~$r$).
\кзадача


====

\задача  Нарисуйте кривые:
\вСтрочку
%\пункт $x=y$;
\пункт $x^2=y^2$;
%\пункт $y=x^2$;
\пункт $х^2y-xy^2=x-y$;
%\пункт $ax^2+by^2=1$, где $a,b$ --- такие числа, что $a>b>0$;
%\пункт $ax^2-by^2=1$, где $a,b$ --- такие числа, что $a>b>0$;
\пункт $y^2=x^3$;\quad
\пункт $y-1=x^3$;\quad
\пункт $y^2-1=x^3$;\quad
\пункт $y^2-x=x^3$;\quad
\пункт $y^2-x^2=x^3$.
\кзадача

\сзадача Нарисуйте кривые:
\вСтрочку
\пункт $x^2=x^4+y^4$;
\пункт $xy=x^6+y^6$;
\пункт $x^3=y^2+x^4+y^4$;
\пункт $x^2y+xy^2=x^4+y^4$.
\кзадача 