% !TeX encoding = windows-1251
\documentclass[a4paper,12pt]{article}
\usepackage{newlistok}

\УвеличитьШирину{1.5cm}
\УвеличитьВысоту{1.5cm}
% \renewcommand{\spacer}{\vfill}

\Заголовок{Метрические пространства: сходящиеся последовательности}
\НомерЛистка{10д-MS2}
\ДатаЛистка{05.2016 г}

\begin{document}
\СоздатьЗаголовок

\опр
Говорят, что последовательность $(x_n)$ точек метрического пространства $(M, d)$ \выд{сходится к $a \in  M$}, если для любого $\ep>0$ найдётся номер $N\in\N$ такой, что если $n>N$, то $d(x_n, a)<\ep$.
%Обозначение: $\limn x_n = a$.
\копр

\задача
Докажите, что последовательность в метрическом пространстве не может иметь двух различных пределов.
\кзадача

\задача
Известно, что $\limn x_n = a$, $\limn y_n = b$. Верно ли, что $\limn d(x_n, y_n) = d(a,b)$?
\кзадача

\задача
Докажите, что если последовательность сходится и предел её лежит внутри некоторого открытого шара, то почти все её члены лежат внутри этого шара.
\кзадача

\задача[Сходимость в $\R^m$]
Рассмотрим арифметическое $m$-мерное пространство $\R^m$ с евклидовой метрикой. Докажите, что $\limn x_n = a$ если и только если $\fa 1\le i\le m$: $\limn x_n^{(i)} = a^{(i)}$ (под $\al^{(i)}$ подразумевается $i$-ая координата точки $\al$).
\кзадача

\задача
Какие последовательности являются сходящимися в\\
\пункт
дискретной метрике;
\пункт
$p$-адической метрике?
\кзадача

\задача
Рассмотрим пространство $M$ ограниченных на отрезке $[a, b]$
функций с равномерной метрикой.
\пункт
Докажите, что если $\limn f_n = g$, то для всех $x \in [a, b]$ имеем $\limn f_n(x) = g(x)$.
\пункт
Верно ли обратное?
\кзадача

\опр
Последовательность $(x_n)$ точек метрического пространства $(M,d)$ называется \выд фундаментальной, если для любого $\ep>0$ найдётся номер $N\in\N$ такой, что если $m,n>N$, то $d(x_m, x_n)<\ep$.
\копр

\задача
\пункт
Докажите, что любая сходящаяся последовательность является фундаментальной.\\
\пункт
Верно ли обратное?
\кзадача

\опр
Метрическое пространство $(M, d)$ называется \выд полным, если любая фундаментальная последовательность в нём сходится.
\копр

\задача
Докажите, что вещественная прямая с естественной метрикой полна.
\кзадача

\задача
Докажите, что пространство $C([a,b])$ с равномерной метрикой является полным.
\кзадача

\опр
Отображение $f\colon M \to M$ из метрического пространства $M$
в себя называется \выд сжимающим, если найдётся такая константа
$0<\theta<1$, что для любых $x,y\in M$: $d(f(x),f(y))<\theta
d(x,y)$.
\копр

\задача
При каких условиях гомотетия на плоскости является сжимающим отображением?
\кзадача

\задача
\пункт
Докажите, что сжимающее отображение $f$ полного метрического пространства $M$ имеет неподвижную точку, то есть $\exi x\in M$: $f(x)=x$.
\пункт
Верно ли это без условия полноты $M$?\\
(Подсказка к пункту \textbf{а}:\,\,{\reflectbox{\hbox{\tiny докажите, что любая последовательность $(f^n(x))$ фундаментальна}}})
\кзадача

\задача
Докажите, что композиция гомотетии с коэффициентом, не равным $\pm 1$ и любого движения имеет неподвижную точку.
\кзадача

\задача[Метод Ньютона]
Пусть функция $\al(x)$ дважды непрерывно дифференцируема (то есть вторая производная непрерывна) на отрезке $[a,b]$, имеет на нём корень $\wt{x}$, причём $\al'(x)\ne0$ всюду на $[a,b]$.
Рассмотрим функцию $f(x) =  x - \dfrac{\al(x)}{\al'(x)}$.
\невСтрочку
\пункт
Докажите, что $\al(\wt{x}) = 0$ тогда и только тогда, когда $f(\wt{x})=\wt{x}$;
\пункт
Докажите, что $f$ и $f'$ непрерывны;
\пункт
Докажите, что найдётся такое $\de>0$, что $f$ на $U_\de(\wt{x})$ осуществляет сжимающее отображение.
\пункт
Что всё это значит и как это применять?
\пункт
Найдите $\sqrt{2}$ с точностью до трёх знаков после запятой.
\кзадача

%\GenXMLW
\ЛичныйКондуит{0.3mm}{6.5mm}
%\СделатьКондуит{8mm}{8mm}

\end{document}
