% !TeX encoding = windows-1251
\documentclass[a4paper, 11pt]{article}
\usepackage{newlistok}
%\documentstyle[11pt, russcorr, listok]{article}
%\newcommand{\0}[1]{\overline{#1}}
%\def\C{\mbox{$\Bbb C$}}

\УвеличитьШирину{1.35truecm}
\УвеличитьВысоту{2.6truecm}
\renewcommand{\spacer}{\vspace{2pt}}
\ВключитьКолонтитул

\begin{document}

\НомерЛистка{26$\frac23$}
\Заголовок{Множества на прямой. Канторово множество}
\ДатаЛистка{01.2016}

\СоздатьЗаголовок

\опр
Множество $M$ \emph{имеет меру ноль}, если для любого
положительного $\varepsilon$ множество $M$ можно покрыть не более
чем счётной системой
интервалов с суммарной длиной не больше $\varepsilon$ (то есть
покрыть такой счётной системой интервалов, что сумма длин любого конечного
набора этих интервалов не больше $\varepsilon$).
\копр

\задача
Докажите, что счётное множество имеет меру ноль.
\кзадача

\задача
Докажите, что
\вСтрочку
\пункт конечное
\пункт счётное
объединение множеств меры ноль имеет меру ноль.
\кзадача

\задача
Верно ли, что определение \лк множества меры ноль\пк\
по сути не изменится, если в определении разрешить
покрытия только конечными системами интервалов?
\кзадача

\задача
Раскрасим прямую так, чтобы точки имели одинаковый цвет,
если и только если
разность между ними рациональна (понадобится несчётное
множество цветов). Рассмотрим какое-нибудь множество $M$, в котором
все точки имеют разный цвет, но при этом все цвета встречаются. Может
ли оно иметь меру ноль?
\кзадача

%\сзадача
%Придумайте множество положительной меры, которое не содержит
%никакого отрезка. Подходит ли множество $M$ из предыдущей задачи?
%\кзадача

\задача
\label{Kantor}
Возьмём отрезок $K_0=[0,1]$. Разделим его на три равные
части и средний интервал $I_1^1=(\frac13, \frac23)$ выкинем.
Первый и третий отрезки образуют множество $K_1$.
Каждый из них разделим на три части
и выкинем средние интервалы $I_1^2=(\frac19, \frac29)$,
$I_2^2=(\frac79, \frac89)$.
Получится множество $K_2$. И так далее: на $n$-м шаге будем
делить каждый из $2^{n-1}$ отрезков, образующих $K_{n-1}$, на три
равные части и выкидывать все средние интервалы
$I_1^n, I_2^n, \dots, I_{2^{n-1}}^n$.
Так получается множество $K_n$, состоящее из $2^n$ отрезков.
Устремим $n$ к бесконечности.
Множество, получающееся в пределе, т.~е. $\bigcap\limits_{n=1}^{\infty} K_n$,
называется \emph{канторовым} (всюду дальше будем обозначать~его~$K$).
%\вСтрочку
\сНовойСтроки
\пункт
Конечно ли это множество? Счётно?
\пункт
Докажите, что канторово множество имеет меру ноль.
\пункт
Является ли оно открытым? Замкнутым?
\кзадача

\задача
%\вСтрочку
\пункт
Докажите, что $\frac14$ принадлежит канторову множеству.
\пункт
Бесконечно ли множество рациональных чисел, принадлежащих канторову
множеству?
\кзадача

\задача
\пункт
Докажите, что множество $\{a+b\,|\,a,b\in K\}$ всевозможных
сумм элементов канторова множества совпадает с отрезком $[0,2]$.
\пункт
Что можно сказать про множество всевозможных разностей
$\{a-b\,|\,a,b\in K\}$?
\кзадача

\опр
Множество $M$ называется \emph{нигде не плотным}, если на каждом
интервале $I$ найдётся интервал $U\subset I$,
такой что $U\cap M=\varnothing$.
\копр

\задача
Докажите, что объединение конечного числа
нигде не плотных множеств нигде не плотно.
\кзадача

\задача
Являются ли нигде не плотными следующие множества:
\вСтрочку
\пункт
$\{\frac1n\,|\,n\in\N\}$;
\пункт
$\Q$;
\пункт
множество бесконечных десятичных дробей, в записи которых
используется только 0~и~1;
\пункт
канторово множество?
\кзадача

\задача
Существует ли непрерывная функция $f\colon K\to[0,1]$, множеством значений
которой является весь отрезок $[0,1]$?
\кзадача

\опр
Объединение счётного числа нигде не плотных множеств
называется \emph{множеством первой категории}.
Если множество не представимо в виде счётного объединения
нигде не плотных, то его называют
\emph{множеством второй категории}.
\копр

\задача
\пункт[Теорема Бэра]
Докажите, что отрезок является множеством второй категории.
\пункт
Какую категорию имеет множество иррациональных чисел?
\кзадача

\задача
\label{setX}
Пронумеруем рациональные числа на отрезке $[0,1]$: $q_1, q_2, q_3,\dots$
Возьмём некоторое $\varepsilon>0$ и рассмотрим множество
$X_\varepsilon=\bigcup\limits_{n=1}^{\infty}U_{\frac{\varepsilon}{2^n}}(q_n)$,
то есть объединение всех интервалов вида
$(q_n-\frac{\varepsilon}{2^n},q_n+\frac{\varepsilon}{2^n})$.
\вСтрочку
\пункт
Докажите, что $[0,1]\setminus X_\varepsilon$ нигде не плотно.
\пункт
Верно ли, что $X_{\frac12}\supset[0,1]$\,?
\пункт
Устремим $\varepsilon$ к нулю, то есть рассмотрим множество $X$~---
пересечение всевозможных $X_\varepsilon$, или
$X=\bigcap\limits_{\varepsilon>0}X_\varepsilon$.
Верно ли, что $X=\Q\cap[0,1]$\,? Зависит ли ответ от выбранной
в начале нумерации рациональных чисел на отрезке?
\кзадача

\задача
Математики Банах и Мазур опять играют в свою игру (см. задачу 18
листка 26$\frac12$). У кого из них есть
выигрышная стратегия в случае, когда $D$~--- это
\вСтрочку
\пункт
нигде не плотное множество;
\пункт
множество первой категории;
\пункт
множество меры ноль;
\пункт
$X_{\varepsilon}$;
\пункт
$X$ (см. задачу \ref{setX})?
\кзадача

\опр
\emph{Замыканием} множества $M$ называется объединение его со всеми
своими предельными точками. Обозначение: $\overline{M}$.
\копр

\задача
Докажите, что
\вСтрочку
\пункт
множество замкнуто тогда и только тогда, когда оно
совпадает со своим замыканием;
\пункт
$\overline{\overline{A}}=\overline{A}$;
\пункт
$\overline{A}$ есть пересечение всевозможных замкнутых множеств, содержащих~$A$.
\кзадача


\vfill
\ЛичныйКондуит{0mm}{5mm}
\ОбнулитьКондуит
\newpage



\задача
\пункт
Докажите, что замыкание нигде не плотного множества нигде не плотно.\\
\пункт
Верно ли, что замкнутое нигде не плотное множество имеет меру ноль?
\кзадача


\опр
Множество $M$ называется \emph{всюду плотным в множестве $A$},
если $A\subset\overline{M}$. Если множество $M$ всюду плотно
на прямой (то есть $\overline{M}=\R$), то обычно про него просто
говорят, что оно всюду плотно.
\копр

\задача
Являются ли следующие множества всюду плотными:\\
\вСтрочку
\пункт
множество всевозможных несократимых дробей, в записи числителя
и знаменателя которых не участвует девятка;
\пункт
множество конечных десятичных дробей;
\пункт
$[0,1]\setminus X$ в отрезке $[0,1]$ (см. задачу \ref{setX})?
\кзадача

\задача
\пункт
Верно ли, что дополнение всюду плотного множества нигде не плотно?\\
\пункт
Верно ли, что дополнение нигде не плотного множества всюду плотно?\\
\пункт
Докажите, что дополнение всюду плотного открытого множества
нигде не плотно.
\кзадача

\задача
\вСтрочку
\пункт
Докажите, что если множество не является нигде не плотным,
то оно всюду плотно в некотором интервале.
\пункт
Что значит, что множество не является всюду плотным?
\кзадача

\задача
Пусть $\alpha$~--- действительное число. Рассмотрим множество $D_\alpha$
всех дробных частей $\{n\alpha\}$ для всевозможных целых $n$.
Пусть $\alpha$ иррационально.
\сНовойСтроки
\пункт
Докажите, что 0~--- предельная точка этого множества.
\пункт
Докажите, что $D_\alpha$ всюду плотно на отрезке $[0,1]$.
\пункт
Что можно сказать про $D_\alpha$, если $\alpha$ рационально?
\кзадача

\задача
Разделим окружность длины 1 на произвольные $m$ дуг и занумеруем их
числами от 1 до $m$. Фиксируем иррациональное
число $a$ и отметим на окружности все точки вида $n\alpha$
радиан (отсчитывая от какой-то начальной точки
против часовой стрелки), где угол $\alpha$ соответствует дуге длины $a$.
Составим бесконечную последовательность $\nu$ из чисел от 1 до $m$:
если точка $n\alpha$ попала внутрь дуги с номером $k$, то запишем
на $n$-м месте последовательности число~$k$~%
\footnote{Точка $n\alpha$ может попасть на границу между соседними
дугами, но не более одного раза для каждой такой
границы, так как $a$ иррационально.
Поэтому в этом случае будем рассматривать последовательность
начиная с того места, после которого $n\alpha$ на границы
дуг уже не попадает.}.
\сНовойСтроки
\пункт
Докажите, что все числа от 1 до $m$ встречаются
в этой последовательности бесконечно много раз.
\пункт
Пусть где-то в последовательности $\nu$ встретился конечный
отрезок $\overline{a_1\dots a_k}$ из чисел от 1 до $m$.
Докажите тогда, что найдётся такое натуральное $l$,
что среди любых $l$ подряд идущих чисел из последовательности $\nu$
найдутся $k$ подряд идущих, составляющие $\overline{a_1\dots a_k}$
(последовательности с таким свойством
называются \emph{почти периодическими}).
\пункт
Будет ли последовательность $\nu$ периодической?
\кзадача


\задача
\пункт
Докажите, что $2^n$ начинается с цифры 7 тогда и только тогда, когда
$\lg7\leq\{n\lg2\}<\lg8$.
\пункт
Докажите, что для любой конечной последовательности
цифр $\overline{a_1\dots a_k}$, где $a_1\ne0$,
найдётся натуральная степень двойки,
начинающаяся с этой последовательности цифр.
\кзадача

\задача
Рассматривается последовательность, $n$-й член которой есть
первая цифра числа $2^n$.\\
\вСтрочку
\пункт
Докажите, что эта последовательность
почти периодическая (см.~задачу 20).
Является ли она периодической?
\пункт
Найдите количество различных \лк слов\пк\
длины 13 (наборов из 13 подряд идущих цифр) в этой последовательности.
\кзадача


\задача
\вСтрочку
\пункт
Возьмём отрезок $[0,1]$ и будем действовать так же, как и при построении
множества $K$, но только делить на пять равных частей
(а не на три) и выкидывать
вторую и четвёртую. Исследуйте свойства получившегося множества.
А если делить на восемь равных частей и выкидывать вторую, третью и шестую?
\пункт
Опять возьмём отрезок $[0,1]$ и будем с ним поступать так же, как при построении
множества $K$, но только на $n$-м шаге будем из каждого оставшегося отрезка
вырезать в середине интервал длины $\frac1n$ от его длины (а не $\frac13$).
Исследуйте свойства получившегося множества.
\кзадача

\задача[И. М. Гельфанд]
\пункт
На плоскости из начала координат в каждом
направлении выпустили по некоторому отрезку.
Пусть $M$~--- объединение этих отрезков.
Известно, что $M$~--- замкнутое множество.
Докажите, что оно имеет непустую внутренность
(непустое множество внутренних точек).\\
\пункт
Обязательно ли в $M$ найдётся круг
с центром в начале координат?
\кзадача

\задача
Докажите, что у счётного замкнутого множества всегда найдётся
изолированная точка.
%Множество $M$ замкнуто, и каждая точка множества $M$
%является предельной для множества $M$. Докажите,
%что $M$ несчётно.
\кзадача

\vfill
\ЛичныйКондуит{0mm}{5mm}
%\GenXMLW
% %\СделатьКондуит{3.5mm}{6mm}

\end{document} 