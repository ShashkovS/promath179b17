% !TeX encoding = windows-1251
\documentclass[a4paper,11pt]{article}
\usepackage{newlistok}
%\documentstyle[11pt, russcorr, listok]{article}

\УвеличитьШирину{1truecm}
\УвеличитьВысоту{3.5truecm}
\hoffset=-2.5truecm
\voffset=-21truemm


\Заголовок{Математическая индукция}
\НомерЛистка{4}
\ДатаЛистка{11.2013}


\begin{document}

\СоздатьЗаголовок

%\vspace*{-3truemm}
%{\footnotesize
%\noindent
%\опр
\noindent
\выд{Математическая индукция} --- это способ доказать
бесконечную серию занумерованных натуральными числами
утверждений за два хода:\qquad
 1) \выд{база индукции:}
доказываем первое утверждение;\\
 2) \выд{шаг индукции:}
доказываем, что при любом натуральном $n$ из $n$-го
утверждения следует $(n+1)$-е.
%\копр

\smallskip


\задача
Докажите, что части, на которые $n$ прямых делят плоскость, можно
раскрасить в два цвета, так чтобы соседние части (имеющие
общий отрезок или луч) были окрашены в разные цвета.
\кзадача

\задача
Докажите, что $1\cdot1!+2\cdot2!+\dots+n\cdot n!=(n+1)!-1$
при любом натуральном $n$.
\кзадача


\задача
В компании из $k$ человек $(k\geq4)$ у каждого появилась новость, известная
лишь ему одному. За один телефонный разговор двое сообщают друг другу
все известные им новости. Докажите, что за $2k-4$ разговора все они
могут узнать все новости.
\кзадача

\задача
Известно, что
$a_1=1$ и  $a_{n+1}=2a_n+1$ при $n\geq1$.
Найдите $a_n$.
\кзадача


\задача
Докажите, что при любом натуральном $n$
\вСтрочку
\пункт $2^n>n$;
\пункт
$\displaystyle{\frac1{1^2}+\frac1{2^2}+\frac1{3^2}+\dots
+\frac1{n^2}\leq 2-\frac1{n}}$.
\кзадача


\задача
Докажите неравенство Бернулли: $(1+a)^n\geq1+na$, если $a\geq-1$
и $n$ --- натуральное число.
\кзадача

\задача
Докажите, что модуль суммы любого числа слагаемых не больше
суммы модулей этих слагаемых.
\кзадача

\задача
Из клетчатого квадрата $2^n\times2^n$ клеток вырезали одну клетку.
Докажите, что полученную фигуру можно разрезать на \лк уголки\пк\
из трёх клеток. (\лк Уголок\пк\ ---  это квадрат $2\times2$ без одной клетки.)
\кзадача


%\задача
%Докажите, что $2^{5n-2}+5^{n-1}\cdot3^{n+1}$ делится на 17
%при любом натуральном $n$.
%\кзадача

\задача
Найдите ошибку в рассуждении:
\лк Докажем, что в любом табуне все лошади одной масти. Воспользуемся
индукцией по числу лошадей в табуне.
Если в табуне всего одна лошадь, то, разумеется, все лошади в этом
табуне одной масти.
Предположим теперь, что в любом табуне из $n$ лошадей все лошади одной
масти. Рассмотрим произвольный табун из $n+1$ лошади.
По предположению индукции любые $n$ лошадей в этом табуне
одной масти. Поэтому все лошади в табуне одной масти.\пк
%что и требовалось доказать.\пк
\кзадача


\задача Верна ли теорема: \лк Если треугольник разбит отрезками
на треугольники, то хотя бы один из треугольников разбиения не
остроугольный\пк?
Вот е\"е доказательство (нет ли в н\"ем ошибки?):\\
{\лк
1. Если треугольник разбит отрезком на два треугольника,
то один из них не остроугольный (ясно).\\
2. Пусть имеется треугольник, как-то разбитый на $n$ треугольников.
Провед\"ем ещ\"е один отрезок, разбив один из маленьких треугольников
на два. Получим разбиение на $n+1$ треугольник, прич\"ем один из
двух новых треугольников будет не остроугольный.
По индукции теорема доказана.\пк}
\кзадача

\задача
На какое максимальное число частей могут разбить плоскость
\пункт $n$ прямых; \пункт $n$ окружностей?
%\кзадача
%
%\сзадача
\спункт На какое максимальное число частей могут разбить пространство
$n$ плоскостей?
\кзадача

\smallskip

{\footnotesize
\noindent
Есть разные варианты индукции. Иногда в качестве шага
приходится проверять, что $n$-е
утверждение верно~\hbox{если}~верны {\em все} предыдущие. Другой
вариант: предположим, что не все утверждения верны. Тогда
есть {\em наименьшее} на\-ту\-ра\-ль\-ное $n$, для которого $n$-е
утверждение неверно. Если из этого выводится противоречие,
то все утверждения верны.

}

\smallskip

\задача
Докажите, что уравнение $n^2=2m^2$ не имеет решений в
натуральных числах.
%(т.~е.~$\sqrt 2$ иррационально).
\кзадача

\задача
Докажите, что любое натуральное число можно представить как сумму нескольких
разных степеней двойки (возможно, включая и нулевую).
\кзадача

\задача
Число $\displaystyle x+\frac1x$ --- целое.
Докажите, что
$\displaystyle x^n+\frac1{x^n}$ --- тоже целое при любом
натуральном $n$.
\кзадача


\задача[Ханойские башни]
Есть детская пирамида с $n$ кольцами и два пустых стержня
той~же~высоты.
Разрешается перекладывать верхнее кольцо с одного стержня на
другой, но нельзя класть~большее кольцо на меньшее.
Докажите, что
\вСтрочку
\пункт можно переложить все кольца на один из пустых стержней;
\пункт можно сделать это за $2^n-1$ перекладываний;
\пункт меньшим числом перекладываний не обойтись.
\кзадача

\сзадача
Докажите, что для любого натурального~$n>3$ число~$n!$ можно разложить на два
множителя, отношение которых будет не меньше~$2/3$ и не больше~$3/2$.
\кзадача

\сзадача
На кольцевой автотрассе стоят несколько машин.
Общего количества бензина в этих машинах
достаточно для того, чтобы одной машине объехать всю трассу.
Докажите, что одна из машин действительно сможет объехать трассу,
забирая по дороге бензин у других машин.
\кзадача

\сзадача
$k$ воров хотят поделить добычу. Каждый уверен, что он поделил бы добычу
на равные части, но остальные ему не верят.  Как действовать ворам,
чтобы после раздела каждый был уверен, что у него не менее $1\over k$
части добычи? Разберите случаи:
\вСтрочку
\пункт
$k=2$;
\пункт
$k=3$;
\пункт
$k$~--- любое.
\кзадача

\сзадача
При каких $n$ гири весом 1, 2, \dots, $n$ кг
можно разложить на три равные по весу кучи?
\кзадача


\сзадача
Двое играют в игру, исход которой не зависит от случая. %Игроки
Ходят по
очереди, по правилам игра длится не более $n$ ходов. Ничьих
нет. Докажите, что у кого-то есть выигрышная стратегия.
\кзадача

\vspace*{-2mm}
\ЛичныйКондуит{0mm}{6mm}
\vspace*{-3mm}
%\СделатьКондуит{7mm}{7mm}


\end{document}
\сзадача
При каких $n$ можно соединить каждые два из данных $n$ сел
односторонним маршрутом так, чтобы из любого села
в любое другое можно было доехать не более чем с одной пересадкой?
\кзадача
