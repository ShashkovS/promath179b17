% !TeX encoding = windows-1251
\documentclass[a4paper,12pt]{article}
\usepackage{newlistok}


\newenvironment{напоминание}{\medskip\textbf{Напоминание.}}{\par}

\ВключитьКолонитул

\УвеличитьВысоту{2cm}
\УвеличитьШирину{1.5cm}
\renewcommand{\spacer}{\vfil}

\Заголовок{Группы преобразований}
%\НомерЛистка{6д}
\ДатаЛистка{10.2014}


\begin{document}
\СоздатьЗаголовок

\begin{напоминание}
Отображение $\varphi\colon X\to Y$ из множества~$X$ в~множество~$Y$ называется \emph{взаимно однозначным} (или \emph{биекцией}), если для каждого элемента $y\in Y$ существует ровно один элемент~$x$ такой, что $\varphi(x)=y$.\par
Преобразование~$\psi$ называется \emph{тождественным}, если для каждого~$x\in X$ выполнено равенство $\psi(x)=x$. Обозначение: $\psi=\id_X$.\par
Отображение $\varphi\colon X\to Y$ называется \emph{обратным} для отображения $\psi\colon Y\to X$, если справедливы равенства $\varphi\circ\psi=\id_Y$ и~$\psi\circ\varphi=\id_X$. Обозначение: $\varphi=\psi^{-1}$\par
Количество элементов во множестве~$X$ обозначается через~$|X|$ или~$\#X$.
\end{напоминание}

\опр
\emph{Преобразованием} множества~$X$ называется любая биекция $\varphi\colon X\to X$. Для множества всех преобразований~$X$ зарезервировано обозначение~$S(X)$.
\копр

\опр
\emph{Группой преобразований} множества~$X$ называется всякая непустая совокупность его преобразований~$G$, удовлетворяющая следующим свойствам:
\begin{items}{-5}
\item[(i)]
$G$~замкнута относительно композиции, то есть для всех $g,h\in G$ верно: $g\circ h\in G$;
\item[(ii)]
$G$~замкнута относительно взятия обратного преобразования, то есть для всех $g\in G$  преобразование~$g^{-1}$ лежит в~$G$.
\end{items}
\копр
\vspace*{-3mm}

\задача
Докажите, что группа преобразований любого множества содержит тождественное преобразование.
\кзадача



\задача
\label{triangle}
Пусть множество~$X$\т это правильный треугольник~$ABC$ (с внутренностью, точки $A$, $B$, $C$ идут по часовой стрелке).
Обозначим через~$s_{a}$, $s_{b}$ и $s_с$ симметрии относительно прямых,
содержащих соответствующие высоты треугольника.
Далее, обозначим через~$r_0$, $r_1$ и $r_2$ повороты вокруг центра треугольника на $0^\circ$, $120^\circ$ и $240^\circ$ против часовой стрелки соответственно.
\невСтрочку
\пункт
Докажите, что $G=\{s_{a},s_{b},s_{c},r_0,r_1,r_2\}$ образует группу преобразований треугольника;
\пункт
Выпишите таблицу умножения в этой группе (например, $s_b\circ s_a = r_1$);
\пункт
Придумайте группу преобразований треугольника, состоящую из трёх преобразований.
\кзадача


\задача
\label{sym}
\пункт
Докажите, что для любого множества~$X$ множество~$S(X)$ является группой;
\\\пункт
Пусть $X$\т конечно, причём~$|X|=n$. Найдите $|S(X)|$.
\кзадача
\замечание
Если множество $X$ конечно, то группа $S(X)$~называется \emph{симметрической группой} и~обозначается~$S_n$.
\кзамечание


\задача
Пусть множество~$X$ является подмножеством прямой, плоскости или пространства.
Рассмотрим множество преобразований $\Isom(X)=\{\varphi\in S(X)\mid \varphi\text{ сохраняет расстояния}\}$.
Докажите, что вне зависимости от~$X$ множество преобразований $\Isom(X)$ является группой.
Эта группа называется \emph{группой движений}~$X$.
\кзадача


\задача
Докажите, что группа движений треугольника совпадает с группой, описанной в задаче \ref{triangle}.
\кзадача



\задача
\label{square}
\пункт
Опишите группу движений квадрата (то есть найдите и опишите все её элементы).
\\\пункт
Придумайте две различных группы преобразований квадрата, состоящих из четырёх преобразований.
\кзадача




\опр
\emph{Порядком элемента}~$g$ группы преобразований~$G$ называется наименьшее натуральное~$k$ такое, что $g^k=\underbrace{g\circ\dots\circ g}_k=\id$. Обозначение: $\ord(g)$.
\копр

\опр
\emph{Порядком группы}~$G$ называется количество элементов в~$G$. Обозначение: $|G|$ или~$\#G$.
\копр


\задача
Найдите порядок каждого элемента групп из задач \ref{square} и~\ref{triangle}.
\кзадача

\задача
Докажите, что в конечной группе каждый элемент имеет конечный порядок.
\кзадача


\задача
\label{ord}
Перечислите все элементы и~их порядки в~группах движений следующих множеств:\\
\пункт
прямоугольник;
\пункт
правильный $m$-угольник;
\пункт
правильный тетраэдр;
\пункт
куб;
\\
\спункт
октаэдр;
\спункт
икосаэдр;
\спункт
додекаэдр.
\кзадача
\noindent\help{Как связаны между собой куб и~октаэдр? Тот же вопрос для икосаэдра и~додекаэдра.}

\замечание
Группа из задачи~\ref{ord}б) называется \emph{группой диэдра} и~обозначается $D_m$.
\кзамечание


%\vfill
%\ЛичныйКондуит{0mm}{6mm}
%\GenXMLW
\newpage
\раздел{Циклические подгруппы и смежные классы}


\задача
Рассмотрим множество $X$ остатков по модулю $n$.
Пусть преобразование $g$ --- домножение на остаток $g$ (то есть $g(0) = 0$, $g(1) = g$, $g(2) = (2g\mod n)$ и \итд).
\невСтрочку
\пункт
Для $n=7$ для каждого $g$ расположите остатки $X$ по кругу и нарисуйте стрелочками, куда переходит каждый элемент $X$ при действии $g$.
\пункт
Нарисуйте аналогичные картинки для остатков по модулю $6$.
\пункт
Для каких $g$ и $x\in X$ цепочка стрелочек, начинающаяся с $x$, заканчивается в $0$?
\пункт
Для каких $g$ и $x\in X$ цепочка стрелочек идёт по кругу, проходя через все ненулевые остатки?
\пункт
Для каких $g$ и $x\in X$ цепочка стрелочек никогда не приводит в 0?
\кзадача


\опр
Любое подмножество $H$ группы $G$, являющееся группой преобразований, называется подгруппой.
Обозначение: $H\subset G$.
\копр

\задача
Пусть $G$ --- группа преобразований некоторого множества $X$ и $g$ --- некоторый её элемент.
\пункт
Докажете, что множество преобразований
$\ha{g} = \hc{g^k\mid k\in\Z} = \hc{e, g, g^{-1}, g\circ g, g^{-1}\circ g^{-1}, ...}$
является подгруппой в $G$.
\\
\пункт
Пусть порядок элемента $g$ равен $k$.
Тогда в подгруппе $\ha{g}$ в точности $k$ элементов.
\кзадача

\опр
Если для некоторого элемента $g$ группы $G=\ha{g}$, то группа $G$ называется \выд циклической,
а про элемент $g$ говорят, что он \выд порождает группу $G$.
\копр


\задача
Пусть $G=\ha{g}$ --- конечная циклическая группа из $n$ элементов.
\невСтрочку
\пункт
Найдите порядок элемента $g^k$;
\пункт
Докажите, что элемент $g^k$ порождает $G$ тогда и только тогда, когда $(n,k)=1$;
\пункт
Докажите, что подгруппа циклической группы --- циклическая.
\кзадача

\опр
Пусть $G$ --- группа преобразований, а $H$ --- её подгруппа.
Будем говорить, что элементы $g_1, g_2\in G$ \выд{сравнимы по модулю $H$},
и писать $g_1\equiv g_2\pmod{H}$, если найдётся элемент $h\in H$ такой, что $g_2=g_1\circ h$.
Это определение обобщает определение сравнимости чисел по модулю $n$.
\копр

\задача
Докажите, что
\пункт
если $g_1\equiv g_2\pmod{H}$, то $g_2\equiv g_1\pmod{H}$;
\пункт
если $g_1\equiv g_2\pmod{H}$ и $g_2\equiv g_3\pmod{H}$,  то $g_1\equiv g_3\pmod{H}$;
\кзадача

\опр
Таким образом все элементы группы $G$ разбиваются на классы, в которых каждые два элемента сравнимы по модулю $H$.
Эти классы называются \выд{левыми смежными классами} группы $G$ по подгруппе $H$.
Класс элемента $g$ обозначается через $gH$.
Множество всех смежных классов группы $G$ по подгруппе $H$ обозначается через $G/H$.
\копр

\задача[Теорема Лагранжа]
Докажите, что если $G$ --- конечная группа, и $H$ --- любая её подгруппа, то $|G| = |H|\cdot|G/H|$.
\кзадача


\задача
  Докажите, что порядок любого элемента группы делит порядок группы.
\кзадача


\задача
  Докажите, что всякая конечная группа простого порядка является циклической.
\кзадача


\опр
Функция, равная количеству натуральных чисел, меньших $n$ и взаимно простых с ним, называется \выд{функцией Эйлера} и обозначается через $\phi(n)$.
\копр


\задача
Пусть $n$ --- произвольное число.
Рассмотрим множество $\Z_n$ остатков по модулю $n$, и группу~$G$, состоящую из остатков, взаимно простых с $n$, действующих на~$\Z_n$ домножениями.
\\\пункт Докажите, что такое множество преобразований образует группу;
%\\\пункт Найдите орбиты действия этой группы;
\\\пункт[теорема Эйлера]
Докажите, что если числа $a$ и $n$ взаимно просты, то $a^{\phi(n)} \equiv 1\pmod{n}$.
\кзадача


\newpage
\раздел{Орбиты и стабилизаторы}


\опр \выд{Орбитой} элемента $x \in X$ при действии группы преобразований $G$ называется множество $\{g(x) \mid g \in G\} \subset X$.
\выдд Обозначение $Gx$.
\копр

\задача
Найдите орбиту каждой точки при действии группы движений\\
\пункт
квадрата;
\пункт
куба;
\пункт
правильного $m$-угольника.
\кзадача



\задача
\пункт
Опишите группу движений единичного круга;
\пункт
Найдите орбиту каждой точки при действии этой группы;
\пункт
Найдите преобразование, не имеющее конечного порядка.
\кзадача

\задача
Докажите, что любые две орбиты либо совпадают, либо не пересекаются. Следует ли отсюда,
что всё множество $X$ есть объединение непересекающихся орбит?
\кзадача

\задача
Докажите, что для любых двух элементов одной орбиты $a,b \in Gx$ найдётся элемент $g \in G$,
такой что $g(a) = b$.
\кзадача


\опр \выд{Стабилизатором\/} элемента $x \in X$ при действии группы преобразований $G$ называется
множество $\{g \mid g(x)=x\} \subset G$.
\выдд Обозначение: $G_x$.
\копр

\задача
Найдите стабилизаторы каждой из точек следующих множеств при действии их групп движений:
\пункт
квадрата;
\пункт
куба;
\пункт
правильного $m$-угольника.
\кзадача

\задача
Рассмотрим группу движений куба $G$. Эта группа также является группой преобразований следующих множеств:
\пункт множества вершин куба;
\пункт множества диагоналей куба;
\пункт множества граней куба;
\спункт множества пар вершин куба.
Опишите орбиты и стабилизаторы во всех случаях.
\кзадача

\задача
Пусть задана группа преобразований $G$ множества $X$. Докажите, что стабилизатор
любого элемента $x\in X$ также является группой преобразований множества $X$.
\кзадача

\задача
Пусть группа $G$ конечна. Докажите, что для любых двух элементов одной орбиты $a,b \in Gx$ выполнено $|G_a| = |G_b|$.
\кзадача

\задача
Пусть группа $G$ конечна. Докажите, что для любого $x \in X$ верно $|G| = |Gx| \cdot |G_x|$.
\кзадача

\задача
Пусть $p$ --- простое число.
Рассмотрим множество $\Z_p$ остатков по модулю $p$ и группу~$G$, состоящую из ненулевых остатков, действующих на~$\Z_p$ домножениями (т.е. $G=\{1, 2,\ldots,p-1\}$ и $g(x) = x\cdot g$).
\\\пункт Найдите орбиты действия этой группы;
\\\пункт[малая теорема Ферма] Докажите, что $a^{p-1} \equiv 1 \pmod{p}$.
\кзадача



\задача
Образует ли группу множество преобразований плоскости, переводящих прямые в прямые? Что это за преобразования?
\кзадача


\newpage
\раздел{Изоморфизмы групп}
\опр
\label{homo}
Пусть $G$ --- группа преобразований множества $X$, а $H$ --- группа преобразований множества $Y$. Группы $G$ и $H$ называются \выд{изоморфными\/}, если найдётся биекция $\ph \from G \to H$, при которой тождественное преобразование переходит в тождественное, обратное --- в обратное, а композиция преобразований --- в композицию преобразований, то есть:\\
(\emph{i}\/) $\ph(\id_X) = \id_Y$;\\
(\emph{ii}\/) для каждого $g \in G$ верно: $\ph(g^{-1}) = (\ph(g))^{-1}$;\\
(\emph{iii}\/) для любых $g_1,g_2\in G$ верно: $\ph(g_1\circ g_2)=\ph(g_1)\circ\ph(g_2)$.\\
Отображение $\ph$ в этом случае называется \выд{изоморфизмом}.
\выдд Обозначение: $G\isom H$, $G\stackrel{\ph}{\isom} H$.
\копр

\задача
Правда ли, что если $G\isom H$, то
\пункт $\#G = \#H$;
\пункт $\#X = \#Y$?
\кзадача

\задача
Пусть $\ph \from G \to H$ --- биекция, такая что выполнено условие (\emph{iii}\/) определения~\ref{homo}. Докажите, что $\ph$ является изоморфизмом.
\кзадача


\задача
Докажите, что следующие группы изоморфны:\\
\пункт группа вращений правильной 4-угольной призмы (не являющейся кубом) и группа движений квадрата;\\
\пункт группа движений куба и группа движений октаэдра;\\
\пункт группа вращений правильного $n$-угольника и группа вычетов по модулю $n$ (см. задачу \ref{gr}в). Эта группа обозначается $\Z_n$ или $\Z/n\Z$;\\
\спункт группа движений тетраэдра и группа вращений куба.
\кзадача


\задача
Пусть $\ph\from G\to H$ --- изоморфизм. Докажите, что
для любого элемента $g\in G$ верно: $\ord(g) = \ord(\ph(g))$;
\кзадача

\задача
Какие из следующих групп изоморфны:\\
1) группа вращений правильного 24-угольника;\\
2) группа движений правильного 12-угольника;\\
3) группа движений правильной 6-угольной призмы;\\
4) группа движений правильного тетраэдра;\\
5) группа $S_4$?
\кзадача


\раздел{Абстрактные группы}


\опр
\выд Абстрактной \выд группой (или просто \emph{группой}) называется множество $G$ с операцией умножения, обладающей следующими свойствами:\\
(\emph{i}\/) $(ab)c = a(bc)$ для любых $a,b,c\in G$ (\emph{ассоциативность});\\
(\emph{ii}\/) существует такое элемент $e\in G$ (\emph{единица}), что $ae=ea=a$ для любого $a\in G$;\\
(\emph{iii}\/) для всякого элемента $a\in G$ существует такой элемент $a^{-1}\in G$ (\emph{обратный элемент}), что $a a^{-1} = a^{-1} a = e$.
\копр

\задача
Докажите, что всякая группа преобразований с операцией композиции является абстрактной группой.
\кзадача

\задача
\label{gr}
Являются ли  следующие множества с указанными операциями группами:\\
\пункт $(\Z, + )$; \пункт $(\R\setminus \hc{0}, \cdot)$; \пункт $(\text{остатки по модулю }5, +)$; \пункт $(\text{остатки по модулю }5, \cdot)$;\\ \пункт $(\text{ненулевые остатки по модулю }5, \cdot)$; \пункт то же самое по модулю 10.
\кзадача

\задача
\пункт
Пусть $G$ --- группа преобразований множества $X$, и $h\in G$.
Докажите, что отображение $L_h \from G \to G$, $g\corr{L_h} h\circ g$ является преобразованием $G$ (такое преобразование называется \выд{левым сдвигом});\\
\пункт
Реализуйте произвольную абстрактную группу как группу преобразования некоторого множества.
\кзадача

%\задача
%Пусть $G$ --- группа преобразований множества $X$. Обозначим через $\wt{G} = \hc{L_h\mid h\in G}$ множество всех левых сдвигов группы $G$. Докажите, что $\wt{G}$ образует группу, изоморфную $G$.
%\кзадача

\задача
Докажите, что в группе может быть только одна единица, только один обратный элемент.
\кзадача

\задача
Докажите, что группы 1) вращений окружности; 2) комплексных чисел, по модулю равных $1$ c операцией умножения; и 3) группа матриц вида $\rbmat{\cos \ph & - \sin \ph\\\sin \ph &\phantom{-}\cos\ph}$ с операцией умножения $\hr{\begin{smallmatrix}a&b\\c&d\end{smallmatrix}}\cdot\hr{\begin{smallmatrix}x&y\\z&t\end{smallmatrix}}=
\hr{\begin{smallmatrix}ax+bz&ay+bt\\cx+dz&cy+dt\end{smallmatrix}}$ изоморфны.
\кзадача



\end{document}




