% !TeX encoding = windows-1251
\documentclass[a4paper,12pt]{article}
\usepackage[mag=910]{newlistok}
%\usepackage{tikz}
%\usetikzlibrary{calc}

%\documentstyle[11pt, russcorr, listok]{article}
\newcommand{\del}{\mathrel{\raisebox{-.3 ex}{${\vdots}$}}}

\УвеличитьШирину{1.4truecm}
\УвеличитьВысоту{2.5truecm}
% \hoffset=-2.5truecm
% \voffset=-25.5truemm


\Заголовок{Прогрессии}
\НомерЛистка{14}
\ДатаЛистка{09.2014}
\renewcommand{\spacer}{\vfill}

\begin{document}

\СоздатьЗаголовок

%\vspace*{-1truemm}

%Последовательность --- это набор занумерованных чисел
%\опр Арифметической прогрессией называется после
%\копр

\задача
Фабрика выпускает наборы из $n>2$ белых слоников различной
величины и массы, стоящих по росту. По стандарту разность масс
соседних слоников должна быть одной и той же. При каких $n$ контролер
гарантированно сможет это проверить с помощью чашечных весов без гирь?
\кзадача

\опр {\it Арифметическая прогрессия\/}  --- это (конечная или бесконечная)
последовательность чисел $\ldots\,,\,a_1,\,a_2,\,a_3,\,\ldots\,$,
в которой разность $d=a_k-a_{k-1}$ между соседними членами $a_k$ и $a_{k-1}$
одинакова для всех $k$; она называется {\it разностью\/} или {\it
приращением\/} прогрессии.
\копр

\задача
Выразите $n$-й член арифметической прогрессии через
первый член и разность.
%Найдите 50-е натуральное число, большее
%90, c остатком 3 от деления на~4.
Найдите 50-е натуральное число среди чисел, больших 90 и имеющих
остаток 3 при делении на 4.
\кзадача

\задача Каждый член некой последовательности (кроме крайних, если
они есть) равен среднему арифметическому двух соседних членов.
%: $a_k=(a_{k-1}+a_{k+1})/2$.
Верно ли, что это %а последовательность ---
арифметическая прогрессия?
Верно ли обратное?
% утверждение?
\кзадача

\задача В некоторой арифметической прогрессии сумма первых
$n$ членов равна сумме первых $m$ членов (где $m<n$).
Докажите, что сумма первых $n+m$ членов этой прогрессии равна нулю.
\кзадача

%\задача Найдите 100-й член последовательности $a_n$, заданной условиями
%$a_1=1$, $a_2=4$, $a_{n}=2a_{n-1}-a_{n-2}$ при $n\geq3$.
%\кзадача


%\задача
%Будет ли арифметической прогрессией последовательность с $k$-м
%членом, равным:
%\вСтрочку
%\пункт
%$\underbrace{1\,1\,\ldots\,1}_{k}$\,,
%\пункт
%$k$-тому натуральному числу, оканчивающемуся на 13\,?
%\кзадача


%\задача
%Какие из перечисленных ниже свойств набора чисел $\OFAM a,n$
%необходимы, а какие --- достаточны для того, чтобы этот набор был
%арифметической прогрессией:
%\сНовойСтроки
%\пункт
%каждый элемент (кроме крайних) равен среднему арифметическому
%двух соседних:
%$a_k=(a_{k-1}+a_{k+1})/2$;
%\пункт
%$2\,a_i=a_{i-2}+a_{i+2}$ при всех $2\leq i\leq n-2$;\hfill
%\пункт
%сумма $a_i+a_{n-i}$ одна и та же для всех $0\leq i\leq n$?
%\кзадача

\задача
Выразите сумму всех членов конечной арифметической
прогрессии $a_1,\ \!a_2,\ \!\ldots,\ \!a_n$
через\\
%\сНовойСтроки
\вСтрочку
\пункт два крайних члена и число слагаемых;
\пункт начальный член, число слагаемых и приращение.
\кзадача

\задача
Найдите сумму всех тр\"ехзначных чисел, оканчивающихся на 7.
\кзадача

%\задача
%Конечная арифметическая прогрессия состоит из целых
%чисел, и ее сумма --- степень двойки. Докажите, что
%количество членов прогрессии --- тоже степень двойки.
%\кзадача

\задача
По строкам и столбцам прямоугольной таблицы %размера
$m\times n$ стоят
арифметические прогрессии. Найдите сумму всех чисел в таблице,
если сумма четыр\"ех угловых чисел равна~$S$.
\кзадача

%\задача
%%Как связаны арифметические прогрессии с $m$-угольными числами Диофанта?
%Что такое {\it $m$-угольные числа\/}?
%Чему равно $n$-тое $m$-угольное число?
%\кзадача

\задача
\пункт
Дан квадратный тр\"ехчлен $f(x)=ax^2+bx+c$.
При каких условиях на коэффициенты $a$, $b$, $c$
найд\"ется такая арифметическая прогрессия
$(a_n)$, что $a_1+\ldots+a_n=f(n)$ при всех натуральных~$n$?\\
\пункт Найдите арифметическую прогрессию,
сумма первых $n$ членов которой равна $2n^2-3n$.
\кзадача


\сзадача
Можно ли натуральный ряд покрыть $k$ арифметическими
прогрессиями с различными целыми разностями, не равными 1,
если
\вСтрочку
\пункт $k=2$;
\пункт $k=3$;
\пункт $k=4$;
\пункт $k=5$?
\кзадача



%\задача
%Известно, что среди членов  некоторой арифметической прогрессии
%$a_1, a_2, a_3, a_4, \dots$ есть числа $a_1^2$, $a_2^2$ и $a_3^2$.
%Докажите, что эта прогрессия состоит из целых чисел.
%\кзадача

\vspace*{-3pt}
\раздел{***}

\vspace*{-8pt}

\опр {\it Геометрическая прогрессия\/}  --- это (конечная или бесконечная)
последовательность ненулевых чисел $\ldots\,,\,a_1,\,a_2,\,a_3,\,\ldots\,$,
в которой отношение $q=a_k/a_{k-1}$ соседних членов
одинаково для всех $k$; оно называется {\it знаменателем\/} прогрессии.
\копр


\задача
Будет ли геометрической прогрессией последовательность, $k$-й
член~\hbox{которой}~ра\-вен\\
\вСтрочку
\пункт
$0,\underbrace{0\ldots0}_{k}3$;
%\пункт
%  $3^{-k}$;
\пункт
 $\underbrace{1\ldots1}_{k}$;
\пункт
 $2^{3k+5}$;
%\пункт
% $\sqrt[k]3$;
\пункт
 $g_k\cdot h_k$, где $(g_k)$, $(h_k)$ --- геометрические~\hbox{прогрессии?}\\
\пункт
Выразите $n$-й член геометрической прогрессии через
первый член~и~\hbox{знаменатель.}
\кзадача

\задача
Квадрат каждого члена некой последовательности (кроме крайних,
если они есть) ненулевой и равен произведению двух соседних членов. %: $a_k^2=a_{k-1}\cdot a_{k+1}$,
Верно ли,
что это %последовательность ---
геометрическая прогрессия?
Верно ли обратное?
\кзадача

\задача Некто приезжает в город с новостью и
сообщает е\"е двоим. Каждый из вновь узнавших новость через 5 минут
сообщает е\"е ещ\"е двоим (которые е\"е не знают) и
т.~д.~(пока все в городе е\"е не узнают).
Через сколько времени новость узнает весь город, если в н\"ем
1\,000\,000 жителей?
\кзадача

\задача Торговец прин\"ес на рынок мешок одинаковых орехов.
Первый покупатель купил 1 орех,~\hbox{второй ---~2,}
третий --- 4, и т.\,д.: каждый следующий %покупатель
покупал вдвое больше орехов, чем предыдущий.
Орехи, купленные последним, весили 50 кг, после чего у торговца
остался 1 орех. Сколько килограммов орехов было у него~вначале?
%(Все орехи одинаковые.)
\кзадача

\задача Найдите суммы:
\вСтрочку
%\пункт $1+3+3^2+\ldots+3^{10}$;
\пункт
$1+x+x^2+\ldots+x^n$;
\пункт
$1-\frac12+\frac14-\frac18+\ldots-\frac1{512}$.
%\кзадача
%
%\задача
\пункт Выразите сумму
всех членов конечной геометрической прогрессии через начальный член $a$,
количество слагаемых $n$ и знаменатель~$q$.
\кзадача

\задача
Даны две бесконечные вправо прогрессии: арифметическая
%$\alpha$
%$a_1,a_2,a_3\dots$
и геометрическая. %$\beta$.
%$b_1,b_2,b_3\dots$,
Известно, что все числа, которые встречаются среди членов геометрической
прогрессии, % $\beta$,
встречаются и среди членов арифметической прогрессии. % $\alpha$.
%причём вторая содержится в первой.
Докажите, что знаменатель геометрической прогрессии --- целое число.
\кзадача

\задача Можно ли покрыть натуральный ряд конечным
числом геометрических прогрессий?
\кзадача


%\vskip-5pt
%\раздел{***}
%
%\vskip-15pt

\vspace*{-3pt}
\раздел{***}

\vspace*{-8pt}


\опр {\it Числа Фибоначчи} -- это члены последовательности
$f_0,f_1,\ldots,$
в которой $f_0=f_1=1$, а
%остальные вычисляются по формуле $u_n=u_{n-1}+u_{n-2}$ (при $n\geq3$).
каждый следующий член равен сумме двух предыдущих:
%:$u_1=u_2=1$,
$f_{n}=f_{n-1}+f_{n-2}$ при всех целых $n\geq2$.
\копр

\задача Вычислите первые 15 чисел Фибоначчи.
\кзадача


\задача Найдите все
\вСтрочку
\пункт
арифметические;
\пункт
геометрические прогрессии, у которых каждый член,
начиная с третьего, равен сумме двух предыдущих.
\кзадача


\задача
%\пункт Является ли последовательность чисел Фибоначчи
%геометрической прогрессией?
Представьте последовательность Фибоначчи в виде
суммы двух геометрических прогрессий,
т.~е.~найдите такие прогрессии $(g_n)$ и $(h_n)$, что $f_n=g_n+h_n$ при
всех целых $n\geq0$.  %Найдите формулу для чисел Фибоначчи.
\кзадача

\vspace*{-2mm}
\ЛичныйКондуит{0mm}{6mm}
%\СделатьКондуит{6mm}{6.5mm}


\end{document}

\задача
\кзадача

\задача
\кзадача

\задача
\кзадача



\vspace*{-5pt}
\раздел{***}

\vspace*{-10pt}

%\noindent
%{\small {\Bf Суммирование разностей.}

\задача
Найдите суммы:\footnote{
Сумму $b_1+b_2+\ldots+b_n$ иногда уда\"ется вычислить,
представив каждое слагаемое в виде разности $b_i=c_{i+1}-c_i$ чисел
некого другого набора (тогда при подстановке в исходную сумму почти все $c_i$
сокращаются). }
\вСтрочку
\пункт
{\small
$\displaystyle{\frac1{1\cdot2}+\frac1{2\cdot3}+\,\cdots\,+\frac1{n\,(n+1)}}$;}
%\пункт
%$1\cdot2+2\cdot3+\ldots+n(n+1)$;
\пункт
%{\small
$\displaystyle{\frac1{a_1a_2}+\frac1{a_2a_3}+\,\cdots\,+\frac1{a_{n-1}a_n}}$
(где $a_1,\ldots,a_n$ --- арифметическая прогрессия с разностью $d$);\\
%\пункт
%$1\cdot1!+2\cdot2!+\ldots+n\cdot n!$ (где $k!=1\cdot2\cdot\ldots\cdot k$);
\пункт
{\small
$\displaystyle{\frac1{1\cdot2\cdot3}+\frac1{2\cdot3\cdot4}+\ldots
 +\frac1{n(n+1)(n+2)}}$;}
\пункт
{\small
$\displaystyle{\frac1{1+\sqrt2}+\frac1{\sqrt2+\sqrt3}+\ldots+\frac1{\sqrt{48}+\sqrt{49}}}$.}
%$1\cdot2\cdot3+2\cdot3\cdot4+\ldots+n(n+1)(n+2)$.
%{\small
%$\displaystyle{\frac1{1\cdot2\cdot3\cdot4}+\frac1{2\cdot3\cdot4\cdot5}+\,\cdots\,
% +\frac1{n\,(n+1)\,(n+2)\,(n+3)}}$.}
\кзадача


\задача
\пункт
Выразите $(n+1)^k$ и разность $(n+1)^k-n^k$ в виде многочленов от $n$
при $k=1,2,3,4,5$.
\пункт
Пользуясь п.~а), найдите сумму $k$-тых степеней первых
$n$ натуральных чисел для $k=0,1,2,3,4$.
\кзадача


\задача
Найдите суммы:
%\footnote{
%Сумму $b_1+b_2+\ldots+b_n$ иногда уда\"ется вычислить,
%представив каждое слагаемое в виде разности $b_i=c_{i+1}-c_i$ чисел
%некого другого набора (тогда при подстановке в исходную сумму почти все $c_i$
%сокращаются). }
\вСтрочку
\пункт
{\small
$\displaystyle{\frac1{1\cdot2}+\frac1{2\cdot3}+\,\cdots\,+\frac1{n\,(n+1)}}$;}
%%\пункт
%%$1\cdot2+2\cdot3+\ldots+n(n+1)$;
\пункт
{\small
$\displaystyle{\frac1{a_1a_2}+\frac1{a_2a_3}+\,\cdots\,+\frac1{a_{n-1}a_n}}$}
(где $a_1,\ldots,a_n$ --- арифметическая прогрессия с разностью $d$).
\кзадача
