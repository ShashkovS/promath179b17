% !TeX encoding = windows-1251
\documentclass[a4paper,12pt]{article}
\usepackage{newlistok}
%\usepackage{tikz}
%\usetikzlibrary{calc}

%\documentstyle[11pt, russcorr, listok]{article}
\newcommand{\del}{\mathrel{\raisebox{-.3 ex}{${\vdots}$}}}

\УвеличитьШирину{1truecm}
\УвеличитьВысоту{2.1truecm}
% \hoffset=-2.5truecm
% \voffset=-1truecm

%{\reflectbox{\hbox{{\tiny Подсказка к а): докажите, что любая последовательность $(f^n(x))$ фундаментальна}}}}

\begin{document}

\Заголовок{Равномощность. Счётные множества}
\Подзаголовок{}
\НомерЛистка{12}
\ДатаЛистка{04.2014}

\СоздатьЗаголовок


\опр
Говорят, что между множествами $A$ и $B$ задано
\выд{взаимно
однозначное соответствие}, если каждому элементу множества $A$
поставлен в соответствие какой-то определенный элемент множества $B$,
причём каждый элемент множества $B$ поставлен в соответствие ровно
одному элементу множества $A$. Множества $A$ и $B$ %называются
\выд{равномощны}, если между ними можно установить взаимно
однозначное соответствие.
%Множества $X$ и $Y$ называют \выд{равномощными},
%если их элементы можно разбить на пары $(x,y)$, где $x\in X$
%и $y\in Y$, причём каждый элемент $x\in X$ образует пару
%ровно с одним элементом $y\in Y$, и наоборот,
%каждый элемент $y\in Y$ образует пару
%ровно с одним элементом $x\in X$.
Обозначение: $|A| = |B|$.
%Говорят ещё, что между $X$ и $Y$ можно установить взаимно однозначное
%соответствие.
\копр

\задача Докажите следующие утверждения:\\
%Докажите:
\вСтрочку
\пункт $\!\!\!$ $|A| = |A|$;
\пункт $\!\!\!$ если $|A| = |B|$, то $|B| = |A|$;
\пункт $\!\!\!$ если $|A| = |B|$ и $|B| = |C|$,~то~\hbox{$|A| = |C|$.}
\кзадача

%\задача
%Обозначение $|X|$ уже встречалось для конечных множеств.
%Не будет ли недоразумений?
%%Когда равномощны два конечных множества?
%\кзадача

\задача
Сколько есть взаимно однозначных соответствий между двумя конечными множествами с одинаковым числом элементов?
\кзадача



\задача
Некоторое число делится на $2$, но не делится на $4$. Докажите, что
количество ч\"етных делителей этого числа равно количеству
его неч\"етных делителей.
\кзадача

\задача
Среди треугольников с целыми сторонами
каких больше:\\
%\вСтрочку
$\!\!\!$
\пункт
$\!\!\!$ периметра 2011 или периметра
2014;
$\!\!\!$ \пункт
$\!\!\!$ периметра 2012 или периметра 2015?
%с целыми сторонами
\кзадача

\задача
Сколько существует
троек целых чисел $(x;y;z)$, для которых\\
%\вСтрочку
\пункт $0<x<y<z<13$;
\пункт~\hbox{$0\leq x\leq y\leq z\leq9$?}
\кзадача

\задача
Даны четыре множества: $\N$, множество чётных натуральных
чисел, $\Z$, множество натуральных чисел без числа 3.
Какие из этих множеств равномощны между собой?
%существует ли
%взаимно однозначное отображение из первого во второе.
%\сНовойСтроки
%\пункт множество натуральных чисел;
%\пункт множество чётных натуральных чисел;
%\пункт множество натуральных чисел без числа 3.
\кзадача

\задача Равномощны ли следующие множества точек:\\
%\вСтрочку
\пункт любые два отрезка различной длины;
\пункт любые два интервала различной длины?
\кзадача


\опр Множество называется \выд{сч\"етным}, если
оно равномощно множеству $\N$.\\
Говорят, что множество $X$ \выд{не более чем сч\"етно,} если $X$ конечно (например, пусто)
или сч\"етно.
\копр

%\задача Докажите, что следующие множества
%сч\"етны:
%\вСтрочку
%\пункт $\Z$;
%\пункт $\{x\in\N\ |\ x\ {\rm делится\ на}\ 9\}$.
%\кзадача

\задача
Любое ли сч\"етное множество можно разбить на 3
непересекающихся сч\"етных множества?
\кзадача

\задача
%Даны три ящика $A$, $B$ и $C$.
В ящике $A$  сч\"етное число орехов,
ящики $B$ и $C$  пусты. Берут 10 орехов из ящика~$A$
и перекладывают их в ящик $B$, после чего берут
один орех из ящика $B$ и перекладывают его в ящик~$C$.
Сколько орехов может оказаться в каждом из ящиков после
бесконечного числа таких действий?
\кзадача


\задача
Докажите, что
\вСтрочку
\пункт
подмножество сч\"етного множества не более чем сч\"етно;\\
\пункт %Докажите, что
если~множества $A$~и~$B$~сч\"ет\-ны, % множества,
то $A \cup B$ тоже
сч\"етно;\\
%объединение двух счетных множест счетно;
\пункт %Докажите, что
объединение непустого конечного или сч\"етного множества сч\"етных множеств сч\"етно.
%\пункт %Докажите, что
%сч\"етное объединение сч\"етных множеств сч\"етно.
%%\пункт Верно ли, что сч\"етное объединение конечных множеств всегда
%сч\"етно?
\кзадача

\задача Докажите, что сч\"етно
\вСтрочку
\пункт множество точек плоскости, координаты
которых --- целые числа;
\пункт множество $\Q$;
\пункт множество пар $A\times B=\{(a,b)\ |\ a\in A,\ b\in B\}$,
где $A$ и $B$ сч\"етны.
\кзадача

\сзадача
Найдите алгебраическое выражение от двух переменных $x$ и $y$,
задающее взаимно однозначное соответствие между
%множеством неотрицательных целых чисел и
множеством точек плоскости с натуральными координатами и $\N$.
\кзадача

\задача Докажите, что сч\"етно \quad
%\сНовойСтроки
\вСтрочку
\пункт множество конечных последовательностей из 0 и 1;\\
\пункт множество предложений русского языка;
\пункт множество конечных подмножеств множества $\N$.
\кзадача

\задача Сч\"етно ли \quad
% следующие множества:
%\сНовойСтроки
\пункт множество точек плоскости, обе координаты которых рациональны;\\
\пункт множество всех треугольников на плоскости, координаты
вершин которых рациональны;\\
\пункт множество всех многоугольников на плоскости, координаты
вершин которых рациональны?
\кзадача

\задача
Сч\"етно ли %следующие множества:
любое бесконечное множество непересекающихся\\
%\сНовойСтроки
\вСтрочку
%Сч\"етно ли \ \ \
\пункт %любое бесконечное множество непересекающихся
интервалов длины более $1$ на~прямой;
\пункт %любое бесконечное множество непересекающихся
интервалов на прямой;
\пункт %любое бесконечное множество непересекающихся
кругов на плоскости;
\пункт %любое бесконечное множество непересекающихся
восьм\"ерок на плоскости
(восьм\"ерка --- это любые две касающиеся внешним образом
окружности);
%\пункт букв \лк{\sf Г}\пк\
%(одного размера) на плоскости;
\спункт %любое бесконечное множество непересекающихся
букв \лк{\sf Т}\пк\
(любых размеров) на плоскости?
%%%(любых размеров) на плоскости?
\кзадача

\задача
Сч\"етно ли множество корней квадратных уравнений с рациональными
коэффициентами?
\кзадача


\сзадача
Может ли множество быть равномощно множеству всех своих подмножеств?
\кзадача


%\ссзадача



\ЛичныйКондуит{0mm}{6mm}



%\СделатьКондуит{6mm}{6.2mm}


\end{document}

