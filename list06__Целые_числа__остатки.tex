% !TeX encoding = windows-1251
\documentclass[a4paper,12pt]{article}
\usepackage{newlistok}
%\documentstyle[11pt, russcorr, listok]{article}
\newcommand{\del}{\mathrel{\raisebox{-.3 ex}{${\vdots}$}}}

\УвеличитьШирину{1truecm}
\УвеличитьВысоту{3.22truecm}
\hoffset=-2.5truecm
\voffset=-27.3truemm

%\documentstyle[12pt, russcorr, listok]{article}

%\УвеличитьПробелы{-2mm}{-1mm}
%\ВосстановитьПробелы
%первый параметр --- это на сколько изменить дополнительный
%промежуток перед пунктом.
%Второй --- на сколько изменить полную ширину пункта (т.е.
%буквы, скобки и пробела после)

%\УвеличитьПромежутки{85}
%\ВосстановитьПромежутки
%параметр --- в столько процентов изменить междустрочные промежутки

\Заголовок{Целые числа: остатки}
\НомерЛистка{6}
\ДатаЛистка{12.2013}

%\renewcommand{\spacer}{\vspace*{0.4\smallskipamount}}
\renewcommand{\baselinestretch}{0.97}
\begin{document}

\СоздатьЗаголовок

%\задача
%Докажите, что Ваше 28-летие будет
%в такой же день недели, в какой Вы родились.
%\кзадача

\smallskip

\опр
Пусть $a$ и $m$ --- целые числа, $m\ne 0$.
\выд{Разделить} $a$ на $m$ \выд{с остатком} значит найти
такие целые числа $k$ (частное) и $r$ (остаток),
что $a = km + r$ и $0\leq r < |m|$.
\копр





%\опр
%Если разность чисел $a$ и $b$ делится на $m$, то говорят, что {\it $a$ %сравнимо с $b$ по модулю $m$} и пишут А?В (mod М).
%\копр

\опр %Пусть $m\in\N$. % --- натуральное число.
Говорят, что \выд{$a$ сравнимо с $b$ по модулю $m$}, если
$a-b$ делится на~$m$. Обозначение: $a\equiv b\!\pmod{m}$.
\копр

\задача
Докажите, что $a\equiv b\!\pmod{m}$ тогда и только тогда, когда $a$ и $b$
дают одинаковые остатки от деления на $m$.
\кзадача

{\narrower{\small
\noindent
{\bf Замечание.}
Иногда, для удобства, остатком от деления $a$ на $m$ называют целое число, не обязательно лежащее в пределах от $0$ до $m$. %сравнимое с $a$ по модулю $m$.
Например, бывает удобно сказать, что число 29 при делении на 6 дает остаток $-1$. Или что число $9N$ дает остаток $2N$ при делении на 7 (число $2N$ может быть и больше 7, но главное, что оно сравнимо с $9N$ по модулю 7).
}

}

\задача
Пусть $a\equiv b\!\pmod{m}$,  $c\equiv d\ \!\pmod{m}$.
Докажите, что сравнения по одному и тому же модулю можно\\ %\qquad
%\сНовойСтроки
\пункт
складывать и вычитать: $a\pm c\equiv b\pm d\!\pmod{m}$;\\
\пункт
умножать: $ac\equiv bd\!\pmod{m}$;\\
%(сравнения по одному и тому же модулю можно умножать);\\
\пункт
возводить в натуральную степень $n$: $a^n\equiv b^n\!\pmod{m}$;\\
\пункт
домножать на любое целое число $k$: $ka\equiv kb\!\pmod{m}$.
\кзадача

\задача
Найдите частные и остатки от деления $2013$ на $23$, $-17$ на $4$ и
$n^2-n+1$~на~$n$.
\кзадача

\задача
Найдите остаток от деления числа $1+31+331+\ldots+3333333331$ на $3$.
\кзадача

\задача
Найдите остаток от деления числа  $1-11+111-1111+\ldots-1111111111$ на 9.
\кзадача

\задача
Найдите остатки от деления на 3 чисел $2N$, $100N$, $2^N$, $100^N$, $2007^N$ (ответ зависит от $N$).
\кзадача

\задача
Найдите остаток от деления  \пункт 10! на 11; \пункт 11! на 12.
\кзадача

\задача
\вСтрочку
\пункт
Какой цифрой оканчивается %число $14^{14}$?
%А число
$8^{18}$?
\пункт
При каких натуральных $k$ верно: $2^k-1\del7$?
\кзадача

\задача
Найдите две последние цифры числа $1999^{2000}$.
\кзадача


\задача
Докажите, что
\пункт $30^{99}+61^{100}$ делится на $31$;
\пункт $43^{95}+57^{95}$ делится на $100$.
\кзадача



\задача
Числа $x$ и $y$ целые, причем $x^2+y^2$ делится на 3.
Докажите, что и $x$ и $y$ делятся на 3.
\кзадача

\задача
Какие целые числа дают при делении на 3 остаток 2,
а при делении на 5 --- остаток 3?
% (и докажите, что других нет).
\кзадача

\задача
Даны 20 целых чисел, ни одно из которых не делится на 5. Докажите, что
сумма двадцатых степеней этих чисел делится на 5.
\кзадача

\задача
Докажите, что остаток от деления простого  числа на 30 есть или простое
число или 1.
\кзадача

\задача
Докажите, что из любых 52 целых чисел всегда можно выбрать два
таких числа, что\\
\вСтрочку
\пункт
их разность делится на 51;
\пункт
их сумма или разность делится на 100.
\кзадача

\задача
Докажите, что из любых $n$ целых чисел всегда можно выбрать несколько,
сумма которых делится на $n$ (или одно число, делящееся на $n$).
\кзадача

\сзадача
Существует ли делящееся на 2013 натуральное число,
состоящее из цифр 0 и 1?
%Найдётся ли натуральное число, все цифры которого только 0 и 1,
%делящееся на 2007?
\кзадача

\задача
Докажите, что существует бесконечно много натуральных чисел, не представимых как сумма трех или менее точных квадратов.
%Подсказка: рассмотреть остатки при делении на 8.
\кзадача

\задача
Шайка разбойников отобрала у купца мешок с монетами. Каждая монета стоит целое число грошей. Оказалось, что какую монету не отложи, оставшиеся монеты можно поделить между разбойниками так, что каждый получит одинаковую сумму. Докажите, что число монет без одной делится на число разбойников в шайке.
\кзадача

\ЛичныйКондуит{0mm}{8mm}

%\СделатьКондуит{7mm}{7mm}
%\СделатьКондуитФ{0mm}{6mm}{10}

%\сзадача Числа $a_1,\dots,a_n$ целые
%Для каждой пары целых чисел $i$ и $j$, где
%$1\leq j<j\leq n$, возьмем число $(a_i-a_j)/(i-j)$
%и перемножим все такие числа. Докажите, что получится
%целое число.
%%
%%Пусть $A$ --- произведение всевозможных разностей $a_i-a_j$, где
%%$1\leq j<j\leq n$, $B$ --- произведение всевозможных разностей $i-j$, где
%%$1\leq j<j\leq n$. докажите, что $A$ делится на $B$.
%\кзадача

\end{document}
