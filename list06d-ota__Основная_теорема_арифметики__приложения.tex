% !TeX encoding = windows-1251
\documentclass[a4paper,11pt]{article}
\usepackage{newlistok}
%\usepackage{tikz}
%\usetikzlibrary{calc}

%\documentstyle[11pt, russcorr, listok]{article}
\newcommand{\del}{\mathrel{\raisebox{-.3 ex}{${\vdots}$}}}

\УвеличитьШирину{1.4truecm}
\УвеличитьВысоту{2.9truecm}
\hoffset=-2.5truecm
\voffset=-25truemm


\begin{document}

\Заголовок{Основная теорема арифметики: приложения}
%Целые числа 3\ \ \ \ \ \ \ \ П Р О Е К Т}
%\Подзаголовок{Простые числа. Основная теорема арифметики}
\НомерЛистка{12}
\ДатаЛистка{09.2010}
\def\a{\alpha}

\СоздатьЗаголовок

%\vspace*{-1.5truemm}

%\опр
%Натуральное число $p>1$ называется \выд{простым}, если оно имеет ровно два
%натуральных делителя: 1 и $p$, в противном случае оно
%называется \выд{составным}.
%\копр

\задача
\вСтрочку
\!\!\!\! \пункт[Решето Эратосфена]\!\!\!
Выпишем в ряд целые числа от 2 до~$n$. Подчеркн\"ем число
2 и сотр\"ем~\hbox{числа,} делящиеся на 2.
Первое неподч\"еркнутое число подчеркн\"ем и сотр\"ем %теперь
числа, делящиеся на него, и т.~д.
%Снова первое невычеркнутое число обвед\"ем в кружок, и т.~д.
Будем~действовать так, пока каждое число от 2 до $n$
не будет либо подч\"еркнуто, либо ст\"ерто.
Докажите, что мы подчерк\-н\"ем в точности простые числа
от~1~до~$n$.\!
\пункт\!\!\! Пусть очередное число, которое мы хотим~\hbox{подчеркнуть, больше $\sqrt{n}$.}
Докажите, что не\-ст\"ер\-тые к этому моменту числа от 2 до $n$ простые.
%натуральное число $a$, большее 1, не делится ни на одно простое число, меньшее $\sqrt{a}$. Докажите, что $a$ простое.
\пункт Какие числа, меньшие $100$, простые?
\кзадача


\задача
Назов\"ем ч\"етное число $n$ \выд{ч\"етнопростым}, если $n$
не раскладывается
в произведение двух ч\"етных чисел. (Например, 6 --- ч\"етнопростое,
а 12 --- нет.)
%Верно ли, что любое ч\"етное
%число единственным образом раскладывается в произведение ч\"етнопростых
%чисел (с точностью до порядка сомножителей)?
Какие пункты задачи 2 будут верны, если
заменить в условии целые числа на ч\"етные, а
простые %числа
--- на ч\"етнопростые?
\кзадача

\задача 
\пункт При каких натуральных $k$ число $(k-1)!$ не делится на $k$?
\пункт При каких нечетных $n=2k+1$ число $k!+(k+1)\cdot\ldots\cdot(2k)$ не делится на $n$?
\кзадача


\задача
Целые числа $a,b,c,d$ таковы, что $ab=cd$. Может ли число $a+b+c+d$ быть простым?
\кзадача

\задача
\вСтрочку
\пункт [Теорема Лежандра] Докажите, что простое число
$p$ входит в каноническое разложение числа $n!$
в степени $[n/p]+[n/p^2]+[n/{p^3}]+\dots$
(где $[x]$ --- это \выд{целая часть} числа $x$).\\
С какого момента слагаемые в этой сумме станут равными нулю?\\
\пункт Сколько у $2000!$ нулей в конце его десятичной записи?
% числа $2000!$?
\пункт Может ли $n!$ делиться на $2^n$ ($n\geq1$)?
\кзадача


\задача
Число $p$ простое. Докажите, что $C_p^k$ делится на
$p$, если $0<k<p$.
\кзадача

%\задача[Малая теорема Ферма]
%Докажите: $n^p-n$ делится на $p$, если $p$ ---
%простое,~\hbox{$n$ --- целое.}
%%Пусть $p$ простое.
%\кзадача

\задача[Малая теорема Ферма]
Пусть $p$ --- простое число, $n$ --- целое число.
Докажите индукцией по $n$, что\\
\вСтрочку
\пункт  $n^p-n$ делится на $p$;
\пункт  если $(n,p)=1$, то $n^{p-1}-1$ делится на $p$.
\кзадача


%\задача
%Для каких $k\in\N$ есть $k$ последовательных
%целых чисел, являющихся составными?
%\кзадача


\сзадача
\вСтрочку
\пункт
Числа $p$ и $q$ простые, $2^{p}-1\del q$. Докажите,
что $q-1\del p$.
\пункт
Простое ли $2^{13}-1$?
\кзадача


\сзадача Может ли быть целым число
\вСтрочку
\пункт
$\displaystyle{\frac{1}{2}+\frac{1}{3}+\frac{1}{4}+\ldots+\frac{1}{n}}$;
\пункт
$\displaystyle{\frac{1}{3}+\frac{1}{5}+\frac{1}{7}+\ldots+\frac{1}{2n+1}}$?
\кзадача



%\сзадача Найдутся ли 100 таких различных натуральных чисел, что для
%любых двух чисел $a$ и $b$ из них
%\пункт $ab$~делится на сумму всех 100 чисел;
%\пункт $a+b$ делится на $a-b$;
%\пункт $(a,b)=|a-b|$?
%\пункт Тридцать~три~богаты\-ря едут верхом по кольцевой дороге против часовой стрелки.
%Могут ли они ехать неограниченно %различными
%долго~с~\hbox{разными} посто\-янными скоростями,
%если на дороге есть только одна точка, %в которой
%где богатыри %имеют возможность
%могут обгонять друг друга?
%\кзадача

\vspace*{-1mm}
\раздел{***}

\vspace*{-2mm}
\задача
Найдите все натуральные числа c
неч\"етным числом натуральных делителей.
\кзадача

\задача Число $n$ натуральное. Докажите, что натуральных делителей
у $n$ меньше, чем $2\sqrt n$.
\кзадача

\задача Пусть $p_1^{\a_1}\cdot \dots \cdot p_k^{\a_k}$ ---
каноническое разложение числа $n$. Обозначим через $\tau(n)$ и $S(n)$ соответственно количество и сумму натуральных делителей~$n$, а
через $\varphi(n)$ --- количество чисел от 1 до $n$, взаимно простых~с~$n$.
\вСтрочку
\пункт Найдите $\tau(p_1^{\alpha_1})$.  \ $\!\!\!$
\пункт Верно ли, что $\tau(ab)=\tau(a)\tau(b)$, если $(a,b)=1$? \ $\!\!$
\пункт Найдите $\tau(n)$.\\
\пункт Найдите $S(p_1^{\alpha_1})$.
\пункт Верно ли, что $S(ab)=S(a)S(b)$, если $(a,b)=1$?
\пункт Найдите $S(n)$.\\
\пункт Найдите $\varphi(p_1^{\alpha_1})$.
\пункт Верно ли, что $\varphi(ab)=\varphi(a)\varphi(b)$, если $(a,b)=1$?
\пункт Найдите $\varphi(n)$.
%\пункт Найдите произведение всех натуральных делителей числа $n$.
\кзадача

\задача
Какие натуральные числа делятся на 30 и имеют ровно
20 натуральных делителей?
\кзадача

%\сзадача [Теорема Эйлера] Пусть $a$ и $m$ натуральные, $(a,m)=1$.
%Докажите, что $a^{\varphi(m)}-1$ делится на $m$.
%\кзадача

\сзадача Число $n$ натуральное. %Рассмотрим
Докажите, что количество упорядоченных пар натуральных
чисел $(u;v)$, где $[u,v]=n$, равно количеству
%наименьшее общее кратное которых равно $n$.
%(Если $u\ne v$, то пара $(u;v)$ считается отличной от пары $(v;u)$.)
%Докажите, что таких пар %существует
%столько же, сколько %существует
натуральных делителей у числа $n^2$.
\кзадача


\сзадача Натуральное число называется \выд{совершенным,}
если оно равно сумме всех своих натуральных делителей, меньших его самого.
Докажите,~что~ч\"етное число $n$ совершенно
%тогда и только тогда, когда  $n=2^{p-1}(2^p-1)$
%для некоторых простых чисел $p$ и $2^{p}-1$.
тогда и только тогда, когда  найдется такое простое $p$,
что $2^p-1$ также простое, и $n=2^{p-1}(2^p-1)$.
\кзадача

\ЛичныйКондуит{0mm}{5mm}

%\СделатьКондуит{4.3mm}{7.5mm}


\end{document}

\раздел{Запас}

\задача
$a$ и $b$ --- натуральные числа.
Докажите, что если $[a,a+5]=[b,b+5]$, то $a=b$.
\кзадача

\задача
Число $n$ натуральное. Пусть $k$ --- наименьшее
целое число, большее 1 и
взаимно простое с каждым из чисел $1$, $2$, \dots, $n$.
Докажите, что $k$ существует и является простым.
\кзадача

\задача %Пусть $n\in\N$.
Сколько двоек в разложении числа
$1001\cdot1002\cdot\ldots\cdot2000$ на простые множители?
\кзадача

\задача Докажите, что при любом целом $n>1$ между $n$ и $n!$ есть простое
число.
\кзадача

\задача
Решите в натуральных числах уравнения:
\вСтрочку
\пункт
$x^y=y^x$;
\пункт
$u^x+u^y=u^z$.
\кзадача

\сзадача Найдите все натуральные $n$,
у которых не меньше чем $\sqrt n$ натуральных делителей.
\кзадача

\сзадача
Для целых $a,b$ и натурального $n$ докажите:  %, что
$b^n(a+b)(a+2b)\cdot\ldots\cdot(a+nb)$ кратно $n!$.
\кзадача


\сзадача Числа $a$, $b$, $c$ и $d$ натуральные, %прич\"ем
$ab=cd$. Может ли число $a+b+c+d$ быть простым?
\кзадача


\end{document} 