% !TeX encoding = windows-1251
\documentclass[a4paper, 12pt]{article}
\usepackage{newlistok}
%\documentstyle[11pt, russcorr, listok]{article}
\newcommand{\0}[1]{\overline{#1}}
\def\C{\mbox{$\Bbb C$}}

\УвеличитьШирину{1.5truecm}
\УвеличитьВысоту{3.5truecm}

\begin{document}


\Заголовок{Основная теорема алгебры}
\НомерЛистка{11д}
\ДатаЛистка{09.2016}

\СоздатьЗаголовок



\centerline{\sl Формулировка}

\vspace{3pt}

{\it
Произвольный многочлен степени $n>0$ с комплексными коэффициентами имеет
ровно $n$ комплексных корней (считаемых со своими кратностями).}

\rm

\vspace{7pt}

\centerline{\sl Обозначения}

\vspace{3pt}

\noindent
$P(z)$ --- некоторый произвольно выбранный многочлен от комплексной
переменной $z$ с~комплексными коэффициентами степени $n>0$.

\noindent
${\Bbb D}(z_0,\rho)$ --- круг с центром в точке $z_0\in\C$ радиуса $\rho$,
т.~е. $\{z\in\C:|z-z_0|\le\rho\}$.

\vspace{7pt}

\centerline{*~*~*}

\vspace{3pt}

\задача [Поведение многочлена на бесконечности] Докажите, что
$|P(z)|\rightarrow+\infty$ при $|z|\rightarrow+\infty$.
\кзадача

\задача
Многочлен $P(z)$ можно рассматривать как функцию комплексной переменной
$z$ с комплексными значениями, а $|P(z)|$
можно рассматривать как функцию комплексной переменной
$z$ с вещественными значениями.\\
\пункт
Дайте определение непрерывности функции комплексной переменной
(аналогично определению для функции вещественной переменной).\\
\пункт
Докажите, что функции $P(z)$ и $|P(z)|$ непрерывны.
\кзадача

\задача [Поведение многочлена в круге]
Докажите, что
$|P(z)|$ ограничен в любом круге (конечного радиуса) и достигает в н\"ем
своих максимума и минимума.\\
{\small (Разрешается решить эту задачу для квадрата со сторонами,
параллельными осям координат, вместо круга --- это немного проще и этого достаточно
для дальнейшего.)}

\кзадача

\задача [Разложение Тейлора] Докажите, что
для любого $z_0\in\C$ существуют такое $k\in\N$ и такие $c_k,c_{k+1},\dots,c_n\in\C$,
что $c_k\ne0$ и для любого $z\in\C$ справедливо равенство
$$
P(z)=P(z_0)+c_k(z-z_0)^k+c_{k+1}(z-z_0)^{k+1}+\dots+c_n(z-z_0)^n\eqno (*)
$$
Представление $P(z)$ в таком виде называется \выд{разложением Тейлора
многочлена $P(z)$ в точке~$z_0$}.
\кзадача

\задача [Поведение многочлена в малой окрестности точки]
Пусть $(*)$ --- разложение Тейлора многочлена $P(z)$ в точке $z_0\in\C$.
\сНовойСтроки
\пункт
Докажите, что  существует такое $\rho>0$, что для любого $z\in{\Bbb D}(z_0,\rho)$,
$z\ne z_0$, справедливо неравенство
$$
|P(z)|<|P(z_0)+c_k(z-z_0)^k|+|c_k(z-z_0)^k| \eqno (**)
$$
\пункт
Пусть для любого $z\in{\Bbb D}(z_0,\rho)$, $z\ne z_0$, выполнено соотношение
$(**)$, и, кроме того, $P(z_0)\ne0$. Докажите, что  существует такое $z_1\in{\Bbb D}(z_0,\rho)$,
что $|P(z_1)|<|P(z_0)|$.
\кзадача

\задача [Поведение многочлена на плоскости]
\сНовойСтроки
\пункт
Докажите, что  $|P(z)|$ достигает на плоскости своего минимума: существует такое
$\mu\ge0$, что $|P(z)|\ge\mu$ при любом $z\in\C$, прич\"ем
найд\"ется такое $z_0\in\C$, что~$|P(z_0)|=\mu$.
\пункт Пусть $\mu$ такое, как в п.~а). Докажите, что  $\mu=0$.
\кзадача

\задача  Докажите, что  всякий %произвольный
многочлен ненулевой степени с комплексными коэффициентами
имеет хотя бы один комплексный корень, и выведите отсюда основную теорему
алгебры.
\кзадача

\задача \пункт Разложите в произведение многочленов не более чем второй
степени с вещественными коэффициентами многочлены $x^4+3x^2+2$, $x^4+4$,
$x^n-1$.\\
\пункт Докажите, что произвольный
многочлен  с вещественными
коэффициентами раскладывается в произведение многочленов не более чем
второй степени с вещественными коэффициентами.
\кзадача


\задача Многочлен $P(x)\in\R(x)$ при всех $x\in\R$ принимает только
неотрицательные значения. Докажите, что его можно представить в виде
суммы нескольких квадратов многочленов с вещественными коэффициентами.
\кзадача

\задача
Докажите, что максимум $|P(z)|$ в круге достигается в некоторой точке
граничной окружности этого круга.
\кзадача

\ЛичныйКондуит{0mm}{8mm}
% %\GenXMLW
%\СделатьКондуит{11mm}{9mm}

%\сзадача Многочлен $P(x)\in\R(x)$ при всех неотрицательных
%$x$ принимает только неотрицательные значения.
%Докажите, что его можно представить в виде  $Q(x)/R(x)$, где
%$Q(x)$ и  $R(x)$ --- многочлены с неотрицательными коэффициентами.
%\кзадача


%\сзадача

%\кзадача

%\сзадача
%\кзадача

\end{document} 