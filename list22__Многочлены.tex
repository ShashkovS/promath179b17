% !TeX encoding = windows-1251
\documentclass[a4paper, 12pt]{article}
\usepackage{newlistok}
%\usepackage{tikz}
%\usetikzlibrary{calc}

%\documentstyle[11pt, russcorr, listok]{article}
\newcommand{\del}{\mathrel{\raisebox{-.3 ex}{${\vdots}$}}}

\УвеличитьШирину{.8truecm}
\УвеличитьВысоту{4.5truecm}
\hoffset=-2.45truecm
\voffset=-13truemm


\begin{document}

\Заголовок{Многочлены. Корни многочленов}
\НомерЛистка{22}
\ДатаЛистка{09.2015}
%\Подзаголовок{}

\СоздатьЗаголовок

%\vfill
%\раздел{Основные определения}


\опр \выд{Многочленом степени $n$ от одной переменной $x$}
называется любое выражение вида
$$
a_nx^n + a_{n-1}x^{n-1} + \dots  + a_0,  %\eqno (*)
$$
где
$n\in\N\cup\{0\}$,
а \выд{коэффициенты}
$a_n,\dots ,a_0$ --- любые числа (даже комплексные), прич\"ем $a_n\ne0$.
Краткое обозначение: $A(x)$ или $A$. Коэффициент $a_n$ называют
\выд{старшим}.
Степень ненулевого многочлена $A$ обозначают $\deg A$.
Число 0 называют \выд{нулевым} многочленом, его степень не определена.
Множества всех многочленов с целыми, рациональными, действительными, комплексными коэффициентами обозначаются соответственно $\Z[x]$, $\Q[x]$, $\R[x]$, $\Cbb[x]$.
\копр

\задача
Определите \выд{сумму} и \выд{произведение} многочленов.
% --- это сумма и
%Докажите, что сумма и произведение многочленов также
%являются многочленами.
%Как найти их коэффициенты?
\кзадача

\задача \вСтрочку
\пункт Пусть $\deg A=10$, $\deg B=\deg C=7$. Какими могут быть
$\deg(A+B)$ и $\deg(B+C)$?\\
\пункт Докажите, что $\deg AB=\deg A+\deg B.$
%Как связана $\deg (A+B)$ с $\deg A$ и $\deg B~?$
\пункт Докажите, что $\deg A(B(x))=\deg A\cdot \deg B$.
\кзадача

\задача Может ли произведение нескольких ненулевых многочленов
быть нулевым многочленом?
\кзадача



%\задача
%\пункт  Докажите, что  для любого натурального числа $n$ и
%любого действительного
%числа $C$ найд\"ется такое число $x$, что $x^n>C(1+x+\dots +x^{n-1})$.
%\пункт  Докажите, что  если  многочлен зада\"ет нулевую функцию, то
%он нулевой.
%\пункт  Докажите, что разные многочлены задают разные  функции.
%\кзадача

%\noindent {\bf Замечание.}
%Таким образом, можно не различать многочлен и задаваемую им функцию.

%\опр \выд{Сумма} и \выд{произведение} многочленов --- это сумма и
%произведение соответствующих функций.
%\копр


%\задача Найдите сумму  и произведение многочленов \\ \вСтрочку
%\пункт $a_3x^3+a_2x^2+a_1x+a_0$  и  $b_2x^2+b_1x+b_0;$
%\пункт  $x^{19}-9x+7$ и  $x^7+99x+1.$
%\кзадача


\noindent
%{\narrower

%}

%\задача
%Старший коэффициент многочлена $A$ положителен.
%Пусть $\deg A>0$.
%Докажите, что последовательность $A(1)$, $A(2)$,  \dots \
%\hbox{бесконечно большая.}
%$\lim\limits_{n\rightarrow+\infty}A(n)=+\infty$.
%\кзадача


%\задача Может ли ненулевой многочлен задавать нулевую функцию?
%\кзадача

\опр %{\bf Замечание.}
Многочлен $A(x)$ зада\"ет функцию, которая сопоставляет каждому числу $s$ %\in\R$
число $A(s)$ (результат подстановки в выражение $A(x)$ числа $s$ вместо переменной $x$).
\копр

\задача
Найдите сумму всех коэффициентов многочлена:\\
\вСтрочку
\пункт $(x-1)^{n}$;
\пункт $(x+1)^{n}$;
\пункт $(x-2)^{n}$;
\пункт $(x+2)^{n}$;
\пункт $(1-x+x^4)^{1000}.$\\
\пункт Найдите сумму коэффициентов при неч\"етных степенях
у многочлена из пункта д).
\кзадача


\раздел{Число корней многочлена}


%\опр Многочлен $A$ \выд{делится} на многочлен $B,$ если
%найд\"ется такой многочлен~$C,$ что $A=BC.$
%\копр

%\опр %{\bf Замечание.}
%Многочлен $A(x)$ зада\"ет функцию $A:\R\rightarrow\R,$
%которая сопоставляет каждому действительному
%числу $s$ %\in\R$
%число $A(s)$ (результат
%подстановки в выражение $A(x)$ числа $s$ вместо переменной $x$).
%\копр

\опр
Число  $s$  называется \выд{корнем} многочлена  $A,$  если  $A(s)=0.$
\копр

%\опр
%Если в результате подстановки числа $r$ в многочлен получился 0,
%то $r$ называют \выд{корнем} этого многочлена.
%\копр

\задача Докажите, что если многочлен $A$ \выд{делится} на многочлен $B,$
то есть существует такой многочлен~$C,$ что $A=BC$, то
все корни $B$ являются корнями $A.$ Верно ли
обратное утверждение?
\кзадача

\задача Делится ли многочлен $x^9-1$ на многочлен $x$?
А на многочлен $x^2-1$?
\кзадача

\задача  Произвольный многочлен $A(x)$ домножили на $(x-1)$. Могут ли у
получившегося многочлена все коэффициенты быть положительными?
\кзадача

\задача
Докажите, что число $s$ --- корень многочлена $A(x)$
если и только если $A(x)$ делится на $x-s$.
%$s$ --- корень $A(x)$.
\кзадача


\задача Пусть $A(1)=A(2)=0.$  Докажите, что  $A(x)$ делится на $(x-1)(x-2).$
\кзадача

\задача Докажите, что число различных корней многочлена $A$ не больше
$\deg A.$
\кзадача

\задача
Могут ли разные многочлены задавать одну и ту же функцию?
\кзадача

\задача Пусть многочлен $A(x)$ таков, что $A(x)=A(-x)$ при любом $x$.
Докажите, что  существует такой многочлен $P(x),$ что
$A(x)=P(x^2)$ при любом $x$.
\кзадача


\задача Можно ли задать многочленом функцию $\sin x$?
\кзадача

\задача Пусть значения многочленов $A$ и $B$
совпадают при $n$ различных значениях переменной, и степени
этих многочленов меньше $n$. Докажите, что тогда $A=B.$
\кзадача

\задача
В скольких точках прямая может пересекать параболу?
\кзадача


%\задача
%%\пункт  Может ли ненулевой многочлен задавать нулевую функцию?
%%\пункт
%Могут ли разные многочлены задавать одинаковые функции?
%\кзадача

\задача
%\пункт  Докажите, что  любой квадратный тр\"ехчлен
%можно представить в виде $a+bx+cx(x-1).$\\
%\пункт Найдите уравнение параболы, проходящей через точки
%$(0;1)$, $(1;2)$ и $(2;4)$.\\
\пункт  Докажите, что  любой многочлен степени 3
представляется в виде
$$a+bx+cx(x-1)+dx(x-1)(x-2).$$
\пункт Найдите такой многочлен $P(x)$ степени 3,
что $P(0)=-8$, $P(1)=5$, $P(2)=6$, $P(3)=1$.
\кзадача

\задача
Даны различные числа $a_1,a_2,\dots ,a_n$ и любые числа $b_1,b_2,\dots ,b_n$.\\
\пункт
Найдите многочлен степени $n-1$, который равен $b_1$ при $x=a_1$ и равен 0 при $x\in\{a_2,\dots,a_n\}$.
\пункт
Докажите, что существует единственный многочлен $P(x)$ степени
меньше $n$ такой, что $P(a_1)=b_1,$  \dots , $P(a_n)=b_n.$
\кзадача



%\ЛичныйКондуит{.1mm}{6mm}{8}
\ЛичныйКондуит{0mm}{6mm}

%\СделатьКондуит{5mm}{7.7mm}

%\GenXMLW

\end{document}

\раздел{Корни многочленов с целыми коэффициентами}


\задача  Докажите, что  если многочлен $A(x)$ с целыми коэффициентами
принимает при $x=0$ и $x=1$ неч\"етные значения, то уравнение $A(x)=0$
не имеет целых решений.
\кзадача

\задача \пункт Ненулевая несократимая дробь $p/q$ --- корень многочлена
$A(x)=a_nx^n + \dots + a_0$ с целыми коэффициентами.
Докажите, что  тогда $a_n$ делится на $q$
и $a_0$ делится на $p$.\\
\пункт Пусть в предыдущем пункте $a_n=1$. Докажите, что
все рациональные корни $A$ --- целые числа.
\кзадача

\задача Найдите все рациональные корни многочленов:\вСтрочку
\пункт $x^3-6x^2+15x-14$;
\пункт $6x^4+19x^3-7x^2-26x+12$.
\кзадача

\задача
\пункт
Коэффициенты многочлена $Q$ рациональны, $Q(\sqrt2)=0$.
Докажите, что $Q(-\sqrt2)=0.$\\
%\кзадача
%
%\задача
\пункт
Найдите ненулевой
многочлен $P$ с целыми коэффициентами и корнем
$\sqrt2+\sqrt3$.
Найдите все корни~$P$.
\кзадача

%\newpage


\раздел{Теорема Виета}



\задача
\пункт Пусть многочлен $a(x)=x^3+ax^2+bx+c$
раскладывается на \выд{линейные} множители (т.~е.~многочлены
первой степени):
$a(x)=(x-\alpha_1)(x-\alpha_2)(x-\alpha_3)$.
Докажите, что справедливы \выд{формулы Виета:}
$$\alpha_1+\alpha_2+\alpha_3=-a, \quad
\alpha_1\alpha_2+\alpha_2\alpha_3+\alpha_3\alpha_1=b,\quad
\alpha_1\alpha_2\alpha_3=-c.$$
\спункт Найдите подобные формулы, если $\deg a(x)=n$, и
$a(x)$ раскладывается на линейные множители.
\кзадача


\задача
\вСтрочку
\пункт Про числа $a,b,c$ известно, что
$a+b+c>0,$ $ab+bc+ac>0,$ $abc>0.$
Докажите, что $a,b$~и~$c$ положительны.
\пункт Пусть  $a+b+c<0,$ $ab+bc+ac<0,$ $abc<0.$
Какие знаки могут иметь числа $a,b,c$?
\кзадача

\сзадача
\вСтрочку
\пункт Пусть число $c \ne 0.$  Докажите, что  многочлен $x^5+ax^2+bx+c$
не может раскладываться на пять линейных множителей.
%т.~е.~не может иметь пять вещественных корней (не обязательно различных).
\пункт Та же задача для многочлена $x^5+ax^4+bx^3+c$.
\кзадача

\задача
Коэффициенты многочлена $(x-a)(x-b)$ целые.
%Пусть $\alpha_1$ и $\alpha_2$ --- все корни квадратного тр\"ехчлена
%с целыми коэффициентами.
Докажите, что число $a^n+b^n$ целое при $n\in\N$.
\кзадача

\сзадача
Найдите первые $n$ цифр после запятой в десятичной записи
числа $(\sqrt{26}+5)^n$.
\кзадача


\раздел{Деление с остатком}

\опр\label{del}  Пусть $A$ и $B$ --- многочлены, причем $\deg B>0.$
Разделить $A$ на $B$ с остатком значит найти такие многочлены
$Q$ и $R,$ что $A=BQ+R,$ где либо $R=0$, либо $\deg R<\deg B$.
\копр

\задача  Разделите с остатком $2x^4-3x^3+4x^2-5x+6$ на $x^2-3x+1$.
\кзадача

\задача
\пункт  Докажите, что  деление многочленов с остатком всегда возможно.
\пункт  Докажите, что  при делении с остатком многочлены $Q$ и $R$
определяются однозначно.
\кзадача

\задача[Теорема Безу] Докажите, что остаток от деления многочлена $A(x)$
на двучлен $x-a$ равен значению многочлена $A(x)$ при  $x=a.$
\кзадача

\задача Остаток от деления $A(x)$ на $x-1$ равен 5, а на $x-3$ равен 7.
Найдите остаток от деления $A(x)$ на $(x-1)(x-3).$
\кзадача

\опр \выд{Наибольшим общим делителем} ({НОД}) двух многочленов,
один из которых ненулевой, называют многочлен
наибольшей степени, делящий оба этих многочлена.
\копр

\задача Насколько однозначно определяется {НОД} двух многочленов?
\кзадача

\задача Найдите {НОД} многочленов: \вСтрочку
\пункт $x(x-1)^3(x+2)$ \ и \  $(x-1)^2(x+2)^2(x+5);$
\пункт $3x^3-2x^2+x+2$ \  и \  $x^2-x+1;$ \
\спункт $x^m-1$ \ и \ $x^n-1;$ \
\спункт $x^m+1$ \ и \ $x^n+1.$
\кзадача

%\задача Докажите, что {НОД} многочленов $A$ и $B$--- это многочлен
%наименьшей степени, который делится на любой общий делитель $A$ и $B$.
%\кзадача



\задача
Пусть $A$ и $B$ --- любые многочлены степени $m$ и $n$ соответственно.
\сНовойСтроки
\пункт Докажите, что
существуют такие многочлены $U$ и $V,$ что {НОД}$(A,B)=AU+BV.$
\спункт  Докажите, что если $m,\ n>0,$
то $U$ и $V$ можно выбрать так, чтобы  $\deg U<n$ и $\deg V<m.$
\пункт Найдите такие $U$ и $V$, если $A$ и $B$ --- многочлены
из пункта б) предыдущей задачи.
\кзадача

%\задача Обозначим многочлены из пункта б) предыдущей задачи как $f$ и $g$.
%Найдите такие многочлены $u$ и $v,$ что {НОД}$(f,g)=fu+gv$,
%прич\"ем $\deg u<\deg g$ и $\deg v< \deg f$.
%\кзадача


\раздел{Дополнительные задачи}

\задача Пусть многочлен $A(x)$ таков, что $A(x)=A(-x)$ при любом $x$.
Докажите, что  существует такой многочлен $P(x),$ что
$A(x)=P(x^2)$ при любом $x$.
\кзадача

\задача
Пусть $p(x)$ --- непостоянный многочлен с целыми коэффициентами.
\сНовойСтроки
\пункт Докажите, что при любом целом числе $n$ либо
$p(n)$ делит $p(n+p(n))$, либо $p(n)=p(n+p(n))=0$.
\пункт Могут ли все числа $p(0)$, $p(1)$, $p(2)$, \dots\  быть простыми?
\кзадача

%\задача Пусть $s_1$, \dots, $s_k$  --- корни многочлена
%$a_nx^n+\dots+a_1x+a_0$.
%Найдите корни многочленов
%\вСтрочку \пункт $(-1)^na_nx^n+...+a_2x^2-a_1x+a_0;$
%\пункт $a_0x^n+a_1x^{n-1}+...+a_{n-1}x+a_n$.
%\кзадача

\задача Коэффициенты произведения двух многочленов с целыми
коэффициентами делятся на~5.  Докажите, что коэффициенты
одного из этих многочленов делятся на~5.
\кзадача

%\задача Пусть $p(x)$ --- многочлен с целыми коэффициентами.
%\сНовойСтроки
%\пункт Докажите, что $a-b$ делит $p(a)-p(b)$  при любых различных
%целых числах $a$ и $b$.
%\спункт Пусть уравнения $p(x)=1$ и $p(x)=3$ имеют целое решение.
%Может ли уравнение $p(x)=2$ иметь два различных целых решения?
%\кзадача

\задача Используя равенство $(1+x)^p(1+x)^q=(1+x)^{p+q}$,
вычислите двумя способами коэффициент при $x^m$ в многочлене $(1+x)^{p+q}$
и решите задачу 34 листка 3.
\кзадача

\задача
Коэффициенты квадратного уравнения $x^2+px+q=0$ изменили не больше, чем
на 0,001. Может ли больший корень уравнения измениться больше,
чем на 1000?
\кзадача

\сзадача
У многочлена $P(x)$ есть  отрицательный коэффициент. Могут ли у всех
его степеней $P^n(x)$ (где $n>1$ --- целое) все коэффициенты быть
положительными?
\кзадача


%--------------------------------------------------------------------------

\end{document}

\задача
\пункт Квадратный тр\"ехчлен $ax^2 + bx + c$ принимает при каждом целом
$x$ целое значение. Верно ли, что среди его коэффициентов
хотя бы один --- целое число?
\пункт Верно ли, что все его коэффициенты --- целые числа?
\кзадача


\задача
Правильные треугольники со сторонами 1, 3, 5, \dots\
расположены в ряд  так, что их основания лежат
на одной прямой вплотную друг к другу.
Докажите, что вершины треугольников, противоположные основаниям,
лежат на некоторой параболе.
\кзадача



\задача \пункт Для каждого $x\in\{-3,-2,-1,-1/2,\ 0,\ 1/2,\ 1,\ 2,\ 3\}$
нарисуйте на плоскости $pOq$ графики прямых, задающихся уравнениями
$x^2+px+q=0.$
\пункт
Напишите уравнение, задающее множество таких точек $(p,q),$ что
квадратный \break
тр\"ехчлен $x^2+px+q$ имеет кратный корень, и изобразите его на
плоскости.
\пункт Докажите, что все прямые из п.~а)
касаются\footnote{В этой задаче будем считать,
что прямая $l$ \выд{касается} некоторой кривой, если эта кривая лежит по одну
сторону от прямой $l$ и имеет с $l$ ровно одну общую точку.}
некоторой кривой. Что это за кривая?
\пункт Укажите на плоскости множества таких точек $(p,q),$ что квадратный
тр\"ехчлен\break $x^2+px+q$ имеет два различных корня, не имеет корней.
\спункт Укажите на плоскости множества таких точек $(p,q),$ что
тр\"ехчлен $x^2+px+q$ имеет на отрезке $[-1;1]$ два различных корня,
кратный корень, не имеет корней.
\кзадача


\end{document} 