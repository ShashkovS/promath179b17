% !TeX encoding = windows-1251
\documentclass[12pt,a4paper]{article}
\usepackage[mag=970]{newlistok}
%\usepackage[mag=990]{newlistok}


\УвеличитьШирину{1truecm}
\УвеличитьВысоту{2.5truecm}
\hoffset=-2.5truecm
\voffset=-25truemm

%\documentstyle[12pt,russcorr,listok]{article}

\Заголовок{Комбинаторика}
\НомерЛистка{3}
\ДатаЛистка{10.2013}


\begin{document}

\СоздатьЗаголовок

\задача В классе учатся 20 человек. Сколькими способами из них можно
выбрать двоих школьников: старосту и ответственного за проездные билеты?
А просто двоих школьников?
\кзадача

\задача
Сколько разных слов (не только осмысленных) можно получить,
переставляя буквы в словах
\вСтрочку
\пункт
{\tt РОК};
\пункт
{\tt КУРОК};
\пункт
{\tt КОЛОБОК};
\пункт
$\underbrace{{\texttt А}{\texttt А}\dots{\texttt А}}_a
\underbrace{{\texttt Б}{\texttt Б}\dots{\texttt Б}}_b$?
\спункт
$\underbrace{{\texttt ы}_1\dots{\texttt ы}_1}_{k_1}
\underbrace{{\texttt ы}_2\dots{\texttt ы}_2}_{k_2}
\,\dots\,\dots\,
\underbrace{{\texttt ы}_m\dots{\texttt ы}_m}_{k_m}$.
\кзадача

\задача
\пункт Сколькими способами можно выбрать тр\"ех дежурных в классе
из 20 человек?\\
\пункт А сколькими способами можно выбрать старосту, его помощника
и тр\"ех дежурных?
\кзадача

\опр \выд{Числом сочетаний из $n$ элементов по $k$} называется количество
способов выбрать $k$ предметов из $n$ различных предметов.
Обозначение: $n\choose k$
или $C_n^k$ (читается \лк це из $n$ по $k$\пк).
\копр

\задача
Докажите, что
\вСтрочку
\пункт
$C_n^k=C_n^{n-k}$;
\пункт
$C_{n+1}^k=C_n^k+C_n^{k-1}$.
\кзадача

\задача
Найдите формулу для $C_n^k$.
\кзадача

\УстановитьГраницы{0cm}{5.3cm}
\задача
\пункт
На рисунке изображен план города
(линии --- это~ули\-цы,  пересечения линий --- перекрестки).
На улицах~\hbox{введено} одностороннее движение: мож\-но
ехать только \лк вверх\пк\ или \лк впра\-во\пк. Сколько разных
маршрутов вед\"ет из точ\-ки $A$ в точ\-ку $B$?
\ВосстановитьГраницы
\пункт
Сколько из этих маршрутов не проходят через отмеченную на плане
точку внутри города?
\кзадача

\newsavebox{\блок}
\sbox{\блок}{\line(0,1){60}}

\vspace*{-4.3truecm}
\setbox5\vbox{
{\hsize 6truecm
\begin{picture}(150,75)
\multiput(150,-10)(15,0){8}%
{\usebox{\блок}}
\multiput(150,-10)(0,15){5}%
{\line(1,0){105}}
\put(135,-15){{$A$}}
\put(150,-10){\circle*{3.5}}
\put(260,45){{$B$}}
\put(255,50){\circle*{3.5}}
\put(195,20){\circle*{3.5}}
\end{picture}

}

}

\centerline{\hfil\copy5}

\vspace*{1.5truecm}

\задача Сколькими способами можно рассадить класс,
если пришло 27 человек, а мест 30?
\кзадача

\задача
Сколькими способами можно высадить в ряд 3 груши и 4 яблони?
\кзадача

\УстановитьГраницы{0cm}{7.3cm}
\опр \выд{Треугольником Паскаля} называют числовой треугольник,
изображенный на рисунке (по краям треугольника стоят единицы,
а каждое из остальных чисел равно сумме двух, стоящих справа
и слева над ним).
\копр
\ВосстановитьГраницы

\vspace*{-2.75truecm}

\setbox3\vbox{
{\hsize 6.5truecm
$$
\begin{array}{ccccccccccc}
&&&&&1\\
&&&&1&&1\\
&&&1&&2&&1\\
&&1&&3&&3&&1\\
&1&&4&&6&&4&&1\\
.&&.&&.&&.&&.&&.
\end{array}
$$

}

}

\centerline{\ \hfill\copy3}

\vspace*{-1.45truecm}

\УстановитьГраницы{0cm}{8cm}
\задача
На рисунке выписаны первые 5 строк треугольника Паскаля.
Напишите следующие 5 строк.
\кзадача
\ВосстановитьГраницы

\задача
Докажите, что $k$-ое число $n$-ой строки равно $C_n^k$
(строки нумеруются сверху вниз, начиная с нуля,
а числа в строках нумеруются слева направо, также начиная с нуля).
\кзадача

\задача
Докажите, что сумма чисел в $n$-ой строке треугольника Паскаля
равна~$2^n$.
\кзадача


\задача
\пункт Раскройте скобки и приведите подобные в выражениях
$(a+b)^2$, $(a+b)^3$, $(a+b)^4$.
\пункт [Бином Ньютона] Раскроем скобки и привед\"ем подобные
в выражении $(a+b)^n$. Возьм\"ем любое слагаемое.
Оно имеет вид $C\cdot a^k\cdot b^{n-k}$ (почему?).
Докажите, что $C=C_n^k$.
\кзадача

\задача
Докажите:
\пункт $C_n^1+2C_n^2+3C_n^3+\dots+nC_n^n=n2^{n-1}$;
\пункт $C_n^0-C_n^1+C_n^2-C_n^3+\dots+(-1)^nC_n^n=0$.
\кзадача

\задача
Возьм\"ем любое число $C$ в треугольнике Паскаля и сложим все
числа, начиная с него и идя по прямой направо-вверх.
Докажите, что сумма равна числу, стоящему под $C$ справа.
%Запишите доказанное утверждение в виде тождества.
\кзадача

\задача
Выведите из задачи 14 формулы для сумм \
$1+\ldots+n$,\ $T_1+\ldots+T_n$,\ ${\textit П}_1+\ldots+{\textit П}_n$.
\кзадача

\сзадача
Как из предыдущей задачи вывести формулы для сумм $1^2+\dots+k^2$,
$1^3+\dots+k^3$, \ldots?
\кзадача

\сзадача
Отметьте в треугольнике Паскаля чётные числа.
В каких строках %треугольника Паскаля
все числа неч\"етные?
\кзадача

\сзадача
Докажите, что
$C_p^0\cdot C_q^m+C_p^1\cdot C_q^{m-1}+\dots+C_p^{m-1}\cdot C_q^1
+C_p^m\cdot C_q^0=C_{p+q}^m$.
\кзадача

\сзадача
Докажите, что любые два не единичных числа из одной строки треугольника Паскаля имеют общий множитель, больший 1.
\кзадача

\vspace*{-0.3truecm}

\раздел{$***$}

\vspace*{-0.2truecm}

\задача
\пункт
В НИИ работают 67 человек. Из них
47 знают английский язык, 35 --- немецкий, и 23 --- оба языка.
Сколько человек в НИИ не знают ни английского, ни немецкого языков?\\
\пункт Пусть кроме этого польский язык
знают 20 человек, английский и польский --- 12, немецкий и
польский --- 11, все три языка --- 5.
Сколько человек не знают ни одного из этих языков?
%\спункт [Формула включений и исключений]
%Решите задачу в общем случае: имеется $m$ языков,
%и для каждого набора языков известно, сколько человек знают все языки
%из этого набора.
\кзадача

\задача
В ряд записали 105 единиц, поставив перед каждой знак \лк$+$\пк.
Сначала изменили знак на противоположный перед каждой третьей единицей,
затем --- перед каждой пятой, а затем --- перед
каждой седьмой. Найдите значение полученного выражения.
\кзадача

\задача \вСтрочку
\пункт
На полке стоят 10 книг. Сколькими способами их можно переставить
так,~чтобы ни одна книга не осталась на месте?
\пункт А если на месте должны остаться ровно 3 книги?
\кзадача


\ЛичныйКондуит{0mm}{6mm}


%\СделатьКондуит{5.6mm}{7mm}
\end{document}
