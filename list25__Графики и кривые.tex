% !TeX encoding = windows-1251
\documentclass[a4paper,12pt]{article}
\usepackage{newlistok}

\УвеличитьШирину{1truecm}
\УвеличитьВысоту{2.5truecm}
% \hoffset=-2.5truecm
% \voffset=-27.3truemm
%\def\hang{\hangindent\parindent}
%\documentstyle[11pt, russcorr, ll]{article}



\Заголовок{Графики и кривые}
\Подзаголовок{}
\НомерЛистка{25}
\ДатаЛистка{11.2015}

%\overfullrule=3pt

\begin{document}

\СоздатьЗаголовок

{\hsize 14.2truecm
\задача
График %функции
$f$ %имеет график,
изображ\"ен на рис.~1.
Нарисуйте %поточнее
графики~\hbox{функций:}\\
\вСтрочку
\пункт   $2\,f(x)$;
\пункт   $f(2x)$;
\пункт   $f(x+2)$;
\пункт   $f(x)+2$;
\пункт   $f(-x)$;
\пункт   $f(|x|)$;\\
\пункт   $|f(-|x|)|$;
%\пункт $3-f(x)$;
\пункт   $1-2\,f(x)$;
%\пункт $f\left(\dfrac{x-3}2\right)$;
\пункт   $f\left(\dfrac{1}2 x-3\right)$;
\пункт   $3-2f(3-2x)$;\\
\пункт   $1-\left|f\left(\dfrac{1-|x|}3\right)\right|$;
%\пункт $x\,f(x)$;
\пункт   $\dfrac1{f(x)}$.
%\пункт $f\left(\dfrac1x\right)$;
%\пункт $\sqrt{f(x)}$.
\кзадача

%\задача
%Решите неравенство
%$\Bigl|\;3\,\bigl|\,1-2\,|x-1|\,\bigr|-5\;\Bigr|\ge2$.
%\кзадача

%\задача
%Постройте графики:
%\вСтрочку
%%\пункт $x^{2}$;
%%\пункт $3-2(x-5)^{2}$;
%\пункт $\sin(1/x)$;
%\пункт $x\sin(1/x)$.
%%\пункт $ax^{2}+bx+c$, %где
%%$a\ne0$.
%\кзадача


%\задача
%Постройте графики:
%\вСтрочку
%\пункт $\dfrac1x$;
%\пункт $1-\dfrac2{x+1}$;
%\пункт $\dfrac{x+1}{x-1}$;
%\пункт $\dfrac{2x+3}{2-3x}$.
%%\пункт $\dfrac{ax+b}{cx+d}$, где $ad\ne bc$.
%\кзадача

\задача
Выразите функцию $g$ через функцию $f$, если известно, что\\
график $g$ получается из графика $f$\\
%\сНовойСтроки
\вСтрочку
\пункт параллельным переносом на вектор $(3,5)$;\\
\пункт сжатием в 2 раза к оси $OY$;\\
\пункт  симметрией относительно оси $OX$;\\
\пункт симметрией относительно прямой $x=3$;\\
\пункт растяжением в 3 раза от прямой $y=-1$;\\
\пункт центральной симметрией относительно начала координат;\\
\пункт гомотетией с коэффициентом 2 относительно начала координат;\\
\пункт гомотетией с коэффициентом $-1/3$ относительно точки $(-3,2)$.
\кзадача


}


\putpict{16.2cm}{7cm}{pct_sample_func_graph}{Рис.~1. График $y=f(x)$}


%\EFIG{graf}{График $y=f(x)$}{50}{40}{0,0}{graf}


\vspace*{-5.2truemm}


\задача
Постройте графики:
\вСтрочку
\пункт $\sin(1/x)$;
\пункт $x\sin(1/x)$;
\пункт $\cos^2x$;
\пункт $\sqrt{\cos x}$.
\кзадача


\опр
Функция $f$ определена на множестве $A\subseteq\R$
и принимает значения в множестве $B\subseteq\R$.~Если
%существует
найд\"ется функция $g$,
которая определена на множестве $B$,
принимает значения в множестве $A$ и
$g(f(x))=x$ для всех $x\in A$, $f(g(y))=y$ для всех $y\in B$, то
$g$ называют \выд{обратной} к функции $f$.
Обозначение: $g=f^{-1}$.
\копр

\задача
%Пусть $g$  --- обратная к $f$ функция.
%\вСтрочку
%\пункт
Докажите, что графики $y=f(x)$  и $y=f^{-1}(x)$ симметричны
относительно прямой $y=x$.
%\пункт
%Какой вид имеет график $x=g(y)$?
\кзадача

\задача
Найдите обратную к функции\\
\вСтрочку
\пункт
\hskip-3pt $2x+3$; \hskip-3pt
\пункт
\hskip-3pt $x^3$; \hskip-3pt
\пункт
\hskip-3pt $1/x$; \hskip-3pt
\пункт
\hskip-3pt $x^3+1$; \hskip-3pt
\пункт
\hskip-3pt $x/(1+x)$; \hskip-3pt
\пункт
\hskip-3pt $\sqrt{1-x^2}$, $x\geq0$. %\in[0;1]$.
\кзадача




%\задача
%Где лежат середины любого семейства параллельных хорд данной
%параболы?
%\кзадача

%\задача
%Дан график
%\вСтрочку
%\пункт $y=x^2$;
%\пункт $y=1/x$;
%оси координат ст\"ерты.
%Как  восстановить их циркулем и линейкой?
%\кзадача


%\сзадача
%Найд\"ется ли функция, у  которой
%\вСтрочку
%\пункт график пересекается с любым кругом на плоскости;\\
%\пункт множество значений
%на каждом отрезке является множеством всех действительных чисел?
%\кзадача




\задача
 Нарисуйте кривые:\\
\вСтрочку
%\пункт $xy=1$;
\пункт $x^2=y^2$;
%\пункт $x^2+y^2=1$;
\пункт $х^2y-xy^2=x-y$;
%\пункт $x^2-y^2=1$;
\пункт $4\,x^2+9\,y^2=36$;
\пункт $16\,x^2-25\,y^2=400$.
\кзадача

\задача
 Нарисуйте кривые:\\
\вСтрочку
\пункт $y^2=x^3$;
\пункт $y=1+x^3$;
\пункт $y^2=1+x^3$;
\пункт $y^2=x+x^3$;
\пункт $y^2=x^2+x^3$.
\кзадача

\задача
%Множество всех решений $(x,y)$ уравнения $Ф(x,y)=0$ представляет собой
%кривую, изображ\"енную на рис.~\ref{curve}. Нарисуйте %(поточнее)
%кривые, заданные уравнениями:
Кривая на рис.~2 задана уравнением $Ф(x,y)=0$.
Нарисуйте %(поточнее)
кривые:\\
\вСтрочку
\пункт   $Ф(x+2,y-1)=0$;
\пункт   $Ф(y,x)=0$;
\пункт   $Ф(-x,-y)=0$;
\пункт   $Ф(2x,y)=0$;\\
\пункт   $Ф\left(\dfrac x2,2y\right)=0$;
\пункт   $Ф(|y|,x)=0$;
\пункт   $Ф(x+y, x-y)=0$.
\кзадача

%\СделатьКондуит{4.6mm}{8mm}



%\end{document}


\vspace*{-1truemm}

{\hsize 14truecm
\задача
%Напишите какое-нибудь
%Найдите уравнение,
Задайте уравнением кривую, получающуюся
из %показанной на рис.~\ref{curve}
кривой $Ф(x,y)=0$\\
\вСтрочку
%\сНовойСтроки
\пункт %параллельным переносом
сдвигом на вектор $(-2,3)$;\\
\пункт растяжением в 2 раза от оси $OX$;\\
\пункт симметрией относительно %оси
$OY$;\\
\пункт симметрией относительно прямой $y=2$;\\
\пункт центральной симметрией относительно начала координат;\\
\пункт гомотетией с коэффициентом 2 относительно начала координат;\\
\пункт поворотом на $45^{\circ}$ вокруг точки $O$;\\ %начала координат;
\пункт сжатием в 3 раза к прямой $x=1$;\\
\пункт гомотетией с коэффициентом $-1/3$ относительно точки $(1,-1)$.
\кзадача

}

\сзадача
Найдите уравнение, задающее\\
%\сНовойСтроки
\вСтрочку
%\пункт
%окружность радиуса $3$ с центром в точке $(-5,2)$;\\
\пункт
гиперболу  $y=1/x$, сдвинутую и пов\"ернутую так, что е\"е центр
находится в точке $(-5,2)$, а асимптоты соста\-вляют углы в $45^{\circ}$ с
%осью
$OX$;\\
\пункт
параболу $y=x^2$, сдвинутую и пов\"ернутую так, что е\"е вершина
находится в точке $(-1,-1)$, а ось ид\"ет под углом  $45^{\circ}$ к %оси
$OX$.
\кзадача


\putpict{16.2cm}{7cm}{pct_complex_cat}{Рис.~2. Кривая $Ф(x,y)=0$}
\vspace*{-7mm}

\ЛичныйКондуит{0mm}{8mm}

%\СделатьКондуит{4.5mm}{7.5mm}

% %\GenXMLW

\end{document}

\vspace*{-7.1truemm}

\раздел{$***$}

\vspace*{-1.55truemm}

\опр \label{epsilon-delta}  {\small\sc (Непрерывность по Кош\'и.)}
%Функция $f:M\to\R$ , где $M \subseteq\R$,
Пусть $M \subseteq\R$. Функция $f:M\to\R$
называется \выд{непрерывной в точке $a\in M$}, если
%Пусть функция $f$ определена на множестве $M\subseteq \R$,
%и пусть $a\in M$.
%и $a$ есть предельная точка $M$.
%Говорят, что $f$ \выд{непрерывна в точке $a$}, если
для каждого числа $\varepsilon>0$
найд\"ется такое число $\delta>0$,
что будет верно утверждение:
\выд{для каждого $x$ из $M$, %удовлетворяющего неравенству
такого что $|x-a|<\delta$,
выполнено неравенство $|f(x)-f(a)|<\varepsilon$}.
Обозначение: $f\in C(а)$.
%(Неформально говоря, непрерывная функция переводит близкие точки в близкие.)
%$$
%\forall \varepsilon > 0\quad \exists \delta > 0:
%\qquad f\left(U_{\delta}(a)\cap M\right)\subset
%U_{\varepsilon}\left(f(a)\right).
%$$
\копр

\задача Запишите без отрицаний: \лк  $f:M \to \R$
\выд{разрывна} (т.~е.~не является непрерывной) в точке $a\in M$\пк.
\кзадача

\задача Укажите множество точек непрерывности %каждой из следующих
функций\footnote{%В этой задаче, как обычно и делается в подобных
%случаях, все
Как обычно, функцию, заданную формулой, мы считаем определ\"енной
всюду, где эта формула имеет смысл.}:
\вСтрочку
\пункт   $x$;
\пункт   $\mathop{\mbox{\rm sgn}}x$;
\пункт   $x^2$;
\пункт   $\{x\}$;
\пункт   $\frac 1x$;
\пункт   $\sqrt x$.
%\спункт $\lim\limits_{n\to \infty}
%\left(\lim\limits_{m\to \infty} \cos^m (2\pi x n!)\right)$.
\кзадача

%\задача Сформулируйте (без отрицаний), что значит, что функция $f:M \to \R$
%\выд{разрывна} (то есть не является непрерывной) в точке $a\in M$?
%\кзадача

\задача Будет ли функция, непрерывная в точке $a$, ограниченной
в некоторой окрестности точки $a$?
\кзадача

\задача Пусть $f:M \to \R$ непрерывна в точке $a\in M$, прич\"ем $f(a) > 0$.
Докажите, что существует такая окрестность $U$ точки $a$, что $f$
положительна на множестве $U \cap M$.
\кзадача

\задача Какие функции окажутся \лк непрерывными\пк, если в определении
2 забыть, что %написать, что
\вСтрочку
\пункт
$\varepsilon>0$;
\пункт
$\delta>0$?
\кзадача


%\опр \label{limit}
%Пусть $a\in M$ --- предельная точка множества $M$.
%Функция $f:M \to \R$, где $M\subset \R$, называется непрерывной в
%точке $a\in M$, если предел функции $f$ в точке $a$ равен $f(a)$.
%
%Если точка $a\in M$ не является предельной точкой множества $M$,
%то функция $f$
%непрерывна в точке $a$ по определению.
%\копр

\опр \label{limit} {\small\sc (Непрерывность по Г\'ейне.)}
Пусть $M \subseteq\R$. Функция $f:M\to\R$
называется \выд{непрерывной в точке $a\in M$}, если
%Пусть функция $f$ определена на множестве $M\subseteq \R$,
%и пусть $a\in M$.
%и точка $a$ принадлежит $M$.
%и $a$ есть предельная точка $M$.
%Говорят, что $f$ \выд{непрерывна в точке $a$}, если
для каждой сходящейся к $a$ последовательности $(x_n)$,
%, все элементы которой отличны от $a$ и принадлежат $M$,
все элементы которой принадлежат $M$,
справедливо равенство $\lim\limits_{n \to \infty} f(x_n)=f(a)$.
Обозначение: $f\in C(а)$.
\копр


\задача Докажите эквивалентность определений~\ref{epsilon-delta} и~\ref{limit}.
%(Напоминание:
%$(A\Rightarrow B)\Leftrightarrow(\overline{B}\Rightarrow\overline{A})$.)
\кзадача


%\задача Рассмотрим функцию  $f(x)=x^2$ на отрезке $[-5;5]$.
%Найдите такое $\delta>0$, чтобы для любых двух точек  $x,y\in[-5;5]$,
%таких, что $|x-y|<\delta$, выполнялось бы неравенство
%$|f(x)-f(y)|<0,01$.
%\кзадача

\задача Функции $f$, $g$ непрерывны в точке $a\in \R$. Докажите: %, что:
\вСтрочку
%\сНовойСтроки
\пункт \hskip-1pt $|f|\in C(a)$;\hskip-1pt
\пункт \hskip-1pt $f\pm g\in C(a)$;% непрерывна в точке $a$;\hskip-1pt
\пункт \hskip-1pt $f\cdot g\in C(a)$; %непрерывна в точке $a$;\hskip-1pt
\пункт \hskip-1pt если %, кроме того,
$g(a) \neq 0$,
то функция $f/g$ определена в некоторой
окрестности точки $a$ и непрерывна в точке $a$.
\кзадача

\задача Докажите непрерывность функции (во всех точках е\"е области
определения):
\вСтрочку
\пункт $x^n$, где $n\in\N$;
\пункт \hskip-1pt многочлен из $\R[x]$; \hskip-1pt
\пункт \hskip-1pt $P(x)/Q(x)$, где $P,Q\in\R[x]$, $Q\ne0$;\hskip-1pt
%\пункт $1/x^n$;
%\пункт $[x]$;
\пункт \hskip-1pt $\root n \of x$,~где~$n\in\N$;\hskip-1pt
\пункт \hskip-1pt $\sin x$;\hskip-1pt
\пункт \hskip-1pt $\cos x$;\hskip-1pt
\пункт \hskip-1pt $\tg x$.
%\пункт $\ctg x$.
%\пункт Как проще всего вывести  из пункта г) непрерывность
%функции $\cos x$?
%\пункт Докажите непрерывность функций $\tg x$ и $\ctg x$.
\кзадача

\задача Придумайте функцию  $f:\R\to\R$, %определ\"енная на $\R$ и\\
\вСтрочку
\пункт всюду разрывную;
\пункт непрерывную лишь в одной точке;
\пункт разрывную в точках вида $1/n$, где $n\in\N$, и только в них;
\спункт разрывную в точках из $\Q$ и только в них.
%\спункт разрывную в рациональных точках и непрерывную в иррациональных.
\кзадача


\сзадача
Существует ли функция, которая
на каждом отрезке принимает все действительные значения?
\кзадача

%\СделатьКондуит{4.6mm}{8mm}



\end{document}

====

\задача  Нарисуйте кривые:
\вСтрочку
%\пункт $x=y$;
\пункт $x^2=y^2$;
%\пункт $y=x^2$;
\пункт $х^2y-xy^2=x-y$;
%\пункт $ax^2+by^2=1$, где $a,b$ --- такие числа, что $a>b>0$;
%\пункт $ax^2-by^2=1$, где $a,b$ --- такие числа, что $a>b>0$;
\пункт $y^2=x^3$;\quad
\пункт $y-1=x^3$;\quad
\пункт $y^2-1=x^3$;\quad
\пункт $y^2-x=x^3$;\quad
\пункт $y^2-x^2=x^3$.
\кзадача

\сзадача Нарисуйте кривые:
\вСтрочку
\пункт $x^2=x^4+y^4$;
\пункт $xy=x^6+y^6$;
\пункт $x^3=y^2+x^4+y^4$;
\пункт $x^2y+xy^2=x^4+y^4$.
\кзадача 