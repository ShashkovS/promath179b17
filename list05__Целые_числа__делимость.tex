% !TeX encoding = windows-1251
\documentclass[a4paper,12pt]{article}
\usepackage{newlistok}
%\documentstyle[11pt, russcorr, listok]{article}
\newcommand{\del}{\mathrel{\raisebox{-.3 ex}{${\vdots}$}}}

%\УвеличитьШирину{1truecm}
%\УвеличитьВысоту{3.22truecm}
%\hoffset=-2.5truecm
%\voffset=-27.3truemm

%\documentstyle[12pt, russcorr, listok]{article}

%\УвеличитьПробелы{-2mm}{-1mm}
%\ВосстановитьПробелы
%первый параметр --- это на сколько изменить дополнительный
%промежуток перед пунктом.
%Второй --- на сколько изменить полную ширину пункта (т.е.
%буквы, скобки и пробела после)

%\УвеличитьПромежутки{85}
%\ВосстановитьПромежутки
%параметр --- в столько процентов изменить междустрочные промежутки

\Заголовок{Целые числа: делимость}
\НомерЛистка{5}
\ДатаЛистка{12.2013}

%\renewcommand{\spacer}{\vspace*{0.4\smallskipamount}}
\renewcommand{\baselinestretch}{0.97}
\begin{document}

\СоздатьЗаголовок

%\задача
%Докажите, что Ваше 28-летие будет
%в такой же день недели, в какой Вы родились.
%\кзадача

\smallskip

\опр
Пусть $n$ и $k$ --- целые числа, $k\ne0$. Говорят, что $n$ \выд{делится}
на~$k$, если найдётся такое целое $m$,
что $n=k\cdot m$. Обозначение: $n\del k$.
При этом говорят ещ\"е, что $n$ \выд{кратно} $k$
или что $k$ \выд{делит}~$n$ ($k\,|\,n$).
\копр

\задача
\пункт
Докажите, что $m(m+1)(m+2)$ делится на 6 при любом целом $m$.\\
\пункт
Докажите, что произведение любых $n$ последовательных
целых чисел делится на $n!$.
\кзадача

\задача Верно ли, что
\пункт
если $n\del k$ и $k\del n$, то $n=k$;\\
\пункт
если $b\del a$ и $c\del a$, но $d\!\not\,\,\del a$, то $b+c\del a$, но
$b+d\!\not\,\,\del a$;\\
\пункт
если $b\del a$ и $c\del b$, то $c\del a$;\\
%\пункт
%если $a\del b$ и $c\del d$, то $ac\del bd$;
\пункт
если $a$ и $b$ не делятся на $c$, то $ab$ не делится на $c^2$?
\кзадача

\задача
Пусть $m,n$ --- целые, и $5m+3n\del11$. Докажите, что\\
\пункт
$6m+8n\del11$;\\
\пункт  $9m+n\del11$.
\кзадача

\задача
Докажите, что
\пункт
$\overline{aaa}$ делится на $37$;\\
\пункт
$\overline{abc}-\overline{cba}$ делится на 99 (где $a$, $b$, $c$ --- цифры).
\кзадача

\задача
\пункт Докажите, что целое число делится на 4 тогда и только тогда,
когда две его последние цифры образуют число, кратное 4.\\
\пункт Найдите и докажите признаки делимости на 2, 5, 8, 10.
\кзадача

\задача
\пункт
Из натурального числа $\overline{a_n\ldots a_1a_0}$ вычли сумму
его цифр $a_n+\ldots+a_1+a_0$. Докажите, что получилось число,
делящееся на 9.\\
\пункт
Выведите из пункта а) признаки делимости на 3 и на 9.
\кзадача

\задача
Сформулируйте и докажите признак делимости на 11.
%По данному числу найдем
%Докажите, что число делится на 11 тогда и только тогда, когда
%сумма его цифр, стоящих на четных местах,
\кзадача


%\задача
%Докажите, что число $11\dots11$, запись которого состоит из $3^n$ единиц,
%делится на $3^n$.
%\кзадача

%\задача
%Числа $a,b,c,d$ --- натуральные. Обязательно ли число
%$\displaystyle\frac{(a+b+c+d)!}{a!\ b!\ c!\ d!}$ целое?
%\кзадача


\задача
\вСтрочку
Может ли $n!$ оканчиваться ровно на 4 нуля? А ровно на 5 нулей?
\кзадача

\задача
Сколькими нулями оканчивается число $11^{100}-1$?
\кзадача

\задача
%\вСтрочку
%\пункт
Целые числа $a$ и $b$ различны. Докажите, что $a^n-b^n\del a-b$
при любом натуральном~$n$.
%Чему равно частное?
%\пункт Докажите, что $a^n+b^n\del a+b$, если $a+b\ne0$ и
%натуральное $n$ нечетно.
\кзадача

\задача
Найдите все целые $n$, при которых число $(n^3+3)/(n+3)$ целое.
\кзадача

\задача
Решите в натуральных числах уравнения:\\
\пункт
$x^2-y^2=31$;\\
\пункт
$x^2-y^2=303$.
\кзадача


\опр
Натуральное число $p>1$ называется \выд{простым}, если оно имеет ровно два
натуральных делителя: 1 и $p$, в противном случае оно
называется \выд{составным}.
\копр

\задача Докажите, что любое натуральное число, большее 1,
либо само простое, либо раскладывается в произведение нескольких
простых множителей.
\кзадача


\задача
\пункт
Даны целые числа $a_1$, \ldots, $a_n$, большие 1.
Придумайте целое число, большее 1, которое не делится ни на одно из чисел
$a_1$, \ldots, $a_n$.\\
\пункт Докажите, что простых чисел бесконечно много.\\
\пункт Докажите, что простых чисел вида $3k+2$ бесконечно много
($k$ --- натуральное).
\кзадача

\задача
\пункт
Могут ли 100 последовательных натуральных чисел
все быть составными?\\
\пункт
Найдутся ли 100 последовательных натуральных чисел,
среди которых ровно 5 простых?
\кзадача


\сзадача
Из чисел 1, 2, 3, \dots , 1000 выбрали произвольным образом 501 число.
Докажите, что среди выбранных чисел найдутся два числа, одно из которых
делится на другое.
\кзадача

\ЛичныйКондуит{0mm}{8mm}

%\СделатьКондуит{6mm}{7mm}
%\СделатьКондуитФ{0mm}{6mm}{10}

%\сзадача Числа $a_1,\dots,a_n$ целые
%Для каждой пары целых чисел $i$ и $j$, где
%$1\leq j<j\leq n$, возьмем число $(a_i-a_j)/(i-j)$
%и перемножим все такие числа. Докажите, что получится
%целое число.
%%
%%Пусть $A$ --- произведение всевозможных разностей $a_i-a_j$, где
%%$1\leq j<j\leq n$, $B$ --- произведение всевозможных разностей $i-j$, где
%%$1\leq j<j\leq n$. докажите, что $A$ делится на $B$.
%\кзадача
\end{document}

\опр
Пусть $a$ и $b$ --- целые числа, $b>0$.
\выд{Разделить} $a$ на $b$ \выд{с остатком} значит найти
такие целые числа $k$ (частное) и $r$ (остаток),
что $a = kb + r$ и $0\leq r < b$.
\копр


\задача
Числа $a$ и $b$ --- целые, $b>0$.
Отметим на числовой прямой все числа, кратные~$b$.
Они разобьют прямую на отрезки длины $b$.
Точка $a$ лежит на одном из них.
Пусть $kb$ --- левый конец этого отрезка.
Докажите, что $k$ --- частное, а
$r = a - kb$ --- остаток от деления $a$ на $b$.
%(и значит, частное и остаток определены однозначно).
\кзадача


\задача
Найдите частные и остатки от деления $2013$ на $23$, $-17$ на $4$ и
$n^2-n+1$~на~$n$.
\кзадача

\задача
\вСтрочку
\пункт
Какой цифрой оканчивается %число $14^{14}$?
%А число
$8^{18}$?
\пункт
При каких натуральных $k$ верно: $2^k-1\del7$?
\кзадача

\задача
Числа $x$ и $y$ целые, причем $x^2+y^2$ делится на 3.
Докажите, что и $x$ и $y$ делятся на 3.
\кзадача

\задача
Какие целые числа дают при делении на 3 остаток 2,
а при делении на 5 --- остаток 3?
% (и докажите, что других нет).
\кзадача

\задача
Даны 20 целых чисел, ни одно из которых не делится на 5. Докажите, что
сумма двадцатых степеней этих чисел делится на 5.
\кзадача

\задача
Докажите, что остаток от деления простого  числа на 30 есть или простое
число или 1.
\кзадача

\задача
Докажите, что из любых 52 целых чисел всегда можно выбрать два
таких числа, что\\
\вСтрочку
\пункт
их разность делится на 51;
\пункт
их сумма или разность делится на 100.
\кзадача

\задача
Докажите, что из любых $n$ целых чисел всегда можно выбрать несколько,
сумма которых делится на $n$ (или одно число, делящееся на $n$).
\кзадача

\сзадача
Существует ли делящееся на 2013 натуральное число,
состоящее из цифр 0 и 1?
%Найдётся ли натуральное число, все цифры которого только 0 и 1,
%делящееся на 2007?
\кзадача
