% !TeX encoding = windows-1251
\documentclass[a4paper, 11pt]{article}
\usepackage{newlistok}
%\documentstyle[11pt, russcorr, listok]{article}
%\newcommand{\0}[1]{\overline{#1}}
%\def\C{\mbox{$\Bbb C$}}

\УвеличитьШирину{1.35truecm}
\УвеличитьВысоту{2.6truecm}


\begin{document}

\НомерЛистка{26$\frac12$}
\Заголовок{Множества на прямой. Открытые и замкнутые множества}
\ДатаЛистка{12.2015}


\СоздатьЗаголовок

%\begin{comment}
%В этом листке изучаются множества,
%являющиеся подмножествами действительной прямой.
%\end{comment}

\опр
{\itshape Окрестностью} точки называется произвольный
содержащий её интервал {\small (всюду в этом листке под интервалами
понимаются в том числе и бесконечные:~открытые лучи и вся прямая)}.
Точка называется
{\itshape внутренней точкой} множества $M$, если
она содержится в $M$ вместе с некоторой своей окрестностью.
\копр

\задача
Найдётся ли множество, у которого
%есть хотя бы одна
%найдётся внутренняя точка?
\пункт
нет внутренних точек;
%\пункт
%все точки внутренние?\\
\пункт
ровно одна внутренняя точка?
%\пункт
%все точки внутренние?
\кзадача

\опр
Множество называется {\itshape открытым}, если каждая его точка
внутренняя.
\копр

\задача
\пункт
%Приведите пример непустого открытого множества.
Докажите, что интервал --- открытое множество.
%Может ли счётное множество быть открытым?
\пункт
Бывают ли счётные открытые множества?
\кзадача

\опр
Точка называется {\itshape предельной точкой} множества $M$,
если в любой её окрестности содержится бесконечное
количество точек из~$M$. Точка называется
{\itshape изолированной точкой} множества~$M$, если она
принадлежит~$M$ и не является для него предельной.
\копр

\задача
Найдите все предельные точки $\!\!$%множества:
\пункт
$\!\Z$;
\пункт
$\!(0,1)$;
\пункт
$\!\{\frac1n\,|\,n\in\N\}$;
\пункт
$\!\{(-1)^n+\frac1n\,|\,n\in\N\}$;
\пункт
$\!\R\setminus\Q$\,;
\пункт
$\!\!\Q$\,; $\!\!$
\пункт
$\!\!\{\frac{m}{2^n}\,|\,n,m\in\Z\}$\,; $\!\!$ %({\itshape множество
%двоично-рациональных чисел})
\пункт
$\!\!$ бесконечных десятичных дробей, в записи которых
%используется
только 0~и~1; $\!\!$
\пункт
$\!\!\{\sin n\,|\,n\in\N\}$.
\кзадача

\задача
Может ли
\пункт
$\N$\,;
\пункт
$\{\frac1n\,|\,n\in\N\}$
быть множеством предельных точек
какого-нибудь множества?
\кзадача

\задача
%Докажите, что если множество $M$ не пусто, то $\sup M$ является
%его предельной точкой.
Верно ли, что %$\sup M$ является предельной точкой множества~$M$?
точная верхняя грань ограниченного множества является его
предельной точкой?
\кзадача

%\задача
%\пункт
%Обязательно ли множество содержит свою предельную точку, если она у него есть?
%\пункт
%У всякого ли множества есть предельная точка?
%\пункт
%Может ли у множества быть ровно одна предельная точка?

\задача
Верно ли, что точка~$x$ предельная для множества~$M$ тогда и только
тогда, когда в любой окрестности~$x$ содержится \
\пункт
хотя бы одна
точка множества~$M$?
\пункт
хотя бы две точки множества~$M$?
% Если нет, то как
%это утверждение надо исправить, чтобы оно стало верным?
\кзадача

%\задача
%\пункт
\опр
$a$ называется {\itshape предельной точкой}
последовательности $(x_n)$, если
%выполняется следующее условие:
$\forall\varepsilon>0\ \forall n\ \exists m>n\ |x_m-a|<\varepsilon$.
\копр

\задача
\пункт
Верно ли, что $a$ является предельной точкой
$(x_n)$, если % тогда и только тогда, когда
$a$ является предельной точкой множества~$\{x_n\,|\,n\in\N\}$?
Верно ли обратное?
\пункт
Докажите, что ограниченная
последовательность имеет предел тогда и только тогда,
когда у неё существует и единственна предельная точка.
\кзадача

\задача
\пункт
Выкинем из множества все изолированные точки. Может ли так
оказаться, что мы ничего не выкинули? Выкинули всё?
\пункт
С полученным множеством повторим ту же самую операцию.
И так далее: из получающегося
после каждого шага множества будем выкидывать все изолированные точки.
Допустим, каждый раз
из множества действительно что-то выкидывают. Может ли это
продолжаться бесконечно долго?
%\пункт
%Тот же вопрос, но выкидываются каждый раз все предельные точки.
%\пункт
%Может ли это продолжаться бесконечно долго, но при этом так,
%чтобы каждая точка
%выкидывалась на каком-то конечном шаге?
\кзадача

\опр
Множество называется {\itshape замкнутым}, если оно содержит
все свои предельные точки.
\копр

\задача
\пункт
%Обязательно ли множество, не являющееся замкнутым, открыто?
%Обязательно ли множество, не являющееся открытым, замкнуто?
Существуют ли множества, не являющиеся ни замкнутыми, ни открытыми?\\
\пункт
Всегда ли дополнение замкнутого множества открыто?
Всегда ли дополнение открытого множества замкнуто?
({\itshape Дополнением} множества $A$ называется разность
$\R\setminus A$. Обозначения: $\overline{A}$.)
\кзадача

\задача
\пункт Докажите, что конечное пересечение
(то есть пересечение конечного числа) и произвольное объединение
(то есть объединение произвольного количества)
открытых множеств открыто.\\
\пункт Докажите, что конечное объединение
и произвольное пересечение замкнутых множеств замкнуто.
\кзадача

\задача
Найдите все множества, являющиеся одновременно открытыми и замкнутыми.
\кзадача

\задача
\пункт
Докажите, что всякое открытое множество можно представить в виде
объединения не более чем счётного числа попарно непересекающихся интервалов.
\пункт
Единственно ли такое представление?
\кзадача

\задача
\пункт
Можно ли представить интервал в виде объединения двух
непересекающихся непустых открытых множеств?
\пункт
А отрезок в виде объединения двух
непересекающихся непустых замкнутых множеств?
\пункт
Можно ли представить прямую в виде объединения попарно непересекающихся
отрезков?
\кзадача

\задача
Докажите, что
\пункт
у %произвольного
любого бесконечного ограниченного множества
есть хотя бы одна предельная точка;
\пункт
%Докажите, что
из любой ограниченной последовательности можно выделить
сходящуюся подпоследовательность.
\кзадача

\задача
({\itshape Компактность отрезка})
Отрезок покрыт произвольной системой
%(или произвольное замкнутое ограниченное множество;
%такие множества называютcя {\itshape компактами})
% \пункт
% интервалов;
% \пункт
% произвольных
открытых множеств.
Докажите, что в этой системе можно выбрать
конечную подсистему, также покрывающую отрезок.
%(Указание: предположите сначала, что каждое следующее
%открытое множество содержит предыдущее.)
% \пункт
% Решите задачу для произвольной (не обязательно счётной)
% системы открытых множеств.
\кзадача

\задача
Останутся ли верными утверждения предыдущей задачи, если заменить
отрезок на интервал?
\кзадача

%\задача
%\пункт
\опр
Непустое множество $M$ называется {\itshape компактом},
если из произвольного покрытия $M$ открытыми множествами
можно выделить конечное подпокрытие.
\копр

\задача
Докажите, что
\пункт
компакты на прямой~--- это в точности непустые
замкнутые ограниченные множества;
\пункт
%Докажите, что
у любой последовательности вложенных компактов
$K_1\supset K_2\supset K_3\supset\dots$ %имеет непустое
пересечение непусто.
\кзадача

%\задача
%Рассмотрим условие \лк существует такое $\varepsilon>0$, что для
%всякого $m$ найдётся такое $n>m$, что
%$|x_n-a|<\varepsilon$\пк. Эквивалентно ли оно одному из следующих?
%\пункт
%$a$~--- предельная точка~$(x_n)$;
%\пункт
%$(x_n)$~ограничена;
%\пункт
%$(x_n)$~имеет предельную точку.

\задача
Пусть $D\subset\R$.
Математики Банах и Мазур играют в бесконечную игру.
Они по очереди выбирают отрезки на прямой,
так чтобы каждый следующий содержался внутри предыдущего. Если в
пересечении полученной последовательности вложенных отрезков
будет точка из множества~$D$, то выиграл Банах, иначе Мазур.
Кто выиграет при правильной игре, если $D$\hspace{2mm}
\пункт
конечно;
\пункт
счётно;
%$\Q$;
%\пункт
%$\R\setminus\Q$;
\пункт
открыто;
\пункт
замкнуто?
\кзадача

%\задача
%Можно ли представить прямую в виде объединения попарно непересекающихся
%отрезков?

\ЛичныйКондуит{0mm}{5mm}
%\GenXMLW

% %\СделатьКондуит{4.5mm}{7.5mm}


\end{document} 