% !TeX encoding = windows-1251
\documentclass[a4paper,12pt]{article}
\usepackage{newlistok}
%\usepackage{tikz}
%\usetikzlibrary{calc}

%\documentstyle[11pt, russcorr, listok]{article}
\newcommand{\del}{\mathrel{\raisebox{-.3 ex}{${\vdots}$}}}

\УвеличитьШирину{.5truecm}
\УвеличитьВысоту{4.5truecm}
\hoffset=-2.45truecm
\voffset=-13truemm


\begin{document}

\Заголовок{Многочлены.}
\НомерЛистка{23}
\ДатаЛистка{09.2015}
%\Подзаголовок{}

\СоздатьЗаголовок


\раздел{Корни многочленов}


\задача  Докажите, что  если многочлен $A(x)$ с целыми коэффициентами
принимает при $x=0$ и $x=1$ неч\"етные значения, то уравнение $A(x)=0$
не имеет целых решений.
\кзадача

\задача \пункт Ненулевая несократимая дробь $p/q$ --- корень многочлена
$A(x)=a_nx^n + \dots + a_0$ с целыми коэффициентами.
Докажите, что  тогда $a_n$ делится на $q$
и $a_0$ делится на $p$.\\
\пункт Пусть в пункте а) дано $a_n=1$. Докажите, что
все рациональные корни $A$ --- целые числа.
\кзадача

\задача Найдите все рациональные корни многочленов:\\
\пункт $x^3-6x^2+15x-14$;
\пункт $6x^4+19x^3-7x^2-26x+12$.
\кзадача

\задача
Пусть $A(x)\in\Q[x]$, $A(\sqrt2)=0$. Докажите, что $A(-\sqrt2)=0.$
\кзадача

\задача
\пункт
Найдите ненулевой
многочлен $P$ с целыми коэффициентами и корнем
$\sqrt2+\sqrt3$.
\пункт Найдите все корни многочлена $P$ из пункта а).
\кзадача


\задача
Найдите многочлен минимальной степени из $\R[x]$, для которого $3-i$, 2 и $1+i$ --- корни.
\кзадача

%\newpage


\раздел{Теорема Виета}



\задача
\пункт Пусть многочлен $P(x)=x^3+ax^2+bx+c$
раскладывается на \выд{линейные} множители (то есть многочлены
первой степени):
$P(x)=(x-\alpha_1)(x-\alpha_2)(x-\alpha_3)$.
Докажите, что справедливы \выд{формулы Виета:}
$$\alpha_1+\alpha_2+\alpha_3=-a, \quad
\alpha_1\alpha_2+\alpha_2\alpha_3+\alpha_3\alpha_1=b,\quad
\alpha_1\alpha_2\alpha_3=-c.$$
\спункт Найдите подобные формулы, если $\deg P=n$ и
$P$ раскладывается на линейные множители.
\кзадача


\задача
\вСтрочку
\пункт Пусть $a+b+c>0,$ $ab+bc+ac>0,$ $abc>0.$
Докажите, что $a,b$~и~$c$ положительны.
\пункт Пусть  $a+b+c<0,$ $ab+bc+ac<0,$ $abc<0.$
Какие знаки могут иметь числа $a,b,c$?
\кзадача

\сзадача
\вСтрочку
\пункт Пусть число $c \ne 0.$  Докажите, что  многочлен $x^5+ax^2+bx+c$
не может раскладываться на пять линейных множителей.
%т.~е.~не может иметь пять вещественных корней (не обязательно различных).
\пункт Та же задача для многочлена $x^5+ax^4+bx^3+c$.
\кзадача

\задача
\пункт Коэффициенты многочлена $(x-a)(x-b)$ целые.
%Пусть $\alpha_1$ и $\alpha_2$ --- все корни квадратного тр\"ехчлена
%с целыми коэффициентами.
Докажите, что $a^n+b^n$ целое при $n\in\N$.
\спункт Найдите первые $n$ цифр после запятой в десятичной записи
числа $(\sqrt{26}+5)^n$.
\кзадача


\раздел{Дополнительные задачи}

\задача
Коэффициенты многочленов $P$ и $Q$ целые. Коэффициенты
их произведения делятся на 5. Докажите, что либо коэффициенты $P$,
либо коэффициенты $Q$ делятся на 5.
\кзадача

\задача Пусть $P$ --- многочлен степени $k$ из $\Cbb[x]$ и $n>k$.
Докажите, что среднее арифметическое значений $P$ в вершинах
правильного $n$-угольника равно значению $P$ в центре $n$-угольника.
\кзадача

\задача
На графике многочлена из $\Z[x]$ отмечены две точки с целыми координатами. Докажите, что если
расстояние между ними --- целое число, то у них одинаковые ординаты. %соединяющий их отрезок параллелен оси абсцисс.
\кзадача

\задача
Даны многочлены положительной степени $P(x)$ и $Q(x)$, причём выполнены тождества $P(P(x))=Q(Q(x))$ и
$P(P(P(x)))=Q(Q(Q(x)))$. Обязательно ли $P(x)$ и $Q(x)$ совпадают?
\кзадача

\задача
При каких $n$ многочлен степени $n$ с нечётными коэффициентами
может иметь $n$ нечётных корней?
\кзадача

\задача
Квадратный трехчлен $ax^2 + bx + c$ при всех целых $x$ принимает целые значения. Верно ли, что среди его коэффициентов
\пункт хотя бы один --- целое число;
\пункт все --- целые числа?
\кзадача

\задача Докажите, что для любого числового многочлена $P(x)$ степени $n$, принимающего при всех целых $x$ целые значения,
существуют такие целые числа $b_0, b_1, \ldots, b_n,$  что
$$
P(x) = b_n C_x^n + b_{n-1} C_x^{n-1} + ... + b_1 C_x^1 + b_0,\quad
\text{где} \quad C_x^i = \frac{x(x-1) ... (x-i+1)}{i!}.
$$
\кзадача


\vspace*{-5mm}
\задача
Многочлен $P(x)$ степени $n-1$ принимает целые значения при $n$ последовательных целых значениях $x$. Докажите, что $P(x)\in Q[x]$ и
$P(n)\in\N$ при всех $n\in\N$.
\кзадача


%\ЛичныйКондуит{.1mm}{6mm}{8}
\ЛичныйКондуит{0mm}{6mm}

%\СделатьКондуит{5mm}{7.7mm}

%\GenXMLW

\end{document}


\раздел{Деление с остатком}

\опр\label{del}  Пусть $A$ и $B$ --- многочлены, причем $\deg B>0.$
Разделить $A$ на $B$ с остатком значит найти такие многочлены
$Q$ и $R,$ что $A=BQ+R,$ где либо $R=0$, либо $\deg R<\deg B$.
\копр

\задача  Разделите с остатком $2x^4-3x^3+4x^2-5x+6$ на $x^2-3x+1$.
\кзадача

\задача
\пункт  Докажите, что  деление многочленов с остатком всегда возможно.
\пункт  Докажите, что  при делении с остатком многочлены $Q$ и $R$
определяются однозначно.
\кзадача

\задача[Теорема Безу] Докажите, что остаток от деления многочлена $A(x)$
на двучлен $x-a$ равен значению многочлена $A(x)$ при  $x=a.$
\кзадача

\задача Остаток от деления $A(x)$ на $x-1$ равен 5, а на $x-3$ равен 7.
Найдите остаток от деления $A(x)$ на $(x-1)(x-3).$
\кзадача

\опр \выд{Наибольшим общим делителем} ({НОД}) двух многочленов,
один из которых ненулевой, называют многочлен
наибольшей степени, делящий оба этих многочлена.
\копр

\задача Насколько однозначно определяется {НОД} двух многочленов?
\кзадача

\задача Найдите {НОД} многочленов: \вСтрочку
\пункт $x(x-1)^3(x+2)$ \ и \  $(x-1)^2(x+2)^2(x+5);$
\пункт $3x^3-2x^2+x+2$ \  и \  $x^2-x+1;$ \
\спункт $x^m-1$ \ и \ $x^n-1;$ \
\спункт $x^m+1$ \ и \ $x^n+1.$
\кзадача

%\задача Докажите, что {НОД} многочленов $A$ и $B$--- это многочлен
%наименьшей степени, который делится на любой общий делитель $A$ и $B$.
%\кзадача



\задача
Пусть $A$ и $B$ --- любые многочлены степени $m$ и $n$ соответственно.
\сНовойСтроки
\пункт Докажите, что
существуют такие многочлены $U$ и $V,$ что {НОД}$(A,B)=AU+BV.$
\спункт  Докажите, что если $m,\ n>0,$
то $U$ и $V$ можно выбрать так, чтобы  $\deg U<n$ и $\deg V<m.$
\пункт Найдите такие $U$ и $V$, если $A$ и $B$ --- многочлены
из пункта б) предыдущей задачи.
\кзадача

%\задача Обозначим многочлены из пункта б) предыдущей задачи как $f$ и $g$.
%Найдите такие многочлены $u$ и $v,$ что {НОД}$(f,g)=fu+gv$,
%прич\"ем $\deg u<\deg g$ и $\deg v< \deg f$.
%\кзадача


\раздел{Дополнительные задачи}

\задача Пусть многочлен $A(x)$ таков, что $A(x)=A(-x)$ при любом $x$.
Докажите, что  существует такой многочлен $P(x),$ что
$A(x)=P(x^2)$ при любом $x$.
\кзадача

\задача
Пусть $p(x)$ --- непостоянный многочлен с целыми коэффициентами.
\сНовойСтроки
\пункт Докажите, что при любом целом числе $n$ либо
$p(n)$ делит $p(n+p(n))$, либо $p(n)=p(n+p(n))=0$.
\пункт Могут ли все числа $p(0)$, $p(1)$, $p(2)$, \dots\  быть простыми?
\кзадача

%\задача Пусть $s_1$, \dots, $s_k$  --- корни многочлена
%$a_nx^n+\dots+a_1x+a_0$.
%Найдите корни многочленов
%\вСтрочку \пункт $(-1)^na_nx^n+...+a_2x^2-a_1x+a_0;$
%\пункт $a_0x^n+a_1x^{n-1}+...+a_{n-1}x+a_n$.
%\кзадача

\задача Коэффициенты произведения двух многочленов с целыми
коэффициентами делятся на~5.  Докажите, что коэффициенты
одного из этих многочленов делятся на~5.
\кзадача

%\задача Пусть $p(x)$ --- многочлен с целыми коэффициентами.
%\сНовойСтроки
%\пункт Докажите, что $a-b$ делит $p(a)-p(b)$  при любых различных
%целых числах $a$ и $b$.
%\спункт Пусть уравнения $p(x)=1$ и $p(x)=3$ имеют целое решение.
%Может ли уравнение $p(x)=2$ иметь два различных целых решения?
%\кзадача

\задача Используя равенство $(1+x)^p(1+x)^q=(1+x)^{p+q}$,
вычислите двумя способами коэффициент при $x^m$ в многочлене $(1+x)^{p+q}$
и решите задачу 34 листка 3.
\кзадача

\задача
Коэффициенты квадратного уравнения $x^2+px+q=0$ изменили не больше, чем
на 0,001. Может ли больший корень уравнения измениться больше,
чем на 1000?
\кзадача

\сзадача
У многочлена $P(x)$ есть  отрицательный коэффициент. Могут ли у всех
его степеней $P^n(x)$ (где $n>1$ --- целое) все коэффициенты быть
положительными?
\кзадача

%\GenXMLW

%--------------------------------------------------------------------------

\end{document}

\задача
\пункт Квадратный тр\"ехчлен $ax^2 + bx + c$ принимает при каждом целом
$x$ целое значение. Верно ли, что среди его коэффициентов
хотя бы один --- целое число?
\пункт Верно ли, что все его коэффициенты --- целые числа?
\кзадача


\задача
Правильные треугольники со сторонами 1, 3, 5, \dots\
расположены в ряд  так, что их основания лежат
на одной прямой вплотную друг к другу.
Докажите, что вершины треугольников, противоположные основаниям,
лежат на некоторой параболе.
\кзадача



\задача \пункт Для каждого $x\in\{-3,-2,-1,-1/2,\ 0,\ 1/2,\ 1,\ 2,\ 3\}$
нарисуйте на плоскости $pOq$ графики прямых, задающихся уравнениями
$x^2+px+q=0.$
\пункт
Напишите уравнение, задающее множество таких точек $(p,q),$ что
квадратный \break
тр\"ехчлен $x^2+px+q$ имеет кратный корень, и изобразите его на
плоскости.
\пункт Докажите, что все прямые из п.~а)
касаются\footnote{В этой задаче будем считать,
что прямая $l$ \выд{касается} некоторой кривой, если эта кривая лежит по одну
сторону от прямой $l$ и имеет с $l$ ровно одну общую точку.}
некоторой кривой. Что это за кривая?
\пункт Укажите на плоскости множества таких точек $(p,q),$ что квадратный
тр\"ехчлен\break $x^2+px+q$ имеет два различных корня, не имеет корней.
\спункт Укажите на плоскости множества таких точек $(p,q),$ что
тр\"ехчлен $x^2+px+q$ имеет на отрезке $[-1;1]$ два различных корня,
кратный корень, не имеет корней.
\кзадача


\end{document} 