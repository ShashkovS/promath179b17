% !TeX encoding = windows-1251
\documentclass[a4paper,12pt]{article}
\usepackage{newlistok}

\ВключитьКолонитул
\УвеличитьВысоту{5mm}
\УвеличитьШирину{2mm}

% \renewcommand{\spacer}{\vspace{.5mm}}

\Заголовок{Гомеоморфизмы, связность и линейная-связность}
\НомерЛистка{MS-6}
\ДатаЛистка{01.2017}

\begin{document}

\СоздатьЗаголовок

{\footnotesize
При изучении топологических пространств часто оказывается, что какие-то их них в некотором смысле одинаковые, то есть задавая интересующие нас вопросы (Является ли оно компактным? Является ли оно связным? Какие на нём бывают непрерывные функции? ...), мы получаем одинаковые ответы. Странно было бы отдельно доказывать компактность дуги окружности, когда уже известна компактность отрезка. Один из способов определить такую "одинаковость" --- это отношение \emph{гомеоморфности}.

}

\опр
    \emph{Гомеоморфизмом} называется взаимно-однозначное отображение топологических пространств $f \colon X \to Y$, такое что $f$ и $f^{-1}$ непрерывны. Если такое отображение существует, то пространства $X$ и $Y$ называются \emph{гомеоморфными}. Обозначение $X \cong Y$.
\копр

\задача
    Докажите, что гомеоморфность является отношением эквивалентности, то есть выполняются свойства:\\
    ({\it i\/}) [Рефлексивность] $X \cong Х$. \\
    ({\it ii\/}) [Симметричность] Если $X \cong Y$, то $Y \cong X$.\\
    ({\it iii\/}) [Транзитивность] Если $X \cong Y$ и $Y \cong Z$, то $X \cong Z$.
\кзадача

\задача
    Пусть $f \colon X \to Y$ --- взаимно-однозначное непрерывное отображение. Верно ли, что $f$ --- гомеоморфизм?
\кзадача

\задача
   Привести пример двух негомеоморфных топологических пространств $X$ и $Y$, таких что существуют взаимно-однозначные непрерывные отображения $f\colon X \to Y$ и $g\colon Y \to Х$.
\кзадача

\опр
    Топологическое пространство называется \emph{хаусдорфовым}, если для любых двух его точек найдутся непересекающиеся окрестности.
\копр

\задача
    \пункт
    Приведите пример нехаусдорфова топологического пространства.
    \пункт
    Доказать, что любое метрическое пространство хаусдорфово.
\кзадача

\задача
    Пусть $X$ --- хаусдорфово компактное топологическое пространство. Докажите, что любой компакт в $X$ замкнут.
\кзадача

\задача
    \пункт Пусть $X, Y$ --- хаусдорфовы компактные топологические пространства. Пусть также $f\colon X \to Y$ непрерывное взаимно-однозначное отображение. Докажите, что $f$ --- гомеоморфизм.
    \пункт Верно ли это утверждение, если не требовать хаусдорфовости пространств?
\кзадача

\опр
    \emph{Путём} в пространстве $X$ называется образ любого непрерывного отображения $f\colon [0,1] \to X$. Точки $f(0)$ и $f(1)$ называются соответственно \emph{началом} и \emph{концом} данного пути.
\копр

\задача
    Докажите, что образ пути при гомеоморфизме также является путём.
\кзадача

\задача
    В этой задаче считаем рассматриваемые пространства метрическими с метрикой $d_2$. Гомеоморфны ли:
    \пункт отрезок и прямая?
    \пункт квадрат и круг?
    \пункт точка и отрезок?
    \пункт интервал и прямая?
    \пункт отрезок и окружность?
    \пункт эллипс и окружность?
    \пункт множество натуральных чисел и множество целых чисел?
    \пункт множество целых чисел и множество рациональных чисел?
\кзадача

\задача
    Разбейте все буквы русского алфавита на классы гомеоморфности.
\кзадача

\vfill
\ЛичныйКондуит{0mm}{6mm}
\ОбнулитьКондуит
\newpage

\опр
    Топологическое пространство называется \emph{связным}, если его нельзя представить как объединение двух непустых непересекающихся открытых подмножеств. Топологическое пространство называется \emph{линейно-связным}, если для любых двух его точек найдётся путь, начинающийся в одной из них и заканчивающийся в другой.
\копр

\задача
    Являются ли связными или линейно-связными следующие метрические пространства с естественной топологией?
    \пункт Пустое множество.
    \пункт Отрезок.
    \пункт $n$-мерный шар.
    \пункт $n$-мерная сфера (множество точек $\R^{n+1}$, задаваемых уравнением $x_1^2 + \dots + x_{n+1}^2 = 1$).
    \пункт Множество рациональных чисел.
    \пункт Множество целых чисел.
    \пункт Подмножество $\R^4$, задаваемое неравенством $x_1x_2 \neq x_3x_4$.
    \пункт Подмножество $\R^4$, задаваемое неравенством $x_1x_2 > x_3x_4$.
\кзадача

\задача
    \пункт
    Правда ли, что любое связное топологическое пространство является линейно-связным?
    \пункт
    Правда ли, что любое линейно-связное топологическое пространство является связным?
    \пункт
    Те же вопросы для метрических пространств.
\кзадача

\задача
    Докажите, что если топологическое пространство обладает свойством связности или линейной связности, то таким же свойством обладает его образ при непрерывном отображении.
\кзадача

\сзадача
    Докажите, что если из $\R^n$ ($n \ge 2$) выбросить конечное или счётное число точек, то оставшееся множество будет связным.
\кзадача

\задача
    \пункт
    Опишите все линейно-связные множества на прямой.
    \пункт
    Опишите все связные множества на прямой.
    \пункт
    Гомеоморфны ли прямая и плоскость?
\кзадача

\задача
    Пусть $X$ --- топологические пространство обладающее следующим свойством: у любой его точки найдётся окрестность, гомеоморфная открытому шару в $\R^n$ (для некоторого $n$). Докажите, что свойства связности и линейной связности для $X$ равносильны.
\кзадача

\задача
    Приведите пример открытого множества $X \subset \R^2$ и точки $x$ его границы, таких что не существует непрерывного отображения $f \colon [0,1] \to \R^2$ с условиями $f(a) \in X$, при $a < 1$ и $f(1) = x$.
\кзадача

\vfill
\ЛичныйКондуит{0mm}{5mm}
%\СделатьКондуитИз{6.2mm}{6.2mm}{sp_TOP.tex}
\end{document} 