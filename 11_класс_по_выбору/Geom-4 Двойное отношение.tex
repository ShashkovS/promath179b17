% !TeX encoding = windows-1251
\documentclass[a4paper,12pt]{article}
\usepackage[mag=980]{newlistok}
\graphicspath{{pdfpict/}}


\УвеличитьВысоту{2.5cm}
\УвеличитьШирину{1.8cm}


\Заголовок{Двойное отношение}
\НомерЛистка{GM-4}
\ДатаЛистка{01.2017}

\expandafter\def\expandafter\normalsize\expandafter{%
    \normalsize
    \setlength\abovedisplayskip{2pt}
    \setlength\belowdisplayskip{2pt}
    \setlength\abovedisplayshortskip{2pt}
    \setlength\belowdisplayshortskip{1pt}
}
\sloppy

\usepackage{ifthen}
\usepackage[noadjust]{marginnote}
\newcommand{\rightpicture}[4]%
{\ifthenelse{\lengthtest{10mm>#3mm}}%
{\marginnote{\hbox to #1 {\hss\includegraphics[scale=#3]{#4}}}[-#2]}%
{\marginnote{\hbox to #1 {\hss\includegraphics[width=#3]{#4}}}[-#2]}}
\newcommand{\leftpicture}[4]%
{\ifthenelse{\lengthtest{10mm>#3mm}}%
{\reversemarginpar\marginnote{\hbox to -#1 {\includegraphics[scale=#3]{#4}\hss}}[-#2]\normalmarginpar}%
{\reversemarginpar\marginnote{\hbox to -#1 {\includegraphics[width=#3]{#4}\hss}}[-#2]\normalmarginpar}}


\begin{document}
\СоздатьЗаголовок


\опр Пусть $A,B,C,D$~--- точки на одной прямой $l$, причём
$\{A,B\}\cap\{C,D\} = \emptyset$. Их \выд{двойным отношением}
называется
  $$
  [A,B,C,D] = \frac{CA}{CB}:\frac{DA}{DB},
  $$
где отношение отрезков берется со знаком. Если одна из точек
является бесконечно удалённой, то двойное отношение равно
\выд{простому} отношению остальных трёх точек, то есть, например,
$[A,B,C,\infty] = CA : CB$, и т.~п.

Пусть $a,b,c,d$~--- прямые в плоскости $\pi$, проходящие через
(конечную) точку $O$, причём $\{a,b\}\cap\{c,d\} = \emptyset$. Их
\выд{двойным отношением} называется
  $$
  [a,b,c,d] = \frac{\sin\angle ca}{\sin\angle cb} : \frac{\sin\angle
  da}{\sin\angle db},
  $$
где на каждой прямой выбрано направление и угол между прямыми~---
это угол, отсчитанный против часовой стрелки от выбранного
направления на первой прямой до выбранного направления на второй
прямой. \копр

\задача Пусть $a,b,c,d$~--- прямые в плоскости $\pi$, проходящие
через точку $O$, причём\break $\{a,b\}\cap\{c,d\} = \emptyset$,
прямая $l\subset\pi$ не проходит через $O$, а точки $A,B,C,D$ суть
точки пересечения прямой $l$ с прямыми $a,b,c,d$ соответственно.
Докажите, что $[a,b,c,d] = [A,B,C,D]$. \кзадача

\задача Докажите, что двойные отношения сохраняются при проективных
преобразованиях. \кзадача

\задача Как меняется двойное отношение $[A,B,C,D]$ при перестановках
точек $A,B,C,D$? Сколько различных значений двойного отношения так
получается (для точек в общем положении)? \кзадача

\УстановитьГраницы{0cm}{75mm}
\rightpicture{0mm}{0mm}{75mm}{proj_geom-5}
\задача Пусть на проективной
плоскости $\bar\pi$ выбраны прямые $k,l,m,n$, никакие три из которых
не конкурентны. Пусть $A = k\cap l$, $B = m\cap n$, $E = k\cap m$,
$F = k\cap n$, $G = l\cap m$, $H = l\cap n$, $C = AB\cap EH$, $D =
AB\cap FG$. Докажите, что $[A,B,C,D] = -1$ (точки $C$ и $D$
\выд{гармонически сопряжены} относительно отрезка $AB$). \кзадача

\задача[Проективная теорема Чевы]\label{Ceva} Пусть $A,B,C$~--- три
точки, не лежащие на одной прямой, $D$ и $E$~--- любые точки, не
принадлежащие прямым $AB$, $AC$ и $BC$. Докажите, что
  $$
  [AD,AE,AB,AC] \cdot [BD,BE,BC,BA] \cdot [CD,CE,CA,CB] = 1.
  $$
\кзадача
\ВосстановитьГраницы

\задача[Теорема Чевы] На сторонах $AB$, $AC$ и $BC$ треугольника
$ABC$ (или на продолжениях этих сторон) выбрали соответственно точки
$C'$, $B'$ и $A'$. Выведите из задачи \ref{Ceva}, что прямые $AA'$,
$BB'$ и $CC'$ конкурентны тогда и только тогда, когда
  $$
  \frac{BA'}{CA'} \cdot \frac{AC'}{BC'} \cdot \frac{CB'}{AB'} = -1.
  $$
(Отношения отрезков берутся со знаком!) \кзадача

\задача[Теорема Чевы в углах] На сторонах $AB$, $AC$ и $BC$
треугольника $ABC$ (или на продолжениях этих сторон) выбрали
соответственно точки $C'$, $B'$ и $A'$. Выведите из задачи
\ref{Ceva}, что прямые $AA'$, $BB'$ и $CC'$ конкурентны тогда и
только тогда, когда
  $$
  \frac{\sin\angle BAA'}{\sin\angle CAA'}\cdot \frac{\sin\angle
  ACC'}{\sin\angle BCC'}\cdot \frac{\sin\angle CBB'}{\sin\angle
  ABB'} = -1.
  $$
(Углы берутся со знаком!) \кзадача

\задача Пусть в окружность вписан шестиугольник $ABCDEF$, причём
  $$
  \frac{AB\cdot CD\cdot EF}{BC\cdot DE\cdot FA} = 1.
  $$
Докажите, что диагонали $AD$, $BE$ и $CF$ конкурентны. \кзадача



\ЛичныйКондуит{0mm}{6mm}

\end{document}
