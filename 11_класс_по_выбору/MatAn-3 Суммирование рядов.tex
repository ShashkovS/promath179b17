% !TeX encoding = windows-1251
\documentclass[a4paper,12pt]{article}
\usepackage{newlistok}

% \renewcommand{\spacer}{\vfil}
\УвеличитьВысоту{1.2cm}
\УвеличитьШирину{1cm}


\sloppy

\Заголовок{Суммирование рядов 1}
\НомерЛистка{MA-3}
\ДатаЛистка{01.2017}

\begin{document}

\СоздатьЗаголовок

\noindent

\опр
Пусть $(a_n)$~--- числовая последовательность.
Формальное выражение
$a_1+a_2+a_3+\ldots=\sum\limits_{n=1}^\infty a_n$
называется {\em рядом}.
Число $s_n=a_1+a_2+\dots+a_n$ называется {\em  $n$-ой частичной
суммой} ряда.\\
Говорят, что ряд $\sum\limits_{n=1}^\infty a_n$
{\em сходится и имеет сумму $A$}, если
существует $\lim\limits_{n\to\infty}s_n=A$.
Тогда пишут $\sum\limits_{n=1}^\infty a_n=A$.
Если предел  $\lim\limits_{n\to\infty}s_n$
не существует, то говорят, что ряд $\sum\limits_{n=1}^\infty a_n$
{\em расходится}.
\копр

\задача
%Пусть $\forall n\ a_n\geqslant0$.
Пусть %$(a_n)$ --- последовательность,
$a_n\geq0$ при $n\in\N$.
%из неотрицательных чисел.
Докажите, что ряд $\sum\limits_{n=1}^\infty a_n$ сходится
тогда и только тогда, когда
%если и только если
ограничено множество его частичных
сумм $\{s_n\ |\ n\in\N\}$, причём в этом случае
$\sum\limits_{n=1}^\infty a_n=\sup\{s_n\ |\ n\in\N\}$.
%\limits_{n\in\N}
\кзадача

\задача
Какие из следующих рядов сходятся? Найдите их суммы.\\
\вСтрочку
%Сходятся ли следующие ряды? Для каждого сходящегося ряда
%найдите его сумму.
\пункт
$\sum\limits_{n=1}^\infty (-1)^n$;
\пункт
$\sum\limits_{n=1}^\infty \frac1{2^n}$;
\пункт
({\em геометрическая прогрессия})
$\sum\limits_{n=1}^\infty \frac1{q^n}$, $q\in\R,\ q\ne0$;\\
\пункт
({\em гармонический ряд})
$\sum\limits_{n=1}^\infty \frac1n$;
\пункт
$\sum\limits_{n=1}^\infty \frac{n}{2^n}$;
\спункт
$\sum\limits_{n=1}^\infty \frac{n^2}{2^n}$;
\пункт
$\sum\limits_{n=1}^\infty \frac1{n(n+1)}$.
\кзадача

\задача
\пункт
Докажите, что если ряд $\sum\limits_{n=1}^\infty a_n$ сходится,
то $\lim\limits_{n\to\infty}a_n=0$. Верно ли обратное?
\пункт
({\em Критерий Коши сходимости ряда.})
Докажите, что ряд $\sum\limits_{n=1}^\infty a_n$ сходится тогда и только
тогда, когда для любого $\varepsilon>0$ существует такое $N$, что
из $n\geqslant m>N$
(где $n,m\in\N$) следует $|a_m+a_{m+1}+\dots+a_n|<\varepsilon$.
\кзадача

%\задача
%Cходится ли ряд $\sum\limits_{n=1}^\infty a_n$, если
%для каждого $p=1, 2, 3,\dots$ выполнено $\lim\limits_{n\to\infty}
%(a_{n+1}+a_{n+2}\hm+\dots+a_{n+p})=0$?



\задача
%\пункт
%Пусть ряды
%$\sum\limits_{n=1}^\infty a_n$ и $\sum\limits_{n=1}^\infty b_n$
%сходятся. Докажите, что тогда ряд $\sum\limits_{n=1}^\infty (\alpha
%a_n+\beta b_n)$ сходится, причём выполнено равенство
%$\sum\limits_{n=1}^\infty (\alpha a_n+\beta b_n)=
%\alpha\sum\limits_{n=1}^\infty a_n+\beta\sum\limits_{n=1}^\infty b_n$.
%\пункт
%Пусть ряд $\sum\limits_{n=1}^\infty a_n$ сходится,
%а ряд $\sum\limits_{n=1}^\infty b_n$ расходится.
%Докажите, что тогда ряд $\sum\limits_{n=1}^\infty (a_n+b_n)$
%расходится.
%\пункт
Верно ли, что если ряды
$\sum\limits_{n=1}^\infty a_n$ и $\sum\limits_{n=1}^\infty b_n$
сходятся, то сходится ряд $\sum\limits_{n=1}^\infty a_nb_n$?
\кзадача

%\сзадача
%Пусть $(a_n)$, $(b_n)$~--- монотонно стремящиеся к нулю последовательности
%положительных чисел, причём $\sum a_n$ и $\sum b_n$ расходятся.
%Всегда ли ряд $\sum \min(a_n, b_n)$ расходится?

\задача
Сходятся ли следующие
ряды:
\вСтрочку
\пункт
$\sum\limits_{n=1}^\infty \frac{(-1)^n}{n}$;
\пункт
$\sum\limits_{n=1}^\infty \frac1{\sqrt{n}}$;
\пункт
$\sum\limits_{n=1}^\infty \frac1{n^2}$;
%\пункт
%$\sum\limits_{n=1}^\infty \frac1{n^n}$;
\кзадача




\задача
Докажите:
\вСтрочку
\пункт
ряд $\sum\limits_{n=1}^\infty \frac1{n!}$ сходится;
\пункт
$\sum\limits_{n=1}^\infty \frac1{n!}=e$;
\пункт
$e-\sum\limits_{n=1}^m \frac1{n!}<\frac1{m!\,m}$;
\пункт
число $e$ иррационально.
\кзадача

\задача
Пусть $a_n\geqslant0$ при всех $n\in\N$ и $\sigma\colon\N\to\N$~---
взаимно однозначное отображение (перестановка натурального
ряда). Тогда $\sum\limits_{n=1}^{\infty}a_n=
\sum\limits_{n=1}^{\infty}a_{\sigma(n)}$ (то есть
если сходится ряд в левой части равенства, то сходится и ряд в правой
части, прич\"ем их суммы равны; если ряд в левой части расходится,
то и ряд в правой части расходится).
\кзадача


\сзадача
Пусть $p_n$ --- $n$-е простое число, $n\in\N$.
\\
\пункт
Докажите, что
$\lim\limits_{n\rightarrow\infty}
\left(\frac1{1-1/p_1^2}\cdot\ldots\cdot\frac1{1-1/p_n^2}\right)=
\sum\limits_{n=1}^\infty \frac1{n^2}$.\\
\пункт Существует ли предел
$\lim\limits_{n\rightarrow\infty}
\left(\frac1{1-1/p_1}\cdot\ldots\cdot\frac1{1-1/p_n}\right)?$
\пункт Сходится ли ряд
$\sum\limits_{n=1}^\infty \frac1{p_n}$?
\кзадача


\сзадача
\вСтрочку
\пункт
Пусть $\gamma_k$ --- сумма ряда
$\sum\limits_{n=2}^{\infty}\frac1{n^k}$.
Найдите сумму $\sum\limits_{k=2}^{\infty}\gamma_k$.\\
\пункт [Эйлер.]
Пусть $A$ --- множество всех целых
чисел, представимых в виде $n^k$, где $n,k$ --- %любые
целые числа, большие 1.
Найдите сумму~\hbox{$\sum\limits_{a\in A}\frac1{a-1}$.}
\кзадача

%\СделатьКондуитИз{6.2mm}{6.2mm}{sp_An.tex}
\ЛичныйКондуит{0mm}{6mm}
% \GenXMLW

\end{document}



\задача
\пункт
({\em Признак сравнения Вейерштрасса.})
Пусть $\sum\limits_{n=1}^\infty a_n$,
$\sum\limits_{n=1}^\infty b_n$ --- ряды с неотрицательными членами.
Пусть найд\"ется такой номер $k$, что при всех $n>k$, $n\in\N$
будет выполнено неравенство
%$\forall n\ b_n\geqslant a_n\geqslant0$
$b_n\geqslant a_n$.
%(говорят, что
%{\em ряд $\sum\limits_{n=1}^\infty b_n$  мажорирует ряд}
%$\sum\limits_{n=1}^\infty a_n$).
Тогда если $\sum\limits_{n=1}^\infty b_n$ сходится, то
$\sum\limits_{n=1}^\infty a_n$ сходится;
если $\sum\limits_{n=1}^\infty a_n$ расходится, то
$\sum\limits_{n=1}^\infty b_n$ расходится.
%Верно ли это утверждение без предположения неотрицательности
%членов рядов?
\пункт
({\em Признак д'Аламбера.})
Пусть члены ряда $\sum\limits_{n=1}^\infty a_n$
положительны, и
существует %предел
$\lim\limits_{n\to\infty}\frac{a_{n+1}}{a_n}=q$.
Если $q<1$, то
ряд сходится, а если %$\lim\limits_{n\to\infty}\frac{a_{n+1}}{a_n}>1$,
$q>1$, то ряд расходится. Что можно сказать о сходимости
ряда, если $q=1$?
%в случае $\lim\limits_{n\to\infty}\frac{a_{n+1}}{a_n}=1$?
\пункт
({\em Признак Коши.})
Пусть члены ряда $\sum\limits_{n=1}^\infty a_n$ неотрицательны,
и существует %предел
%Если существует предел
$\lim\limits_{n\to\infty}\sqrt[n]{a_n}=q$.
Если $q<1$, то
ряд сходится, а если
%$\lim\limits_{n\to\infty}\sqrt[n]{a_n}>1$,
$q>1$, то ряд расходится. Что можно сказать о сходимости ряда,
%в случае $\lim\limits_{n\to\infty}\sqrt[n]{a_n}=1$?
если $q=1$?
\пункт
Приведите пример сходящегося ряда
с положительными членами, к которому применим признак Коши,
но не применим признак д'Аламбера. Бывает ли наоборот?
\кзадача

\задача
Исследуйте ряды на сходимость:\\
\вСтрочку
\пункт
$\sum\limits_{n=1}^\infty \frac1{n^p}$;
\пункт
$\sum\limits_{n=2}^\infty \frac1{n\ln n}$;
\пункт
$\sum\limits_{n=1}^\infty \frac{1\cdot3\cdot5\cdot\ldots\cdot(2n-1)}
{2\cdot4\cdot6\cdot\ldots\cdot2n}$;
\пункт
$\sum\limits_{n=1}^\infty \frac{n^k}{a^n}$;
\пункт
$\sum\limits_{n=1}^\infty \frac{a^n}{n!}$;
\пункт
$\sum\limits_{n=1}^\infty \frac1{\binom{2n}{n}}$;
%\пункт
%$\sum\limits_{n=1}^\infty \frac{n^{n+\frac1n}}{(n+\frac1n)^n}$;
\пункт
$\sum\limits_{n=1}^\infty (1-\cos\frac{x}n)$.
\кзадача


\задача
\пункт
({\em Теорема Лейбница.})
Пусть $a_n>0$ при всех $n\in\N$, и кроме того, $a_1\geqslant a_2\geqslant
a_3\geqslant\dots$, $\lim\limits_{n\to\infty}a_n=0$.
Тогда знакочередующийся ряд $a_1-a_2+a_3-a_4+a_5-\dots$ сходится.
\пункт
Верно ли утверждение теоремы без условия монотонности $(a_n)$?
\кзадача

\опр
Ряд $\sum\limits_{n=1}^\infty a_n$ называется
{\em абсолютно сходящимся}, если сходится ряд
$\sum\limits_{n=1}^\infty|a_n|$.
\копр

\задача
Докажите, что абсолютно сходящийся ряд сходится.
\кзадача

\задача
Пусть ряд $\sum\limits_{n=1}^\infty a_n$ абсолютно
сходится. Тогда абсолютно сходится произвольный ряд
$\sum\limits_{n=1}^\infty b_n$, полученный из него
перестановкой слагаемых, причём
$\sum\limits_{n=1}^\infty b_n=\sum\limits_{n=1}^\infty a_n$.
\кзадача

\опр
Ряд $\sum\limits_{n=1}^\infty a_n$ называется
{\em условно сходящимся}, если он сходится,
но ряд $\sum\limits_{n=1}^\infty|a_n|$ расходится.
\копр

\задача
Пусть ряд $\sum\limits_{n=1}^\infty a_n$ сходится условно.
\сНовойСтроки
\пункт
Докажите, что ряд, составленный из его положительных
(или отрицательных) членов, расходится.
\пункт
({\em Теорема Римана.})
Докажите, что ряд $\sum\limits_{n=1}^\infty a_n$ можно превратить
перестановкой слагаемых как в расходящийся ряд, так и в сходящийся
с произвольной наперёд заданной суммой.
\пункт
Докажите, что можно так сгруппировать члены ряда
$\sum\limits_{n=1}^\infty a_n$ (не переставляя их),
что ряд станет абсолютно сходящимся.
\спункт
Пусть $\sum\limits_{n=1}^\infty a_n$~--- ряд, составленный
из комплексных чисел, $S$ --- множество всех перестановок $\sigma$
натурального ряда, для которых ряд $\sum\limits_{n=1}^\infty a_{\sigma(n)}$
сходится. Каким может быть множество
$\{\sum\limits_{n=1}^\infty a_{\sigma(n)}\ |\ \sigma\in S\}$?
\кзадача

\задача
Исследуйте ряды на абсолютную и условную сходимость:\\
\вСтрочку
\пункт
$\sum\limits_{n=1}^\infty \frac{(-1)^{n+1}}{n^p}$;
\пункт
$\sum\limits_{n=1}^\infty \frac{(-1)^{n+1}}{n^{p+\frac1n}}$;
%\пункт
%$\sum\limits_{n=1}^\infty \frac{(-1)^n}{\sqrt[n]n}$;
%\пункт
%$\sum\limits_{n=1}^\infty \frac{(-1)^n}{x+n}$;
\пункт
$\sum\limits_{n=1}^\infty \frac{(-1)^{[\sqrt{n}]}}{n}$;
\пункт
$\sum\limits_{n=1}^\infty \sin n^2$;
\спункт
$\sum\limits_{n=1}^\infty \frac{\sin n}{n}$.
\кзадача

\задача
Пусть $s$ --- сумма ряда
%известно, что
$\sum\limits_{n=1}^\infty \frac{(-1)^{n+1}}{n}$. %=\ln 2$.
Найдите суммы\\
\вСтрочку
\пункт
$1+\frac13-\frac12+\frac15+\frac17-\frac14+\frac19+\frac1{11}-\frac16+\ldots$\,;
\пункт
$1-\frac12-\frac14+\frac13-\frac16-\frac18+\frac15-\frac1{10}-\frac1{12}+\ldots$\,.\\
\пункт
Переставьте члены ряда $\sum\limits_{n=1}^\infty \frac{(-1)^{n+1}}{n}$
так, чтобы он стал расходящимся.
%\спункт
%Докажите равенство
%$\sum\limits_{n=1}^\infty \frac{(-1)^{n+1}}{n}=\ln 2$.
\кзадача


%\сзадача
%Докажите:
%\пункт
%$1+\frac12+\frac13+\frac14+\dots+\frac1n=C+\ln n+\varepsilon_n$,
%где $C$~--- константа ({\em постоянная Эйлера}),
%$\lim\limits_{n\to\infty}\varepsilon_n=0$;
%\пункт
%({\em тождество Каталана})
%$1-\frac12+\frac13-\frac14+\dots+\frac1{2n}=
%\frac1{n+1}+\frac1{n+2}+\dots+\frac1{2n}$;
%\пункт
%$\sum\limits_{n=1}^\infty \frac{(-1)^{n+1}}{n}=\ln 2$.

\задача
Существует ли такая последовательность $(a_n)$, $a_n\ne0$ при $n\in\N$,
что ряды $\sum\limits_{n=1}^\infty a_n$ и
$\sum\limits_{n=1}^\infty \frac1{n^2a_n}$ сходятся?
Можно ли выбрать такую последовательность из
положительных чисел?
\кзадача


\сзадача
Существует ли такая последовательность $(a_n)$, что
ряд $\sum\limits_{n=1}^\infty a_n$ сходится, а ряд
$\sum\limits_{n=1}^\infty a_n^3$ расходится?
\кзадача



%\сзадача
%Найдётся ли такая перестановка $\sigma$ натурального ряда, что
%для неё существует сходящийся ряд, который она переставляет
%в расходящийся, и существует расходящийся ряд, который она
%переставляет в сходящийся?

\сзадача
Пусть функция $f\colon\R\to\R$ такова, что для любого сходящегося
ряда $\sum a_n$ ряд $\sum f(a_n)$ сходится. Докажите, что тогда найд\"ется
такое число $C\in\R$, что $f(x)=Cx$ в некоторой окрестности нуля.
\кзадача

\end{document} 