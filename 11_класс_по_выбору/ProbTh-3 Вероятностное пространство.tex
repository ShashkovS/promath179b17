% !TeX encoding = windows-1251
\documentclass[a4paper,12pt]{article}
\usepackage[mag=930]{newlistok}

\УвеличитьШирину{1.5truecm}
\УвеличитьВысоту{2.4truecm}

\Заголовок{Вероятностные пространства}
\НомерЛистка{PT-3}
\ДатаЛистка{01.2017}
%\renewcommand{\spacer}{\vfill}
\ВключитьКолонитул
\sloppy

\begin{document}
\СоздатьЗаголовок
\setlength{\abovedisplayskip}{3pt}
\setlength{\belowdisplayskip}{-10pt}
% \setlength{\abovedisplayshortskip}{0pt}
% \setlength{\belowdisplayshortskip}{0pt}

{\small
Пусть мы проводим серию испытаний или экспериментов, в результате
которых могут наблюдаться различные \выд{исходы (элементарные события)}, зависящие от случая.
Пример --- бросание игральной кости, здесь элементарное событие --- выпадение одного из чисел $1, 2,\ldots,6$.
Некоторые совокупности элементарных событий называются \выд{событиями}.
Пример события --- выпадение четного числа очков на игральной кости.
Если некоторое событие при многократном повторении испытания происходит примерно с частотой $p$, говорят, что вероятность данного события равна $p$.

Если множество возможных исходов конечно, и все исходы равновероятны,
то подсчёт вероятности событий сводится к подсчёту числа исходов, содержащихся в событии.

Если же не все исходы равновероятны, то наша интуиция может нас подвести,
поэтому требуется некоторая строгость в определениях и утверждениях.
Случаи, когда множество исходов бесконечно, требуют особой строгости,
так как небольшая неосторожность может вести к парадоксам или противоречиям.
Примером этому может служить следующая

}

\задача
    Пусть $p$ --- вероятность того, что случайно выбранная точка в круге находится на расстоянии, меньшем половины радиуса.
    Придумайте, как получить $p=1/2$ и $p=1/4$.
\кзадача


{\small
\опр \выд{Вероятностным пространством} называется тройка $(\Omega ,\A, P)$, где
\begin{items}{-4}
    \item[$\Omega $] --- некоторое множество (\выд{пространство элементарных
    событий});
    \item[$\A$] --- совокупность подмножеств множества $\Omega$,
    называемых \выд{событиями}, обладающая следующими свойствами:
    \begin{nums}{-4}
        \item $\es \in \A$, $\Omega \in \A$,
        \item если $A \in \A$, то событие $\overline {A}$ \выд противоположное событию $A$ (происходящее тогда и только тогда, когда не происходит событие $A$) лежит в $\A$;
        \item если $A,B \in \A$, то \выд сумма событий $A\cup B$ (происходящее тогда и только тогда, когда происходит хотя бы одно из событий $A, B$) лежит в $\A$;
        \item если $A,B \in \A$, то \выд произведение событий $A\cap B$ (происходящее тогда и только тогда, когда происходит и $A$, и $B$) лежит в $\A$;
    \end{nums}
    %iii)$ если $\Omega$ бесконечно, и если $A_i \in \A, i=1,2,\dots $, то
    %$\bigcup\limits_{i=1}^{\infty} A_i \in \A $  и
    %$\bigcap\limits_{i=1}^{\infty} A_i \in \A $
    \item[$P$] --- числовая функция $P:\A\to {\Bbb R}$ (называемая \выд{вероятностью}, или \выд{вероятностной мерой}), такая что
    \begin{nums}{-4}
        \item $P(\es)=0$, $P(\Omega)=1$, $P(A)\ge 0$ для любого $A \in \A$;
        \item \выд{(аддитивность вероятностной меры)} если $A\cap B=\es$ (т.~е.~события $A$ и $B$ \выд{несовместны}), то $P(A\cup B)=P(A)+P(B)$;
    \end{nums}
\end{items}
%$iii)$ [счетная аддитивность вероятностной меры] в случае бесконечного
%$\Omega$ если $A_i \in \A, i=1,2,\dots $ и $A_i\cap A_j=\es,  i\ne
%j$, (т.е.  $\{A_i\}$ --- попарно несовместны), то
%$P(\bigcup\limits_{i=1}^{\infty}A_i)=\sum\limits_{i=1}^{\infty}P(A_i)$.\\
%События, вероятность которых равна 1, называются \выд{достоверными}.
\копр
}

\vspace*{-3mm}
\ссзадача
    Если множество $\Omega$ бесконечно, то требуются дополнительные аксиомы, касающиеся суммы и произведения бесконечного числа событий.
    Попробуйте их придумать и сформулировать.
\кзадача

\задача
    \пункт Постройте вероятностное пространство, отвечающее бросанию четырех игральных костей.
    \пункт Найдите вероятность выпадения при бросании четырёх костей хотя бы одной шестерки.
\кзадача

\задача
    Пусть $(\Omega ,\A, P)$ --- вероятностное пространство.
    Докажите, что
    \пункт
        вероятность любого события не превосходит 1;
    \пункт
        если $A, B\in\A$ --- события, причём $A\subset B$, то $P(A)\le P(B)$.
\кзадача

\задача
    Рассмотрим %любое
    конечное множество $\Omega $ из $k$ элементов.
    Пусть $\A$ --- множество $2^\Omega$ всех подмножеств $\Omega$.
    Для каждого $X\in\A$ положим $P(X)=|X|/k$.
    Докажите, что тройка $(\Omega ,\A, P)$ образует вероятностное пространство.
\кзадача


\задача
    Из множества всех последовательностей длины~$n$, состоящих из цифр~$0,\,1,\,2$, случайно выбирается одна.
    Найдите вероятность того, что в последовательности ровно $m_0$~нулей, $m_1$~единиц и $m_2$~двоек.
\кзадача

% \задача
%     \пункт[Выборка без возвращения]
%         В урне $M$ черных и $N$ белых шаров.
%         Наугад выбрано $n$ шаров, причем после взятия из урны шар не возвращается назад.
%         Какова вероятность вытащить ровно $m$ белых шаров?
%     \пункт[Выборка с возвращением]
%         Та же задача с тем отличием, что после каждого взятия шар кладется на место.
% \кзадача


\задача
Юра выучил 3 билета из 30. На экзамене все билеты
%(каждый~---~в одном экземпляре)
лежат на столе, студенты
по очереди тянут билеты, вытянутые билеты убирают со стола.
Каким выгоднее тянуть билет Юре? %, чтобы %иметь наибольшую вероятность
%вытянуть выученный билет?
\кзадача



\задача[Схема Бернулли]
    Проводятся $n$ опытов, в каждом опыте может произойти определенное событие (<<успех>>) с вероятностью $p$ (или не произойти --- <<неудача>> --- с вероятностью $q=1-p$), после чего подсчитывается количество успехов.
    Постройте вероятностное пространство, соответствующее этому эксперименту.
\кзадача



% \задача
%     Пусть вероятность попасть под машину, переходя улицу в неположенном месте, равна 0{,}01.
%     Какова вероятность остаться целым, сто раз перейдя улицу в неположенном месте?
% \кзадача

\задача
\пункт Пусть вероятность попасть под машину при переходе улицы в неположенном
месте равна 0,01. Какова вероятность остаться целым, сто раз
перейдя улицу в неположенном месте?
\пункт Как связана эта вероятность с числом $e$?
Вычислите её поточнее.
\кзадача

\задача[Геометрическое распределение]
    Проводится сколь угодно длинная серия опытов, в каждом из которых может произойти событие (<<успех>>) с вероятностью $p$, или событие <<неудача>> (с вероятностью $q=1-p$), до тех пор, пока не произойдёт успех.
    Подсчитывается количество испытаний до наблюдения первого <<успеха>>.
    Постройте вероятностное пространство, соответствующее этому эксперименту.
\кзадача

\задача
Про некий вид бактерий известно,
что каждая бактерия через минуту после своего появления
на свет делится с вероятностью $p_k$ на $k$ потомков, где $k=0,1,\dots,10$.
При этом $p_0$ --- это вероятность смерти
бактерии через минуту после рождения.
Докажите, что вероятность $x$ того, что весь род, начавшийся с данной
бактерии,  когда-либо целиком вымрет, удовлетворяет уравнению
$x=p_0+p_1x+p_2x^2+\dots+p_{10}x^{10}$.
\кзадача


% \сзадача [Задача о разорении]
% Игрок, имеющий $n$ монет, играет против казино, которое имеет
% неограниченное число монет. За одну игру игрок либо проигрывает монету,
% либо выигрывает с вероятностью 0,5. Он играет, пока не разорится. Какова
% вероятность разориться ровно за $m$ игр?
% \кзадача

\задача %[Сумасшедшая старушка]
Каждый из $n$ пассажиров купил по билету на $n$-местный самолет.
Первой зашла сумасшедшая старушка и села на случайное место.
Далее, каждый вновь вошедший занимает свое место, если оно свободно;
иначе занимает случайное. Какова вероятность того,
что последний пассажир займет свое место?
\кзадача



\ЛичныйКондуит{0mm}{6mm}
\ОбнулитьКондуит
\newpage


\раздел{Условная вероятность}

\задача
    Пусть $B$ --- событие, обладающее ненулевой вероятностью.
    \невСтрочку
    \пункт
        Дайте определение условной вероятности $P(A\mid B)$ события $A$ при условии $B$.
    \пункт
        Докажите, что тройка $(\Omega, \A, P_B)$, где $P_B$ --- условная вероятность, является вероятностным пространством.
\кзадача

\задача
%    Пусть вероятность рождения мальчика равна $1/2$.
    Какова вероятность того, что в семье два мальчика, если один из детей --- мальчик?
\кзадача

\задача
    Вероятность попадания в цель при отдельном выстреле равна $0{,}2$.
    Какова вероятность поразить цель, если в $2\%$ случаев выстрел не происходит из-за осечки?
\кзадача

\опр
    События $A$ и $B$ называются \выд{независимыми}, если $P(AB)=P(A)\cdot P(B)$.
    События $A_1,\dots,A_n$ называются \выд{независимыми в совокупности}, если
    для любых $1\le i_1<i_2<\dots<i_k\le n$ выполнено равенство
    $P(A_{i_1}A_{i_2}\dots A_{i_k})=P(A_{i_1})\cdot P(A_{i_2})\cdot\dots\cdot P(A_{i_k})$.
\копр

\задача
    Из колоды в 52 карты выбирается наудачу одна карта. Независимы ли события\\
    \вСтрочку
    \пункт
        \лк выбрать вальта\пк\ и \лк выбрать пику\пк;
    \пункт
        \лк выбрать вальта\пк\ и \лк не выбрать даму\пк?
\кзадача

\задача
    Пусть $A$ и $B$ независимы.
    \пункт Верно ли, что $P(B\mid A)=P(B)$?
    \пункт Выразите $P(A\ {\text и} \ B)$ через $P(A)$ и $P(B)$.
    \пункт Верно ли, что независимы события $A$ и \лк не $B$\пк?
    \пункт Тот же вопрос про события \лк не $A$\пк\ и \лк не $B$\пк.
\кзадача

\задача
    Следует ли из попарной независимости нескольких событий их независимость в
    совокупности?
\кзадача

\задача[Теорема умножения вероятностей]
    Пусть $A_1,A_2,\ldots,A_n$ --- события, вероятности которых больше~0.
    Докажите, что
    $$
    P(A_1 A_2\dots A_n) = P(A_1)\cdot P(A_2\mid A_1)\cdot P(A_3 \mid A_1 A_2)\cdot\ldots\cdot P(A_n \mid A_1\dots A_{n-1}).
    $$
\кзадача

\опр
    События $A$ и $B$ \выд{несовместны}, если они не могут произойти одновременно ($A\cap B=\es$).
\копр

\задача[Формула полной вероятности]
    Пусть $H_1,H_2,\ldots,H_n$ --- попарно несовместные события (<<гипотезы>>),
    причем $H_1\cup H_2 \cup \dots \cup H_n=\Omega$.
    Докажите, что для любого события $B$
    $$
    P(B)=\sum\limits_{i=1}^{n}P(H_i)\cdot P(B\mid H_i).
    $$
\кзадача

\задача
    Два охотника одновременно выстрелили одинаковыми пулями в медведя.
    Медведь был убит одной пулей.
    Как поделить охотникам шкуру, если вероятность попадания у первого --- $0{,}3$, а у второго --- $0{,}6$?
\кзадача

\задача
Три завода выпускают одинаковые изделия. Первый производит 50\%
всей продукции, второй --- 20\%, третий --- 30\%.
Первый завод выпускает 1\% брака, второй --- 8\%, третий --- 3\%.
Выбранное наугад изделие --- бракованное. Какова вероятность
того, что оно %изготовлено на втором заводе?
со второго завода?
\кзадача

% \задача[Формула Байеса]
%     Пусть $H_1,H_2,\ldots,H_n$ --- попарно несовместные события (<<гипотезы>>).
%     Предположим, стало известно, что событие $A$ произошло.
%     Тогда
%     $$
%     P(H_i \mid A) = \dfrac{P(H_i)\cdot P(A\mid H_i)}{P(A)}.
%     $$
% \кзадача


\задача
Пусть при рентгеновском обследовании вероятность обнаружить туберкулез у больного туберкулезом равна 0,9, а вероятность принять здорового человека за больного равна 0,01. Доля больных туберкулезом по отношению ко всему населению равна 0,001.
С какой вероятностью человек здоров, если \пункт он был признан больным при обследовании;
%Там ответ --- 91,7 %
\пункт
он был признан больным при двух независимых обследованиях?
%Там ответ 10.98
\кзадача

\задача
    В первой урне 2 белых и 6 чёрных шаров, во второй --- 4 белых и 2 чёрных. Из первой урны наудачу переложили 2 шара во вторую, после чего из второй урны наудачу достали один шар.\\
    %\невСтрочку
    \пункт
        Какова вероятность того, что этот шар белый?
    \пункт
        Шар, взятый из второй урны, оказался белым. Какова вероятность того, что из первой урны во вторую были переложены 2 белых шара?
\кзадача



\задача
    Из 100 симметричных монет одна фальшивая (с двумя орлами).
    Выбрали случайно монету,~бросили 5 раз: выпали все орлы.
    С какой вероятностью,  если её бросить ещё 10 раз, снова выпадут все орлы?
\кзадача

\раздел{Геометрические вероятности}

\noindent
При решении требуется построить соответствующее бесконечное
вероятностное пространство.

\задача
Палку случайно ломают на 3 части. С какой вероятностью из них можно сложить~треугольник?
\кзадача

\задача [Задача Бюффона]
На плоскость, разлинованную параллельными прямыми на расстоянии
$a$ друг от друга, случайно брошена игла длиной $l<a$. Найти вероятность
пересечения иглы с какой-нибудь прямой.
\кзадача

\задача [Парадокс Бертрана]
С какой вероятностью случайная хорда некой данной окружности будет больше
стороны правильного треугольника, вписанного в эту окружность?
\кзадача

\задача
Монету радиусом $r$ и толщиной $d$
бросают на горизонтальную поверхность (соударение неупругое).
Какова вероятность того, что монета упадет на ребро? %(Соударение считается неупругим.)
\кзадача


\ЛичныйКондуит{0mm}{6mm}
% \GenXMLW
\end{document}

