% !TeX encoding = windows-1251
\documentclass[a4paper,12pt]{article}
\usepackage{newlistok}

\УвеличитьВысоту{2.3cm}
\УвеличитьШирину{1.4cm}

\Заголовок{Пространство-время в классической теории}
\НомерЛистка{STO-1}
\ДатаЛистка{01.2017}

\sloppy

\begin{document}
\СоздатьЗаголовок

{\footnotesize
Перед тем, как изучать специальную теорию относительности,
необходимо научиться мыслить в терминах пространства-времени,
а также освоить некоторые вещи из линейной алгебры.

Соответственно сначала мы посмотрим на классическую теорию в терминах пространства-времени,
затем изучим необходимые основы линейной алгебры и уже тогда перейдём к СТО.

Чтобы говорить о движении каких-либо объектов, нам нужно ввести систему координат.
Обычно это три координаты в пространстве и одна координата --- время. После того, как координаты введены, мы можем изучать динамику тел.

Далее возникает естественный вопрос: что произойдёт, если взять другую систему координат,
ведь координаты в нашем пространстве-времени нельзя выбрать инвариантно, то есть не привязываясь к существующим там объектам.
Так возникает понятие \выд{инерциальной системы отсчёта.}

Выберем какую-нибудь систему координат (три пространственных координаты $(x,y,z)$ и временную координату $t$).
Эту систему координат будем называть \выд{системой отсчёта лаборатории}.
Системе отсчёта лаборатории всегда будет противопоставляться система отсчёта \лк ракеты\пк.
В ракете есть свои часы и своя метровая линейка,
они были взяты из лаборатории перед стартом и не отличались от своих копий в лаборатории.
С помощью этих часов и линейки мы можем найти координаты любого события в пространстве-времени в координатах ракеты.

В нашей ракете мы будем проводить ровно те же опыты, что в лаборатории.
Мы постулируем, что результаты экспериментов, проведённых в лаборатории и ракете, подчиняются одним и тем же законам
(этот постулат --- результат множества проведённых экспериментов).
Оказывается, для определения достаточно рассматривать опыты с равномерным движением.
\spacer

}

\опр
Система отсчёта называет \выд{инерциальной} в некоторой области пространства-времени, если во всей этой области с некоторой данной степенью точности любая первоначально покоившаяся частица сохраняет своё состояние покоя, а любая частица, изначально двигавшаяся, сохраняет своё движение без изменение величины и направления скорости.
\копр

\опр
\выдд{Принцип относительности}: все законы физики одинаковы во всех инерциальных системах отсчёта.
Или по-другому, невозможно отличить одну инерциальную систему отсчёта от другой с помощью законов физики.
Это не значит, что величины должны быть одинаковыми по своим \выд{численным значениям} (например, время между событиями ровно 1\,с), но должны удовлетворять тем же \выд{законам}.
\копр



\задача
  Для измерений будем использовать обыкновенную линейку и обыкновенный секундомер.
  Рассмотрим систему отсчёта, связанную с МКС (международной космической станцией).
  Приведите пример области пространства-времени, где эта система отсчёта заведомо инерциальна, и заведомо неинерциальна.
\кзадача

{\footnotesize
\spacer
Итак, пространство с выбранными пространственными (не обязательно всеми тремя, возможно и с одной или двумя) и временной координатой будем называть \выд{пространством-временем}.
Яблоко в момент времени $5$ и координатами $(2,-3,10)$ будем описывать точкой $(2,-3,10,5)$. Равноускоренное падение этого яблока вниз (вдоль третьей координаты) из этой точки --- это множество точек $(2,-3,10-g\tau^2/2,5+\tau)$. Равномерное движение мотоциклиста --- например, набор точек $(30t,40t,150,t)$. Множество точек в пространстве-времени, соответствующих данной частице во все моменты времени, называется \выд{мировой линией} частицы.
\spacer

}

\задача
    Опишите мировую линию
    \пункт покоящейся частицы;\\
    \пункт равномерно двигающейся частицы;\\
    \пункт Как выглядит в пространстве-времени покоящийся стержень?\\
    \пункт Равномерно без вращения двигающийся стержень;\\
    \пункт Вращающийся на месте стержень;
\кзадача


\соглашение
Для простоты будем считать,
оси ракеты сонаправлены с осями лаборатории, начало координат ракеты (то есть сама ракета) движется вдоль первой оси со скоростью $u$, в момент времени $0$ центры систем отсчёта ракеты и лаборатории совпадают.
\ксоглашение


\задача
  \пункт
    Опишите мировую линию лаборатории (она находится в точке $(0,0,0)$ в своей системе отсчёта) в координатах лаборатории и в координатах ракеты.\\
  \пункт
    В системе отсчёта ракеты табуретка летит со скоростью $\vec{w}$ вдоль второй оси. Опишите мировую линию табуретки в системе отсчёта лаборатории и ракеты.
\кзадача

\задача
  Опишите множество точек в пространстве-времени в координатах лаборатории и в координатах ракеты, которые соответствуют покоящемуся в лаборатории стержню длины 1 направленному вдоль \пункт оси $Ox$; \пункт оси $Oy$; \пункт лежащему в плоскости $xOy$ под углом $\ph$ к~оси~$Ox$.
\кзадача

\ЛичныйКондуит{0mm}{6mm}
% \GenXMLW

\end{document}
