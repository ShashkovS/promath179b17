% !TeX encoding = windows-1251
\documentclass[a4paper,12pt]{article}
\usepackage{newlistok}

\УвеличитьВысоту{2.3cm}
\УвеличитьШирину{1.4cm}
\newcommand{\wa}{\overrightarrow}

\Заголовок{Линейные и аффинные функции}
\НомерЛистка{STO-2}
\ДатаЛистка{01.2017}

\sloppy

\begin{document}
\СоздатьЗаголовок

\задача
Как перелётным птицам проще лететь: по ветру или против ветра? (в каком смысле \лк проще\пк следует понять самостоятельно)
\кзадача

\задача
  Астрономы считают, что все галактики разлетаются
  прямолинейно по направлениям от нашей со скоростями,
  пропорциональными расстояниям до них. Означает ли это, что наша
  галактика --- центр вселенной?
\кзадача

\задача
  Крючок безмена заменили на более тяжёлый и одновременно параллельно
  сдвинули вниз шкалу, так чтобы нуль совпал с новым положением
  стрелки. Будет ли безмен после этого правильно измерять вес?
\кзадача

\опр
\выд{Векторным пространством $\R^n$} называется множество всевозможных наборов $(x_1\sco x_n)$ действительный чисел вместе с операциями сложения  $(x_1\sco x_n) + (y_1\sco y_n) = (x_1+y_1\sco x_n+y_n)$ и умножения на числа $\la\,(x_1\sco x_n) = (\la x_1\sco \la x_n)$.
\копр

\опр
  Отображение $\R^m\corr{f}\R^n$ называется {\it линейным\/}, если для всех векторов
  $x\in\R^m$, $y\in\R^m$ и всех чисел $\la,\mu\in\R$ выполняется
  равенство $f(\la x+\mu y)=\la\,f(x)+\mu\,f(y)$.
  Отображение $\R^m\corr{g}\R^n$ называется {\it аффинным\/}, если существует $a\in\R^m$, такое что
  отображение $x\longmapsto g(x+a)-g(a)$ линейно.
  \\
  Будем опускать лишние скобки в выражении $f((x_1\sco x_m))$ и писать просто $f(x_1\sco x_m)$.
  \\Например, будем писать $f(x)$ вместо $f((x))$ для $x\in\R^1$, $f(x, y)$ вместо $f((x, y))$ для $(x,y)\in\R^2$.
\копр

\задача Являются ли следующие отображения аффинными или линейными?:\\
\пункт $f\colon\R^1\to\R^1$, $f(x)=(0)$;
\пункт $f\colon\R^1\to\R^1$, $f(x)= (x^2+1)$;
\пункт $f\colon\R^1\to\R^2$, $f(x)= (57x,179x+57)$;\\
\пункт $f\colon\R^2\to\R^1$, $f(x_1,x_2)=-3(x_1 - x_2)$;
\пункт $f\colon\R^2\to\R^2$, $f(x_1,x_2)=(x_2 - x_1 - 1, x_1)$;\\
\пункт $f\colon\R^2\to\R^3$, $f(x_1,x_2)=(x_1, x_1+x_2+1, x_1^2+x_2^2)$?
\кзадача


\задача
  Изменим в определении аффинного отображения фразу \лк существует
  $a\in\R^m$\пк\ на фразу \лк для любого $a\in\R^m$\пк. Будет ли
  новое определение эквивалентно исходному?
\кзадача


\задача
На плоскости фиксированы три точки: $O$, $A$ и $B$. Нарисуйте множество точек $\la \wa{OA}+\mu\wa{OB}$ при \пункт $\la+\mu = 1$; \пункт $\la,\mu>0$.
\кзадача


\задача
  Пусть линейное отображения $\R^2\corr{f}\R^2$ переводит базисные векторы
  $e_1=(1,0)$ и $e_2=(0,1)$ в векторы $(a,c)$ и $(b,d)$ соответственно.
  Куда оно переведёт вектор $(x,y)$?
\кзадача

\задача
  Опишите все линейные и все аффинные отображения\\
  \пункт $\R^n\to\R^1$\hfill
  \пункт $\R^1\to\R^n$\hfill
  \пункт $\R^2\to\R^2$\hfill
  \пункт $\R^n\to\R^m$
\кзадача


\задача
Докажите, что множество всех линейных отображений $f\colon\R^n\to\R^m$ образует коммутативную группу по сложению (то есть сложение коммутативно, ассоциативно и имеет обратный элемент).
%Обозначение: $\Hom(\R^n,\R^m)$ (от слова \лк гомоморфизм\пк);
\кзадача

\задача
Пусть задано некоторое биективное отображение $f\colon \R^m\to\R^m$. Известно, что точка в $\R^m$ движется равномерно и прямолинейно тогда и только тогда, когда её образ движется равномерно и прямолинейно.
Докажите, что преобразование $f$ аффинно.
\кзадача

\задача
Докажите, что в классической механике преобразование координат между инерциальными системами отсчёта аффинно.
\кзадача


\опр
Набор векторов $\hc{v_1\sco v_n}\subset\R^m$ называется базисом, если для любого вектора $w\in\R^m$ найдётся единственный набор чисел $\hc{\la_1\sco\la_n}$ (который называется координатами вектора $w$ в этом базисе) такой, что
\vspace*{-2mm}
$$w = \la v_1 + \ldots + \la_n v_n.$$
\vspace*{-4mm}
\копр

\задача
\пункт
Опишите все базисы в $\R^1$;
\пункт
Докажите, что в любом базисе в $\R^2$ ровно два вектора.
\спункт
Докажете, что в любом базисе в $\R^m$ ровно $m$ векторов.
\кзадача

\ЛичныйКондуит{0mm}{6mm}

%\СделатьКондуитИз{6.2mm}{6.2mm}{sp_STO.tex}

% \GenXMLW
\end{document}

\задача
Как перелётным птицам проще лететь: по ветру или против ветра? (в каком смысле \лк проще\пк следует понять самостоятельно)
\кзадача


{\footnotesize
Для изучения сложных вопросов необходимо изучить преобразования, связывающие две различные инерциальные системы отсчёта.

Навсегда зафиксируем некоторую инерциальную систему отсчёта и будем называть её \выд{системой отсчёта лаборатории}. Другая система\footnote{}, называется \выд инерциальной, если любая точка двигается в ней равномерно и прямолинейно тогда и только тогда, когда она двигается равномерно и прямолинейно в системе отсчёта лаборатории.

\noindent{\bf Обозначения:}
Для того, чтобы не путать разные системы координат, мы будем использовать следующие обозначения: центр координат системы отсчёта \лк лаборатории\пк (\лк неподвижной\пк) будем обозначать через $O$, вектора, вдоль которых измеряются координаты --- $(e_1,e_2,e_3)$ или $(e_x,e_y,e_z)$ в зависимости от ситуации, и, собственно, значения координат --- через $(x^1,x^2,x^3)$ или через $(x,y,z)$ соответственно. Для системы координат \лк ракеты\пк точно такие же обозначения, но со штрихами: $O'$, $(e_{1'},e_{2'},e_{3'})$, $(e_{x'},e_{y'},e_{z'})$, $(x^{1'},x^{2'},x^{3'})$ и $(x',y',z')$ соответственно.

Преобразования, превращающие данную инерциальную (то есть систему координат \лк лаборатории\пк) систему отсчёта в другую инерциальную называются преобразованиями Галилея.



}



\задача
Как может выражаться координаты центра системы отсчёта ракеты через координаты лаборатории и время?
\кзадача

\задача
Докажите, что если центр системы отсчёта ракеты неподвижен, то и вся система отсчёта ракеты неподвижна.
\кзадача

\опр
Преобразование $f$ называется \выд линейным, если для любых двух векторов $v$ и $w$ и любых двух чисел $\la$  и $\mu$ верно: $f(\la v + \mu w) = \la f(v) + \mu f(w)$.
\копр

\задача
Предположим, что $O'=O$. Докажите, что преобразование координат линейно.
\кзадача

\задача
Докажите, что множество всех преобразований Галилея






\end{document}
