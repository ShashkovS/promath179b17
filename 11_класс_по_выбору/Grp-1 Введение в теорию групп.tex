% !TeX encoding = windows-1251
\documentclass[12pt]{article}
\usepackage{newlistok}

\УвеличитьВысоту{2.1cm}
\УвеличитьШирину{1.1cm}

% \renewcommand{\spacer}{\vspace{1mm}}
\begin{document}
	\Заголовок{Введение в теорию групп --- основные понятия}
	\НомерЛистка{GR-1}
	\ДатаЛистка{01.2017}
	\СоздатьЗаголовок
	\опр
	Группой $(G, \cdot)$ называется множество $G$ с заданной на нем бинарной операцией $\cdot$ , то есть отображением $G\times G \rightarrow G$, значение которого на элементах $a$ и $b$ из $G$ обозначается $a\cdot b$ или $ab$, со следующими свойствами
	\eqn{a(bc)=(ab)c \;\; \forall a, b, c\in G}
	\eqn{\exists \, e \in G: \; \forall a \in G \; ea=ae=a}
	\eqn{\forall a \in G \; \exists \, a^{-1} \in G: \; aa^{-1}=a^{-1}a=e}
		Иными словами, существует нейтральный элемент (иногда он называется просто единицей группы и обозначается цифрой 1), и для каждого элемента существует обратный к нему. Свойство (1) называют ассоциативностью.
	\копр
	\УвеличитьПромежутки{94}
	\замечание
	Когда операция в группе $(G, \cdot)$ оговорена заранее, то такую группу обозначают просто $G$.
	\кзамечание
	
		\задача
		Проверить, являются ли группами следующие множества с указанными операциями:
		\пункт $(\Z,+)$;
		\пункт $(\Z \backslash \{0\},\cdot)$;
		\пункт $(\R,\cdot)$;
		\пункт $(\R \backslash \{0\},\cdot)$;
		\пункт $(\Z_m,+)$;
		\пункт $(\Z_m\backslash \{ 0 \},\cdot)$;
		\пункт $S_n$ с операцией композиции;
		

		\спункт Множество $\{a,b,c,d\}$ с операцией $\times$ и таблицей умножения $\tab{|c||c|c|c|c|}{\hline $\times$&a&b&c&d \\ \hline\hline a&a&b&c&d \\ \hline b&b&a&d&c \\ \hline c&c&d&a&b \\ \hline d&d&c&b&a \\ \hline}$
		\кзадача

	\задача
	\пункт Докажите, что в уравнениях в группе можно сокращать, то есть если $ac=bc$ или $ca=cb$, то $a=b$.
	\кзадача

	\взадача
	\пункт Докажите, что нейтральный элемент в группе единственный.
	\пункт Докажите, что для каждого элемента группы обратный элемент единственный.
	\пункт Докажите, что в группе уравнения $ax=b$ и $xa=b$ имеют единственные решения и найдите их.
	\пункт Найдите обратный к элементу $ab$.
	\кзадача

	%\опр
	%Недогруппой с правым сокращением называется множество $S$ с заданной на нем бинарной операцией $\cdot$ , то есть отображением $S\times S \rightarrow S$, значение которого на элементах $a$ и $b$ из $S$ обозначается $a\cdot b$ или $ab$, со следующими свойствами
	%\eqn{a(bc)=(ab)c \;\; \forall a, b, c\in S}
	%\eqn{\exists \, e \in S: \; \forall a \in S \; ae=a}
	%\eqn{\forall a \in S \; \exists \, a^{-1} \in S: \; aa^{-1}=e}
	%Иными словами, существует ПРАВЫЙ нейтральный элемент, и для каждого элемента существует ПРАВЫЙ обратный к нему.
	%\копр
	\сзадача
	Докажите, что если в определении группы заменить свойства (2) и (3), на <<существует элемент $e$ такой, что для любого $a\in G \; ae=a$ (правая единица) и для любого элемента $a$ существует такой элемент $a^{-1}$, что $aa^{-1}=e$ (правый обратный)>>, то определенный объект останется группой.
	\кзадача

	\раздел {Группы преобразований}

	\опр
	Группой преобразований множества $M$ называется любой непустой набор $G$ биекций этого множества на себя со следующими условиями:
	\equ{g, h\in G \Longrightarrow g\circ h \in G}
	\equ{g\in G \Longrightarrow g^{-1}\in G}
	То есть множество биекций $G$ замкнуто относительно композиции и взятия обратного отображения.
	\копр

	\задача
	Покажите, что группа преобразований $G$ произвольного множества $M$ является группой (относительно какой операции?), укажите в явном виде единицу этой группы.
	\кзадача

	\опр
	Группой диэдра $D_n$ называется совокупность преобразований правильного $n$-угольника, являющихся движениями (движение сохраняет расстояние между точками).
	\копр
		
	\опр
	Группа $G$ называется абелевой или коммутативной, если $\forall \, a,b \in  G \; ab=ba$.
	\копр
	
	
	\задача
	Покажите, что группа диэдра действительно является группой.
	\кзадача
	
	\задача
	Какие из групп задачи 1 абелевы? Верно ли, что группа диэдра абелева?
	\кзадача
	
	\задача
	\пункт Перечислите все элементы группы $D_3$
	\пункт Перечислите все элементы группы $D_4$
	\пункт Перечислите все элементы группы $D_n$
	\кзадача
	
    \vfill
	\ЛичныйКондуит{0mm}{6mm}
	\ОбнулитьКондуит
\newpage

	\раздел {Подгруппы, обыкновенные}
	
	\опр
	Подгруппой $H$ группы $G$ называется такое непустое подмножество элементов группы $G$, что оно само является группой относительно операции в группе $G$.
	\копр
	
	\задача
	Опишите все подгруппы группы $D_3$
	\кзадача
	
	\задача
	Рассмотрим элемент $g$ группы $G$ и множество всех его степеней -- $\{ \dots g^{-2}, g^{-1}, g^0=e, g, \dots \}$. Докажите, что это множество является подгруппой в $G$.
	\кзадача
	
	\опр
	Циклической группой называется такая группа $G$, что существует элемент $g$ такой, что любой элемент группы $G$ является некоторой степенью элемента $g$. Обозначение $G=\langle g \rangle $
	\копр

	\задача Какие из этих групп являются циклическими?
	\пункт $(\Z,+)$;
	\пункт $D_4$;
	\пункт $(\Q \backslash \{0\},\cdot)$;
	\пункт $(\Z_m, +)$;
	\спункт $(\Z_p \backslash \{0\},\cdot)$, где $p$ -- простое число.
	\кзадача
	
	
	\раздел{Смежные классы по подгруппе}

	\опр
	Пусть задана группа $G$ и ее подгруппа $H$, $g \in G$. Тогда назовем левым смежным классом элемента $g$ по подгруппе $H$ множество элементов вида $gh$, где $h \in H$. Обозначение $gH$. Аналогично определяем правый смежный класс элемента $g$ ($Hg$).
	\копр
	
	\задача
	Рассмотрим группу $S_3$ и ее подгруппу $\{(1,2), id\}$. Выпишите все левые и правые смежные классы по этой подгруппе и покажите, что левые и правые смежные классы совпадают не у всех элементов.
	\кзадача
	
	\задача
	Пусть есть два левых смежных класса $gH$ и $g_1H$. Докажите, что тогда они либо не пересекаются, либо совпадают.
	\кзадача
	
	\задача
	Сопоставим элементам $gh_1$, $gh_2\dots$ смежного класса $gH$ элементы $g_1h_1$, $g_1h_2\dots$ смежного класса $g_1H$, не пересекающегося с $gH$. Докажите, что это отображение смежных классов как множеств биективно.
	\кзадача
	
	\опр
	Порядком конечной группы $G$ называется число ее элементов. Обозначение $\ord G$ или $|G|$.
	\копр

	\опр
	Порядком элемента $g$ конечной группы $G$ называется $\min \{ k \in \N : g^k=e\}$. Обозначение $\ord g$.
	\копр
	
	\задача
	Докажите, что $g^k=e \Longleftrightarrow \ord g|k$.
	\кзадача
	
	\задача
	\пункт Докажите, что множество элементов группы $G$ есть дизъюнктное объединение различных левых смежных классов по данной подгруппе $H$.
	\пункт[Теорема Лагранжа] Пусть группа $G$ конечна. Докажите, что порядок группы $G$ делится на порядок любой ее подгруппы $H$.
	\пункт В условиях предыдущего пункта докажите, что порядок $G$ делится на порядок любого своего элемента.
	\кзадача
	
	\задача
	Докажите, что конечная группа является циклической тогда и только тогда, когда существует элемент, порядок которого совпадает с порядком группы.
	\кзадача
	
	\задача
	Пусть циклическая группа $\langle g \rangle$ имеет порядок $k$. Рассмотрим циклическую подгруппу этой группы, порожденную $m$-й степенью $g$, то есть $\langle g^m\rangle$. В каких случаях эта подгруппа будет совпадать со всей группой?
	\кзадача
	
	\задача
	Докажите, что любая группа порядка $p$, где $p$ -- простое число, циклическая.
	\кзадача
	
	\задача
	Объясните решение задачи 13[32] на языке теории групп.
	\кзадача
	
	\задача
	Докажите теорему Эйлера: $a^{\varphi(m)}\equiv1$ (mod $m$), где $\varphi(m)$ -- количество чисел, не превосходящих $m$ и взаимнопростых с $m$ c помощью теории групп.
	\кзадача

\vfill
	\ЛичныйКондуит{0mm}{6mm}

\GenXMLW


\end{document}
