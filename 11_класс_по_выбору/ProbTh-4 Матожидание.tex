% !TeX encoding = windows-1251
\documentclass[a4paper,12pt]{article}
\usepackage[mag=930]{newlistok}

\УвеличитьШирину{1.5truecm}
\УвеличитьВысоту{2.4truecm}

\Заголовок{Математическое ожидание}
\НомерЛистка{PT-4}
\ДатаЛистка{01.2017}
%\renewcommand{\spacer}{\vfill}
\ВключитьКолонитул
\sloppy

\begin{document}
\СоздатьЗаголовок
\setlength{\abovedisplayskip}{3pt}
\setlength{\belowdisplayskip}{-10pt}


\vspace{1mm}
\опр
Пусть дано конечное вероятностное пространство $\Omega$.
\выд {Случайной величиной} называется любая функция $\xi:\Omega\to\R$. \emph{Математическим ожиданием} случайной величины $\xi$ называется число
$$E\xi = \sum_{\omega \in \Omega} \xi(\omega) P(\omega).$$
\копр
\vspace{-1mm}

\задача
Чему равно матожидание числа, выпадающего на игральном кубике?
\кзадача

\задача
Предположим, случайная величина $\xi$ принимает значения в $\N$. Покажите, что формулу для её матожидания можно переписать так: $E\xi = \sum_{i=1}^\infty i \cdot P(\xi = i)$ или $E\xi = \sum_{i=1}^\infty P(\xi \ge i)$.
\кзадача

\задача
Монетка падает орлом вверх
\пункт с вероятностью $1/2$;
\пункт с вероятностью $1/3$.
Найдите матожидание числа подбрасываний этой монетки до выпадения первого орла.
\кзадача

\задача
Школьнику нужно правильно ответить на вопрос с $n$ вариантами ответа. Он ничего не знает и каждый раз даёт случайный ответ. Найдите матожидание числа его попыток до правильного ответа, если
\пункт он не запоминает даже, какие ответы он уже давал;
\пункт всё-таки запоминает.
\кзадача

\сзадача
Вдоль дороги стоит $n$ фонарей. Дорожная служба меняет все перегоревшие фонари как только перегорают два фонаря подряд. Каждый фонарь перегорает независимо от других. Найдите матожидание числа фонарей, которые придётся поменять при очередной замене.
\кзадача

\задача[Линейность матожидания\footnote{Имеется в виду техника подсчета матожиданий, подробно описанная в этой (надеюсь, уже вам известной) книге: \лк Кормен, Лейзерсон, Ривест, Штайн. Алгоритмы: построение и анализ\пк, глава 5.}]
Пусть случайная величина $\xi$ раскладывается в сумму некоторого числа более простых величин: $\xi = \xi_1 + \xi_2 + \ldots + \xi_p$. Докажите что тогда $E\xi = E\xi_1 + E\xi_2 + \ldots + E\xi_p$.
\кзадача

\задача
50 мужчин и 50 женщин случайно рассаживаются за круглый стол. Назовем мужчину довольным, если рядом с ним сидит женщина. Для каждого мужчины введем случайную величину равную 1, если он окажется доволен, и 0 иначе. Найдите матожидание
\пункт такой случайной величины;
\пункт числа довольных мужчин.
\кзадача

\задача
Собралось k случайных людей. Найдите матожидание числа пар людей с совпадающими днями рождения (для простоты можно считать, что никто не родился 29 февраля).
\кзадача

\задача
Если человек стоит в очереди минуту, будем говорить, что бесцельно затрачена одна человеко-минута. В очереди в банке стоит восемь человек, из них пятеро планируют простые операции, занимающие 1 минуту, а трое планируют операции, занимающие 10 минут. 
Рассмотрим суммарное количество бесцельно затраченных человеко-минут, найдите его
\пункт наименьшее и наибольшее возможное значения;
\пункт математическое ожидание, при условии, что порядок людей в очереди случаен.
\кзадача

\задача
Каждый из $n$ людей положил в мешок по подарку, затем их перемешали и каждый вытащил подарок для себя. Найдите матожидание числа тех, кому достался подарок, который они сами принесли.
\кзадача

\задача
$n$ претендентов на должность в случайном порядке приходят на собеседование. 
Если в результате собеседования выясняется, что новый претендент лучше того, кто в данный момент занимает должность, первого нанимают, а последнего --- увольняют.
\пункт С какой вероятностью $k$-й по силе претендент будет нанят в какой-либо момент.
\пункт Найдите матожидание числа увольнений.
\кзадача

\задача
Чтобы сгенерировать перестановку чисел от 1 до $n$, возьмём число 1 и случайно выберем число, в которое оно переходит. Потом случайно выберем число, в которое переходит только что выбранное число и т.д., пока цикл не замкнётся. Будем строить этим методом цикл за циклом, начиная с наименьшего из ещё не выбранных чисел.
\пункт Докажите что все перестановки получатся с равной вероятностью.
\пункт Чему равна вероятность того, что первый цикл имеет длину $m$.
\пункт Найдите матожидание числа циклов в случайной перестановке.
\пункт Найдите матожидание числа пассажиров, сидящих не на своих местах, в задаче \выдд 16.16 .
\кзадача

\сзадача
В каждую жвачку вложен один из $n$ вкладышей (каждый встречается с вероятностью $1/n$).
Сколько в среднем надо купить жвачек, чтобы собрать полную коллекцию вкладышей?
\кзадача



\ЛичныйКондуит{0mm}{6mm}
% \GenXMLW
\end{document}

