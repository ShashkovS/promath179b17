% !TeX encoding = windows-1251
\documentclass[a4paper,12pt]{article}
\usepackage[mag=970]{newlistok}

\ВключитьКолонитул
\УвеличитьШирину{1.5truecm}
\УвеличитьВысоту{2.7truecm}
\sloppy
\renewcommand{\spacer}{\vfill}

\begin{document}

\Заголовок{Элементы теории вероятностей}
\НомерЛистка{PT-5}
\ДатаЛистка{2017}

\СоздатьЗаголовок

\раздел{Случайные величины. Закон больших чисел}

\опр
Пусть $\Omega $ --- конечное вероятностное пространство. Числовая функция
$\xi:\Omega\to {\Bbb R}$ называется \выд{случайной величиной}.
\копр

\опр
Пусть случайная величина $\xi$ принимает значения $x_1,x_2,\dots ,x_m$ с
вероятностями $p_1,p_2,\dots ,p_m$ соответственно (т.~е.~вероятность события
"$\xi =x_i$" равна $p_i$). Тогда число $M\xi = \sum\limits_{k=1}^{m}x_k
p_k$ называется \выд{математическим ожиданием} случайной величины~$\xi$.
Математическое ожидание равно среднему значению величины $\xi$.\\
Число $D(\xi)=M(\xi - M\xi)^2$ называется \выд{дисперсией} случайной величины
$\xi $. Дисперсия  характеризует уклонение случайной величины от ее среднего
значения.
\копр

\задачан{14}
Человек, имеющий $n$ ключей, хочет отпереть свою дверь, испытывая
ключи независимо один
от другого в случайном порядке. Найдите математическое
ожидание и дисперсию числа испытаний, если неподошедшие ключи\\
\вСтрочку
\пункт не исключаются из дальнейших испытаний;
\пункт если они исключаются.
\кзадача

\задача
\пункт Докажите, что для любых случайных величин $\xi $ и $\eta $ выполнено
равенство $M(\xi +\eta )=M\xi + M\eta$.
\пункт Докажите, что $D\xi = M\xi ^2 - (M\xi )^2$
\кзадача

\задача
В расписании движения автобусов на остановке
\лк Университет\пк\ написано, что
средний интервал движения автобуса \No 57 равен 35 минут, а средний интервал
движения автобуса \No 661 равен 20 минут. Сколько времени в среднем нужно
ждать один из этих автобусов?
\кзадача




\опр
Пусть случайные величины $\xi$ и $\eta$ принимают значения $x_1,x_2,\dots ,x_m$
и  $y_1,y_2,\dots ,y_n$ соответственно. Если для любых $1\le i\le m$ и  $1\le
j\le n$ выполнено равенство
$P(\{\xi=x_i\} \cap \{\eta=y_j\})=P(\{\xi=x_i\})\cdot
P(\{\eta =y_j \})$, то случайные величины
$\xi$ и $\eta$ называются \выд{независимыми}.
\копр


\задача
Пусть $\xi$ и $\eta$ --- независимые случайные величины. Докажите:
\сНовойСтроки
\пункт $M(\xi \eta)=M\xi M\eta$.
Верно ли это равенство для зависимых случайных величин?
\пункт $D(\xi + \eta)=D\xi + D\eta$.
Верно ли это равенство для зависимых случайных величин?
\кзадача


\задача [Неравенство Чебышева]
Докажите, что  для любого положительного $\alpha $ выполнено соотношение
$P\{|\xi - M\xi|\ge \alpha\}\le \frac{D\xi}{\alpha ^2}$.
\кзадача

\задача
Схемой Бернулли называется последовательность $n$  независимых испытаний
с двумя возможными исходами --- "успех" и "неудача", причем вероятность
успеха в каждом испытании равна $p$. Обозначим $\mu _n$ случайную
величину, равную числу успехов при $n$ испытаниях в схеме Бернулли.
\сНовойСтроки
\пункт Найдите $M\mu_n $, $D\mu_n $.
\пункт [Закон больших чисел для схемы Бернулли] Докажите, что для любого
$\alpha>0 $ выполнено:
$\lim\limits_{n\to\infty }P\{|\frac{\mu_n -M\mu_n}{n}|< \alpha\}=1$.
Последнее утверждение означает, что при увеличении числа испытаний частота
выпадения успехов стремится к вероятности $p$.
\кзадача

\задача
Рассмотрим схему Бернулли с вероятностью успеха, равной $1/2$ (например,
$n$-кратное подбрасывание монеты). Пусть $\alpha$ и $\beta$ --- положительные
числа. Обозначим через $P_n(\alpha, \beta)$ вероятность того, что при $n$
испытаниях число успехов заключено между $n/2-\alpha  \sqrt {n}$ и
$n/2+\beta  \sqrt {n}$. Найдите предел $P_n(\alpha, \beta)$ при
$n\to \infty$.
\кзадача

\ЛичныйКондуит{0mm}{6mm}
\ОбнулитьКондуит
\newpage

\раздел{Случайные блуждания}

\задача
Предположим, что мы находимся в целочисленной точке
горизонтальной прямой
 и каждую секунду сдвигаемся с вероятностью $1/2$
на 1 вправо или влево.
\сНовойСтроки
\пункт Найдите число способов попасть из начала координат
в точку с координатой $x$ через
$t$ секунд $(t\ge x\ge 0)$.
\пункт [Принцип отражения] Докажите, что число способов попасть
через $t$ секунд из точки $x>0$
в точку $y>0$, не проходя через начало координат, равно числу способов
попасть через $t$ секунд из точки $-x$ в точку $y$.
\пункт Найдите число способов попасть из начала координат в точку $ x>0$
через $t$ секунд, не проходя при этом второй раз через начало координат.
\пункт Найдите число способов двигаться из начала координат $t$ секунд, не
проходя при этом второй раз через начало координат.
\кзадача

\задача
Пусть в начальный момент времени мы находимся в начале координат.
\сНовойСтроки
\пункт Найдите вероятность $u_{2t}$ возвращения в начало координат через $2t$
секунд.
\пункт Обозначим $f_{2t}$ вероятность первого возвращения в начало координат
через $2t$ секунд. Докажите, что $f_{2t}=u_{2t-2}-u_{2t}$.
\пункт Докажите, что случайное блуждание на прямой возвратно,
т.~е.~что выйдя из
начала координат, мы вернемся в него с вероятностью 1.
\пункт Докажите, что выйдя из начала координат, мы с вероятностью 1
достигнем каждой целочисленной точки.
\кзадача

\задача
Аналогично предыдущему определяется случайное блуждание на плоскости и в
пространстве. В каждую секунду производится сдвиг на 1 в  направлении,
параллельном одной из координатных осей. Вероятности сдвига по всем
направлениям равны $1/4$ в случае плоскости и $1/6$
в случае пространства. Также обозначим $u_{2t}$
и $f_{2t}$ соответственно вероятности возвращения и первого возвращения
в начало координат через $2t$ секунд.
\сНовойСтроки
\пункт Найдите $u_{2t}$ в случае блуждания на плоскости и в пространстве.
\пункт Докажите, что $u_{2t}=\sum\limits_{k=1}^{t} f_{2k}u_{2t-2k}$.
\спункт Докажите, что в случае блуждания на плоскости   ряд
$\sum\limits_{t=0}^{\infty} u_{2t}$ расходится и блуждание возвратно.
\спункт Докажите, что при блуждании в пространстве ряд
$\sum\limits_{t=0}^{\infty} u_{2t}$
сходится и вероятность возврата строго меньше 1.
\кзадача

\vspace*{-5mm}
\раздел {Дополнительные задачи}

\задача
Двое бросают монету --- один 10 раз, другой --- 11. Какова вероятность того,
что у второго орлов выпало больше, чем у первого?  \кзадача

\задача [Задача о баллотировке] Предположим, что на выборах кандидат
$P$ набрал $p$ голосов, а кандидат $Q$ набрал $q$ голосов, причем
$p>q$. Найдите вероятность того, что при последовательном подсчете голосов
$P$ все время был впереди $Q$.
\кзадача


\сзадача %[Сумасшедшая старушка]
Каждый из $n$ пассажиров купил по билету на $n$-местный самолет.
Первой зашла сумасшедшая старушка и уселась на случайное
место. Далее, каждый вновь пришедший занимает свое место, если оно свободно;
в противном случае он занимает случайное место. Какова вероятность того, что
последний пассажир займет свое место?
\кзадача

\задача
Датчик случайных чисел может выдавать конечное число чисел, каждое число ---
с определенной вероятностью. Скажем, что один датчик круче другого, если
с вероятностью большей $1/2$ выданное им число больше числа, выданного другим
датчиком. Можно ли изготовить 3 датчика $A$, $B$ и $C$ так, чтобы $A$
был круче $B$, $B$ был круче $C$, а $C$ был круче $A$?
\кзадача

\сзадача [Выбор невесты]
Царь желает выбрать самую красивую невесту из $100$ претенденток.
Процедура выбора  невесты состоит в следующем: претендентки в случайном
порядке приходят к царю, и в момент прихода очередной претендентки
 царь может объявить
ее своей невестой (царь заранее не знаком с претендентками, но легко
упорядочивает девушек по красоте). Докажите, что царь может выбрать самую
красивую с вероятностью, большей $1/3$.
\кзадача

\сзадача
В жевачку вложен с вероятностью $1/n$ один из $n$ вкладышей. Какое количество
жевачек нужно в среднем купить, чтобы собрать полную коллекцию вкладышей?
\кзадача

\ЛичныйКондуит{0mm}{6mm}


% \GenXMLW
\end{document} 