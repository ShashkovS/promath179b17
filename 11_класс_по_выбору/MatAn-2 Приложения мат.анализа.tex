% !TeX encoding = windows-1251
\documentclass[12pt]{article}
\usepackage{newlistok}

\УвеличитьШирину{15truemm}
\УвеличитьВысоту{15truemm}
\ВключитьКолонитул
% \pagestyle{empty}
% \renewcommand{\spacer}{\vfil}

\Заголовок{Приложения математического анализа}

\НомерЛистка{MA-2}
\ДатаЛистка{01.2017}


%\sloppy

\begin{document}

%\pagestyle{empty}

\СоздатьЗаголовок

\задача Спидометр на велосипеде считывает время, за которое колесо совершает один
оборот.
Какую величину в действительности он показывает и какую стремится показать?
Зависит ли от скорости точность спидометра?
\кзадача


\задача
На координатной плоскости дан единичный отрезок $AB$.
Для каждого $\alpha$ из промежутка $[0;\pi]$ пусть $f(\alpha)$ ---
длина проекции отрезка $AB$ на прямую, выходящую из начала координат под
углом $\alpha$ к оси абцисс.
\выд{Вариацией} $AB$ называется среднее значение проекций $AB$
по всем направлениям, то есть число
$$
K=\frac{1}{\pi}\int_0^{\pi}f(\alpha)\, d\alpha.
$$
Найдите $K$.
\кзадача


\задача
\пункт Пусть точка движется по прямой так, что в момент времени $t$ она имеет координату $x(t)$.
Определите скорость и ускорение точки в момент времени $t_0$. Какой должна быть функция $x(t)$?
\сНовойСтроки
\пункт Шпанская мушка летает по комнате так, что расстояние от неё до двух соседних стен
и пола  в момент времени $t$ с --- это $x(t)$ м, $y(t)$ м и $z(t)$ м соответственно. Найдите скорость мушки.
\пункт Пусть мушка летает по окружности радиуса $R$ со скоростью $v$. Найдите её ускорение.
\пункт Пусть $x(t) = 4t$ м, $y(t) = 2t^2$ м, $z(t) = \frac{2t^3}{3}$ м. Найти расстояние, которое пролетит мушка за минуту.
\пункт Найти длину произвольного куска параболы $y=x^2$.
\пункт Определите длину произвольной кривой $\gamma \colon [a,b] \to \R^n$, $t \to (x_1(t),\ldots,x_n(t))$, где функции $x_i$ дифференцируемы. Проверьте, что для отрезка получается обычная длина.
\кзадача


\задача
Найти площадь фигуры, ограниченной кривыми $ax = y^2$, $ay=x^2$.
\кзадача

\задача Пусть пара дифференцируемых функций $(x(t),y(t))$, $0\leqslant t \leqslant T$ задаёт замкнутую
несамопересекающуюся кривую, причём для любого $x_0$ существуют не более 100
чисел $t_i$, таких что $x(t_i) = x_0$. Кривая ограничивает область площади $S$.
Доказать, что
\vspace{-3mm}
$$S = \left| \int\limits_0^T y(t) x'(t) dt \right|$$
\кзадача

\задача
Окружность радиуса $R$ катится по прямой с угловой скоростью $\omega$. На окружности зафиксировали точку. Кривая, по которой движется эта точка, называется \выд{циклоидой}. Задайте кривую параметрически (то есть в виде $(x(t),y(t))$) и найдите площадь одной арки циклоиды.
\кзадача

\задача Найти массу проволоки длиной 100 м, если известно что плотность проволоки на расстоянии $x$ м от конца равна $\rho(x)$ кг/м.
\кзадача

\задача Пусть на прямой установлено несколько точечных весов с массами $m_i$ и координатами $x_i$. Найти центр масс этой системы.
\кзадача

\задача Найти центр масс стержня длины 10 м, если его плотность изменяется по закону $\rho(x) = 6 + 0,3x$ (кг/м), где $x$ --- расстояние до одного из его
концов.
\кзадача

\задача Суточные расходы при плавании судна состоят из двух частей: постоянной, равной $a$ р., и переменной, возрастающей пропорционально кубу скорости с
коэффициентом пропорциональность $\alpha$. При какой скорости $v$ плавание судна будет наиболее экономичным, то есть затраты на один километр пути будут
минимальными?
\кзадача


\опр Пусть для каждой пары $(x,p) \in \Omega \subset \R^2$ определено число $f(x,p)$. Тогда говорят, что на множестве $\Omega$ задана функция двух
переменных $x$ и $p$.
\копр

\задача Определите непрерывность в точке для функций двух переменных.
\кзадача


\ЛичныйКондуит{0mm}{6.5mm}
\ОбнулитьКондуит
\newpage





\опр Пусть на множестве $\left\{ (x,p)\in \R^2 \mid x\in[a,b],\, p \in [\varphi(x), \psi(x)] \right\}$ задана  непрерывная ограниченная функция $f(x,p)$
. Тогда можно определить интеграл с параметром:
$$F(x) := \int\limits_{\varphi(x)}^{\psi(x)} f(x,p)\, dp$$
\vspace*{-3mm}
\копр



\задача
Найти массу квадратной пластины размера $1\times 1$, если её плотность на расстоянии $x$ и $y$ от соседних сторон равна $x^2y + y^2x + x^3\cos y$.
\кзадача

\задача
Найти объём тела, ограниченного поверхностями $\frac{x^2}{a^2} + \frac{y^2}{b^2} = 1$, $z = c^2x$, $z = 0$.
\кзадача


\задача
Доказать, что объём тела, образованного вращением вокруг оси $Oy$ плоской фигуры, заданной условием $0 \leqslant a \leqslant x \leqslant b,\, 0\leqslant y \leqslant y(x)$, где $y(x)$
--- непрерывная функция, равен $V = 2 \pi \int\limits_a^b x y(x) \, dx$.
\кзадача

\задача
\пункт Найти объём шара радиуса $R$.
\сНовойСтроки
\пункт Определить центр масс однородного полушария радиуса $R$.
\пункт Найти площадь сферы радиуса $R$.
\спункт Найти объём четырёхмерного шара радиуса $R$ (фигуры, заданной уравнением $x_1^2 + x_2^2 + x_3^2 + x_4^2 \leqslant R^2$).
\спункт Найти объём пятимерного шара радиуса $R$.
\спункт Найти объём шестимерного шара радиуса $R$.
\кзадача


\сзадача
С какой силой материальная бесконечная прямая постоянной плотности $\mu_0$ притягивает материальную точку массы $m$, находящуюся на расстоянии  $a$ от
этой прямой?
\кзадача

\сзадача
Найти кинетическую энергию цилиндра высоты $h$ радиуса $R$ постоянной плотности $\rho$, вращающегося вокруг своей оси с угловой скоростью $\omega$.
\кзадача

\опр Функция $(\ln |f(x)|)' = \frac{f'(x)}{f(x)}$ называется \выд{логарифмической производной} функции $f$.
\копр

\задача
Найти все решения дифференциального уравнения $f'(x) = f(x)$.
\кзадача

\сзадача
Скорость распада радия в каждый момент времени пропорциональна его наличному количеству. В начальный момент был 1 кг радия. Найти с точностью до 50 лет
время, за которое распадётся 0,999 кг радия, если известно, что через 1600 лет его количество уменьшится в два раза.
\кзадача

\сзадача
Для остановки речных судов у пристани с них бросают канат, который наматывают на столб,
стоящий на пристани. Какая сила будет тормозить судно, если канат делает три витка вокруг столба,
коэффициент трения каната о столб равен $\frac13$, и рабочий на пристани тянет за свободный конец
каната с силой $10\cdot g$ Н? ($g$ --- ускорение свободного падения) Скорость верёвки считать постоянной.\\
({\sl Указание:}
Сила трения $F_{тр} = \mu \cdot N$, $N$ можно найти для куска каната радианной меры
$\Delta \varphi$, а силу можно выразить как функцию радианной меры
угла $\varphi$.)
\кзадача





\ссзадача
В ванну площади $1$ м$^2$ со скоростью $0,25$ л/с течёт вода. В стенке ванной сделано
сливное отверстие радиуса $2,3$ см. Расстояние от края борта до середины отверстия равно $10$ см.
Пренебрегая различием уровня воды внизу и вверху отверстия найти,
через какое время зальёт соседей, если вначале вода уже у середины отверстия?\\
({\sl Напоминание:}
Согласно закону Торричелли скорость истечения жидкости из сосуда равна $v = c\sqrt{2gh}$, где $g$ --- ускорение свободного падения, $h$ --- высота уровня
жидкости над отверстием,
$c=0,6$ --- опытный коэффициент.)
\кзадача


\ЛичныйКондуит{0mm}{6.5mm}
% \GenXMLW

\end{document}



