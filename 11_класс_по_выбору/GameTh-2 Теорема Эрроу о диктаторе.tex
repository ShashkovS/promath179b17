% !TeX encoding = windows-1251
\documentclass[a4paper,12pt]{article}
\usepackage{newlistok}

\УвеличитьВысоту{2.1cm}
\УвеличитьШирину{1.1cm}


\Заголовок{Теорема Эрроу о диктаторе}
\НомерЛистка{GT-2}
\ДатаЛистка{01.2017}


\begin{document}

\СоздатьЗаголовок


Пусть перед обществом из $n$ человек стоит $l$ альтернатив, и каждый
из участников упорядочивает их в соответствии со своими
предпочтениями, а по результатам опроса определяется коллективное
решение. Это означает, что набору $P = \{<_1, <_2, \ldots, <_n\}$
мнений\footnote{\лк меньше\пк\ = \лк хуже\пк}~--- строгих линейных
порядков на множестве альтернатив $A$~--- сопоставляется нестрогий
линейный порядок $\leq_P$ на $A$ (альтернативам разрешается быть
равнозначными с точки зрения общества).

\опр Набор $<_1, <_2, \ldots, <_n$ называется {\em профилем}, а
отображение из множества профилей в множество нестрогих линейных
порядков на $A$~--- {\em функционалом общественного выбора}. \копр

\опр Функционал общественного выбора мы будем называть {\em
справедливым}, если он обладает следующими свойствами:
\\
{\em эффективность по Парето}: если в профиле $P$ для некоторых
альтернатив $x$ и $y$ $x<_k y$ для любого участника $k$, то $x\leq_P
y$ и не $y\leq_P x$;
\\
{\em независимость от посторонних альтернатив}: если в профиле $P$
каждый участник сравнил альтернативы $x$ и $y$ так же, как в профиле
$P'$, то $x\leq_P y \Longleftrightarrow x\leq_{P'} y$. \копр


\задача \пункт При каких $l$ правило простого большинства (общество
считает, что $x\leq y$ если и только если хотя бы для половины
участников $x < y$) задает функционал общественного выбора? \пункт
Является ли этот функционал эффективным? Независимым от посторонних
альтернатив? \кзадача

\задача Пусть в данном профиле альтернатива заняла $i$-е по счету
место в предпочтениях $j$-го участника. Присвоим ей рейтинг~---
сумму мест в предпочтениях всех участников, и упорядочим
альтернативы по возрастанию рейтинга (такой способ формирования
коллективного решения называется {\em правилом Борда}).
\сНовойСтроки \пункт Задает ли правило Борда функционал
общественного выбора? \пункт Является ли правило Борда эффективным?
Независимым от посторонних альтернатив? \кзадача

\задача \пункт Придумайте свой способ формирования коллективного
решения. При каких $l$ он является справедливым? \пункт Зачем в
определении справедливого функционала требовать независимость от
посторонних альтернатив? \кзадача

\задача Определите, что такое диктаторский функционал. Является ли
он эффективным? Независимым от посторонних альтернатив? \кзадача

\noindent {\bf Теорема Эрроу о диктаторе.} Пусть число альтернатив
больше двух. Тогда справедливый функционал общественного выбора
является диктаторским.

\smallskip

Всюду ниже будем считать, что задан функционал общественного выбора,
удовлетворяющий условиям теоремы Эрроу. Будем обозначать $P_{xy} =
\{k: x <_k y\}$.

\задача[лемма о нейтральности] Пусть для некоторых альтернатив
$x,y,z \ P_{xy} = P_{xz}$, и $x \leq_P y$. Тогда $x \leq_P z$.
\кзадача

\задача[лемма об экстремальной альтернативе] Назовем альтернативу
$a$ {\em экстремальной} в профиле $P$, если для каждого участника
$a$ либо лучшая, либо худшая. Докажите, что в коллективном мнении
экстремальная альтернатива займет либо первое, либо последнее место.
\кзадача

\задача[доказательство теоремы Эрроу] Рассмотрим серию профилей,
отличающихся друг от друга только мнением участников о некоторой
альтернативе $a$: в первом профиле все считают $a$ лучшей
альтернативой, во втором профиле участник 1 считает $a$ худшей, а
остальные~--- лучшей, в третьем профиле участники 1 и 2 считают $a$
худшей, а остальные~--- лучшей, и т.д. В последнем, $(n + 1)$-м
профиле все будут считать $a$ худшей. \сНовойСтроки \пункт Исходя из
значений функционала на этих профилях, угадайте, кто диктатор.
\пункт Докажите, что он действительно диктатор.


\end{document}