% !TeX encoding = windows-1251
\documentclass[a4paper,12pt]{article}
\usepackage[mag=1000]{newlistok}
\graphicspath{{pdfpict/}}

\ВключитьКолонитул
\УвеличитьВысоту{12mm}
\УвеличитьШирину{8mm}


\Заголовок{Полярное преобразование}
\НомерЛистка{GM-5}
\ДатаЛистка{01.2017}

\expandafter\def\expandafter\normalsize\expandafter{%
    \normalsize
    \setlength\abovedisplayskip{2pt}
    \setlength\belowdisplayskip{2pt}
    \setlength\abovedisplayshortskip{2pt}
    \setlength\belowdisplayshortskip{1pt}
}
\sloppy

\usepackage{ifthen}
\usepackage[noadjust]{marginnote}
\newcommand{\rightpicture}[4]%
{\ifthenelse{\lengthtest{10mm>#3mm}}%
{\marginnote{\hbox to #1 {\hss\includegraphics[scale=#3]{#4}}}[-#2]}%
{\marginnote{\hbox to #1 {\hss\includegraphics[width=#3]{#4}}}[-#2]}}
\newcommand{\leftpicture}[4]%
{\ifthenelse{\lengthtest{10mm>#3mm}}%
{\reversemarginpar\marginnote{\hbox to -#1 {\includegraphics[scale=#3]{#4}\hss}}[-#2]\normalmarginpar}%
{\reversemarginpar\marginnote{\hbox to -#1 {\includegraphics[width=#3]{#4}\hss}}[-#2]\normalmarginpar}}


\begin{document}
\СоздатьЗаголовок



\опр Пусть на плоскости $\pi$ задана окружность $\Gamma$ радиуса $r$
с центром $O$. Для любой точки $A\ne O$ \выд{полярой} точки $A$
называется прямая, перпендикулярная прямой $OA$, расстояние от
которой до точки $O$ равно $r^2/OA$. Полярой точки $O$ называется
бесконечно удалённая прямая проективной плоскости $\bar\pi$. Полярой
бесконечно удалённой точки $A\in\bar\pi$ называется прямая,
проходящая через точку $O$ и перпендикулярная ко всем (конечным)
прямым, проходящим через~$A$. \копр

\задача Докажите, что каждая прямая на проективной плоскости
является полярой ровно одной точки. (Эта точка называется
\выд{полюсом} данной прямой.) \кзадача

\опр Отображение проективной плоскости в себя, переводящее каждую
точку в её поляру и каждую прямую в её полюс, называется
\выд{полярным преобразованием}. \копр

\опр Будем говорить, что точка $P$ и прямая $l$ проективной
плоскости \выд{инцидентны}, если точка $P$ лежит на прямой $l$.
\копр

\задача Докажите, что полярное преобразование плоскости сохраняет
\выд{инцидентность} (то есть, если точка и прямая инцидентны, то их
образы при полярном преобразовании также инцидентны). \кзадача


\УстановитьГраницы{0cm}{55mm}
\rightpicture{0mm}{0mm}{55mm}{proj_geom-6}
\задача[Задача о поляре] Пусть
$A\in(\bar\pi\setminus\Gamma)$. Проведём через точку $A$ любые две
прямые, одна из которых пресекает окружность $\Gamma$ в точках $K$ и
$L$, а другая --- в точках $M$ и $N$. Обозначим через $B$ точку
пересечения прямых $KN$ и $LM$. Докажите, что точка $B$ лежит на
поляре точки $A$, причем все точки этой поляры можно получить таким
построением. \кзадача

\задача Докажите, что полярное преобразование сохраняет двойные
отношения. \кзадача

\задача[Проективная двойственность] Пусть верная теорема проективной
геометрии сформулирована в терминах точек, прямых, инцидентности и
двойных отношений. Докажите, что если в формулировке заменить точки
на прямые, а прямые на точки, то снова получится верная теорема
проективной геометрии. \кзадача
\ВосстановитьГраницы

\задача[Проективная теорема Менелая]\label{Menelaus} В какое
утверждение переводит проективная двойственность проективную теорему
Чевы? \кзадача

\задача[Теорема Менелая] На сторонах $AB$, $AC$ и $BC$ треугольника
$ABC$ (или на продолжениях этих сторон) выбрали соответственно точки
$C'$, $B'$ и $A'$. Выведите из задачи \ref{Menelaus}, что точки
$A'$, $B'$ и $C'$ коллинеарны тогда и только тогда, когда
$$
\frac{BA'}{CA'}\cdot\frac{AC'}{BC'}\cdot\frac{CB'}{AB'}=1.
$$
(Отношения отрезков берутся со знаком!) \кзадача

\задача В какое утверждение переводит проективная двойственность
теорему Паппа? \кзадача

\задача В какое утверждение переводит проективная двойственность
теорему Дезарга? \кзадача


\vfill
\ЛичныйКондуит{0mm}{6mm}
\ОбнулитьКондуит
\newpage

\раздел{Конические сечения}

\опр \выд{Коническим сечением} или \выд{коникой} называется образ
окружности под действием проективного преобразования. \копр

\сзадача Докажите, что эллипс, гипербола и парабола являются
коническими сечениями. (Указание: рассмотрите сечения кругового
конуса различными плоскостями). \кзадача

\задача Докажите, что любая коника~--- либо эллипс, либо парабола,
либо гипербола. \кзадача

\задача Пусть $A,B,C,D$~--- точки на конике $\Gamma$, причём
$\{A,B\}\cap\{C,D\} = \emptyset$. Докажите, что для любой другой
точки $E\in\Gamma$ двойное отношение $[EA,EB,EC,ED]$ одно и то же.
\кзадача

\задача[Задача о бабочке] Через середину $C$ произвольной хорды $AB$
окружности проведены две хорды $KL$ и $MN$ (точки $K$ и $M$ лежат по
одну сторону от $AB$). Отрезок $KN$ пересекает $AB$ в точке $P$.
Отрезок $LM$ пересекает $AB$ в точке $Q$. Докажите, что $PC = QC$.
\кзадача

\задача[Теорема Паскаля] Докажите, что точки пересечения
противоположных сторон шестиугольника, вписанного в коническое
сечение, коллинеарны. \кзадача

\задача[Теорема Брианшона] Докажите, что три диагонали, соединяющие
противоположные вершины шестиугольника, описанного около конического
сечения, конкурентны. \кзадача

\задача Докажите, что точка пересечения диагоналей описанного
четырёхугольника совпадает с точкой пересечения прямых, соединяющих
точки касания противоположных сторон. \кзадача


\раздел{Разное}

\задача С помощью одной линейки проведите касательную к окружности
(центр которой не отмечен) через данную точку, \сНовойСтроки \пункт
лежащую вне окружности; \пункт лежащую на окружности. \кзадача

\задача Где лежат середины любого семейства параллельных хорд данной
параболы? \кзадача

\задача Дан график \вСтрочку \пункт $y = x^2$; \пункт $y = 1/x$; оси
координат стёрты. Как  восстановить их циркулем и линейкой? \кзадача

\задача В окружность с центром $O$ вписан четырехугольник, его
диагонали пересекаются в точке $P$, а продолжения противоположных
сторон~--- в точках $Q$ и $R$. Докажите, что высоты треугольника
$PQR$ пересекаются в точке $O$. \кзадача

\задача Четырехугольник $ABCD$ описан около окружности с центром в
точке $O$. Прямые $AB$ и $CD$ пересекаются в точке $P$, а прямые
$AD$ и $BC$~--- в точке $Q$, причем отрезки $BP$ и $DQ$ пересекаются
в точке $A$; $M$~--- основание перпендикуляра, опущенного из точки
$O$ на $PQ$. Докажите, что \сНовойСтроки \пункт точка пересечения
диагоналей четырехугольника $ABCD$ лежит на прямой $OM$; \спункт
углы $\angle BMO$ и $\angle DMO$ равны. \кзадача

\vfill
\ЛичныйКондуит{0mm}{6mm}

\end{document}
