% !TeX encoding = windows-1251
\documentclass[a4paper,12pt]{article}
\usepackage[mag=980]{newlistok}
\graphicspath{{pdfpict/}}


\УвеличитьВысоту{2.5cm}
\УвеличитьШирину{1.8cm}


\Заголовок{Аффинная геометрия}
\НомерЛистка{GM-1}
\ДатаЛистка{01.2017}



\usepackage{ifthen}
\usepackage[noadjust]{marginnote}
\newcommand{\rightpicture}[4]%
{\ifthenelse{\lengthtest{10mm>#3mm}}%
{\marginnote{\hbox to #1 {\hss\includegraphics[scale=#3]{#4}}}[-#2]}%
{\marginnote{\hbox to #1 {\hss\includegraphics[width=#3]{#4}}}[-#2]}}

\begin{document}
\СоздатьЗаголовок


\раздел{Аффинные преобразования плоскости}

\rightpicture{0mm}{3mm}{45mm}{affine_geom-1}
\УстановитьГраницы{0cm}{4.5cm} \опр Пусть в пространстве заданы две
плоскости $\pi$ и $\pi'$, параллельные или непараллельные между
собой.

Пусть $l$~--- прямая, не параллельная ни $\pi$, ни $\pi'$.
\выд{Параллельной проекцией $\pi$ на $\pi'$ вдоль $l$} называется
отображение, сопоставляющее каждой точке $P\in\pi$ такую точку
$P'\in\pi'$, что прямая $PP'$ параллельна прямой $l$ (см.~рис.
справа).

Любое отображение плоскости $\pi$ на плоскость $\pi'$, которое можно
представить в виде композиции параллельных проекций, называется
\выд{аффинным}. Аффинное отображение плоскости $\pi$ на себя
называется \выд{аффинным преобразованием}. \ВосстановитьГраницы
\копр

\задача Докажите, что следующие преобразования являются аффинными:
\вСтрочку \пункт параллельный перенос; \пункт осевая симметрия;
\пункт поворот; \пункт любое движение. \кзадача

\задача Зададим на плоскости прямоугольную систему координат.
Докажите, что следующие отображения являются аффинными
преобразованиями: \вСтрочку \пункт $(x,y)\mapsto(a x,y)$, где
$a\ne0$; \пункт гомотетия с центром в начале координат; \пункт любое
преобразование подобия; \пункт $(x,y)\mapsto(x + by, y)$, где
$b$~--- любое число; \пункт $(x,y)\mapsto(ax + by + \alpha, cx + dy
+ \beta)$, где $ad - bc \ne 0$. \кзадача

\задача Докажите, что аффинные преобразования \вСтрочку \пункт
переводят прямые в прямые; \пункт переводят отрезки в отрезки;
\пункт переводят параллельные прямые в параллельные прямые; \пункт
сохраняют отношения длин отрезков, лежащих на параллельных прямых;
\пункт переводят параллелограммы в параллелограммы; \пункт сохраняют
отношения площадей. \кзадача

\задача Пусть $ABC$ и $A'B'C'$~--- два произвольных треугольника.
Докажите, что существует ровно одно аффинное преобразование,
переводящее треугольник $ABC$ в треугольник $A'B'C'$ с сохранением
порядка вершин. \кзадача


\vspace{-4mm}
\раздел{Применения аффинных преобразований}

\задача Используйте аффинные преобразования для доказательства того,
что три медианы любого треугольника пересекаются в одной точке.
\кзадача

\задача Используйте аффинные преобразования для доказательства \лк
замечательного свойства трапеции\пк: в любой трапеции точка
пересечения диагоналей, точка пересечения продолжений боковых сторон
и середины оснований лежат на одной прямой. \кзадача

\задача На плоскости даны две параллельные прямые $l$ и $l'$.
\сНовойСтроки \пункт Отрезок $AB$ прямой $l$ разделите пополам при
помощи одной линейки. \пункт Через данную точку $M$ проведите при
помощи одной линейки прямую, параллельную прямым~$l$~и~$l'$.
\кзадача

\задача Пусть $M$, $N$ и $P$~--- точки, расположенные на сторонах
$AB$, $BC$ и $CA$ треугольника $ABC$ и делящие эти стороны в
одинаковых отношениях $\biggl($то есть $\displaystyle\frac{AM}{MB} =
\frac{BN}{NC} = \frac{CP}{PA}\biggr)$. Докажите, что \сНовойСтроки
\пункт точка пересечения медиан треугольника $MNP$ совпадает с
точкой пересечения медиан треугольника $ABC$; \пункт точка
пресечения медиан треугольника, образованного прямыми $AN$, $BP$ и
$CM$, совпадает с точкой пересечения медиан треугольника $ABC$.
\кзадача

\задача Пусть у четырехугольника $ABCD$ никакие две стороны не
параллельны. Докажите, что прямая, соединяющая середины его
диагоналей, делит пополам отрезок, соединяющий точки пересечений
продолжений противоположных сторон. \кзадача

\задача Докажите, что с помощью только односторонней линейки без
делений и карандаша нельзя опустить перпендикуляр на данную прямую.
\кзадача

\задача Выпуклый пятиугольник $P$ гомотетичен пятиугольнику,
построенному на серединах его сторон. Обязательно ли тогда $P$~---
правильный? \кзадача

\задача На сторонах $AB$, $BC$ и $CA$ треугольника $ABC$ выбрали
соответственно точки $K$, $L$ и $M$ так, что $AK : KB = BL : LC = CM
: MA = 1 : \sqrt3$. Прямые $AL$, $BM$ и $CK$ пересекаются в точках
$A'$, $B'$ и $C'$, образуя новый треугольник, на сторонах которого
аналогичным образом выбирают точки $K'$, $L'$, $M'$ и получают
треугольник $A''B''C''$, и так далее. Докажите, что на каком-то шаге
мы получим треугольник, подобный исходному. \кзадача

\ЛичныйКондуит{0mm}{6mm}

\end{document}
