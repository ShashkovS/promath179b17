% !TeX encoding = windows-1251
\documentclass[a4paper,11pt]{article}
\usepackage{newlistok}

\УвеличитьВысоту{2.5cm}
\УвеличитьШирину{1.8cm}

\Заголовок{Математические модели некоторых систем голосования}
\НомерЛистка{GT-1}
\ДатаЛистка{01.2017}


\begin{document}

\СоздатьЗаголовок


Пусть перед обществом из $n$ человек стоит задача выбора одной
альтернативы из данных $l$ возможностей. Пусть каждый участник
упорядочивает эти альтернативы по возрастанию предпочтительности для
него, то есть его мнение задаётся некоторой \выд{перестановкой} этих
альтернатив. (Для простоты считаем, что никакие две альтернативы для
участника не безразличны). Упорядоченный набор из $n$ таких
перестановок назовём \выд{профилем индивидуальных предпочтений}.
Множество всех возможных профилей обозначим $\Pi$.

Систему голосования можно представить в виде чёрного ящика, который,
получая на вход профиль индивидуальных предпочтений, выдаёт
некоторую альтернативу (она будет считаться \лк наилучшей для
общества\пк). Математически это можно описать как отображение
(функцию) из множества  $\Pi$ (или некоторого его подмножества, если
чёрный ящик \лк срабатывает\пк не всегда) во множество альтернатив
$A$.

В простом случае, когда имеются ровно $2$ альтернативы, часто
применяется \выд{правило абсолютного  большинства}: выбирается
альтернатива, предпочтительная для более чем половины участников.
Достоинства: очевидны. Недостатки: Во-первых, не даёт результата при
равенстве голосов. Во-вторых, не учитывает \лк силы предпочтения\пк.
Например, если для участника $p_1$ крайне важно выбрать альтернативу
$a_1$, а для $p_2$ и $p_3$ альтернатива $a_2$ лишь незначительно
предпочтительнее, чем $a_1$, то иногда можно было бы и согласиться с
$p_1$.

\задача Большинством голосов признано, что утверждение $B$ истинно,
и что из $B$ следует $C$. Можно ли быть уверенным, что по мнению
большинства истинно также и $C$ ? \кзадача

Рассмотрим случай $l > 2$. Можно взять любую пару альтернатив и
сравнить их по правилу большинства (\лк дуэль\пк). Для простоты
будем полагать, что \лк ничьих\пк не бывает.

\задача \label{duel} Рассмотрим следующую \лк дуэльную\пк процедуру
при $l > 2$. На голосование ставится произвольная пара кандидатов,
проигравший выбывает. Затем выигравший ставится на голосование с ещё
не рассмотренным кандидатом. И так далее, пока не останется
единственный кандидат, который и объявляется победителем. Верно ли,
что результат этой процедуры не может зависеть от порядка выбора пар
для \лк дуэлей\пк, если каждый участник голосует в соответствии со
своими неизменными предпочтениями? \кзадача

\задача \пункт Всегда ли отношение \лк $a_1$ предпочтительнее для
большинства, чем $a_2$\пк транзитивно? \пункт Пусть $l = 3$. Верно
ли, что указанное отношение транзитивно тогда и только тогда, когда
результат описанной в задаче \ref{duel} процедуры не зависит от
порядка постановки на голосование? А если $l > 3$? \кзадача

\опр Альтернатива, которая выигрывает подобную дуэль с любой другой
альтернативой, называется \выд{победителем по Кондорсе}. \копр

\задача \label{part} Съезд партии выбирает председателя из
кандидатур $A$, $B$ и $C$, которые имеют $44$, $30$ и $26$
сторонников соответственно. Если в первом туре никто не наберёт
больше $50\%$, то будет второй тур. В нём не будет участвовать
кандидат, набравший минимальное количество голосов. Известно, что
при непопадании своего кандидата во второй тур сторонники $C$ будут
голосовать за $B$, а голоса сторонников $B$ разделятся поровну между
$A$ и $C$. Кто выиграет выборы? Достаточно ли данных для
определённого ответа? \кзадача

\опр Если участник голосует в соответствии со своими истинными
предпочтениями, такое голосование назовём \выд{искренним}. Если же
он голосует вопреки своим предпочтениям (но с тем, чтобы в конечном
итоге достичь лучших с его точки зрения результатов), такое
голосование назовём \выд{тактическим}. \копр

\задача \label{part2} Дополнительно известно, что если бы $A$ не
прошёл во второй тур, то его голоса перешли бы к $С$. Пусть выборы
проводятся по новой системе: каждый избиратель заполняет бюллетень
лишь $1$ раз, при этом ранжируя все кандидатуры в порядке
предпочтения, а затем бюллетени обрабатываются по дуэльной
процедуре. Какой из кандидатов может победить, если первая дуэль
проводится между $A$ и $B$? \кзадача

\задача Решается вопрос постройки нового супермаркета в деревне, где
все дома довольно плотно расположены вдоль длинной улицы. На этой
улице имеется $l$ подходящих мест. Место выбирается голосованием,
при этом каждый хочет, чтобы магазин был поближе к его дому.
(Примечание: вместо длинной улицы можно рассмотреть одномерный
политический спектр \лк правые-левые\пк). \пункт Если сравнить
каждую пару мест путём дуэльного голосования, обязательно ли
полученное отношение предпочтения будет транзитивным? \пункт
Укладываются ли в модель \лк линейного политического спектра\пк
условия задач \ref{part} и \ref{part2}? \кзадача

\задача Пусть $l = 3$ и при данном профиле индивидуальных
предпочтений существует победитель по Кондорсе. Может ли он
проиграть при дуэльной процедуре? \кзадача

\сзадача Предложите систему голосования (более) устойчивую к
тактическому голосованию. \кзадача

\ЛичныйКондуит{0mm}{6mm}

\end{document}
