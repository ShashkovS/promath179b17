% !TeX encoding = windows-1251
\documentclass[a4paper,11pt]{article}
\usepackage{newlistok}

\def\hang{\hangindent\parindent}
\def\binom#1#2{\left({#1\atop #2}\right)}

\УвеличитьВысоту{2.5cm}
\УвеличитьШирину{1.5cm}
%\renewcommand{\spacer}{\vfil}

\НомерЛистка{AG-1}
\Заголовок{Аксиомы поля}
\ДатаЛистка{01.2017}

\begin{document}
\СоздатьЗаголовок

\опр
\выд{Полем} называется любое множество $\Bbbk$, на котором заданы операции
\emph{сложения} ($+$) и \emph{умножения} ($\cdot$),
удовлетворяющие следующим условиям (аксиомам поля):
\begin{itemize}
\item[(A1)]
Для любых $a,b\in\Bbbk$ выполнено равенство $a+b=b+a$
(\emph{коммутативность сложения}).
\item[(A2)]
Для любых $a,b,c\in\Bbbk$ выполнено равенство $(a+b)+c=a+(b+c)$
(\emph{ассоциативность сложения}).
\item[(A3)]
В $\Bbbk$ существует такой элемент $0$, что для любого $a\in\Bbbk$
выполнено равенство
$a+0=a$ (\emph{существование нуля}).
\item[(A4)]
Для любого $a\in\Bbbk$ существует такой $b\in\Bbbk$, что $a+b=0$
(\emph{существование противоположного элемента}: такой элемент $b$
называется \emph{противоположным} к $a$ и обозначается $-a$).
\item[(M1)]
Для любых $a,b\in\Bbbk$ выполнено равенство $a\cdot b=b\cdot a$
(\emph{коммутативность умножения}).
\item[(M2)]
Для любых $a,b,c\in\Bbbk$ выполнено равенство $(a\cdot b)\cdot
c=a\cdot (b\cdot c)$ (\emph{ассоциативность умножения}).
\item[(M3)]
В $\Bbbk$ существует такой элемент $1$, не равный нулю, что для
любого $a\in\Bbbk$ выполнено равенство $a\cdot 1=a$
(\emph{существование единицы}).
\item[(M4)]
Для любого $a\in\Bbbk$, не равного нулю, существует такой
$b\in\Bbbk$, что $a\cdot b=1$ (\emph{существование обратного
элемента}: такой элемент $b$ называется \emph{обратным} к $a$ и
обозначается $\dfrac1a$ или~$a^{-1}$).
\item[(AM)]
Для любых $a,b,c\in\Bbbk$ выполнено равенство $a\cdot(b+c)=a\cdot
b + a\cdot c$ (\emph{дистрибутивность умножения относительно
сложения}).
\end{itemize}
\копр


\задача
Пусть $\Bbbk$ --- поле. Докажите, что
\сНовойСтроки
%\вСтрочку
\пункт
в $\Bbbk$ есть только один ноль;
\пункт
у каждого элемента только один противоположный;
\пункт
для любого $a\in\Bbbk$ выполнено равенство
$-(-a)=a$;
\пункт
для любых $a,b\in\Bbbk$ уравнение $a+x=b$ имеет ровно одно решение
в $\Bbbk$ (оно обозначается $b-a$; таким образом, в поле
определена операция \emph{вычитания}).
\кзадача


\задача
Пусть $\Bbbk$ --- поле. Докажите, что
\сНовойСтроки
%\вСтрочку
\пункт
в $\Bbbk$ есть только одна единица;
\пункт
у каждого ненулевого элемента только один обратный;
\пункт
для любого ненулевого $a\in\Bbbk$ выполнено равенство
$(a^{-1})^{-1}=a$;
\пункт
для любого $b\in\Bbbk$ и любого ненулевого $a\in\Bbbk$ уравнение
$a\cdot x=b$ имеет ровно одно решение в $\Bbbk$ (оно обозначается
$\dfrac b a$; таким образом, в поле определена операция
\emph{деления} на ненулевые элементы).
\кзадача

\задача
Пусть $\Bbbk$ --- поле. Докажите, что
\сНовойСтроки
\пункт
для любого $a\in\Bbbk$ выполнено равенство $a\cdot 0=0$;
\пункт
если $a\cdot b=0$, то $a=0$ или $b=0$.
\спункт
Останется ли верным утверждение пункта б), если исключить из аксиом поля
аксиому М4?
\кзадача

\задача
Пусть $\Bbbk$ --- поле. Докажите, что для любого $a\in\Bbbk$
выполнены равенства
\сНовойСтроки
%\вСтрочку
\пункт
$a\cdot (-1)=-a$;
\пункт
$(-a)\cdot (-a)=a\cdot a$;
\пункт
$(-a)^{-1}=-(a^{-1})$, если $a\ne0$.
\кзадача

\задача
Пусть $\Bbbk$ --- поле. Докажите, что
%\сНовойСтроки
%\пункт
для любых $a,c\in\Bbbk$ и любых ненулевых $b,d\in\Bbbk$ выполнено
равенство
\вСтрочку
\пункт
$\dfrac a b\cdot\dfrac c d=\dfrac{a\cdot c}{b\cdot d}$;
\пункт
%для любых $a,c\in\Bbbk$ и любых ненулевых $b,d\in\Bbbk$ выполнено
%равенство
$\dfrac a b+\dfrac c d=\dfrac{a\cdot d+b\cdot c}{b\cdot d}$.
\кзадача

\задача
Какие из следующих числовых множеств являются полями:
%множество
$\mathbb N$,
% натуральных чисел, множество
$\mathbb Z$,
% целых чисел, множество
$\mathbb Q$?
%рациональных чисел?
\кзадача
\задача
Является ли полем множество чисел вида $a+b\sqrt2$, где $a,b\in\Q$
(сложение и умножение обычные)?
%(с обычными операциями сложения и умножения чисел)?
\кзадача


\задача
%Является ли полем множество %рациональных
Пусть $\R(x)$ --- множество %дробей
$\displaystyle{\left\{\frac{P(x)}{Q(x)}\
\Bigl|\ P(x),Q(x)\in\R[x],\ Q(x)\ {\rm
не\ равен\ нулевому\ многочлену}\right\}}$.\\
Является ли $\R(x)$ полем (с обычным сложением и умножением)?
% является полем.
\кзадача



\задача Существует ли поле из
\вСтрочку
\пункт
одного элемента;
\пункт
двух элементов;
\пункт
трёх элементов?
\кзадача

\сзадача Постройте поле
%\пункт
%\вСтрочку
из $p$ элементов,
где $p$ --- произвольное простое число. %;
%\пункт из четыр\"ех элементов.
\кзадача

\сзадача Постройте поле из четырёх элементов.
\кзадача

%%\СделатьКондуитИз{6.2mm}{6.2mm}{sp_AlG.tex}
\ЛичныйКондуит{0mm}{6mm}

\GenXMLW
\end{document}

\раздел{Аксиомы упорядоченного поля}

\опр
Множество $E$ называется \emph{линейно упорядоченным}, если на нём
задано отношение \лк меньше или равно\пк{} (то есть известно, для
каких $a,b\in E$ выполнено неравенство $a\le b$), причем выполнены
следующие \emph{аксиомы порядка}:
\begin{itemize}
\item[(O1)]
Для любого $a\in E$ выполнено неравенство $a\le a$.
\item[(O2)]
Если  $a,b\in E$, $a\le b$ и $b\le a$, то $a=b$.
\item[(O3)]
Если  $a,b,c\in E$, $a\le b$ и $b\le c$, то $a\le c$.
\item[(O4)]
Для любых $a,b\in E$ выполнено неравенство $a\le b$ или выполнено
неравенство $b\le a$.
\end{itemize}

Вместо $a\le b$ пишут также $b\ge a$, а запись $a<b$ (или, что то же
самое, $b>a$) означает, что $a\le b$ и $a\ne b$.
\копр

\опр
Поле $\Bbbk$ называется \emph{упорядоченным полем}, если множество
$\Bbbk$ линейно упорядочено, причём выполнены следующие аксиомы:
\begin{itemize}
\item[(AO)]
Если  $a,b,c\in\Bbbk$ и $a\le b$, то  $a+c\le b+c$.
\item[(MO)]
Если  $a,b,c\in\Bbbk$, $0\le c$ и $a\le b$, то  $a\cdot c\le
b\cdot c$.
\end{itemize}
\копр

\задача
Пусть $\Bbbk$ --- упорядоченное поле. Докажите, что
\сНовойСтроки
\пункт
если $a\le b$ и $c\le d$, то $a+c\le b+d$;
\пункт
если $0\le a\le b$ и $0\le c\le d$, то $a\cdot c\le b\cdot d$;
\пункт
$1>0$.
\кзадача

\задача
\пункт
Пусть $\Bbbk$ --- упорядоченное поле, а $P$ --- множество его
\emph{положительных} элементов, то есть $P=\{a\in\Bbbk\mid a>0\}$.
Докажите, что выполнены следующие свойства:
\begin{itemize}
\item[(P1)]
Для любого $a\in\Bbbk$ верно ровно одно из следующих
утверждений:\quad $a\in P$;\quad $a=0$;\quad $-a\in P$.
\item[(P2)]
Если  $a,b\in P$, то $a+b\in P$ и $a\cdot b\in P$.
\end{itemize}
\спункт
Пусть $\Bbbk$ ---  поле, $P\subset\Bbbk$ ---
подмножество, удовлетворяющее условиям (P1) и (P2). Докажите, что
поле $\Bbbk$ можно сделать упорядоченным так, что $P$ будет
множеством положительных элементов, причём отношение порядка $\le$
однозначно определяется множеством~$P$.
\кзадача

\задача
Докажите, что в любом упорядоченном поле бесконечно много
элементов.
\кзадача

\задача\label{caninj}
Пусть $\Bbbk$ --- упорядоченное поле. Поставим в соответствие
каждому натуральному числу $n$ следующий элемент поля $\Bbbk$:
$\overline{n} =\underbrace{1+\dots+1}_{n}$
(сумма $n$ экземпляров единицы поля $\Bbbk$).
\сНовойСтроки
\пункт
Докажите, что для любых $m$ и $n$ имеют место равенства
$\overline{m}+\overline{n}=\overline{m+n}$ и
$\overline{m}\cdot\overline{n}=\overline{mn}$.
\пункт
Докажите, что если $m\le n$, то $\overline{m}\le\overline{n}$.
\пункт
Докажите, что если $m\ne n$, то $\overline{m}\ne\overline{n}$.
\пункт
Докажите, что каждому \emph{целому} числу $n$ можно поставить в
соответствие элемент $\overline{n}\in\Bbbk$ так, что все пункты
а)--в) останутся верными.
\пункт
Докажите, что то же самое можно сделать для всех
\emph{рациональных} чисел.
\кзадача

\medskip

\noindent
{\bf Замечание.} В дальнейшем мы будем писать $n$ вместо
$\overline{n}$, считая, тем самым, что все на\-ту\-ра\-ль\-ные (целые,
рациональные) числа являются элементами любого
упорядоченного поля~$\Bbbk$.




\end{document} 