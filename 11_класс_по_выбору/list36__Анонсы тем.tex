% !TeX encoding = windows-1251
\documentclass[a4paper,12pt]{article}
\usepackage{newlistok}

\УвеличитьШирину{.5cm}
\УвеличитьВысоту{.5cm}
%\renewcommand{\spacer}{\vfil}
%\sloppy

\def\exec{\makebox[25pt]{\put(-5,3){\line(0,1){3}}\put(-5,3){\vector(1,0){11}}}}


\begin{document}

%\newpage
%\ОбнулитьДанные
\Заголовок{Математический анализ: приложения}
\НомерЛистка{36:1}
\ДатаЛистка{01.2017}
\СоздатьЗаголовок

\раздел{Программа}
\begin{nums}{-2}
\item Вычисление скоростей, длин, площадей, объёмов;
\item Решение дифференциальных уравнений для решения физических задач
\end{nums}

\раздел{Примеры задач}

\задача Пусть пара непрерывно дифференцируемых функций $(x(t),y(t))$, $0\leqslant t \leqslant T$ задаёт замкнутую
несамопересекающуюся кривую. Кривая ограничивает область площади $S$.
Доказать, что
\vspace*{-3mm}
$$
S = \left| \int\limits_0^T y(t) x'(t) dt \right|.
$$
\vspace*{-7mm}
\кзадача

\задача
Доказать, что объём тела, образованного вращением вокруг оси $Oy$ плоской фигуры $0 \leqslant a \leqslant x \leqslant b,\, 0\leqslant y \leqslant y(x)$, где $y(x)$
--- непрерывная функция, равен $V = 2 \pi \int\limits_a^b x y(x) \, dx$.
\кзадача



\задача
\пункт Найдите объём шара радиуса $R$.
\пункт Определите центр масс однородного полушария радиуса $R$.
\пункт Найдите площадь сферы радиуса $R$.
\спункт Найдите объём четырёхмерного шара радиуса~$R$.
\сспункт Найдите объём $n$-мерного шара радиуса~$R$.
Докажите, что при достаточно большом $n$ почти весь объём шара заключён в тонком слое на границе.
\кзадача



\сзадача
С какой силой материальная бесконечная прямая постоянной плотности $\mu_0$ притягивает материальную точку массы $m$, находящуюся на расстоянии  $a$ от
этой прямой?
\кзадача

\сзадача
Найти кинетическую энергию цилиндра высоты $h$ радиуса $R$ постоянной плотности $\rho$, вращающегося вокруг своей оси с угловой скоростью $\omega$.
\кзадача

\сзадача
Для остановки речных судов у пристани с них бросают канат, который наматывают на столб,
стоящий на пристани. Какая сила будет тормозить судно, если канат делает три витка вокруг столба,
коэффициент трения каната о столб равен $\frac13$, и рабочий на пристани тянет за свободный конец
каната с силой $10\cdot g$ Н? ($g$ --- ускорение свободного падения) Скорость верёвки считать постоянной.\\
({\sl Указание:}
Сила трения $F_{тр} = \mu \cdot N$, $N$ можно найти для куска каната радианной меры
$\Delta \varphi$, а силу можно выразить как функцию радианной меры
угла $\varphi$.)
\кзадача



{\bf Литература}: Фихтенгольц Г.М. "Основы математического анализа в 2-х томах"





\newpage
\ОбнулитьДанные
\Заголовок{Математический анализ --- Ряды}
\НомерЛистка{36:2}
\ДатаЛистка{01.2017}
\СоздатьЗаголовок

\раздел{Программа}
\begin{nums}{-2}
\item Числовые ряды, их сходимость.
\item Ряды Тейлора, функциональные ряды.
\item Формальные степенные ряды.
\item Производящие функции.
\end{nums}

\раздел{Примеры задач}
\задача
Рассмотрим ряд $\sum\limits_{i=1}^{\infty} x_i$. Сумма $S_n=\sum\limits_{i=1}^{n} x_i$ называется $n$-ой частичной суммой ряда. Ряд называется сходящимся к $S$, если существует $\lim\limits{n\to\infty} S_n = S$.
\кзадача
\задача
Докажите, что ряд $\sum\limits_{i=1}^{\infty} \dfrac{1}{i}$ расходится.\\
\кзадача
\задача
Докажите, что ряд $\sum\limits_{i=1}^{\infty}\dfrac{1}{i^2}$ сходится. (Его сумма равна $\dfrac{\pi^2}{6}$).
\кзадача
\задача
Докажите, что в каждой точке $x$ сходится ряд $\sum\limits_{i=0}^{\infty} \dfrac{x^i}{i!} = e^x$;
\кзадача
\задача
Найдите такой ряд $F$, что $(1-t)\cdot F = 1$.
\кзадача
\задача
При каких условиях на степенной ряд $F$ разрешимо уравнение $X^2 = F$ относительно неизвестного степенного ряда $X$?
\кзадача
\задача
Сколькими способами можно заплатить 1 рубль копейками, алтынами (трехкопеечными монетами) и пятаками (пятикопеечными монетами)?
\кзадача
\задача
Найдите все решения дифференциального уравнения $F'(s)=F(s)$.
\кзадача

{\bf Литература}: С.К. Ландо "Введение в дискретную математику"













\newpage
\ОбнулитьДанные
\Заголовок{Элементы теории вероятностей}
\НомерЛистка{36:3}
\ДатаЛистка{01.2017}
\СоздатьЗаголовок

\раздел{Программа}
\begin{nums}{-2}
\item События и их вероятности;
\item  Опыт с непрерывным пространством элементарных событий;
\item Строгое определение вероятности. Аксиоматика Колмогорова;
\item Условные вероятности. Формула полной вероятности. Формула Байеса;
\item Независимость. Схема Бернулли
\end{nums}

\раздел{Примеры задач}
\задача
  Тест состоит из 10 вопросов, по 4 варианта ответа на каждый, причём только один из них правильный. Если к каждому вопросу подбирать случайный ответ, то какова вероятность ответить верно \пункт на все 10 вопросов; \пункт ровно на 5 вопросов; \пункт не менее, чем на 5 вопросов?
\кзадача

\задача
  В теннисном турнире участвуют 32 спортсмена, причём силы всех спортсменов постоянны, а более сильный всегда выигрывает у более слабого. Найдите вероятность того, что в финале встретятся два самых сильных спортсмена, если:
  \невСтрочку
    \пункт Перед началом турнира создаётся сетка и спортсмены случайным образом распределяются по ней;
    \пункт Перед началом каждого тура спортсмены случайным образом разбиваются на пары, победители которых проходят в следующий тур.
\кзадача

\задача
  Метровую линейку случайным образом разрезают ножницами. Найдите вероятность того, что длина обрезка составит на менее 80 см.
\кзадача

\задача[Теорема умножения вероятностей]
  Пусть $A_1, A_2,\ldots, A_n$ --- события, вероятность которых больше 0. Докажите, что $\Pf(A_1A_2\ldots A_n)=\Pf(A_1)\cdot\Pf(A_2|A_1)\cdot\Pf(A_3|A_1A_2)\cdot\ldots\cdot\Pf(A_n|A_1\ldots A_{n-1})$.
\кзадача

\задача[Формула полной вероятности]
  Пусть $H_1,H_2,\ldots,H_n$ --- попарно несовместимые события, причём $H_1\cup H_2\cup\ldots\cup H_n=\Om$. Докажите, что $\fa B\in\Us\exec\Pf(B)=\sum\limits_{i=1}^n\Pf(H_i)\cdot\Pf(B|H_i)$.
\кзадача


\задача[Формула Байеса]
  Пусть \hfill $H_1,H_2,\ldots,H_n$ \hfill --- \hfill попарно \hfill несовместимые \hfill события, \hfill причём\\$H_1\cup H_2\cup\ldots\cup H_n=\Om$. Предположим, стало известно, что событие $A$ произошло. Докажите, что тогда \equ{\Pf(H_i|A)=\frac{\Pf(H_i)\cdot\Pf(A|H_i)}{\sum\limits_{k=1}^n\Pf(A)\cdot\Pf(A|H_k)}.}
\кзадача

\задача[Задача о разорении]
Игрок, имеющий $n$ монет, играет против казино, имеющего неограниченное количество монет. За одну игру игрок либо проигрывает монету, либо выигрывает с вероятностью $1/2$. Он играет, пока не разорится. Найдите вероятность разориться ровно за $m$ игр.
\кзадача


\задача
Радиоактивная бактерия делится на две таких же бактерии с вероятностью 0.6, а с вероятностью 0.4 погибает. Вначале в лаборатории живет одна радиоактивная бактерия. С какой вероятностью
популяция радиоактивных бактерий вымрет через некоторое количество поколений?
\кзадача


\задача
$n$ претендентов на должность в случайном порядке приходят на собеседование. Если в
результате собеседования выясняется, что новый претендент лучше того, кто в данный момент занимает
должность, первого нанимают, а последнего — увольняют. а) С какой вероятностью $k$-й по силе
претендент будет нанят в какой-либо момент. б) Найдите матожидание числа увольнений.
\кзадача


% \раздел{Литература}
















\newpage
\ОбнулитьДанные
\Заголовок{Алгебраическая теория чисел}
\НомерЛистка{36:4}
\ДатаЛистка{01.2017}
\СоздатьЗаголовок

\раздел{Программа}
\begin{nums}{-2}
\item Цепные дроби: цепные дроби, алгоритм Евклида и решения диофантовых уравнений;
 наилучшие приближения вещественных чисел при помощи цепных дробей; квадратичные
 иррациональности и периодические цепные дроби;  алгебраические числа и их приближения.
\item Геометрические построения и  поля алгебраических чисел: построения одним циркулем, теорема Маскерони; задачи о трисекции угла и удвоении куба и квадратичные расширения полей.
гауссовы числа; круговые поля.
\item Алгебраические уравнения  и теория Галуа: решение уравнений 3 и 4 степени, группа Галуа
 уравнения; проблема разрешимости уравнения в радикалах;
 уравнения деления круга и построения Гаусса правильных многоугольников.
\end{nums}

\раздел{Примеры задач}

\задача
\пункт Сформулируйте и докажите утверждение: всякое вещественное число может быть
единственным образом записано в виде цепной дроби
$$x=a_0+\dfrac{1}{a_1+\dfrac{1}{a_2+\dfrac{1}{a_3+\ldots}}},$$
где все $a_k$ - целые, и $a_k>0$ при $k>0$.
\\
\пункт при этом конечным цепным дробям соответствуют рациональные числа, а периодическим -  квадратичные иррациональности, т.е., решения уравнений  $$b_2x^2+b_1x+b_0=0, \qquad b_i\in\Z$$
\кзадача

\задача
 Убедитесь, что дробь
 $$1+\dfrac{1}{1+\dfrac{1}{1+\dfrac{1}{1+\ldots}}}$$
 представляет золотое сечение.
\кзадача

\задача
 Число Архимеда $\dfrac{22}{7}$ является первым нетривиальным приближением числа
 $\pi$ при помощи цепных дробей. Оно является наилучшим приближением $\pi$  с точностью до второго
 знака после запятой. Постройте следующее за архимедовым приближение $\pi$ цепными дробями.
\кзадача

\задача  Постройте с помощью циркуля и линейки\\
\пункт правильный 5- угольник\\
\пункт правильный 17- угольник (это в свое время проделал Гаусс).
\кзадача

\раздел{Литература}

В.И.Арнольд, Цепные дроби

М.М.Постников, Теория Галуа















\newpage
\ОбнулитьДанные
\Заголовок{Специальная теория относительности}
\НомерЛистка{36:5}
\ДатаЛистка{01.2017}
\СоздатьЗаголовок

\раздел{Введение}

Большинство парадоксальных и противоречащих интуитивным представлениям о мире эффектов, возникающих при движении со скоростью, близкой к скорости света, предсказывается именно специальной теорией относительности. Самый известный из них — эффект замедления хода часов, или эффект замедления времени. Часы, движущиеся относительно наблюдателя, идут для него медленнее, чем точно такие же часы у него в руках.

В специальной теории относительности координаты приписываются не частицам, а элементарным событиям, то есть включают в себя время.

\раздел{Приблизительная программа}
\begin{nums}{-3}
\item Линейная алгебра: группа движений плоскости и пространства. Запись преобразований матрицами. Композиция преобразований и произведение матриц.
\item Принцип относительности Галилея. Принцип относительности Эйнштейна. Предпосылки  для специальной теории относительности. Описания и результаты опытов.
\item Пространство-время. Преобразования пространства-времени, группа Преобразований Лоренца.
\item Сокращение размеров: парадокс шеста и сарая.
\item Углы в пространстве-времени. Сложение скоростей.
\item Разные задачи и парадоксы.
\end{nums}



\раздел{Примеры задач}
\задача
Известно, что скорость света постоянна во всех системах отсчёта. При какой скорости будет наблюдаться (относительно "лаборатории") сокращение объёма вдвое?
\кзадача

\задача[Парадокс поезда]
Пусть на поезде, движущемся со скоростью, близкой к скорости света (такой поезд, видимо, стоит ожидать раньше всего в Японии (если где-нибудь ещё не научатся значительно влиять на скорость света)), едут три человека: $A$ в голове, $O$ --- в середине и  $B$ --- в хвосте поезда. На земле около пути стоит четвёртый человек $O'$. В тот самый момент, когда $O$ проезжает мимо $O'$, сигналы ламп от $A$ и $B$ достигают $O$ и $O'$.
Покажите, что на вопрос "Кто раньше включил фонарь" наблюдатели $O$ и $O'$ дадут различные ответы. Объясните этот парадокс качественно.
\кзадача

\задача
\пункт
Покажите, что если два события происходят одновременно и в одном и том же месте в одной системе отсчёта, то они будут одновременными в любой другой системе отсчёта.
\пункт
Покажите, что если два события происходят одновременно в разных точках в одной системе отсчёта, то они не будут одновременными ни в какой другой системе отсчёта.
\кзадача

\задача
Как синхронизировать часы на спутнике и на Земле? (требуются точные расчёты)
\кзадача

\задача
Известно, что скорость света в среде (например, в воздухе) меньше скорости света в вакууме. Оказывается, что "светящиеся" частицы, двигающиеся в среде быстрее скорости света, создают световой "конус Маха". Вычислите угол этого конуса для электронов, летящих в воздухе, если скорость электронов составляет $\dfrac{999\,999}{1\,000\,000}$ скорости света в вакууме, а скорость света в воздухе --- "всего" $\dfrac{999\,710}{1\,000\,000}$.
\кзадача

\раздел{Предостережение}
Большинство задач специальной теории относительности очень сложны: требуют длительных аккуратных расчётов, противоречат нашей привычной логике.

Для того, чтобы разобраться, придётся изучить несколько дополнительных тем. Описание действительно интересных эффектов не дастся без труда.

Если нет уверенности в своих силах и любопытстве, то не стоит даже начинать!















\newpage
\ОбнулитьДанные
\Заголовок{Алгебраическая геометрия}
\НомерЛистка{36:6}
\ДатаЛистка{01.2017}
\СоздатьЗаголовок

\раздел{Программа}
\begin{nums}{-2}
\item Многочлены от двух переменных.

\item Теорема Безу.

\item Приложение к геометрии.

\item Разные задачи алгебраической геометрии


\end{nums}

\раздел{Примеры задач}

\задача Докажите, что любой многочлен из $\R[x,y]$
однозначно (с точностью до множителей из~$\R$)
раскладывается в произведение неприводимых над  $\R$
\hbox{многочленов.}
\кзадача

\задача  Пусть $A$, $B$ --- различные многочлены из $\R[x,y]$.
Может ли система $A(x,y)=0$, $B(x,y)=0$ иметь
конечное число решений,  бесконечное число решений?
\кзадача

\задача  Еще Исаак Ньютон заметил следующий интересный факт, называемый
\выд{теоремой Безу:}\/
\выд{если $A(x,y)$ и $B(x,y)$ --- ненулевые взаимно простые многочлены,
то система
$A(x,y)=0$, $B(x,y)=0$ имеет не более $\deg A\cdot \deg B$ решений.}\/
Докажите
теорему Безу  для произвольного ненулевого многочлена $A$,
взаимно простого с многочленом $B$.
\кзадача

\задача
Пусть никакие три из точек $A$, $B$, $C$, $D$, $E$ на плоскости не
лежат на одной прямой. Докажите, что
через эти точки %существует
проходит ровно одна коника.
\кзадача

\задача  [Теорема Паскаля] Пусть вершины шестиугольника
$ABCDEF$
лежат на кривой, задающейся неприводимым многочленом второй степени.
Докажите, что тогда
точки пересечения прямых $AB$ и $DE$, $BC$ и $EF$, $CD$ и $FA$
лежат на одной прямой
(смотрите рисунок справа).
\кзадача

\раздел{Литература}

Н.Б.Васильев. Гексаграммы Паскаля и кубические кривые.

"Квант" \No 8-1987, с.2;

И.Р.Шафаревич. Основы алгебраической геометрии. М., Наука, 1988, Т.1.




















\newpage
\ОбнулитьДанные
\Заголовок{Комплексный анализ}
\НомерЛистка{36:7}
\ДатаЛистка{01.2017}
\СоздатьЗаголовок

\раздел{Программа}
\begin{nums}{-2}
\item Дифференцирование комплексных функций;
\item Интегрирование комплексных функций по кривой;
\item Формула Ньютона – Лейбница;
\item Лемма Гурса и её следствия;
\item Бесконечная дифференцируемость функций, имеющих производную;
\end{nums}

\раздел{Примеры задач}


\задача[формула Ньютона\ч Лейбница]
Пусть $f$ непрерывна в области $D$, и существует пер\-во\-об\-раз\-ная $F$ к $f$ в области $D$;
$\ga=\hc{z(t) \vl t \in [a,b]}$\т кусочно\д гладкая кривая в $D$, соединяющая точки $A$ и $B$.
Тогда
\eqn{\ints{\ga}f(z)\,dz=F(B)-F(A).}
\кзадача


\задача
  Если интеграл от непрерывной функции $f$ по любому замкнутому контуру в об\-лас\-ти~$D$ равен нулю, то
$f$ обладает первообразной.
\кзадача  

\задача[лемма Гурса]
Пусть $\tri$\т треугольник, лежащий в области $D$ вместе с внутренностью, и $f$ голоморфна в
окрестности $\tri$. Тогда
\eqn{\ints{\tri}f(z)\,dz=0.}
\кзадача



\задача[интегральная теорема Коши]
Если $D$\т односвязная область, и $f(z)$ голоморфна в $D$, а $\ga$\т кусочно\д гладкий контур в $D$, то
\eqn{\ints{\ga}f(z)\,dz=0.}
\кзадача  




\задача[Теорема Лиувилля]
Если $f(z)$ голоморфна в $\Cbb$ и ограничена, тогда $f(z) \equiv \const$.
\кзадача  


\задача[Основная теорема алгебры]
Любой многочлен $P(z) \in \Cbb[z]$ положительной степени имеет в $\Cbb$ хотя бы один корень.
\кзадача  


\задача
Докажите, что функция комплексно дифференцируема в области $D$ тогда и только тогда, когда
для $\fa a \in D$ найдётся $r > 0$ такое, что в круге $\De(a,r)$ функция $f(z)$ разлагается в ряд
\eqn{f(z) = \sumnzi c_n(z-a)^n.} 
\кзадача  



\раздел{Литература}

Б.В.Шабат. "Введение в комлексный анализ."

А.С.Мищенко, А.Т.Фоменко. "Курс дифференциальной геометрии и топологии"

Э.Б.Винберг. "Курс алгебры"



















\end{document}
