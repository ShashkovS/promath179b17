% !TeX encoding = windows-1251
\documentclass[a4paper,12pt]{article}
\usepackage[mag=980]{newlistok}
\graphicspath{{pdfpict/}}


\УвеличитьВысоту{2.5cm}
\УвеличитьШирину{1.8cm}


\Заголовок{Проективная геометрия}
\НомерЛистка{GM-3}
\ДатаЛистка{01.2017}



\usepackage{ifthen}
\usepackage[noadjust]{marginnote}
\newcommand{\rightpicture}[4]%
{\ifthenelse{\lengthtest{10mm>#3mm}}%
{\marginnote{\hbox to #1 {\hss\includegraphics[scale=#3]{#4}}}[-#2]}%
{\marginnote{\hbox to #1 {\hss\includegraphics[width=#3]{#4}}}[-#2]}}
\newcommand{\leftpicture}[4]%
{\ifthenelse{\lengthtest{10mm>#3mm}}%
{\reversemarginpar\marginnote{\hbox to -#1 {\includegraphics[scale=#3]{#4}\hss}}[-#2]\normalmarginpar}%
{\reversemarginpar\marginnote{\hbox to -#1 {\includegraphics[width=#3]{#4}\hss}}[-#2]\normalmarginpar}}


\begin{document}
\СоздатьЗаголовок



\rightpicture{0mm}{3mm}{43mm}{proj_geom-1}
\УстановитьГраницы{0mm}{33mm}
\опр Пусть в пространстве заданы две плоскости $\pi$ и $\pi'$,
параллельные или непараллельные между собой.

\leftpicture{0mm}{0mm}{43mm}{proj_geom-2}
\УстановитьГраницы{43mm}{33mm}
Пусть $O$ --- точка,
не лежащая ни на $\pi$, ни на $\pi'$. \выд{Центральной проекцией
$\pi$ на $\pi'$ с центром $O$} называется отображение,
сопоставляющее каждой точке $P\in\pi$ точку $P'\in\pi'$ пересечения
прямой $OP$ с плоскостью $\pi'$ (см.~рис.~справа).

\УстановитьГраницы{43mm}{43mm}
Пусть $l$~--- прямая, не параллельная ни $\pi$, ни $\pi'$.
\выд{Параллельной проекцией $\pi$ на $\pi'$ вдоль $l$} называется
отображение, сопоставляющее каждой точке $P\in\pi$ такую точку
$P'\in\pi'$, что прямая $PP'$ параллельна прямой $l$
(см.~рис.слева).
%\hbox{\hspace{0.45truecm}\copy3\hspace{-9.1truecm}\copy5}
\копр

\ВосстановитьГраницы

\задача Опишите область определения и область значений центральной
проекции; параллельной проекции. \кзадача


\опр Пусть $\pi$~--- плоскость. Добавим к каждой прямой на ней
\выд{\лк бесконечно удалённую\пк\ точку}, причём будем считать, что
\лк бесконечно удалённые\пк\ точки у параллельных прямых совпадают,
а у непараллельных~--- различны. Скажем также, что \лк бесконечно
удалённые\пк\ точки всех прямых составляют \выд{\лк бесконечно
удалённую\пк\ прямую}. То, что получилось, называется
\выд{проективной плоскостью} $\bar\pi$. \копр

\задача Докажите, что любые две различные прямые на проективной
плоскости имеют единственную общую точку, а через любые две
различные точки на проективной плоскости проходит единственная
прямая. \кзадача

\задача Докажите, что центральная проекция $\pi$ на $\pi'$ с центром
$O$ продолжается до взаимно однозначного отображения $\bar\pi$ на
$\bar\pi'$, переводящего прямые в прямые (оно называется
\выд{центральной проекцией $\bar\pi$ на $\bar\pi'$ с центром $O$}).
Аналогично для параллельной проекции. \кзадача

\опр Любое отображение $\bar\pi$ на себя, которое можно представить
в виде композиции центральных и параллельных проекций, называется
\выд{проективным преобразованием}. \копр

\задача Докажите, что с помощью проективного преобразования
$\bar\pi$ можно перевести любые две точки в \лк бесконечно
удалённые\пк. \кзадача

\задача Докажите, что с помощью проективного преобразования
$\bar\pi$ на $\bar\pi$ можно перевести любые три различные
\выд{коллинеарные} (лежащие на одной прямой) точки в любые другие
три различные коллинеарные точки. \кзадача

\задача Докажите, что отрезок нельзя разделить пополам с помощью
одной линейки. \кзадача

\задача Докажите, что с помощью проективного преобразования
$\bar\pi$ можно перевести любую четвёрку точек, никакие три из
которых не коллинеарны, в любую другую четвёрку точек с тем же
условием. \кзадача

\ссзадача Докажите, что любое взаимно однозначное преобразование
проективной плоскости в себя, переводящее прямые в прямые,
проективно. \кзадача

\rightpicture{-3mm}{7mm}{63mm}{proj_geom-3}
\УстановитьГраницы{0mm}{63mm}
\задача[Теорема Паппа] Пусть вершины
шестиугольника $ABCDEF$ лежат попеременно на двух прямых
(см.~рис.~справа). Докажите, что точки пересечения противоположных
сторон этого шестиугольника коллинеарны. \кзадача
\ВосстановитьГраницы

\УстановитьГраницы{0cm}{70mm}
\rightpicture{-3mm}{5mm}{68mm}{proj_geom-4}
\задача[Теорема Дезарга] Пусть заданы
два треугольника $ABC$ и $A'B'C'$, причём прямые $AA'$, $BB'$ и
$CC'$ \выд{конкурентны} (пересекаются в одной точке). Докажите, что
точки пересечения соответственных сторон треугольников $ABC$ и
$A'B'C'$ коллинеарны (см.~рис.~справа). \кзадача
\ВосстановитьГраницы

\задача Верна ли теорема, обратная теореме Дезарга? \кзадача



\ЛичныйКондуит{0mm}{6mm}

\end{document}
