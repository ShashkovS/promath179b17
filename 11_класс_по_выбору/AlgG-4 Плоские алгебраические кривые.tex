% !TeX encoding = windows-1251
\documentclass[a4paper,12pt]{article}
\usepackage{newlistok}

\УвеличитьВысоту{2.3cm}
\УвеличитьШирину{1cm}

\НомерЛистка{AG-4}
\ДатаЛистка{01.2017}
\Заголовок{Плоские алгебраические кривые}

\begin{document}
\СоздатьЗаголовок
\раздел{Часть 1. Многочлены от двух переменных}

\опр  \выд{Одночленом от двух переменных}\/ $x$ и $y$ (над $\R$)
называется выражение
вида $ax^my^n$, где $a\in\R$, $m,n\in\Z^+$.
Сумма нескольких одночленов такого~вида
(с приведенными подобными)
называется \выд{многочленом от двух переменных}\/ $x$ и $y$.
Сумма и произведение многочленов от двух переменных определяются аналогично
сумме и произведению многочленов от одной переменной.
Множество всех многочленов от $x$, $y$ (над $\R$) обозначают $\R[x,y]$.
\копр

\задача Дайте определение степени многочлена  $A\in\R[x,y]$
(обозначается $\deg A$).
\кзадача

\задача Пусть $A(x,y)$, $B(x,y)$ --- ненулевые многочлены.\\
\вСтрочку
\пункт Докажите, что $\deg AB=\deg A+\deg B$.
\пункт Что можно сказать о величине $\deg (A+B)$?
\кзадача

\задача Дайте определение
деления с остатком для многочленов от двух переменных.
Всегда ли такое деление возможно?
\кзадача

\задача Дайте определение неприводимого (над $\R$)
многочлена из $\R[x,y]$. %от двух переменных.
\кзадача

\задача Докажите неприводимость многочленов:
\вСтрочку
\пункт $x^2+y^2-1$; \пункт $y^2-x$; \пункт $xy-1$.
\кзадача

%\vspace*{-10pt}
\задача Докажите, что множество
$\displaystyle{\R(y)=\left\{\frac{P(y)}{Q(y)}\
\Bigl|\ P(y),Q(y)\in\R[y],\ Q(y)\ne0\right\}}$
(с обычными операциями сложения и умножения) является полем.
\кзадача

\задача Рассмотрим множество многочленов от $x$ над полем $\R(y)$,
т.~е.~множество $\R(y)[x]$.
Каждый его элемент записывается в виде
$$a_n(y)x^n+a_{n-1}(y)x^{n-1}+...+a_1(y)x+a_0(y),\eqno(*)$$
$\hbox{где}\ n\in\Z^+,\ a_i(y)\in\R(y)\ \hbox{при}\ i=\overline{0,n}.$
Дайте определение неприводимого (над $R(y)$)
многочлена из $\R(y)[x]$. Верна ли для многочленов из $\R(y)[x]$
теорема о единственности разложения на неприводимые сомножители?
\кзадача

\опр Запишем произвольный многочлен $A(x,y)\in\R[x,y]$ в виде $(*)$,
где уже\break $a_i(y)\in\R[y]$ при $i=\overline{0,n}$.
Скажем, что  $A(x,y)$ является \выд{примитивным (по $x$)},
если многочлены $a_n(y),\ \dots,\ a_0(y)$  взаимно просты.
\копр

\задача Докажите, что произведение двух примитивных (по $x$) многочленов
также является примитивным (по $x$)  многочленом.
\кзадача

\задача Докажите, что если многочлен из $\R[x,y]$ неприводим,
то он неприводим и как многочлен из $\R(y)[x]$. Верно ли обратное?
\кзадача

\задача Докажите, что любой многочлен из $\R[x,y]$
однозначно (с точностью до множителей из~$\R$)
раскладывается в произведение неприводимых над  $\R$
\hbox{многочленов.}
\кзадача

\ЛичныйКондуит{0mm}{6mm}
\end{document} 