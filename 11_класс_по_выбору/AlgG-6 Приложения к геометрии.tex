% !TeX encoding = windows-1251
\documentclass[a4paper,11pt]{article}
\usepackage[mag=950]{newlistok}
\graphicspath{{pdfpict/}}

\УвеличитьВысоту{2.5cm}
\УвеличитьШирину{1.8cm}

\НомерЛистка{AG-6}
\ДатаЛистка{01.2017}
\Заголовок{Плоские алгебраические кривые}

\begin{document}
\СоздатьЗаголовок

\vspace*{-2mm}
\раздел{Часть 3. Приложения к геометрии}
\vspace*{-2mm}




\задача [Замена координат] На плоскости $Oxy$ рассмотрим две
непараллельные  прямые, задающиеся  уравнениями
$l_1(x,y)=0$, $l_2(x,y)=0$, где $l_1$ и $l_2$ --- многочлены первой степени.
Примем их точку пересечения  за новое начало координат $O_1$,
а сами прямые за новые оси координат:  $l_1$ за ось $z$,
а $l_2$ за ось $t$.
Тогда прямые $l_1$ и $l_2$ будут иметь в новых координатах
простые уравнения $t=0$, $z=0$ соответственно.
\сНовойСтроки
\пункт  Докажите, что можно так выбрать базисные вектора на осях
$O_1z$, $O_1t$, что новые координаты $(z,t)$ точки будут
вычисляться через ее старые координаты $(x,y)$
по формулам $z=l_2(x,y)$, $t=l_1(x,y)$.
\пункт Докажите, что из уравнений $z=l_2(x,y)$, $t=l_1(x,y)$
можно выразить старые координаты $x$ и $y$ через новые $z$ и $t$.
\пункт Пусть $A(x,y)$ --- многочлен. Подставив в него
вместо $x$ и $y$ их выражения через $z$ и $t$, получим запись многочлена
$A$  в новых координатах $z$, $t$.
Докажите, что при этом степень многочлена не
изменится: $\deg A(x,y)=\deg A(z,t)$.
\кзадача

%\s
%{\rm Мораль:} степень многочлена не зависит от выбора системы координат.

\задача Докажите, что задаваемая уравнением %$x^2-4xy+4y^2-x-y+1=0$,
$z^2+2zt+t^2+\sqrt2z-\sqrt2t +1=0$ кривая имеет ось симметрии.
\кзадача


\опр Кривую, задающуюся многочленом второй степени, будем называть
\выд{коникой.}
\копр

\задача \label{abcd}
Пусть никакие три из точек $A$, $B$, $C$, $D$ плоскости $Oxy$ не лежат
на одной прямой. Пусть прямые $AB$, $BC$, $CD$, $DA$ задаются многочленами
первой степени
$l_1(x,y)$, \ $m_1(x,y)$, \ $l_2(x,y)$, \ $m_2(x,y)$ соответственно. Докажите, что  любую конику,
проходящую через  точки $A$, $B$, $C$, $D$, можно задать уравнением
вида
$\lambda l_1l_2+\mu m_1m_2=0$, где $\lambda,$ $\mu\in\R$.
% --- некоторые действительные числа.
\кзадача

\задача
\вСтрочку
\пункт Пусть никакие три из точек $A$, $B$, $C$, $D$, $E$ на плоскости не
лежат на одной прямой. Докажите, что
через эти точки %существует
проходит ровно одна коника.  \\ %, проходящая через эти точки.\\
\пункт Докажите, что в пункте а) достаточно потребовать, чтобы никакие четыре
из точек $A$, $B$, $C$, $D$, $E$ не лежали на одной прямой.
\кзадача

\задача В обозначениях задачи~\ref{abcd} выясните, является ли четырехугольник
$ABCD$ вписанным в окружность, если
\вСтрочку
\пункт $l_1=6x-y+1$, \ $m_1=3x+55y-388$, \ $l_2=x-9$, \ $m_2=x+9y-9$;
\пункт $l_1=4x-5y-35$, \ $m_1=7x+5y+35$, \ $l_2=83x-93y+415$, \ $m_2=x+y-11$.
\кзадача


{\УстановитьГраницы{7.2cm}{0cm}
\задача  Пусть три красные прямые пересекают
три синие прямые в девяти черных
точках
(на рисунке слева красные прямые изображены сплошными
линиями, а синие --- пунктирными).
Докажите, что если восемь из этих черных точек
лежат на некоторой \выд{кубической} кривой
(то есть на кривой, задающейся многочленом третьей степени),
то и оставшаяся девятая черная точка  лежит на той же кубической кривой.
\кзадача
}

\vspace*{-2.3cm}
\putpict{-4.8cm}{-1.5cm}{alg_curve_pask_papp_des-1}{}
\vspace*{17mm}

\опр  \выд{Шестиугольником}\/ будем называть всякую замкнутую шестизвенную
ломаную, никакие три из шести вершин которой не лежат на одной прямой.
\копр
\ВосстановитьГраницы

{\УстановитьГраницы{0cm}{6.3cm}
\задача  [Теорема Паскаля] Пусть вершины шестиугольника
$ABCDEF$
лежат на кривой, задающейся неприводимым многочленом второй степени.
Докажите, что тогда
точки пересечения прямых $AB$ и $DE$, $BC$ и $EF$, $CD$ и $FA$
лежат на одной прямой
(смотрите рисунок справа).
\кзадача
\ВосстановитьГраницы
}

\vspace*{-1.5cm}
\putpict{16cm}{-.4cm}{alg_curve_pask_papp_des-2}{}
\vspace*{10mm}

\задача  Сформулируйте и докажите теорему, обратную к теореме Паскаля.
\кзадача

\medskip
\medskip

{\УстановитьГраницы{8.1cm}{0cm}
\задача  [Теорема Паппа] Пусть точки $A$, $B$, $C$ и $A'$, $B'$, $C'$
лежат на прямых $l$ и $l'$ соответственно.
Докажите, что тогда точки пересечения прямых
$AB'$ и $A'B$, $BC'$ и $B'C$, $CA'$ и $C'A$ лежат на одной прямой
(см.~рис.~слева).
\кзадача
\ВосстановитьГраницы
}

\vspace*{-1.5cm}
\putpict{2cm}{-.3cm}{alg_curve_pask_papp_des-3}{}
\vspace*{15mm}


{\УстановитьГраницы{0cm}{8.2cm}
\задача [Теорема Дезарга] Пусть никакие три из точек $A$, $B$, $C$,
$A'$, $B'$, $C'$ не лежат на одной прямой.
Докажите, что прямые $AA'$, $BB'$
и $CC'$ пересекаются в одной точке,
тогда и только тогда, когда точки пересечения прямых $AB$ и $A'B'$,
$BC$ и $B'C'$, $CA$ и $C'A'$ лежат на одной прямой (см.~рис.~справа).
\кзадача
\ВосстановитьГраницы
}

\vspace*{-2.2cm}
\putpict{15cm}{-.4cm}{alg_curve_pask_papp_des-4}{}
\vspace*{17mm}


{\УстановитьГраницы{7cm}{0cm}
\опр Назовем два шестиугольника
\выд{сопряженными}, если один из
них образован
точками пересечения неглавных диагоналей
другого. (Например, изображенные на рисунке
слева черные шестиугольники сопряжены.)
Из определения следует, что для каждого шестиугольника имеется
ровно два с ним сопряженных.
\копр
\ВосстановитьГраницы
}

\vspace*{-2cm}
\putpict{2cm}{-.7cm}{alg_curve_pask_papp_des-5}{}
\vspace*{17mm}

\задача[С.А.Дориченко] Докажите, что  главные диагонали шестиугольника
пересекаются в одной точке тогда и только тогда, когда
точки пересечения противоположных сторон сопряженного
с ним шестиугольника лежат на одной прямой.
\кзадача

\ЛичныйКондуит{0mm}{6mm}
% \GenXMLW

\end{document}
