% !TeX encoding = windows-1251
\documentclass[a4paper,12pt]{article}
\usepackage{newlistok}

\УвеличитьВысоту{1.5cm}
\УвеличитьШирину{1.5cm}

\ВключитьКолонитул
\Заголовок{Теория вероятностей\т 1}
\Подзаголовок{Крайне наивная теория вероятностей :)}
\НомерЛистка{PT-1}
\ДатаЛистка{01.2017}

\begin{document}
\СоздатьЗаголовок

При анализе различных экспериментов или событий часто возникает желание предсказать результат.
Теория вероятностей занимается созданием и разбором моделей, призванных упростить или, по крайней мере, формализовать данную задачу. Разработка конкретной модели для решения конкретной задачи --- сугубо личное дело каждого.
В связи с этим, набор <<общепринятых>> терминов пестрит разнообразием и неоднозначностью.\\
Тем не менее, есть некоторое множество моделей, с которыми имеет смысл ознакомиться,
%и, при желании, приобрести,
а часть из них ещё и позволяет предсказать результат опытов, являющихся наиболее распространёнными.

%В повседневной жизни нам регулярно приходится сталкиваться с опытами, процессами или событиями, точное предсказание результатов которых является желанным, но невыполнимым. Такими событиями являются, например, какое число выпадет на игральном кубике, какая комбинация карт будет получена в следующем коне, какого качества утюг, выставленный на витрине.\\
%Теория вероятностей помогает получить приближённые ответы на подобные вопросы.

\опр
    \выдж Исходом события будем называть \выд любой из возможных результатов проводимого испытания.
\копр

%Каждый раз, рассматривая событие, необходимо отдельно объяснять, что называть исходом в данном случае и какие исходы вообще могут быть. Так, например, в случае утюга на витрине можно рассматривать исходы "утюг бракованный", "утюг качественный", "утюг первоклассный", ... В свою очередь, исход "утюг бракованный" можно рассматривать как самостоятельное событие со своими исходами --- "внешний дефект", "разрыв провода", \dots\\

В некоторых случаях, здравый смысл или жизненный опыт могут подсказать, что шансы на появление любого исхода из некоторого множества одинаковы, или, что то же самое, многократное повторение исходов приведёт к тому, что различные исходы встретятся примерно одинаковое количество раз. Так, при подбрасывании монеты мы говорим, что шансы на выпадение <<орла>> или <<решки>> равны, а если монету подбросить достаточное количество раз, то количество выпавших <<орлов>> будет примерно равно количеству выпавших <<решек>>.

\опр
  Говорят, что исходы \выдж равновероятны, если количества их появлений при многократном повторении опыта примерно равны.\\
  \выдж Вероятностью исхода называют отношение количества его появлений к общему количеству проведённых испытаний при достаточно большом общем количестве испытаний.
\копр

Какие исходы считать равновероятными и как велики <<многократное>> и <<достаточно большое>> количества повторений --- вопрос договоренности между людьми, обсуждающими данный опыт. %Например, считается общепринятым, что выпадения <<орла>> и <<решки>> при подбрасывании обычной монетки равновероятны. Как и равновероятны все $2^{1000}$ последовательности из тысячи <<орлов>> и <<решек>>, которые могут получиться при тысячекратном подбрасывании монеты, в том числе --- и 1000 <<орлов>>, хотя в то же время, мы скажем, что 1000 --- достаточно большое число, при котором количество <<орлов>> и <<решек>> будет примерно равно. А вот все возможные регистрационные номера проезжающей мимо машины --- скорее вопрос отдельной договорённости. Например, все номера можно считать равновероятными, а можно сказать, что номер региона <<02>> встречается в Москве несколько реже, чем номер <<97>>.

\опр
  Пусть в случае проведения опыта уже определились, какие его исходы мы рассматриваем. Устраивающие нас исходы мы будем называть \выдж благоприятными, все остальные --- \выдж неблагоприятными. Множество всех благоприятных исходов --- \выдж событие.\\
  Вероятность события равна сумме вероятностей всех благоприятных исходов.
  В случае, \выд{если все исходы равновероятны}, \выдж{вероятностью события} будет отношение числа благоприятных исходов к общему числу исходов.
\копр


\задача
  Игральный кубик бросают дважды. Найдите вероятности следующих событий: \пункт оба раза выпало одно и то же число; \пункт число, выпавшее во второй раз, оказалось больше первого; \пункт сумма чисел после двух бросков больше 5.
\кзадача

\задача
  В очередь в случайном порядке становятся Аня, Боря, Варя и Гена. Определите вероятности следующих событий: \пункт Аня стоит первой; \пункт Аня стоит рядом с Борей; \пункт Аня стоит раньше Бори и Вари; \пункт Аня стоит раньше Бори, а Варя --- после Гены.
\кзадача

\сзадача
  У Пети есть погнутая монета. 
  Каким образом можно оценить вероятность выпадения <<орла>> и <<решки>> на этой монете при её подбрасывании?
  Сколько раз нужно подбросить монету, чтобы при повторении опыта доля решек изменилась не более на 1\%
  с вероятностью 99\%?
\кзадача


\задача
  Из пруда, в котором плавает 50 щук, выловили 18, пометили и вернули обратно. На следующий день из пруда выловили 7 щук. Какова вероятность того, что более половины щук, выловленных во второй день, окажется помеченными?
\кзадача

\задача
  Набор домино состоит из 28 костей, на которых встречаются все возможные пары чисел от 0 до 6. Какова вероятность, что две случайно выбранные кости можно будет приложить друг к другу согласно правилам?
\кзадача

\vfill
\ЛичныйКондуит{0mm}{6mm}
\ОбнулитьКондуит
\newpage

\задача
  При игре в покер игроку раздаётся 5 карт. Ниже перечислены все возможные игровые комбинации в порядке убывания достоинства. Найдите вероятность появления некоторых из них на ваш вкус в случае колоды из 52 карт. Если комбинация подходит для двух --- она идёт в зачёт только более сильной.
  \невСтрочку
    \пункт Роял-флэш --- пять старших карт одной масти;
    \пункт Стрит-флэш --- пять последовательных карт одной масти. Туз может как начинать, так и заканчивать порядок, но не может быть в середине;
    \пункт Каре --- четыре карты одного достоинства;
    \пункт Фул-хаус --- три карты одного достоинства и две карты другого достоинства;
    \пункт Флэш --- пять карт одной масти;
    \пункт Стрит --- пять последовательных карт любых мастей. Туз может как начинать, так и заканчивать порядок, но не может быть в середине;
    \пункт Сет --- три карты одного достоинства;
    \пункт Две пары --- две карты одного достоинства и две карты другого достоинства;
    \пункт Пара --- две карты одного достоинства;
    \пункт Кикер --- ни одна из вышеперечисленных комбинаций.
\кзадача

\задача
  Петя достаёт 6 карт из колоды в 36 карт. Вася называет произвольную масть. Что больше --- вероятность того, что Вася назовёт масть какой-то карты из петиного набора или что карты такой масти в петином наборе нет? Во сколько раз больше?
\кзадача

\задача
  Тест состоит из 10 вопросов, по 4 варианта ответа на каждый, причём только один из них правильный. Если к каждому вопросу подбирать случайный ответ, то какова вероятность ответить верно \пункт на все 10 вопросов; \пункт ровно на 5 вопросов; \пункт не менее, чем на 5 вопросов?
\кзадача

\задача
  В урне находится 10 белых и 7 чёрных шаров. Наугад выбирается 8 шаров. Какова вероятность, что среди них окажется ровно 5 белых шаров, если после взятия из урны шар \пункт не возвращается назад; \пункт возвращается назад?
\кзадача

\задача
  В теннисном турнире участвуют 32 спортсмена, причём силы всех спортсменов постоянны, а более сильный всегда выигрывает у более слабого. Найдите вероятность того, что в финале встретятся два самых сильных спортсмена, если:
  \невСтрочку
    \пункт Перед началом турнира создаётся сетка и спортсмены случайным образом распределяются по ней;
    \пункт Перед началом каждого тура спортсмены случайным образом разбиваются на пары, победители которых проходят в следующий тур.
\кзадача

\задача
  Учитель составляет контрольную на два варианта, случайным образом выбирая для каждого из них 6 задач из данных 12. Найдите наиболее вероятное количество задач, встречающихся как в первом варианте, так и во втором.
\кзадача

\задача
  У Саши есть ящик с обувью, в котором лежат 3 одинаковых пары сапог. Саша наугад выбирает два сапога. Какова вероятность того, что он выберет себе пару?
\кзадача


% \GenXMLW





\vfill
\ЛичныйКондуит{0mm}{6mm}
\end{document}



