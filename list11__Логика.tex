% !TeX encoding = windows-1251
\documentclass[a4paper,11pt]{article}
\usepackage{newlistok}
%\documentstyle[11pt, russcorr, listok]{article}
\newcommand{\del}{\mathrel{\raisebox{-.3 ex}{${\vdots}$}}}

\УвеличитьШирину{1.3truecm}
\УвеличитьВысоту{3.1truecm}
\hoffset=-2.5truecm
\voffset=-27.2truemm

\Заголовок{Логика}
%и цепные дроби}
\НомерЛистка{10}
\ДатаЛистка{04.2014}

\renewcommand{\spacer}{\vfil}

\begin{document}

\СоздатьЗаголовок


\задача
\кзадача

Несколько слов про множества.
Множество – совокупность элементов, обладающих некоторым общим свойством.
Если элемент   принадлежит множеству  , то пишут  , если не принадлежит – то  .
Если все элементы множества   принадлежат множеству  , то говорят, что   – подмножество   и пишут  .
Множество, не содержащее ни одного элемента, называется пустым и обозначается  .
Множество всех элементов, принадлежащих  и  , называется пересечением   и  , и обозначается  .
Множество всех элементов, принадлежащих либо , либо , называется объединением   и  , и обозначается  .
Множество всех элементов, принадлежащих множеству  , но не принадлежащих множеству  , называется разностью множеств   и  , и обозначается  .

Задача 1. Часто множества изображают в виде кругов на плоскости (диаграммы Эйлера-Венна). Посмотрите на следующие диаграммы Эйлера-Венна и запишите, каким множествам соответствуют заштрихованные области. Для записи используйте символы  ,  ,   и скобки.

Задача 2. Каждый третий политик – бизнесмен, а каждый четвертый бизнесмен – политик. Кого больше, политиков или бизнесменов?

Далее будем рассматривать утверждения про элементы множества. Будем считать, что для конкретного элемента утверждение либо верно, либо нет. Например, утверждение «натуральное число m нечетно» верно для чисел 1, 3, 5, …
Заметим, что каждому утверждению   про элемент множества   соответствует некоторое подмножество  , для которого утверждение верно.

Задача 3. Перед футбольным матчем команд "Север" и "Юг" было дано пять прогнозов:  
1) ничьей не будет; 2) в ворота "Юга" забьют;  3) "Север" выиграет;  4) "Север" не проиграет; 5) в матче будет забито ровно 3 гола. После матча выяснилось, что верными оказались ровно три прогноза. С каким счётом закончился матч?
Задача 4. Число   натуральное. Среди утверждений 1)  , 2)  , 3)  , 4)  , 5)   три верных и два неверных. Чему равно  ?
Задача 5. Проиллюстрируйте на диаграммах Эйлера-Венна следующие ситуации:
А) Если для   верно утверждение , то верно и   (другими словами, для   из   следует  ).
Б) Для  из утверждения следуют   и  , но утверждение не следует из  , а   не следует из  .
Задача 6. Проиллюстрируйте на диаграммах Эйлера-Венна следующие ситуации:
А) Для того чтобы для   было верно утверждение  , достаточно, чтобы было верно утверждение  .
Б) Для того чтобы для   было верно утверждение  , необходимо, чтобы было верно утверждение  .
Задача 7. Расставьте вместо многоточий слова «необходимо», «достаточно», «необходимо, но не достаточно», «достаточно, но не необходимо», «необходимо и достаточно» так, чтобы получились верные суждения.
1)	Для того чтобы число   делилось на 5, …, чтобы его десятичная запись кончалась цифрой 0.
2)	Для того чтобы число   делилось на 9, …, чтобы сумма цифр его десятичной записи делилась на 3.
3)	Для того чтобы треугольник   был равнобедренным, …, чтобы углы при основании были равны.
4)	Для того чтобы параллелограмм   был ромбом, …, чтобы диагонали делили пополам внутренние углы.
5)	Для того чтобы параллелограмм   был квадратом, …, чтобы его стороны были равны.
Определение. Будем говорить, что утверждение   является отрицанием к утверждению  , если   верно для тех и только тех  , для которых не верно  .
Задача 8. Покажите, что если для   из   следует  , то из   следует  . 
Объясните, как действует принцип доказательства от противного.
Задача 9. Рассмотрим утверждения вида «  и  » (обозначается  ) и «  или  » (обозначается  ). Докажите следующие теоремы (правила де Моргана): А) Утверждение   равносильно утверждению  
Б) Утверждение   равносильно утверждению  .
Задача 10. Однажды принцесса сказала: «Хочу, чтобы мой муж был красивый, сильный, не был глупым или некрасивым, или чтобы был некрасивым, но не был сильным и глупым». Упростите данное утверждение.
Задача 11. Рассмотрим утверждения вида «для любого  верно  » (обозначается  ) и «существует   такой, что верно  » (обозначается  ).  Постройте отрицания к этим утверждениям. Можно ли сказать, что эти утверждения и отрицания к ним будут всегда истинны или ложны? А если предположить, что   или   могут зависеть от элемента  ?


Задача 12. Запишите символически (с помощью кванторов  ,   и обозначений для множеств) утверждения:
А) В каждом классе найдется ученик, который решил хотя бы одну задачу из контрольной.
Б) Найдется класс, в котором каждый ученик решил хотя бы одну задачу из контрольной.
В) Существует такая задача, что в каждом классе хотя бы один ученик ее решил.
Г) Для каждой задачи есть класс, в котором все ученики ее решили.
Задача 13. Постройте отрицания к следующим утверждениям:
А) В каждом классе каждый ученик изучает, по крайней мере, один из двух языков – английский или французский.
Б) В каждом городе есть магазин, в котором нет хлеба, и никто из продавцов не знает, когда он будет.
Задача 14*. Попытайтесь формализовать фразу «ученики должны показывать свои тетради учителям». Удается ли это сделать единственным способом? Какие тут возможны варианты?

\ЛичныйКондуит{0mm}{6mm}

%\СделатьКондуит{7mm}{7mm}

\end{document}

\кзадача 