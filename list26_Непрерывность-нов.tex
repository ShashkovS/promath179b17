% !TeX encoding = windows-1251
\documentclass[a4paper,12pt]{article}
\usepackage{newlistok}

\УвеличитьШирину{1truecm}
\УвеличитьВысоту{2.5truecm}
% \hoffset=-2.5truecm
% \voffset=-27.3truemm
%\def\hang{\hangindent\parindent}
%\documentstyle[11pt, russcorr, ll]{article}



\Заголовок{Непрерывность: основные определения и теоремы}
\Подзаголовок{}
\НомерЛистка{26}
\ДатаЛистка{12.2015}

%\overfullrule=3pt

\begin{document}

\СоздатьЗаголовок




\опр %\label{epsilon-delta}
{\small\sc (Непрерывность по Кош\'и.)}
%Функция $f:M\to\R$ , где $M \subseteq\R$,
Пусть $M \subseteq\R$. Функция $f:M\to\R$
называется \выд{непрерывной в точке $a\in M$}, если
если для любой окрестности ${\cal V}$ точки $f(a)$
найдется такая окрестность ${\cal W}$ точки $a$, что при всех $x$ из ${\cal W}\cap M$
число $f(x)$ лежит в $\cal V$. Обозначение: $f\in C(а)$.\\
Аналогично можно дать определение непрерывности {\it по Гейне} (сделайте это!).\\
Если $f$ непрерывна в каждой точке $a\in M$, то говорят, что $f$ \выд{непрерывна на $M$}, и пишут
\hbox{$f\in C(M)$.}
%Пусть функция $f$ определена на множестве $M\subseteq \R$,
%и пусть $a\in M$.
%и $a$ есть предельная точка $M$.
%Говорят, что $f$ \выд{непрерывна в точке $a$}, если
%(Неформально говоря, непрерывная функция переводит близкие точки в близкие.)
%$$
%\forall \varepsilon > 0\quad \exists \delta > 0:
%\qquad f\left(U_{\delta}(a)\cap M\right)\subset
%U_{\varepsilon}\left(f(a)\right).
%$$
\копр

\задача Запишите без отрицаний: \лк $f:M \to \R$
\выд{разрывна} (не является непрерывной) в точке $a\in M$\пк.
\кзадача

\задача В каких точках непрерывны %каждой из следующих
функции:
%\footnote{%В этой задаче, как обычно и делается в подобных
%%случаях, все
%Как обычно, функцию, заданную формулой, мы считаем определ\"енной
%всюду, где эта формула имеет смысл.}:
\вСтрочку
\пункт   $x$;
\пункт   $\mathop{\mbox{\rm sgn}}x$;
\пункт   $x^2$;
\пункт   $\{x\}$;
\пункт   $\frac 1x$;
\пункт   $\sqrt x$.\\
{\small (Как обычно, функцию, заданную формулой, мы считаем определ\"енной
всюду, где эта формула имеет смысл.)}
%\спункт $\lim\limits_{n\to \infty}
%\left(\lim\limits_{m\to \infty} \cos^m (2\pi x n!)\right)$.
\кзадача

%\задача Сформулируйте (без отрицаний), что значит, что функция $f:M \to \R$
%\выд{разрывна} (то есть не является непрерывной) в точке $a\in M$?
%\кзадача

\опр
Пусть $M\subseteq\R$.
Говорят, что $f:M\rightarrow\R$ \выд{ограничена на}
$M$, если найд\"ется такое число $k$, что $|f(x)|<k$ при всех $x\in M$.
\копр

\задача Будет ли функция, непрерывная в точке $a$, ограниченной
на некоторой окрестности точки~$a$?
\кзадача

\задача Пусть $f:M \to \R$ непрерывна в точке $a\in M$, прич\"ем $f(a) > 0$.
Докажите, что существует такая окрестность $U$ точки $a$, что $f$
положительна на множестве $U \cap M$.
\кзадача

%\задача Какие функции будут \лк непрерывными\пк, если в определении
%1 забыть, что %написать, что
%\вСтрочку
%\пункт
%$\varepsilon>0$;
%\пункт
%$\delta>0$?
%\кзадача


%\опр \label{limit}
%Пусть $a\in M$ --- предельная точка множества $M$.
%Функция $f:M \to \R$, где $M\subset \R$, называется непрерывной в
%точке $a\in M$, если предел функции $f$ в точке $a$ равен $f(a)$.
%
%Если точка $a\in M$ не является предельной точкой множества $M$,
%то функция $f$
%непрерывна в точке $a$ по определению.
%\копр

\задача
Пусть функция $f$ определена в некоторой окрестности $\cal U$ точки $a$. Докажите, что $f$ непрерывна в точке $a$ тогда и только тогда, когда
$\lim\limits_{x \to a} f(x)=f(a)$.
\кзадача



%\задача Рассмотрим функцию  $f(x)=x^2$ на отрезке $[-5;5]$.
%Найдите такое $\delta>0$, чтобы для любых двух точек  $x,y\in[-5;5]$,
%таких, что $|x-y|<\delta$, выполнялось бы неравенство
%$|f(x)-f(y)|<0,01$.
%\кзадача

\задача Пусть $f, g\in C(a)$. Докажите, что:
\вСтрочку
%\сНовойСтроки
\пункт $|f|\in C(a)$;
\пункт $f\pm g\in C(a)$;% непрерывна в точке $a$;\hskip-1pt
\пункт $f\cdot g\in C(a)$; %непрерывна в точке $a$;\hskip-1pt
\пункт если %, кроме того,
$g(a) \neq 0$,
то функция $f/g$ %определена в некоторой окрестности точки $a$ и
непрерывна в точке $a$.
\кзадача

\задача Докажите непрерывность функции (на е\"е области
определения):
\вСтрочку
\пункт $x^n$, где $n\in\N$;
\пункт \hskip-1pt многочлен из $\R[x]$; \hskip-1pt
\пункт \hskip-1pt $P(x)/Q(x)$, где $P,Q\in\R[x]$, $Q\ne0$;\hskip-1pt
%\пункт $1/x^n$;
%\пункт $[x]$;
\пункт \hskip-1pt $\root n \of x$,~где~$n\in\N$;\hskip-1pt
\пункт \hskip-1pt $\sin x$;\hskip-1pt
\пункт \hskip-1pt $\cos x$;\hskip-1pt
\пункт \hskip-1pt $\tg x$.
%\пункт $\ctg x$.
%\пункт Как проще всего вывести  из пункта г) непрерывность
%функции $\cos x$?
%\пункт Докажите непрерывность функций $\tg x$ и $\ctg x$.
\кзадача

\задача Придумайте определ\"енную на $\R$ функцию $f$, множество точек разрыва которой есть\\
\вСтрочку
\пункт $\R$;
\пункт $\R$ без одной точки;
\пункт $\{1/n\ | \ n\in\N\}$;
\спункт $\Q$.
%\спункт разрывную в рациональных точках и непрерывную в иррациональных.
\кзадача




\раздел{***}

\задача
\пункт
Пусть $f\in C([a;b])$, прич\"ем
$f(a) > 0$, $f(b) <0$. Найд\"ется ли такое $\gamma\in (a,b)$, что
$f(\gamma)=0$?
%\кзадача
\пункт
%\задача
{\small\sc (Теорема о промежуточном значении).}
Пусть $f\in C([a;b])$, прич\"ем  $f(a) < f(b)$.
Докажите, что для любого $k \in [f(a), f(b)]$  найд\"ется такая
точка $\gamma\in [a,b]$, что $f(\gamma) = k$.
\кзадача



\задача Докажите: любой многочлен неч\"етной степени из $\R[x]$
%с действительными коэффициентами
имеет хотя бы один %действительный
корень из $\R$.
\кзадача

\задача {\small\sc (Теорема Л.~Бр\'ауэра о неподвижной точке для отрезка).}
Пусть $f\in C([0;1])$ и все значения функции $f$ содержатся в
отрезке~$[0;1]$. Докажите, что уравнение $f(x)=x$ имеет корень.
\кзадача



%\задача Многоугольник $M$ и точка $A$ лежат в одной плоскости.
%Докажите, что существует прямая $l'$, проходящая через точку $A$ и
%разбивающая $M$ на два равновеликих многоугольника.
%\кзадача

%\сзадача На сковороде лежат два блина (выпуклые многоугольники). Докажите,
%что можно провести
%один прямолинейный разрез так, чтобы каждый блин разделился пополам.
%\кзадача


%\опр \выд{Промежутком} называют любой
%отрезок, полуинтервал, интервал, открытый или замкнутый луч на прямой,
%а также всю прямую.
%\копр

%\задача
%Непостоянная функция $f$ определена и непрерывна на промежутке
%$I\subseteq\R$.  Докажите, что множество значений этой функции
%на $I$ также является промежутком.
%\кзадача

%\опр
%Пусть $M\subseteq\R$.
%\копр


\задача Функция непрерывна на отрезке $I$. Всегда ли она
%на этом отрезке
\вСтрочку
\пункт ограничена на $I$;
\пункт достигает своего наибольшего и наименьшего значений на $I$?
\пункт Та же задача, если $I$ --- интервал или прямая.
%Верны ли утверждения пунктов а), б) для функции, непрерывной
%на интервале или на прямой?
%\кзадача
%
%\задача
%Непостоянная функция $f$ определена и непрерывна на множестве
%$I\subseteq\R$.  Каким может быть множество значений этой функции
%на $I$, если $I$
\пункт Каким может быть множество значений непрерывной функции на
отрезке; интервале; прямой?
\кзадача


%\задача {\small\sc (Теорема о монотонной функции.)}
%Функция $f$ непрерывна на промежутке $I\subseteq\R$.
%Докажите, что $f$ обратима на $I$ если и только если
%$f$ строго монотонна на $I$, прич\"ем обратная функция строго
%монотонна и непрерывна на промежутке $f(I)$.
%\кзадача


\задача Пусть $I$ --- {\em промежуток}, то есть отрезок, интервал, полуинтервал, луч или вся прямая.
Функция $f$ непрерывна на $I\subseteq\R$. Докажите, что $f$ обратима на $I$ если и только если
$f$ строго монотонна на $I$. Докажите, что при этом
$f^{-1}$
%обратная функция
строго монотонна и непрерывна (на %отрезке %$f(I)$).
$[\min\limits_{x\in I} f(x);\max\limits_{x\in I} f(x)]$).
\кзадача

\задача Докажите непрерывность %функций
$\arcsin x$ и $\arctg x$
(на их области определения).
\кзадача

%\задача Сформулируйте и докажите теорему о непрерывности композиции
%%двух
%непрерывных функций.
%%(Композицией функций $f(x)$ и $g(x)$ называется функция $f(g(x))$.)
%\кзадача

%Пусть $A$ и $B$ --- множества из $\R$
%\задача
%Функция $f(x)$ непрерывна в точке $a$, функция $g(x)$ непрерывна
%в точке $f(a)$. Докажите, что функция $g(f(x))$ непрерывна в точке $a$.
%Пусть $f\in C(a)$, $g\in C(f(a))$.
%Докажите, что $g\circ f\in C(a)$.
%\кзадача

\задача
Пусть $A,B\subseteq\R$, и функции $f:A\rightarrow B$, $g:B\rightarrow\R$
непрерывны. Докажите, что %функция
%$g\circ f:A\rightarrow\R$ непрерывна.
$g\circ f\in C(A)$.
\кзадача



\задача
Пусть $f,g\in C(\R)$, прич\"ем $f(x)=g(x)$ для
любого $x\in\Q$. Докажите, что %тогда
$f=g$.
%(иначе говоря,
%непрерывная функция определяется своими значениями в рациональных точках.)
\кзадача

\задача
Найдите все $f\in C(\R)$, %удовлетворяющие условию:
такие что
$f(x+y)=f(x)+f(y)$ для любых $x,y\in\R$.
\кзадача

\раздел{Равномерная непрерывность}

\опр
Функция $f:M\rightarrow\R$ %, где $M\subseteq\R$,
называется \выд{равномерно непрерывной на} %множестве}
$M$, если для каждого $\varepsilon>0$ найд\"ется такое $\delta>0$, что
для любых $x,y\in M$,
%расстояние между которыми меньше $\delta$, %будет
таких что $|x-y|<\delta$,
выполнено %неравенство
$|f(x)-f(y)|<\varepsilon$.
\копр

\задача
%Обязательно ли
Функция $f$ равномерно непрерывна на $M\subseteq\R$.
Обязательно ли тогда $f$ непрерывна на $M$?
%будет непрерывной на этом множестве?
\кзадача


\задача
\вСтрочку
Какие из %следующих
функций $x^2$, $\sqrt x$, $1/x$ равномерно непрерывны
\пункт
на $[1;\infty)$;
\пункт
на $(0;1)$?
%\вСтрочку
%\пункт
%$x^2$;
%\пункт
%$\sqrt x$;
%\пункт
%$1/x$.
%\пункт
%$x$
%\кзадача
%
%
%\задача
%\пункт
%Придумайте функцию, которая
%%\вСтрочку
%непрерывна на $\R$, но не равномерно непрерывна на $\R$.
%\пункт
%Придумайте функцию, которая
%непрерывна на $(0;1)$, но не равномерно непрерывна на $(0;1)$.
%\кзадача
%\сзадача
\пункт
{\small\sc (Теорема К\'антора).}
Докажите: непрерывная на отрезке функция равномерно непрерывна на~н\"ем.
\кзадача

%\СделатьКондуит{4.6mm}{8mm}

\ЛичныйКондуит{0mm}{8mm}

%\СделатьКондуит{4.5mm}{7.5mm}

% %\GenXMLW


\end{document}

====

\задача  Нарисуйте кривые:
\вСтрочку
%\пункт $x=y$;
\пункт $x^2=y^2$;
%\пункт $y=x^2$;
\пункт $х^2y-xy^2=x-y$;
%\пункт $ax^2+by^2=1$, где $a,b$ --- такие числа, что $a>b>0$;
%\пункт $ax^2-by^2=1$, где $a,b$ --- такие числа, что $a>b>0$;
\пункт $y^2=x^3$;\quad
\пункт $y-1=x^3$;\quad
\пункт $y^2-1=x^3$;\quad
\пункт $y^2-x=x^3$;\quad
\пункт $y^2-x^2=x^3$.
\кзадача

\сзадача Нарисуйте кривые:
\вСтрочку
\пункт $x^2=x^4+y^4$;
\пункт $xy=x^6+y^6$;
\пункт $x^3=y^2+x^4+y^4$;
\пункт $x^2y+xy^2=x^4+y^4$.
\кзадача 