% !TeX encoding = windows-1251
\documentclass[a4paper,11pt]{article}
\usepackage{newlistok}
%\documentstyle[11pt, russcorr, listok]{article}
\newcommand{\del}{\mathrel{\raisebox{-.3 ex}{${\vdots}$}}}

\УвеличитьШирину{1.4truecm}
\УвеличитьВысоту{3.1truecm}
\hoffset=-2.6truecm
\voffset=-27truemm
\renewcommand{\spacer}{\vfil}

\Заголовок{Графы}
\НомерЛистка{7}
\ДатаЛистка{01.2014}

\begin{document}

\СоздатьЗаголовок

\vspace*{-1truemm}
\опр
\выд{Граф} задан, если задано конечное множество его \выд{вершин},
и для каждой пары разных вершин известно, связаны они \выд{ребром}
или нет. Граф удобно изображать как множество точек на плоскости,
некоторые пары которых соединены линиями
(точки --- вершины, линии --- р\"ебра).
Бывают и графы с \выд{кратными} рёбрами
(пара вершин может соединяться несколькими рёбрами)
и \выд{петлями} (вершина может соединяться сама~с~собой).
%Ребро, соединяющее некоторую вершину саму с собой, называется
%\выд{петл\"ей}.\\ Р\"ебра, соединяющие одну и ту же пару вершин,
%называются \выд{параллельными} или \выд{кратными}.\\
%Граф, в котором нет петель и параллельных р\"ебер,
%называется \выд{простым}.\\
\копр

%\задача
%Про три компании, из шести человек каждая, известно следующее.
%Шестеро человек сели за круглый стол так, что знакомы были только сидящие рядом и напротив. Затем они пересели на две лавочки так, что на одной лавочке были незнакомые люди, а на разных --- знакомые.
%Каждый среди них знал троих, но среди любых трёх было хотя бы двое незнакомых.
%Могло ли такое быть?
%Могла ли это быть одна и та же компания?
%\кзадача


\задача
\вСтрочку
\пункт
Ид\"ет Петя,  а навстречу  ему 5 человек.
Докажите, что среди них найдутся либо трое,
знакомых с Петей, либо трое, незнакомых с Петей.
\пункт
Докажите, что среди любых 6 человек найдутся
либо 3 попарно знакомых, либо 3 попарно незнакомых человека.
\пункт
А если есть всего 5 человек?
\кзадача

\задача
На плоскости отметили 17 точек и соединили каждые две из них
цветным отрезком: красным, желтым или зел\"еным.
Докажите, что
%\вСтрочку
%\пункт из каждой отмеченной точки
%выходит не меньше 6 одноцветных отрезков;
%\пункт
найдутся три точки в вершинах одноцветного треугольника.
\кзадача

\задача
Выпишите в ряд цифры от 0 до 9 так, чтобы каждое число, составленное из
любых двух соседних цифр, делилось на 7 или на 13.
\кзадача


\задача
В углах доски $3\times3$ стоят кони: 2 белых (в соседних
углах) и 2 чёрных. Можно ли за~\hbox{несколько}~ходов (по шахматным
правилам) поставить их так, чтобы в любых двух
соседних углах
стояли кони разного цвета?
\кзадача

\задача
В стране 15 городов, каждый соедин\"ен
дорогами не менее, чем с семью другими.
Докажите, что из любого
города можно проехать в любой другой напрямую или
через один промежуточный город.
\кзадача

\задача
Группа островов соединена мостами.
% так, что от каждого острова можно добраться до любого другого.
Турист обошёл
все острова, пройдя по каждому мосту %ровно
один раз. На острове
Светлом он побывал трижды. Сколько мостов ведёт со
Светлого, если турист
\вСтрочку
\пункт
не с него начал и не на нём закончил?
\пункт
с него начал, но не на нём закончил?
\пункт
с него начал и на нём закончил?
\кзадача

\задача
Из столицы выходит 101 авиалиния, из
города Дальний --- одна, а из остальных городов по 100. Докажите,
что из столицы можно долететь в Дальний (возможно, с пересадками).
\кзадача


\опр
\выд{Степень} ${\rm deg}\, V$ вершины $V$  --- это  число
выходящих из не\"е р\"ебер (петли считаются дважды).
\копр

\задача
\вСтрочку
%\пункт
%У каждого из 179 марсиан три руки. Смогут ли все они взяться за руки
%так, чтобы свободных рук не осталось? А если бы марсиан было 180?
%\пункт
%Можно ли соединить 77 телефонов между собой так,
%чтобы каждый был соединен ровно
%с 15 другими?
\пункт
Как связаны сумма степеней вершин любого графа
и количество его р\"ебер?\\
\пункт
Верно ли, что число вершин неч\"етной степени любого графа ч\"етно?
\кзадача

\задача
На какое наименьшее число частей надо разделить проволоку длиной 12 см, чтобы
из них можно было сделать каркас куба со стороной 1 см? Полученные части можно %только
изгибать и скреплять друг с другом.
%Можно ли из куска проволоки длиной 120 см
%сделать каркас куба со стороной 10 см, не ломая проволоки?
\кзадача


\сзадача
У Пети 28 одноклассников, %прич\"ем
они имеют разное
%различное
число друзей
в %этом
классе. Сколько из них дружит~\hbox{с Петей?}
\кзадача

%\vspace*{-5truemm}
\задача
%\вСтрочку
\пункт
На чаепитие собрались 25 школьников. Каждый принес по 2 пирожных.
Все пирожные раз\-ло\-жи\-ли на 25 тарелок (по 2 на тарелку).
Докажите, что, как бы ни были размещены пирожные,
можно так раздать тарелки школьникам, что каждому
достанется хотя бы одно пирожное, которое он сам принес.\\
\спункт  А если каждый принёс по 10 пирожных (и их разложили
по 10 штук на тарелку)?
\кзадача



\опр
\выд{Путь} в графе --- это последовательность вершин
$V_1$, $V_2$, \dots, $V_{n+1}$ и соединяющих соседние вершины рёбер
$V_1 V_2$, $V_2V_3$, \dots, $V_n V_{n+1}$.
%, в~\hbox{которой} каж\-дые две соседние
%вершины соединены ребром.
%Последовательность р\"ебер $V_1 V_2$, $V_2V_3$, \dots, $V_n V_{n+1}$
%также называют пут\"ем.
Если $V_1=V_{n+1}$, то путь называется \выд{циклическим},
если при этом р\"ебра пути различны --- \выд{циклом},
а если ещ\"е и вершины разные (кроме $V_1$ и $V_{n+1}$) ---
\выд{простым циклом}.\\
Граф называется \выд{связным}, если каждые две его вершины соединены пут\"ем.
\копр

\задача [Эйлеровы графы]
Дан связный граф, в котором степень любой вершины чётна. Докажите, что
\вСтрочку
\пункт в графе есть простой цикл;
\пункт рёбра графа можно разбить на несколько циклов
(возможно, с общими вершинами, но без общих рёбер);
\пункт условие равносильно тому, что в графе есть цикл, содержащий все~р\"ебра.
\кзадача

\задача
\вСтрочку
В некой стране $N$ городов, некоторые из которых соединены дорогами.
Из любого города можно добраться в любой другой
ровно одним способом (двигаясь по дорогам и нигде не разворачиваясь назад).
\\
\пункт Докажите, что в стране есть город, из которого ведёт ровно
одна дорога.
\пункт
Сколько дорог в этой стране?
\пункт
Одну дорогу закрыли на ремонт. Можно ли теперь попасть из любого
города в любой другой?
\кзадача

\опр
Граф называется \выд{ориентированным}, если на каждом ребре указано
направление.
%Пара вершин может при этом соединяться двумя рёбрами
%разных направлений.
\копр

\задача
В турнире каждая команда сыграла с каждой по разу.
Ничьих не было. Всегда ли можно расположить команды в таком
порядке, чтобы первая команда выиграла у второй,
вторая --- у третьей, и т.~д.?
\кзадача

\задача
\вСтрочку
\пункт
Строка из 36 нулей и единиц начинается с 5 нулей.
Среди пятёрок подряд стоящих цифр~встречаются все 32 возможные комбинации.
Найти 5 последних цифр строки.
\пункт \hspace*{-3.2mm} {\bf *} \ Почему такая строка есть?
\кзадача

\задача
Каждый из 450 депутатов дал пощёчину ровно одному своему
коллеге. Докажите, что из них можно выбрать 150
человек,  среди которых никто никому не давал пощёчины.
\кзадача

\сзадача
Схема проезда по городу представляет собой граф (ребра --- улицы,
вершины --- перекрёстки). Назовём ребро $AB$ \выд{перешейком}, если
любой путь, соединяющий $A$ и $B$, содержит ребро $AB$.
%от $A$ до $B$ можно проехать единственным путём --- по ребру $AB$.
Докажите, что можно ввести на всех улицах, кроме перешейков, одностороннее
движение (а на перешейках --- двустороннее) так, чтобы от любого
перекрёстка можно было доехать до любого другого, не нарушая правил.
\кзадача



\ЛичныйКондуит{0mm}{8mm}

%\СделатьКондуит{6.9mm}{7mm}

\end{document}

\задача
У царя Гвидона было три сына (и больше детей не было).
Из его потомков сто имело по два сына, а остальные умерли бездетными.
Сколько потомков было у царя Гвидона?
\кзадача

\сзадача [Теорема Холла] В некоторой компании $n$ юношей.
При каждом $k$ от 1 до $n$ верно утверждение:
для любых $k$ юношей в компании число девушек,
знакомых хотя бы с одним из этих $k$ юношей, не меньше $k$.
Можно ли женить всех юношей на знакомых девушках?
\кзадача

