% !TeX encoding = windows-1251
\documentclass[a4paper,11pt]{article}
\usepackage{newlistok}
%\usepackage{tikz}
%\usetikzlibrary{calc}

%\documentstyle[11pt, russcorr, listok]{article}
\newcommand{\del}{\mathrel{\raisebox{-.3 ex}{${\vdots}$}}}

\УвеличитьШирину{1.4truecm}
\УвеличитьВысоту{2.9truecm}
\hoffset=-2.5truecm
\voffset=-25truemm


\begin{document}

\Заголовок{Простые числа. Основная теорема арифметики}
\НомерЛистка{15}
\ДатаЛистка{10.2014}
\def\a{\alpha}

\СоздатьЗаголовок

%\vspace*{-1.5truemm}

%\опр
%Натуральное число $p>1$ называется \выд{простым}, если оно имеет ровно два
%натуральных делителя: 1 и $p$, в противном случае оно
%называется \выд{составным}.
%\копр

\задача
\вСтрочку
\!\!\!\! \пункт[Решето Эратосфена]\!\!\!
Выпишем в ряд целые числа от 2 до~$n$. Подчеркн\"ем число
2 и сотр\"ем~\hbox{числа,} делящиеся на 2.
Первое неподч\"еркнутое число подчеркн\"ем и сотр\"ем %теперь
числа, делящиеся на него, и т.~д.
%Снова первое невычеркнутое число обвед\"ем в кружок, и т.~д.
Будем~действовать так, пока каждое число от 2 до $n$
не будет либо подч\"еркнуто, либо ст\"ерто.
Докажите, что мы подчерк\-н\"ем в точности простые числа
от~1~до~$n$.\!
\пункт\!\!\! Пусть очередное число, которое мы хотим~\hbox{подчеркнуть, больше $\sqrt{n}$.}
Докажите, что не\-ст\"ер\-тые к этому моменту числа от 2 до $n$ простые.
%натуральное число $a$, большее 1, не делится ни на одно простое число, меньшее $\sqrt{a}$. Докажите, что $a$ простое.
\пункт Какие числа, меньшие $100$, простые?
\кзадача

\опр
Назовём {\it идеалом} в множестве целых чисел $\Z$ любое подмножество $I$ с такими свойствами:\\
1) если $a\in I$ и $b\in I$, то и $a+b\in I$ (сумма любых двух чисел из идеала также принадлежит этому идеалу);\\
2) если $a\in I$, $n\in \Z$, то $na\in I$ (умножая число из идеала на {\it любое целое}, мы получаем число из этого идеала).
\копр

\задача
Верно ли, что разность любых двух чисел из идеала также принадлежит этому идеалу?
\кзадача

\задача
Какие из следующих множеств являются идеалами в $\Z$:
\пункт $\Z$;
\пункт $\N$;
\пункт множество чётных целых чисел;
\пункт множество нечётных целых чисел;
\пункт $\{0\}$:
\пункт множество чисел, делящихся на 17.
\кзадача

\задача[Теорема об идеалах в $\Z$]
Пусть $r$ --- наименьшее положительное число, принадлежащее идеалу $I$. Докажите, что 
\пункт любое число из $I$ делится на $r$; 
\пункт $I$ состоит из всех целых чисел, делящихся на $r$.
\кзадача

\задача
Пусть $a$ и $b$ --- целые числа, не равные одновременно нулю, $J$ --- множество чисел вида $ax+by$, где $x$ и $y$ целые, и $r$ --- наименьшее положительное число в $J$.
Докажите, что\\
\пункт $J$ --- идеал;
\пункт $r$ делится на $(a,b)$; 
\пункт $a$ и $b$ делятся на $r$;
\пункт $r=(a,b)$, то есть $J$ состоит из всех чисел, делящихся на $(a,b)$.
\кзадача

\задача
Пусть $a$ и $b$ --- целые числа, причем $(a,b)=1$. Докажите, что\\
\пункт найдутся такие целые $x$ и $y$, что $ax+by=1$; 
\пункт если $ca$ делится на $b$, где $c$ --- целое, то $c$ делится на $b$.
\кзадача


\задача[Основная теорема арифметики] Докажите следующие утверждения:
\сНовойСтроки
\vspace*{-3pt}
\пункт если $p$ --- простое число, $m$ и $n$ --- целые, и $mn\del p$,
то либо $m\del p$, либо $n\del p$;
\пункт для каждого целого $n>1$ найдутся такие простые
$p_1,\dots,p_k$, что $n=p_1\cdot \dots \cdot  p_k$; %($k$ --- натуральное);
\пункт[каноническое разложение] Для каждого целого $n>1$ найдутся такие
различные простые $p_1,\dots,p_k$ и натуральные $\a_1,\dots,\a_k$, что
$n=p_1^{\a_1}\cdot \dots \cdot p_k^{\a_k}$; %($k$ --- натуральное);
\пункт разложения из пунктов б) и в) единственны с точностью до порядка
сомножителей.
\кзадача

\задача
Назов\"ем ч\"етное число $n$ \выд{ч\"етнопростым}, если $n$
не раскладывается
в произведение двух ч\"етных чисел. (Например, 6 --- ч\"етнопростое,
а 12 --- нет.)
%Верно ли, что любое ч\"етное
%число единственным образом раскладывается в произведение ч\"етнопростых
%чисел (с точностью до порядка сомножителей)?
Какие пункты задачи 2 будут верны, если
заменить в условии целые числа на ч\"етные, а
простые %числа
--- на ч\"етнопростые?
\кзадача

\задача  Числа $a$, $b$, $c$, $n$ натуральные, $(a,b)=1$, $ab=c^n$.
Найдутся ли такие целые $x$ и $y$, что~$a=x^n$, $b=y^n$?
%Верно ли, что $a=x^n$ %и $b=y^n$
%для некоторого целого $x$?
%натуральных чисел $x$ и $y$?
\кзадача

\задача
Решите в натуральных числах уравнение $x^{42}=y^{55}$.
\кзадача

\задача
Найдите каноническое разложение числа \вСтрочку
\пункт 2010; \пункт 2011; \пункт 17!; \пункт $C_{20}^{10}$.
\кзадача

\задача При каких натуральных $k$ число $(k-1)!$ не делится на $k$?
\кзадача

\задача
\вСтрочку
\пункт [Теорема Лежандра] Докажите, что простое число
$p$ входит в каноническое разложение числа $n!$
в степени $[n/p]+[n/p^2]+[n/{p^3}]+\dots$
(где $[x]$ --- это \выд{целая часть} числа $x$).\\
С какого момента слагаемые в этой сумме станут равными нулю?\\
\пункт Сколько у $2000!$ нулей в конце его десятичной записи?
% числа $2000!$?
\пункт Может ли $n!$ делиться на $2^n$ ($n\geq1$)?
\кзадача


\задача
Число $p$ простое. Докажите, что $C_p^k$ делится на
$p$, если $0<k<p$.
\кзадача

%\задача[Малая теорема Ферма]
%Докажите: $n^p-n$ делится на $p$, если $p$ ---
%простое,~\hbox{$n$ --- целое.}
%%Пусть $p$ простое.
%\кзадача

\задача[Малая теорема Ферма]
Пусть $p$ --- простое число, $n$ --- целое число.
Докажите, что\\
\вСтрочку
\пункт  $n^p-n$ делится на $p$;
\пункт  если $(n,p)=1$, то $n^{p-1}-1$ делится на $p$.
\кзадача


%\задача
%Для каких $k\in\N$ есть $k$ последовательных
%целых чисел, являющихся составными?
%\кзадача


\сзадача
\вСтрочку
\пункт
Числа $p$ и $q$ простые, $2^{p}-1\del q$. Докажите,
что $q-1\del p$.
\пункт
Простое ли $2^{13}-1$?
\кзадача


\сзадача Может ли быть целым число
\вСтрочку
\пункт
$\displaystyle{\frac{1}{2}+\frac{1}{3}+\frac{1}{4}+\ldots+\frac{1}{n}}$;
\пункт
$\displaystyle{\frac{1}{3}+\frac{1}{5}+\frac{1}{7}+\ldots+\frac{1}{2n+1}}$?
\кзадача


\vspace*{-1mm}
\раздел{***}

\vspace*{-2mm}
\опр \выд{Наименьшим общим кратным} ненулевых целых чисел $a$ и $b$
называется наименьшее натуральное число, которое делится на $a$ и на $b$.
Обозначение: $[a,b]$ или НОК$(a,b)$.
\копр

%\задача
%Докажите, что $[a,b]$ существует и единственно
%для любых ненулевых целых $a$ и $b$.
%\кзадача

\задача
\вСтрочку
\пункт
Как, зная канонические разложения %на множители
%натуральных
чисел $a$ и $b$, найти
$(a,b)$ и $[a,b]$?
\пункт
Найдите $[192,270]$.
\пункт
Докажите, что $ab=(a,b) \cdot [a,b]$.
\пункт
Верно ли, что $[a,b]/a$ и $[a,b]/b$ взаимно~просты?
\кзадача

%\задача Верно ли, что \вСтрочку
%\пункт $[ca,cb]=c[a,b]$ при $c>0$;
%\пункт $[a,b]/a$ и $[a,b]/b$ взаимно~просты?
%\кзадача

\задача Докажите, что любое общее кратное
целых чисел $a$ и $b$ делится на $[a,b]$.
\кзадача

%\задача
%Докажите, что $ab=(a,b) \cdot [a,b]$ для любых натуральных
%чисел $a$ и $b$.
%\кзадача

\задача
Про натуральные числа $a$ и $b$
известно, что $(a,b)=15$, $[a,b]=840$. Найдите $a$ и $b$.
\кзадача


%\задача
%Может ли %наименьшее общее кратное чисел
%НОК$(1, 2, \dots, n)$
%быть в $2008$ раз больше, чем %наименьшее общее кратное чисел
%НОК$(1, 2, \dots, m)$?
%\кзадача

\задача
Найдите
$\displaystyle{\frac{{\rm НОК}(1,\,2,\,3,\,\dots\,,\,99)}{{\rm НОК}(2,\,4,\,6,\,\dots\,,\,200)}}$.
\кзадача

%\сзадача Найдутся ли 100 таких различных натуральных чисел, что для
%любых двух чисел $a$ и $b$ из них
%\пункт $ab$~делится на сумму всех 100 чисел;
%\пункт $a+b$ делится на $a-b$;
%\пункт $(a,b)=|a-b|$?
%\пункт Тридцать~три~богаты\-ря едут верхом по кольцевой дороге против часовой стрелки.
%Могут ли они ехать неограниченно %различными
%долго~с~\hbox{разными} посто\-янными скоростями,
%если на дороге есть только одна точка, %в которой
%где богатыри %имеют возможность
%могут обгонять друг друга?
%\кзадача



\ЛичныйКондуит{0mm}{5mm}

%\СделатьКондуит{4.3mm}{6.5mm}


\end{document}

\раздел{Запас}

\задача
$a$ и $b$ --- натуральные числа.
Докажите, что если $[a,a+5]=[b,b+5]$, то $a=b$.
\кзадача

\задача
Число $n$ натуральное. Пусть $k$ --- наименьшее
целое число, большее 1 и
взаимно простое с каждым из чисел $1$, $2$, \dots, $n$.
Докажите, что $k$ существует и является простым.
\кзадача

\задача %Пусть $n\in\N$.
Сколько двоек в разложении числа
$1001\cdot1002\cdot\ldots\cdot2000$ на простые множители?
\кзадача

\задача Докажите, что при любом целом $n>1$ между $n$ и $n!$ есть простое
число.
\кзадача

\задача
Решите в натуральных числах уравнения:
\вСтрочку
\пункт
$x^y=y^x$;
\пункт
$u^x+u^y=u^z$.
\кзадача

\сзадача Найдите все натуральные $n$,
у которых не меньше чем $\sqrt n$ натуральных делителей.
\кзадача

\сзадача
Для целых $a,b$ и натурального $n$ докажите:  %, что
$b^n(a+b)(a+2b)\cdot\ldots\cdot(a+nb)$ кратно $n!$.
\кзадача


\сзадача Числа $a$, $b$, $c$ и $d$ натуральные, %прич\"ем
$ab=cd$. Может ли число $a+b+c+d$ быть простым?
\кзадача


\end{document} 