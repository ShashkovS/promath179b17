% !TeX encoding = windows-1251
\documentclass[a4paper, 12pt]{article}
\usepackage{newlistok}
%\documentstyle[11pt, russcorr, listok]{article}
\newcommand{\0}[1]{\overline{#1}}
\def\C{\mbox{$\Bbb C$}}

\УвеличитьШирину{1.5truecm}
\УвеличитьВысоту{3.5truecm}

%\hoffset=-.25truecm
%\voffset=.5truecm
%\УвеличитьШирину{5cm}
%\УвеличитьВысоту{9cm}
%\pagestyle{empty}



\begin{document}


\Заголовок{Производная. Выпуклость и точки перегиба}
\НомерЛистка{30}
\ДатаЛистка{09.2016}

\СоздатьЗаголовок


%\опр {\it Надграфиком} функции $f:[a,b]\rightarrow\R$ называется
%подмножество плоскости $\{(x,y)\in\R^2\ |\ x\in [a,b],y\geqslant f(x)\}$.
%\копр

%\опр Функция $f:[a,b]\rightarrow\R$
%называется {\it выпуклой вниз} на отрезке $[a,b]$,
%если е\"е надграфик --- выпуклое множество.
%\копр

\раздел{Выпуклые функции}

\опр Функция $f:I\rightarrow\R$
%$f$, определ\"енная на интервале $(a;b)$,
называется \выд{выпуклой вниз} на промежутке $I$, если для каждого отрезка
$[x_1;x_2]\subseteq I$ выполнено: в каждой точке этого отрезка $f(x)\leqslant L(x)$, где $L$ --- прямая,
соединяющая точки $(x_1;f(x_1))$ и $(x_2;f(x_2))$.
Если всегда (кроме концов отрезков) верно строгое неравенство $f(x)<L(x)$, говорят о {\em строгой} выпуклости вниз.
Аналогично вводится понятие (строгой) выпуклости вверх.
\копр

\задача Пусть функция $f$ определена на интервале $(a,b)$.
Докажите, что $f$ выпукла вниз на $(a,b)$ тогда и только тогда, когда выполнено любое из следующих условий:\сНовойСтроки
\пункт
надграфик $f$ на $(a;b)$,
то есть $\{(x,y)\in\R^2\ |\ x\in(a,b),y\geqslant f(x)\}$
--- выпуклое множество;
\пункт
$\alpha\cdot f(x)+(1-\alpha)\cdot f(y)\geqslant
f(\alpha\cdot x+(1-\alpha)\cdot y)$
для любых $x,y\in(a,b)$ и любого $\alpha\in[0,1]$;
\пункт
{\it (неравенство Йенсена)}
$%\displaystyle
\frac{\alpha_1 f(x_1)+\dots+\alpha_n f(x_n)}{\alpha_1+\dots+\alpha_n}
\geqslant
f\left(\frac{\alpha_1 x_1+\dots+\alpha_n x_n}{\alpha_1+\dots+\alpha_n}\right)
$
для любых чисел
$x_1,\dots,x_n\in(a,b)$ и любых положительных чисел
$\alpha_1,\dots,\alpha_n$.
% \спункт
% $
% f\left(\frac{x_1+x_2}2\right)\leq\frac{f(x_1)+f(x_2)}2
% $
% для любых $x_1$, $x_2$ из $(a;b)$ в случае
% непрерывной на $(a;b)$ функции $f$.
\кзадача


\задача
Пусть $f$ положительна и выпукла вниз на $[a;b]$.
Обязательно ли $1/f$ выпукла вверх на $[a;b]$?
\кзадача


%\задача Приведите пример функции, выпуклой вниз на $[0,1]$.
%Нарисуйте е\"е надграфик.

\задача
Докажите, что если функция $f$ выпукла вниз на $(a,b)$, то %выполнено неравенство
$\frac{f(x)-f(x_1)}{x-x_1} \leq \frac{f(x_2)-f(x)}{x_2-x}$ при любых $x_1 < x < x_2$, где $x,x_1,x_2 \in (a,b)$.
\кзадача


\задача Пусть функция $f$ дважды дифференцируема ($f$ и $f'$ дифференцируемы) на
интервале $(a,b)$.  Докажите, что $f$ выпукла вниз на $(a,b)$ если и только если выполнено любое из следующих условий:\\
\вСтрочку
\пункт
$f'$ монотонно неубывает на интервале $(a,b)$;\qquad
\пункт
$f''(x)\geqslant0$ для любого $x\in(a,b)$;\\
\пункт
любая касательная $l$ к графику $f$ расположена не выше его:
$f(x)\geqslant l(x)$ при всех $x\in(a,b)$.
\кзадача

\задача Найдите промежутки выпуклости вверх и выпуклости вниз
следующих функций:\\
\вСтрочку
\пункт $\sin x$;
\пункт $x^2$;
\пункт $x^3$;
\пункт $x^4$;
%\пункт $\ln x$;
\пункт $\sqrt{|x|}$;
\пункт  $5x^4+7x^3$;
\пункт $\sin x+\cos x$;
\пункт $(x(x-1))^{-1}$;
\пункт $x^2+\frac1x$.
\кзадача

\задача Докажите, что %следующие
%неравенства:
%\сНовойСтроки
\вСтрочку
\пункт
%\displaystyle{
$\left({x_1+\dots+x_n\over n}\right)^2\leqslant
{x_1^2+\dots+x_n^2\over n}$; %для любых чисел $x_1,\dots,x_n$;\\
\пункт %[неравенство Коши-Буняковского]
$(x_1y_1+\dots+x_ny_n)^2\leqslant(x_1^2+\dots+x_n^2)(y_1^2+\dots+y_n^2)$.
%\спункт
%$\sin x\sin y\sin z\leqslant 3\sqrt{3}/8$, если $x,y,z$ --- углы
%некоторого треугольника.
\кзадача

\задача
Докажите для положительных $x_1,\ldots,x_n$ неравенство Коши: $\frac{x_1+\ldots+x_n}n\geq \root n \of {x_1\ldots x_n}.$\\
{\em Указание:} вам поможет функция $\ln$.
\кзадача


\задача
Что больше: $\root 3 \of{60}$ или $2+\root 3 \of 7$?
\кзадача

\сзадача
Пусть $f$ выпукла (вниз или вверх) на $(a,b)$. \пункт Докажите, что $f(x)$ непрерывна на $(a,b)$.
\пункт Верно ли, что $f$ имеет в каждой точке из $(a,b)$ правую и левую касательные?\\
\пункт Докажите, что $f$ дифференцируема на $(a,b)$ везде кроме счётного числа точек.
 \кзадача

\раздел{Точки перегиба}

\опр Точка $x_0$ называется {\it точкой перегиба} функции $f$, если
существует $\varepsilon>0$ такое, что $f$ строго выпукла вниз
на $(x_0-\varepsilon,x_0)$ и строго выпукла вверх на $(x_0,x_0+\varepsilon)$
(или наоборот).
\копр

\задача Пусть $f$ дважды дифференцируема в некой окрестности точки~$x_0$.
\сНовойСтроки
\пункт
Пусть $x_0$ --- точка перегиба функции $f$.
Верно ли, что $f''(x_0)=0$? Верно ли обратное? %утверждение?
%\пункт
%Докажите, что если $x_0$ --- точка перегиба $f$, то
%$x_0$ --- точка локального экстремума~$f'$. %Верно ли обратное?
\пункт
Докажите, что $x_0$ --- точка перегиба $f$ если и только если
$f''$ меняет знак в точке $x_0$.
\кзадача

\задача
Нарисуйте графики функций из задачи $5$
и найдите точки перегиба этих функций.
\кзадача

\задача
Пусть $f$ дважды дифференцируема в некоторой окрестности точки $x_0$,
прич\"ем $f'(x_0)=0$ и\\
\вСтрочку
\пункт
$f''(x_0)>0$;
\пункт
$f''(x_0)<0$.
Имеет ли $f$ в $x_0$ локальный экстремум, и если да, то какого типа?
\кзадача

%\задача
%Сколько перегибов у графика $y=(x+1)/(x^2+1)$?
%Лежат ли они на одной прямой?
%\кзадача

\задача
Сколько перегибов у графика $y=(x+1)/(x^2+1)$?
Лежат ли они на одной прямой?
\кзадача

\раздел{Асимптоты}

\опр Прямая $y=kx+b$ называется %(\выд{наклонной\/})
\выд{асимптотой\/}
%\footnote{асимптоты вида $y=b$ называются также {\it горизонтальными\/}}
графика функции $y=f(x)$, %при $x\to+\infty$,
если\break %$f(x)$ определена при всех $x\gg0$
%существует $\lim\limits_{x\to+\infty}(f(x)-(kx+b))=0$.
$f(x)-(kx+b)\to0$ при $x\to+\infty$ или $x\to-\infty$.
Прямая $x=x_0$ называется {\it вертикальной асимптотой\/}
графика функции $y=f(x)$, % при $x$, стремящемся к $x_{0}$ слева,
%(соотв. справа),
если $f(x)\rightarrow \infty$ при $x\rightarrow x_0$ (справа или слева).
%$\lim\limits_{x\rightarrow x_{0}-0}f(x)=\infty$.
%(соотв. $\lim\limits_{x\searrow x_{0}}f(x)=\infty$).
\копр

% \задача
% Дайте определение асимптот графика функции $y=f(x)$
% при $x\to-\infty$ и при $x\rightarrow x_{0}+0$.
% \кзадача

\задача
Пусть $y=f(x)$ имеет асимптоту
$y=kx+b$ при $x\to+\infty$. Найдите $\lim\limits_{x\to+\infty}\frac{f(x)}{x}$
и $\lim\limits_{x\to+\infty}(f(x)-kx)$.
%Как найти $k$ и $b$?
\кзадача

\задача
Пусть существует $\lim\limits_{x\rightarrow+\infty}\frac{f(x)}x=1$.
Обязательно ли тогда функция $f(x)$ имеет асимптоту?
\кзадача

%\задача
%Нарисуйте (с точным указанием всех асимптот) графики функций:\\
%\вСтрочку
%\пункт $x+\frac1x$;
%\пункт $x^2+\frac1x$;
%\пункт $\frac{x+3}{2-x}$;
%\пункт $\frac{x^{2}-4\,x+3}{x+1}$;
%\пункт $\frac{(x+1)(x-2)(x+3)}{x^{2}+1}$;
%\пункт $\sqrt{x\,(1+x)}$.
%\кзадача

%\задача
% Постройте (с полным исследованием) %\footnote{т.~е. как можно более
% точным указанием области определения, множества значений,
% промежутков возрастания - убывания, локальных экстремумов,
% направлений выпуклости, перегибов и асимптот})
% Постройте (с полным исследованием) графики следующих функций:\\
% \вСтрочку
% \пункт $x+\frac1x$;
% \пункт $\frac{x+3}{2-x}$;
%\пункт $\frac{x^{2}-4\,x+3}{x+1}$;
%\пункт $\frac{(x+1)(x-2)(x+3)}{x^{2}+1}$;
% \пункт $\sqrt{x\,(1+x)}$;
% \пункт $x\arctg x$;
% \пункт $\frac{x}{(x+1)^{2}}$;
% \пункт  $\root 3 \of{9-x^3}$;
% \пункт $\frac{x^3}{1-x^2}$;
%\пункт $\frac{1-2\,x+4\,{x}^{2}}{1-2\,x+2\,{x}^{2}}$;
%\пункт $\frac{\arcsin x}{\sqrt{1-x^2}}$;
%\пункт $\sqrt[3]{\frac{x^2}{x+1}}$;
% \спункт $\frac{\cos x}{\cos2x}$.
%\пункт $\frac{x^3+x^2-x+2}{x^2+x}$.
% \кзадача

\ЛичныйКондуит{0mm}{8mm}


%\СделатьКондуит{4.7mm}{7.8mm}
% %\GenXMLW


\end{document}

\раздел{Разное}

\задача
Докажите, что если выпуклая функция $f:\R\rightarrow\R$ ограничена,
то она постоянна.
\кзадача

\задача
Любая ли выпуклая функция $f:\R\rightarrow\R$ имеет в каждой
точке правую и левую касательные?
\кзадача

\сзадача Пусть $f\colon\R\to\R$~--- дважды дифференцируемая функция
на отрезке $[0,a]$,  $f(0)=f(a)=0$ и $f''$
непрерывна на отрезке $[0,a]$.%\\
\вСтрочку
%\сНовойСтроки
\пункт
Докажите, что при
$a=\pi$ справедливо утверждение:  \лк{}существует такая точка
$\xi\in(0,a)$, что $f''(\xi)+f(\xi)=0$\пк.
\пункт
Верно ли утверждение предыдущего пункта для $a=3$?
\кзадача


\сзадача
Пусть $P(x)$ --- многочлен ненулевой степени, разлагающийся на линейные
множители с действительными коэффициентами, прич\"ем $P'(0)=P''(0)=0$.
Докажите, что $P(0)=0$.
\кзадача

\ссзадача
Функция $f$ определена и бесконечно дифференцируема на $\R$,
обозначим $f^{(n)}$ ее $n$-тую производную.
Пусть для каждого $a\in\R$ найдется такое $n\in\N$, что
$f^{(n)}(a)=0$. Докажите, что $f$ --- многочлен.
\кзадача

\ЛичныйКондуит{0mm}{5mm}


%\СделатьКондуит{4.7mm}{7.8mm}

\end{document}

\задача
Под каким углом пересекаются кривые:
\вСтрочку
\пункт
$y=x^2$ и $x=y^2$;
\пункт
$y=\sin x$ и $y=\cos x$?
\кзадача

\задача
%\пункт
Параллельный пучок лучей, падающий на параболу $y=x^2$ по
вертикали сверху, отражается от не\"е по закону
\лк угол падения равен углу отражения\пк.
Докажите, что все лучи этого пучка после первого отражения
пройдут через одну и ту же точку, и найдите эту точку.
%\пункт
%Решите эту задачу для произвольной параболы $y=ax^2+bx+c$, где $a>0$.
\кзадача

