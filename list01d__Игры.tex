% !TeX encoding = windows-1251
\documentclass[a4paper,12pt]{article}
\usepackage{newlistok}
%\documentstyle[11pt, russcorr, listok]{article}

%\УвеличитьШирину{1.2truecm}
%\УвеличитьВысоту{3.5truecm}
\УвеличитьШирину{1.1truecm}
\УвеличитьВысоту{2.5truecm}
\hoffset=-2.5truecm
\voffset=-22truemm


\Заголовок{Игры}
\НомерЛистка{1д}
\ДатаЛистка{09.2013}

\begin{document}

\СоздатьЗаголовок

{\small {\noindent} Если %Вам кажется, что
в задаче ничего не спрашивается, то
ответьте, какой игрок может всегда выигрывать,
как бы ни играл другой?}

\раздел{I. Ответный ход.}

{\small В задачах этого раздела можно указать стратегию
игрока, который выигрывает. Чтобы её найти, полезно бывает
рассмотреть частные случаи или упростить задачу (например, решить задачу
4 сначала для полосок $1\times3$, $1\times4$, $\dots$).
}

\vspace*{1truemm}



\задача
\вСтрочку
В коробке $n$ конфет. Двое по очереди берут себе из коробки
1, 10 или 11 конфет. Выигрывает взявший последнюю.
\пункт Кто выигрывает при $n=1,2,\dots,30$?
\пункт  А при $n=100$?
\кзадача

\задача
В куче $2009$ камней. Двое по очереди берут по
\вСтрочку
\пункт 1 или 2;
\пункт 1 или $m$;
\пункт 1, 2 или $m$ камней.
Выигрывает взявший последний камень.
\кзадача

\задача На крайней правой клетке доски $1\times 20$ стоит фишка.
Два игрока по очереди сдвигают эту фишку (вправо или влево)
на любое число клеток, которое еще не встречалось при
выполнении предыдущих ходов. Проигрывает тот, кто не может
сделать ход.
\кзадача


\задача
Имеется полоска $1\times2013$.
\вСтрочку
\пункт В двух;
\пункт в трёх;
\пункт в $n$ самых правых клетках стоят фишки (по одной в клетке).
Игрок на своём ходу должен одну из фишек переставить влево
на любую незанятую клетку. Кто не может сходить, тот
проиграл.
\кзадача

\vspace*{.5mm}
{\small
%Иногда помогает \выд{анализ с конца}: можно рассмотреть
Назовём позиции, из которых игрок выигрывает одним ходом,
\выд{выигрышными}. Если игрок своим ходом обязательно попадает
в такую позицию, то он находится в \выд{проигрышной} позиции
(после его хода соперник выиграет). % одним ходом).
Если же игрок своим ходом может попасть в проигрышную позицию,
то он находится в выигрышной позиции (сможет выиграть).
Последовательно находя выигрышные и проигрышные позиции,
\выд{начиная с конца},
можно узнать, кто выиграет и найти стратегию.
}

\vspace*{1mm}

\задача
В коробке лежат 300 спичек. Двое по очереди
берут из коробка не более половины имеющихся в нём спичек.
Проигрывает тот, кто не может сделать ход.
\кзадача


\задача
Ферзь стоит в левом нижнем углу клетчатой доски $10\times12$. За один ход его
можно передвинуть на любое число клеток вправо, вверх или по диагонали
\лк вправо-вверх\пк. Играют двое, ходят по очереди. Выигрывает тот,
кто поставит ферзя в правый верхний угол.
\кзадача


\раздел{II. Симметрия.}

{\small Иногда игрок выигрывает с помощью
\лк симметричной стратегии\пк: например, дублирует ход предыдущего
игрока.}

\vspace*{1truemm}

\задача
Есть две кучи камней:
\вСтрочку
\пункт в каждой по 20;
\пункт в одной --- 30, в другой --- 20.
Двое по очереди берут любое число камней из
любой кучки (но не из двух сразу). Выигрывает взявший последний камень.
\пункт А если есть три кучи по 20 камней?
\пункт А если есть четыре кучи по~20~камней?
\кзадача

\задача
Двое по очереди кладут пятаки на круглый
стол так, чтобы они не накладывались друг на друга. Кто не может
сходить, тот проиграл.
\кзадача

\задача
Двое играют на доске $m\times n$. В первом
столбце стоят фишки первого, а в последнем столбце стоят фишки
второго. На своём ходу игрок может передвинуть свою фишку в
строке, не отрывая её от доски, то есть фишка не может
перепрыгнуть через фишку противника. Кто не может сходить, тот
проиграл.
\кзадача

\задача
\вСтрочку
У ромашки \пункт 12; \пункт 11 лепестков. За ход
разрешается оборвать один или два рядом растущих лепестка.
Выигрывает сорвавший последний лепесток.
\кзадача

\задача
[Брюссельская капуста.] В каждый момент на плоскости нарисовано несколько
\лк кочанов\пк: точек с четырьмя хвостами каждая, причём часть
\лк кочанов\пк\ принадлежат одному игроку,
часть --- другому; некоторые из хвостов  соединены линиями так, что из
каждого хвоста выходит не более  одной  линии, и никакие две линии не
пересекаются. Двое ходят по очереди. Ход состоит в том, что игрок проводит
ещё одну линию между двумя свободными хвостами своих точек так,
что все
ограничения остаются в силе, и на  этой  линии ставит новую свою точку с
двумя свободными хвостами по  разные  стороны новой линии. Сначала
на плоскости нарисованы $n$ кочанов одного игрока
и $n$ кочанов другого. Проигрывает тот, кто не может сделать ход.
\кзадача

\break

\раздел{III. Геометрия.}


%\vspace*{-2truemm}
%{\small .}

%\vspace*{1truemm}

\задача
В клетчатом квадрате $100\times100$ двое по
очереди ставят фигурки.
Первый ставит квадрат $2\times2$, второй
--- уголок из трёх клеток (так, что
фигурки занимают целое число клеток и не перекрываются).
Кто не может сходить, тот проиграл.
\кзадача

\задача
На клетчатой доске $2013\times2013$ в центре
стоит фишка. Двое по очереди передвигают фишку на одну из соседних
(по стороне) клеток, если эта клетка ранее ни разу не была занята
фишкой. Кто не может сделать ход, тот проиграл.
\кзадача

\задача
На бесконечной доске двое играют в крестики-нолики.
Выигрывает тот, кто поставит 5 своих знаков в ряд
по вертикали или горизонтали.
Докажите, что при правильной игре второй
\вСтрочку
\пункт не выигрывает;
\пункт не проигрывает.
\кзадача

\задача
В комплекте для игры в домино 28 разных костяшек $2\times1$.
Все костяшки состоят из двух клеток $1\times1$, на каждой клетке выбито от
одной до шести точек или ничего не выбито.
Двое по очереди берут по костяшке и строят из них разветвляющуюся цепочку
(очередную костяшку можно приложить к одной из клеток цепочки тем концом,
на котором выбито то же число точек, что и на этой клетке цепочки).
Проигрывает тот, кто не может сделать ход.
Кто выиграет, если игроки видят все костяшки?
\кзадача

\раздел{IV. Передача хода.}

{\small В задачах этого раздела можно узнать, кто выигрывает
при правильной игре, не указывая %конкретную
стратегию.} % для выигрывающего игрока.}

\vspace*{1truemm}

\задача[Двойные шахматы] Двое играют в шахматы, но каждый делает
по два хода сразу. Докажите, что у второго нет выигрышной стратегии.
\кзадача


\задача
[Игра \лк Щёлк\пк]
Есть прямоугольная шоколадка, разделённая бороздками на дольки.
Двое по очереди выбирают любую ещё не съеденную дольку и съедают
её вместе со всеми дольками,
расположенными от выбранной не ниже и не левее.
Съевший последнюю дольку проигрывает.
%Рассмотрите случаи размера шоколадки а)~$2\times8$; б)~$8\times8$.
%В прямоугольном городе $n\times m$, разбитом
%улицами на кварталы $1\times1$, двое чиновников по очереди
%устанавливают на перекрёстках фонари. Каждый фонарь освещает угол
%в $90^\circ$, стороны которого идут вдоль улиц вправо и вверх. Правила игры
%такие, что каждый новый установленный фонарь должен освещать хотя
%бы один ранее не освещённый квартал. Проигрывает тот, кто не может
%сделать ход.
\кзадача

\задача
На доске написаны числа 1, 2, 3, \dots, 1000. Играют двое,
ходят по очереди. За ход игрок вычеркивает какое-нибудь число
и все его делители. Проигрывает тот, кто не может сделать ход.
\кзадача

\задача
Фома и Ерёма делят кучу из 25 монет в 1, 2, 3, \dots, 25
алтынов.  Каждым ходом один из них выбирает монету из кучи, а другой
говорит, кому её отдать. Первый раз выбирает Фома, далее --- тот, у кого
сейчас больше алтынов, при равенстве --- тот же, кто в прошлый раз.  Может ли
Фома действовать так, чтобы в итоге обязательно получить больше алтынов,
чем Ерёма,
или Ерёма всегда сможет~Фоме~\hbox{помешать?}
\кзадача

\раздел{V. Разное.}

%\vspace*{-2truemm}
%{\small .}

%\vspace*{1truemm}

\задача
В ряд по  горизонтали расположены 179 клеток.  На  самой
левой стоит белая фишка, на самой правой ---  черная.  Двое  ходят  по
очереди. Каждый двигает свою фишку на 1, 2, 3, 4 или 5 клеток  вперед  или
назад по свободным клеткам. Перепрыгивать через фишку соперника  нельзя.
Кто не может сделать ход --- проигрывает.
\кзадача


\задача
Двое играют в следующую игру. Есть кучка камней.
Первый каждым своим ходом берет 1 или 10 камней.
Второй каждым своим ходом берет $m$ или $n$ камней.
Ходят по очереди, начинает первый.
Тот, кто не может сделать
хода, проигрывает. Известно, что при любом
начальном количестве камней первый  всегда
может играть так, чтобы выиграть (при любой игре второго).
Какими могут быть $m$ и $n$?
\кзадача

\задача
Король за ход может поставить по крестику в любые две свободные клетки
бесконечного листа бумаги. Министр за ход может поставить нолик в любую
свободную клетку. Король хочет поставить 100 крестиков в ряд.
Может ли министр ему помешать?
\кзадача


\задача [Игра Ним] Имеется три кучки камней:
\вСтрочку
\пункт 7, 8 и 9;
\пункт любые кучки. Двое по очереди берут любое количество камней из
одной кучки. Выигрывает взявший последний камень.
\кзадача

\задача
Первоначально на доске написано число 2009!
(то есть $1\cdot2\cdot3\cdot\ldots\cdot2009$).
Два игрока ходят по очереди. Игрок в свой ход вычитает
из написанного  числа какое-нибудь натуральное число,  которое
делится не более  чем  на 20 различных простых чисел (так,  чтобы
разность была неотрицательна), записывает  на  доске эту разность,
а старое число стирает. Выигрывает тот, кто получит 0.
Кто из играющих --- начинающий, или его соперник, --- может %гарантировать
обеспечить
себе победу, и как ему %следует
играть?
\кзадача

\end{document}
\раздел{Запас (задачи, которые предлагается исключить)}


\задача
В неокрашенной клетчатой полоске $1\times2007$ двое по
очереди закрашивают два или три соседних квадратика
(которые еще не закрашены). Проигрывает тот, кто
не может сделать ход.
\кзадача

\задача
Двое по очереди выписывают на доске числа. На
$3k+1$-ом ходу можно писать только единицу, на $3k+2$-ом --- 1 или
2, на $3k$-ом --- 1, 2 или 3. Выигрывает тот, после чьего хода
сумма чисел на доске станет равной 90.
\кзадача

\задача
Лежит куча из $n$ камней. Двое по очереди
либо делят любую одну кучку на две, либо берут один камень из
какой-нибудь кучки. Выигрывает взявший последний камень.
\кзадача

