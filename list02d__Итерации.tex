% !TeX encoding = windows-1251
\documentclass[a4paper,12pt]{article}
\usepackage{newlistok}
%\documentstyle[11pt, russcorr, listok]{article}

%\УвеличитьШирину{1.2truecm}
%\УвеличитьВысоту{3.5truecm}
\УвеличитьШирину{.3truecm}
\УвеличитьВысоту{.2truecm}
\hoffset=-2.45truecm
\voffset=-23truemm

%\ВключитьКолонтитул


\Заголовок{Итерации}
\НомерЛистка{2д}
\ДатаЛистка{10.2013}

\begin{document}

\СоздатьЗаголовок

\задача
Над цепью озер летела стая %белых
гусей.
На каждом озере садилась половина гусей и еще
пол гуся, а остальные летели дальше. Все гуси сели на
семи озерах. Сколько гусей было в стае?
\кзадача

\задача
В двух сосудах находится по 1 л воды.
Из первого переливают половину имеющейся в н\"ем воды во второй,
затем из второго переливают треть имеющейся в н\"ем воды в
первый, затем из первого переливают четверть имеющейся в н\"ем воды во второй и т.~д.
Сколько воды будет в каждом сосуде после 100 переливаний?
\кзадача

%\задача
%В каждом из двух сосудов находится по $A$ литров воды.
%Из первого сосуда переливают половину имеющейся в нем воды во второй,
%затем из второго переливают треть имеющейся в нем воды в первый,
%затем из первого переливают четверть имеющейся в нем воды во второй,
%и т.~д. Сколько воды окажется в каждом из сосудов после 100
%переливаний?
%\кзадача

\задача
К 50 чёрным бактериям попадает белая бактерия.
Ежесекундно одна белая бактерия убивает одну чёрную, % бактерию,
после чего все бактерии делятся надвое. Докажите, что %рано или поздно
все чёрные бактерии будут убиты. %, и выясните,
Когда?
%В какой момент это произойдет?
\кзадача

\задача
Последовательность начинается с $3^{1000}$,
каждый следующий равен сумме цифр предыдущего.
Докажите, что члены последовательности сначала
%до некоторого места
уменьшаются, %но, начиная с некоего члена,
%до тех пор, пока не станут %, начиная с некоторого,
а потом равны одному и тому же числу (какому?). %Найдите это число.
\кзадача

\задача
Тысяча ребят стоят по кругу и выбирают водящего.
Они считаются так: первый остается~в круге, следующий
(по часовой стрелке) выходит из круга, следующий %за ним
остается, следующий выходит, и т.~д., через одного по кругу.
Круг сужается, пока в нём не останется~один человек.
На каком месте он стоял сначала?
%, если всего было
%\вСтрочку
%\пункт 16; %человек;
%\пункт 17; %человек;
%\пункт 35; %человек;
%\пункт 500 ребят?
\кзадача


\задача
Бесконечную строку нулей и единиц ~\hbox{0110100110010110\dots}
составили так. Сначала написали нуль.
Затем сделали бесконечное число шагов.
На каждом шаге к уже написанному куску~строки %последовательности
дописывали новый кусок той же длины, получаемый из него
заменой всех нулей на единицы, а единиц --- на нули.
\вСтрочку
\пункт Какая цифра стоит в строке на 1000-м месте?
\пункт Периодическая ли  эта строка (начиная с какого-то места)?
%периодической?
\кзадача

\задача
%Представим себе, что
На бесконечном листе клетчатой бумаги какие-то 100 клеток \лк заболели\пк. Каждый час одновременно происходят такие изменения: если клетка больна, а две клетки, снизу и слева от неё, здоровы, то она выздоравливает;
если клетка здорова, а две клетки,
снизу и слева от неё, больны, то она заболевает
(остальные клетки не меняются). %остаются такими, как были).
Докажите, что через некоторое время все клетки будут здоровы.
\кзадача


\задача
\вСтрочку
\пункт
На доске написаны натуральные числа $x$ и $y$.
Петя пишет на бумажку одно из этих чисел,~а~на доске
уменьшает другое число на 1. С новыми двумя числами на доске
он снова проделывает ту же операцию, и т.д., пока
одно из чисел на доске не станет нулём.
Чему будет в этот момент равна сумма чисел на бумажке?
\пункт Та же задача, но Петя повторяет такую операцию:
когда на доске $a$ и $b$, где $a\leq b$,
он
пишет на бумажку $a^2$, а затем заменяет
числа на доске числами $a$ и $b-a$.
\пункт %Каков геометрический смысл этой задачи?
Каков геометрический смысл этих процессов?
%Есть ли у задач из пунктов а) и б) геометрический смысл?
\кзадача


\задача
Перед шеренгой из $N$ солдат стоит капрал и командует: \лк Нале-ВО!\пк.
По %этой
команде~часть солдат поворачиваются налево, остальные ---
направо. Затем %через
каждую секунду каждые два солдата, стоящие
лицом друг к другу, поворачиваются друг к другу затылками.
%\вСтрочку
%\пункт
%Докажите, что когда-то %через конечное время
%движение прекратится.
%\пункт
%Оцените,
Через сколько секунд движение заведомо прекратится?
%это~\hbox{заведомо произойдёт?}
\кзадача




%\задача
%\вСтрочку
%\пункт
%Как с помощью только двух сосудов объёмом 7 и 11 л, крана с водой
%и раковины, куда можно выливать лишнюю воду, набрать ровно
%2 л воды?
%\пункт
%Пусть сначала есть сосуды объёмом $m$ и $n$ л, где
%$(m,n)=1$ и $m\leq n$. Можно ли набрать любое целое число литров
%от~1~до~$m$?
%\кзадача

\vfill
\ЛичныйКондуит{0mm}{6mm}
\ОбнулитьКондуит
\newpage

\задача
Аня, Боря и Витя сидят по кругу за столом и едят орехи. Сначала все
орехи у Ани. Она делит их поровну между Борей и Витей, а остаток (если
он есть) съедает. Затем всё повторяется: каждый следующий (по часовой
стрелке) делит имеющиеся у него орехи поровну между  соседями, а остаток
(если он есть) съедает. Орехов больше 3.
Докажите, что
\вСтрочку
\пункт хотя бы один орех будет съеден;
\пункт не все орехи будут съедены.
\кзадача

%\задача
%На листе был написан 0. К нему дописали 1, потом  дописали 10,
%потом --- 1001, и т.~д.:
%каждый раз дописывали к уже имеющейся строке цифр
%новую строку, получаемую из имеющейся заменой
%всех нулей на единицы, а единиц --- на нули.
%Так сделали бесконечное число раз и по\-лу\-чи\-ли бесконечную
%строку:~\hbox{0110100110010110\dots.}
%\пункт Какая цифра стоит в этой строке на 1000-м месте?
%\пункт Будет ли эта строка периодической (начиная с какого-то места)?
%\кзадача





\задача
По кругу стоят $n$ корзин, в одной %из них
лежит яблоко,
а остальные --- пусты. За ход %разрешается
можно из любой непустой
корзины забрать  %одно
яблоко, а в две соседние с ней корзины
добавить по яблоку (запас яблок очень большой).
При каких $n$ удастся %можно добиться того,
%чтобы во всех корзинах стало поровну яблок?
сделать число яблок в корзинах одинаковым?
\кзадача

\задача
За круглым столом сидят 7 гномов. Перед каждым стоит кружка,
в некоторые налито~молоко. Один из гномов разливает
все своё молоко в кружки остальных поровну. Затем его сосед справа
делает то же самое, и т.~д. Когда последний (седьмой) гном разлил остальным своё молоко, в каждой кружке оказалось исходное количество молока. Всего в кружках 3 литра молока. Сколько молока было в каждой кружке сначала?
\кзадача

\задача
Дано несколько белых и чёрных точек, некоторые соединены отрезками.
Назовём точку особой, если более половины соединённых с ней точек %окрашены
другого цвета.
%За один ход разрешается выбрать любую точку, цвет которой
%отличается от цвета большинства  соединённых с ней точек,
Если есть особые точки, выбирают
любую из них и
перекрашивают в противоположный
цвет. % большинства соединённых с ней точек.
Докажите, что %после нескольких таких
в какой-то момент
%перекрашивания когда-то прекратятся.
особых точек не останется.
\кзадача

\задача
Есть два %тр\"ехлитровых
больших сосуда. В одном --- 1 л воды, в другом --- 1 л
2\%-го раствора соли. %Разрешается
Можно переливать любую
часть жидкости из одного сосуда в другой (и перемешивать).
%Можно
Удастся ли за несколько таких переливаний получить 1,5\%-й
раствор в сосуде, где вначале была вода?
\кзадача

\задача
Гномы некой страны живут в белых и красных домиках. Ежегодно
те гномы, у кого больше~поло\-вины друзей жили последний год в домиках другого цвета, меняют цвет домика (а другие --- не меняют).~Дока\-жите, что с какого-то момента цвет одних
домиков %вовсе
не будет меняться,
а других --- будет меняться ежегодно. %каждый год.
\кзадача

\задача
На некоторых клетках доски $10\times10$ стоят фишки. За %один
ход Петя
одновременно ставит новые фишки на все пустые клетки,
у которых хотя бы две соседние (по стороне) клетки заняты фишками.
Он делает~ходы, пока добавляются новые фишки. % по этим правилам.
\вСтрочку
\пункт Приведите пример расстановки фишек,
при которой Петя~сделает более 40 ходов.
\пункт Можно ли так расставить
фишки, чтобы Петя сделал более 60 ходов?
\пункт А более~64~ходов?
\кзадача

\задача
На бесконечную белую плоскость посадили ограниченную чёрную кляксу. Каждую секунду все точки меняют свой цвет по такому закону. Точка становится чёрной, если больше половины площади круга радиуса 1 с центром в ней --- чёрная, иначе становится белой. Может ли клякса жить вечно?
\кзадача


%\задача
%Дан баллон объёмом 100 л с газом под давлением 100 атм. и два пустых баллона %по 50 л. Как с помощью простых домашних средств перекачать газ из первого %баллона в два других, не потеряв давления?
%Есть 10 кг грязного белья и 10 л воды. При погружении любой части белья в %любую часть воды грязь распределяется в их общей массе равномерно. Возможно %ли уменьшить количество грязи в белье \пункт в 2,5 раза; \пункт в 3 раза?
%\кзадача

\vspace*{-2mm}
\ЛичныйКондуит{0mm}{6mm}
\vspace*{-3mm}
%\GenXMLW

%\СделатьКондуит{6.5mm}{7mm}


%\ЛичныйКондуит{0mm}{7mm}

%\СделатьКондуит{8mm}{7mm}

\end{document}

\break

\задача[Исследовательская задача]
На столе у чиновника Министерства Околичностей лежит $n$ томов
Британской энциклопедии, сложенных в несколько стопок.
Стопки лежат на столе в один ряд.
Каждый день, приходя на работу, чиновник берет по одному тому
из каждой стопки, образует из них новую стопку, которую кладет
в начало ряда,
%располагает стопки по количеству томов (в невозрастающем порядке)
и записывает в ведомость
количество томов в каждой стопке. Например, если в первый день в ведомости
записано $(8,3,1,1)$, то на следующий день запись будет
$(4,7,2)$, потом --- $(3,3,6,1)$, $(4,2,2,5)$ и т.~д.
\сНовойСтроки
\пункт Пусть $n=36$. Разложите книги %на стопки
так, чтобы чиновник %всегда %ежедневно
делал в ведомости одну и ту же запись.
\пункт Что будет записано в ведомости на 31-й день,
если в первый день там записано $(4,4,4)$?\
\пункт Что чиновник запишет через месяц, если $n=6$?
(Начальное разбиение на стопки~\hbox{неизвестно.)}  % произвольным.)
%\УстановитьГраницы{5mm}{5mm}
{\small \indent Чтобы проследить за путём конкретной книги,
будем считать, что чиновник берёт %самую
нижнюю книгу из
первой стопки, на неё кладёт %самую
нижнюю книгу из второй стопки,
и т.~д. Каждая книга имеет две координаты: текущий номер её стопки
и высота внутри стопки. Всё это удобно изображать
на клетчатой бумаге в первой координатной четверти:
книге отвечает %соответствует
закрашенная клетка с теми же координатами.}
%\ВосстановитьГраницы
\УстановитьГраницы{0cm}{9.5truecm}
\пункт Докажите, что действие чиновника можно описать так:
он отрезает нижнюю строчку от закрашенной фигуры, сдвигает то,
что осталось, на одну клетку вправо и вниз, а отрезанную строчку поворачивает
на $90^\circ$ (превращая её в первый столбик), см. рис.;
затем он, возможно, сдвигает некоторые
столбики влево (чтобы не было пустых столбиков).
\ВосстановитьГраницы
\пункт Какой путь проделала книга $(2,4)$ из пункта б) этой задачи?
\пункт Докажите, что при действиях чиновника сумма координат
каждой книги либо не изменяется, либо уменьшается.
\пункт Докажите, что, начиная с какого-то момента, стопки будут
располагаться по числу книг в невозрастающем порядке,
и каждая книга, начиная с этого момента, будет двигаться по циклу.
\пункт Докажите, что если $n=1+2+3+\dots+k$ для некоторого $k$,
то, начиная с какого-то момента,
чиновник ежедневно будет записывать в ведомость одно и то же.
Что именно?
\пункт Докажите, что если $n\ne1+2+3+\dots+k$ ни для какого $k$,
то, начиная с какого-то момента, записи в ведомости
будут циклически повторяться.
\пункт Докажите, что период $t$, с которым будут повторяться записи
в предыдущем пункте, удовлетворяет условию
$(t-1)t<2n<t(t+1).$
\кзадача

\vspace*{-3truemm}
\раздел{***}

\vspace*{-2truemm}


%\задача
%Гномы из Сказочной страны живут в белых и красных домиках. Ежегодно
%они одновременно красят свои домики, но меняют цвет
%домика только те гномы, у кого больше половины друзей
%жили последний год в домиках другого цвета.
%Докажите, что наступит год, начиная с которого цвет некоторых
%домиков %вовсе
%не будет меняться,
%а остальных --- будет меняться ежегодно. %каждый год.
%Вокруг поляны стоят 12 домиков, покрашенных в белый и красный цвета,
%в которых живут 12 гномов. В январе первый гном красит свой дом
%в тот цвет, в который окрашены дома большинства его друзей
%(или в прежний, если ).
%В феврале это же делает второй (по часовой стрелке) гном,
%в марте --- третий и т.д. Докажите, что наступит момент, после которого
%цвет дома у каждого гнома перестанет меняться.
%\кзадача


\задача
Дано число 1. За ход разрешается
умножить имеющееся число на 2 или прибавить к нему~1. За какое
наименьшее число ходов можно получить число
\вСтрочку
\пункт 10;
\пункт 1000?
\кзадача




\задача
Все натуральные числа выписали подряд без промежутков на бесконечную
ленту:\\ 1234567891011\dots.
Затем ленту разрезали на полоски по 7 цифр в каждой. Докажите, что
любое 7-значное число встретится
\вСтрочку
\пункт  хотя бы на одной полоске;
\пункт  на бесконечном числе полосок.
\кзадача



\end{document}
