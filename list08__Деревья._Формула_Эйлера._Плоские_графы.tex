% !TeX encoding = windows-1251
\documentclass[a4paper,11pt]{article}
\usepackage{newlistok}
%\usepackage{tikz}
%\usetikzlibrary{calc}

%\documentstyle[11pt, russcorr, listok]{article}
\newcommand{\del}{\mathrel{\raisebox{-.3 ex}{${\vdots}$}}}

\УвеличитьШирину{1.4truecm}
\УвеличитьВысоту{3.3truecm}
\hoffset=-2.5truecm
\voffset=-27truemm

\Заголовок{Деревья. Формула Эйлера. Плоские графы}
%Графы 2}
%\Подзаголовок{Деревья. Формула Эйлера. Плоские графы}

\НомерЛистка{8}
\ДатаЛистка{01.2014}

\begin{document}

\thispagestyle{empty}

\СоздатьЗаголовок

%\vspace*{-1truemm}

\задача
Докажите, что граф является \выд{деревом}
(то есть связным и без циклов) если и только если каждые
две его вершины соединены ровно одним пут\"ем с различными р\"ебрами.
\кзадача

\задача
Верно ли, что в дереве с более чем одной вершиной найдутся
%\вСтрочку
%\пункт одна;
%\пункт
две \выд{висячие}
вершины? (Вершина называется \выд{висячей}, если из не\"е выходит ровно одно
ребро.)
\кзадача

\задача
$N$-угольник разбит на треугольники несколькими диагоналями, не
пересекающимися нигде, кроме вершин.
Построим граф, соответствующий этому разбиению: отметим внутри каждого
треугольника точку (это будут вершины
графа) и будем соединять две точки ребром ровно в том случае,
когда соответствующие точкам треугольники имеют общую сторону. Докажите, что \quad
%\сНовойСтроки
\вСтрочку
\пункт построенный граф будет деревом;\\
\пункт %Докажите, что
хотя бы у двух
треугольников разбиения две стороны
совпадают со сторонами $N$-угольника (при $N>3$).
\кзадача


\задача
Как связаны число вершин и число р\"ебер произвольного дерева?
\кзадача

\опр
Граф $O$ называется {\it остовом} связного графа $G$,
если $О$ имеет те же вершины, что и $G$,
получается из $G$ удалением некоторых р\"ебер и является деревом.
\копр

\задача
Всякий ли связный граф имеет остов? Может ли граф иметь несколько остовов?
\кзадача

\задача Волейбольная сетка имеет вид прямоугольника размером $50\times600$
клеток. Какое наибольшее число вер\"евочек можно перерезать так, чтобы
сетка не распалась на куски?
\кзадача

\задача
Всегда ли в связном графе можно удалить некоторую вершину
вместе со всеми выходящими из не\"е р\"ебрами так, чтобы граф
остался связным?
\кзадача

%\задача
%Докажите, что любой граф можно нарисовать в пространстве так,
%чтобы его рёбра не пересекались нигде, кроме вершин.
%\кзадача

\опр \выд{Плоским графом}  называется граф, который можно нарисовать
на плоскости так, что его
р\"ебра не будут пересекаться (нигде, кроме вершин). При этом
граф разделит плоскость на части (одна из которых
неограничена), все они называются \выд{гранями} графа.
\копр


\задача
Докажите, что связный плоский граф является эйлеровым
если и только если его грани можно раскрасить в два цвета так,
чтобы любое ребро принадлежало границам двух граней разного цвета.
\кзадача

\задача [Формула Эйлера]
Докажите, что для каждого связного плоского графа с $в$ вершинами,
$р$ р\"ебрами и $г$ гранями имеет место равенство: \ $в-р+г=2$.
\кзадача

\опр
Граф без кратных рёбер и петель называется \выд{простым}.
Простой граф называется \выд{полным}, если любые две его различные вершины
соединены ребром.
\копр

\задача
Для каких простых плоских графов верны неравенства:
\вСтрочку
\пункт $2р\geq3г$;
\пункт $р\leq3в-6$?
\кзадача



\задача
Является ли плоским полный граф с пятью вершинами?
\кзадача

\задача
Можно ли построить три дома, вырыть три колодца и соединить тропинками каждый
дом с каждым колодцем так, чтобы тропинки не пересекались?
\кзадача

%\задача
%Как изменится формула Эйлера, если связный граф с непересекающимися
%р\"ебра\-ми нарисовать не на плоскости, а на \пункт сфере?
%\пункт торе (бублике, или сфере с ручкой)?
%\пункт кренделе (сфере с двумя ручками)?
%\пункт сфере с $g$ ручками?
%\кзадача

\задача
Пусть $Г$ --- любой простой плоский граф. Докажите, что
\вСтрочку
%\сНовойСтроки
\пункт
в графе $Г$ есть вершина степени меньше~6;
\пункт вершины графа $Г$ можно %правильно
раскрасить
в 6 или менее цветов так, что никакие две вершины одного цвета
не будут соединены ребром;
\спункт можно ли так же раскрасить граф не более чем в 5 цветов?
\кзадача

%\задача
%Можно ли разбить какой-нибудь выпуклый шестиугольник на несколько
%меньших выпуклых
%шестиугольников так, чтобы любые два шестиугольника разбиения
%\кзадача


\задача
Можно ли разбить какой-нибудь шестиугольник на выпуклые шестиугольники
так, чтобы выполнялось условие: границы любых двух из этих шестиугольников
(включая исходный) либо не имеют общих точек,
либо имеют только общую вершину или общую сторону?
\кзадача

\задача
На плоскости отмечено несколько точек, никакие три %из которых
не лежат на одной прямой. Двое %играют в такую игру: они
по очереди соединяют какие-то две ещ\"е не соедин\"енные
точки отрезком так, чтобы отрезки не пересекались нигде, кроме
отмеченных точек. Кто не может сделать ход --- проиграл.
Зависит ли исход от того, как играют соперники?
\кзадача

\задача Докажите формулу Эйлера
\вСтрочку
\пункт для произвольного
связного графа с непересекающимися р\"ебра\-ми,
нарисованного на сфере;
\пункт для произвольного
выпуклого многогранника.
\кзадача

\задача
Дан выпуклый многогранник, грани которого являются $n$-угольниками,
и в каж\-дой вершине сходится $k$ граней. Докажите, что
$1/n+1/k=1/2+1/r$, где $r$ --- число его р\"ебер.
\кзадача

\vspace*{.1truecm}

\putpicture{0mm}{-22mm}{pct_graph_tet}
\putpicture{40mm}{-17mm}{pct_graph_cub}
\putpicture{80mm}{-12mm}{pct_graph_oct}
\putpicture{120mm}{-7mm}{pct_graph_dod}
\putpicture{160mm}{-2mm}{pct_graph_ico}

\vspace*{.2truecm}

\задача
Выпуклый многогранник называют \выд{правильным}, если
все его грани ---~\hbox{правильные}\break $n$-угольники, и
в каждой его вершине сходится $k$ граней. Докажите,
что любой такой многогранник --- либо
тетраэдр, либо куб, либо октаэдр, либо додекаэдр,
либо икосаэдр (см.~рис.).
\кзадача




\ЛичныйКондуит{0mm}{6mm}

% %\СделатьКондуит{7mm}{7mm}

\end{document}
