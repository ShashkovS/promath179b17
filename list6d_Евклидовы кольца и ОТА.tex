% !TeX encoding = windows-1251
\documentclass[a4paper,12pt]{article}
\usepackage[mag=940]{newlistok}

\УвеличитьВысоту{2.1truecm}
\УвеличитьШирину{0.8truecm}

\renewcommand{\spacer}{\vspace{.7mm minus .5mm}}
\renewcommand{\spacer}{\vfill}

\Заголовок{Евклидовы кольца и основная теорема арифметики}
\НомерЛистка{6д}
\ДатаЛистка{10.2014}

\begin{document}
\СоздатьЗаголовок

\small
Математики --- люди ленивые и не любят много раз доказывать один и тот же факт в разных случаях.
Поэтому встретив что-то полезное, они стремятся доказать его в максимальной общности,
чтобы одним махом объять как можно больше случаев.
Для этого иногда приходится анализировать множество примеров, когда факт \лк работает\пк,
искать общее в них и откидывать несущественное.
В данном листке мы будем заниматься теорией делимости в евклидовых кольцах.
Наиболее известными примерами евклидовых колец являются целые числа и многочлены от одной переменной.
Есть и другие примеры.
Если говорить коротко, то евклидово кольцо --- это произвольное множество \лк чисел\пк,
которые можно складывать, умножать и делить с остатком.
Впрочем, тут есть тонкости, о которых будет речь ниже.

\normalsize
\опр
\лк Число\пк $x$ называется \выд обратимым, если найдётся такое число $y$, что $xy=1$.
\копр

\задача
Найдите все обратимые числа в кольцах\\
\пункт целых чисел;
\пункт рациональных чисел;
\пункт многочленов вещественной переменной.
\кзадача

\опр
Необратимое \лк число\пк $p$ называется \выд простым, если оно не может быть представлено в виде $p=ab$,
где $a$ и $b$ --- необратимые элементы.
\копр

\small
Основной целью листка является доказательство достаточно общей теоремы:
\теорема[Основная теорема арифметики]
В евклидовом кольце любой необратимый ненулевой элемент может быть разложен на простые множители,
причём это разложение единственно с точностью до перестановки множителей и умножения их на обратимые элементы.
\ктеорема

Оказывается, бывают такие \лк числа\пк, что в них разложение на простые множители совсем не единственно,
а в особенно клинических случаях некоторые ненулевые необратимые элементы вообще не могут быть разложены на простые множители.

Приведём пример \лк чисел\пк, в которых разложение на простые не единственно.
Для этого рассмотрим всевозможные выражения вида $\hc{a + b\sqrt{-5} \mid a,b\in\Z}$.
Такие числа обозначаются через $\Z[i\sqrt{5}]$.
Их можно складывать (покомпонентно) и умножать (при этом $\sqrt{-5}\cdot\sqrt{-5} = -5$).
Оказывается, число $6$ в них имеет два разложения: $6=2\cdot 3 = (1+\sqrt{-5})\cdot(1-\sqrt{-5})$.
Можно показать (упражнение), что каждое из чисел $2$, $3$, $1+\sqrt{-5}$ и $1-\sqrt{-5}$ является простым.
Кроме того, число $6$ делится и на $2$, и на $1+\sqrt{-5}$, но совершенно не делится на $2+2\sqrt{-5}$ (упражнение).
На этой прискорбной ноте начнём разбираться с тем, что же такое евклидово кольцо, и почему там такого не бывает.

\normalsize
\опр
\выд{Кольцом}%
\footnote{Если быть точным, то кольцо должно обладать только первыми пятью свойствами. 
В данном листке мы рассматриваем только коммутативные ассоциативные кольца с единицей.} 
называется множество $K$ с операциями сложения и умножения, обладающими следующими свойствами:
\vspace*{-1mm}
\begin{nums}{-5}
\item $a+b=b+a$ для любых $a,b\in K$ (\emph{коммутативность сложения\/});
\item $a+(b+c)=(a+b)+c$ для любых $a,b,c\in K$ (\emph{ассоциативность сложения\/});
\item в $K$ существует такой элемент 0 (\emph{нуль\/}), что $a+0=a$ для любого $a\in K$;
\item для всех $a\in K$ существует такой элемент $-a\in K$, что $a+(-a)=0$ (\emph{противоположный\/});
\item $a(b+c)=ab+ac$  для любых $a,b,c\in K$ (\emph{дистрибутивность\/});
\item $ab=ba$ для любых $a,b\in K$ (\emph{коммутативность умножения\/});
\item $a(bc)=(ab)c$ для любых $a,b,c\in K$ (\emph{ассоциативность умножения\/});
\item в $K$ существует такой элемент 1 (\emph{единица\/}), что $a\cdot1=a$ для любого $a\in K$;
\end{nums}
\копр
\vspace*{-3mm}
%\item если $ab=0$ для некоторых $a,b\in K$, то либо $a=0$, либо $b=0$ (\emph{отсутствие делителей нуля\/});

\small
Оказывается, мы уже много раз встречались с кольцами.
Например, числовые множества $\Z$, $\Q$, $\R$ являются коммутативными ассоциативными кольцами с единицей относительно
обычных операций сложения и умножения. Множество $2\Z$ чётных целых чисел тоже является кольцом, однако уже без единицы.
Множество многочленов, множество всех функций на числовой прямой, множество функций на любом подмножестве прямой тоже являются кольцами относительно обычных операций сложения и умножения функций.
Кольца могут быть конечными, например множество остатков по модулю $n$ также является кольцом.

\normalsize

\задача
\пункт
Докажите, что в любом кольце нуль и единица единственны;
\\\пункт
Докажите, что если в кольце хотя бы два элемента, то $0\ne1$;
\кзадача

\опр
Говорят, что элемент $a$ кольца $K$ \выд делится на элемент $b\in K$ (или что $b$ \выд делит $a$), 
если существует такой элемент $q\in K$, что $a=qb$.
\копр

\задача
Докажите, что $a\dv b$ и $b\dv a$ одновременно тогда и только тогда, когда $a=bc$, где элемент $c$ обратим.
Элементы $a$ и $b$ в этом случае называют \выд ассоциированными.
\кзадача

\опр
Ненулевой элемент $a$ кольца $K$ называется \выд{делителем нуля}, 
если найдётся такой ненулевой элемент $b\in K$, что $ab=0$.
\копр

\задача
Докажите, что в кольце без делителей нуля возможно сокращение:
из того, что $ac = bc$ и $c\ne 0$, следует, что $a=b$.
\кзадача

\задача
Приведите пример кольца с делителями нуля.
\кзадача

\newpage
\задача
Рассмотрим кольцо функций на множестве $X$.
В каких случаях в нём есть делители нуля?
\кзадача

\опр
Функция $N:K\setminus\hc{0}\to\Z_+$, которая каждому ненулевому элементу кольца ставит в соответствие неотрицательное число,
называется \выд нормой, если
\begin{nums}{-5}
\item $N(ab)= N(a)$, если $b$ обратим, и $N(ab) > N(a)$ иначе;
\item для любых $a,b\in K$, где $b\ne 0$ существуют такие $q,r\in K$, что $a=bq+r$ и либо $r=0$, либо $N(r) < N(b)$.
То есть в кольце возможно деление с остатком. Его единственности не требуется. 
\end{nums}
\vspace*{-3mm}
\копр

\задача
\пункт Докажите, что модуль целого числа определяет норму на целых числах;
\\\пункт Докажите, что степень многочлена определяет норму на многочленах от одной переменной;
\кзадача

\опр
Кольцо без делителей нуля, для которого существует норма, называется \выд евклидовым.
\копр

\соглашение
Далее говоря \лк число\пк мы имеем в виду элементы некоторого евклидового кольца $K$.
\ксоглашение


\опр
Если число~$d$ делит числа~$a$ и~$b$, то $d$~называется \выд{общим делителем} чисел $a$ и~$b$.
Любой из общих делителей чисел $a$ и~$b$ с наибольшей нормой называется \выд{наибольшим общим делителем} $a$ и~$b$
(обозначение: $(a,b)$). В~том случае, когда $(a,b)=1$, говорят, что числа $a$ и~$b$ \выд{взаимно простые}.
\копр

\соглашение
Пусть $a$ и~$b$\т два фиксированных числа.
В данном листке через $I$ будем обозначать множество всех чисел,
представимых в~виде $ax+by$ (где $x$ и~$y$\т любые числа).
\ксоглашение


\vspace*{-3mm}
\setcounter{problemnum}{10}
\ввзадача[о сумме идеалов]
Пусть $d$\т число с наименьшей нормой в~$I$. Докажите, что
\невСтрочку
\пункт
каждое число из $I$ делится на любой общий делитель чисел $a$ и~$b$ (а~значит, и~на $(a,b)$);
\пункт
каждое число из $I$ делится на~$d$;
\пункт
$d=(a,b)$;
\пункт
число $d=(a,b)$ является числом с наименьшей нормой, делящимся на любой общий делитель~$a$~и~$b$.
\кзадача


\vspace*{-8mm}
\ввзадача[Алгоритм Евклида]
Пусть $a$ и $b$\т два фиксированных числа. Будем последовательно заменять число с большей нормой на остаток от деления на число с меньшей. Докажите, что:
\невСтрочку
\пункт
все числа, которые мы будем получать, лежат в множестве $I$;
\пункт
в некоторый момент мы получим пару $(d, 0)$, $d\ne 0$;
\пункт
$d$ является наибольшим общим делителем $a$ и~$b$;
\пункт
Как именно для данных чисел~$a$ и~$b$ при помощи алгоритма Евклида искать такие числа~$x$ и~$y$, что $ax+by=(a,b)$?
\кзадача
\vspace*{-3mm}

\задача
\пункт
Докажите, что для любого числа~$k$ выполнено $(ka,kb)=k\cdot(a,b)$.\\
\пункт
Докажите, что если $m$\т общий делитель чисел~$a$ и~$b$, то $(a/m,b/m)=(a,b)/m$.
\кзадача

\задача
Докажите, что числа~$a$ и~$b$ взаимно просты тогда и~только тогда, когда существуют такие числа~$x$ и~$y$, что $ax+by=1$.
\кзадача

\задача
Даны числа $a$, $b$ и $c$, причём $(a,b) = 1$. Докажите, что \\
\пункт
если $ac \dv b$,  то $c \dv b$;
\пункт
если $c \dv a$ и $c \dv b$,  то $c \dv ab$.
\кзадача

\vspace*{-3mm}
\ввзадача[Основная теорема арифметики] Докажите следующие утверждения:
\сНовойСтроки
\пункт для каждого необратимого ненулевого числа $n$ найдутся такие простые числа
$p_1,\dots,p_k$, что $n=p_1\cdot \dots \cdot  p_k$; %($k$ --- натуральное);
\пункт[каноническое разложение] Для каждого необратимого ненулевого числа $n$ найдутся такие
различные простые $p_1,\dots,p_k$ и натуральные $\al_1,\dots,\al_k$, что
$n=p_1^{\al_1}\cdot \dots \cdot p_k^{\al_k}$; %($k$ --- натуральное);
\пункт разложения из пунктов а) и б) единственны с точностью до порядка
сомножителей и домножения на обратимые элементы.
\кзадача
\vspace*{-3mm}

\задача
Даны числа $a, b, c \in K$, $n\in\N$, $(a,b) = 1$, $ab = c^n$.
Найдётся ли такое число $x$, что~$a = x^n$?
\кзадача

\задача
Решите уравнение $x^{42} = y^{55}$.
\кзадача

\задача
Найдите каноническое разложение целого числа
\пункт
$2013$,
\пункт
$1002001$,
\пункт
$17!$,
\пункт
$C_{20}^{10}$.
\кзадача


\опр
\emph{Общим кратным} ненулевых чисел~$a$ и~$b$ называется ненулевое число, которое делится как на~$a$, так и~на~$b$. Любое общее кратное с наименьшей нормой называется \emph{наименьшим общим кратным} чисел~$a$ и~$b$. Обозначение:~$[a,b]$.
\копр

\взадача
Пусть $a=p_1^{\al_1}\cdot p_2^{\al_2}\cdot\ldots\cdot p_n^{\al_n},\,\,b=p_1^{\beta_1}\cdot p_2^{\beta_2}\cdot\ldots\cdot p_n^{\beta_n}$, причём $\al_i,\,\beta_i\geqslant0$.\\
\пункт
Найдите~$(a,b)$ и~$[a,b]$.
\пункт
Докажите, что $ab=(a,b)\cdot[a,b]$.
\кзадача

\взадача
Докажите, что любое общее кратное чисел~$a$ и~$b$ делится на~$[a,b]$.
\кзадача

\задача
Верно ли, что
\пункт
$[ca,cb]=c\cdot[a,b]$, если ненулевое число $c$ необратимо;
\пункт
числа $[a,b]/a$ и~$[a,b]/b$ взаимно просты?
\кзадача

\задача
Про натуральные числа $a$ и $b$ известно, что $(a,b) = 15$, $[a,b] = 840$. Найдите $a$ и $b$.
\кзадача


\ЛичныйКондуит{0mm}{6mm}
\vspace*{-3mm}%\GenXML

\end{document}




