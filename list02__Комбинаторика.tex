% !TeX encoding = windows-1251
\documentclass[a4paper,12pt]{article}
\usepackage[mag=970]{newlistok}

\УвеличитьШирину{1truecm}
\УвеличитьВысоту{2.5truecm}
% \hoffset=-2.5truecm
% \voffset=-25truemm


\Заголовок{Комбинаторика}
\НомерЛистка{2}
\ДатаЛистка{09.2013}

\renewcommand{\spacer}{\vfill}

\begin{document}



\СоздатьЗаголовок

\задача
В школьной столовой 5 кранов для умывания. Каждый может быть закрыт или
открыт. Сколькими способами может течь вода в столовой?
\кзадача

\задача Некое современное здание имеет форму куба, стоящего на
четырёх колоннах. Имеется 6 красок. Сколькими способами можно
покрасить грани здания этими красками в 6 цветов? (Каждая грань
красится целиком в один цвет, разные грани красятся в разные цвета.)
\кзадача

\задача
\пункт В заборе 20 досок, каждую надо покрасить в синий,
зелёный или жёлтый цвет, причём соседние доски красятся в разные
цвета. Сколькими способами это можно сделать? \\
\пункт А если
требуется ещё, чтобы хоть одна из досок обязательно была синей?
\кзадача

\задача
\вСтрочку
\пункт
Сколько можно составить различных
(не обязательно осмысленных) слов из $k$ букв,
используя русский алфавит?
\пункт
А если потребовать, чтобы буквы в словах не повторялись?\\
\пункт
Сколькими способами можно переставить буквы в слове из $k$ различных букв?
\кзадача


\задача
\пункт Сколько существует 10-значных чисел, не содержащих
цифру 1?\\
\пункт Сколько из них содержит цифру 9 (хотя бы одну)?
\кзадача

\задача
Сколько раз в записи целых чисел от 1 до 222222 встречается цифра 0?
\кзадача

\задача
\пункт
Десять девушек водят хоровод. Сколькими способами они могут встать
в круг?\\
\пункт
Сколько ожерелий можно составить из 10 различных бусин?\\
\спункт А если в ожерелье всего 3 белых и 7 синих бусин?
\кзадача

\задача
\пункт
Сколько строк можно составить из 0 и 1, чтобы в каждой строке было 10 цифр?
\пункт
На дереве растут 10 яблок. Сколькими способами можно сорвать
несколько из них?
\кзадача

\задача Меню в школьном буфете постоянно и состоит из $n$ разных
блюд. Петя хочет~каждый день выбирать себе завтрак по-новому
(за раз он может съесть от 0 до $n$ разных блюд).\\
\вСтрочку
\пункт Сколько дней ему удастся это делать?
\пункт Сколько блюд он съест за это время? \\
\пункт Вася решил последовать
примеру Пети, но съедать каждый день нечётное число блюд. Сколько
дней ему удастся это делать?
\пункт  Сколько блюд он съест за это время?
\кзадача

\задача Найдите коэффициенты при $x^{17}$ и $x^{18}$ после раскрытия
скобок и приведения подобных членов в выражении $(1+x^5+x^7)^{20}$.
\кзадача


\задача
\пункт Сколькими способами можно расставить на шахматной
доске 8 различных ладей так, чтобы они не били друг друга?
\пункт
Тот же вопрос про 8 неразличимых ладей.
%Ответом в предыдущем пункте является квадрат некоторого числа.
%Объясните это явление.
\кзадача

%\задача
%Сколькими способами можно расставить на шахматной доске
%8~неразличимых ладей так, чтобы они не били друг друга?
%\кзадача

\задача
\пункт
Фабрика игрушек выпускает разноцветные кубики. У всякого
кубика каждая грань окрашена целиком одной из шести красок,
имеющихся на фабрике, причём все цвета присутствуют.
%разные грани одного кубика окрашены разными красками.
Сколько видов кубиков выпускает фабрика?
\пункт
Та же задача, но есть всего пять красок.
%А если имеется всего 5 красок?
\кзадача

\задача Решите пункт а) предыдущей задачи, заменив куб на тетраэдр
(и 6 цветов на 4).
\кзадача

\сзадача Даны два одинаково окрашенных кубика $1\times1\times1$
из пункта а) задачи 12. Сколько видов параллелепипедов %размерами
$1\times1\times2$
можно получить, склеивая эти два кубика (по грани)?
%Фабрика из пункта а) предыдущей задачи начала выпуск параллелепипедов
%$1\times1\times2$, склеивая по два из выпускаемых ею кубиков.
%Сколько получится различных видов новой игрушки?
\кзадача


\задача
На окружности отмечены десять различных точек. Сколько можно провести
незамкнутых несамопересекающихся ломаных с вершинами во всех этих точках?
\кзадача


\задача
\пункт Какое наибольшее число неразличимых слонов можно
расставить на шахматной доске так, чтобы они не били друг друга?
\пункт Докажите, что число способов такой расстановки~--- квадрат
целого числа.
\спункт Найдите это число. (Сначала решите
%такую же
задачу для досок $2\times2$, $4\times4$\ldots)
\кзадача

\задача Сколько существует строк из 20 цифр, в которых встречаются
только нули и единицы, причём никакие два нуля не стоят рядом?
\кзадача

\сзадача
В таблицу размера $m\times n$ записывают числа
$+1$ и $-1$ так, чтобы произведение чисел в каждой строке и в каждом
столбце равнялось 1. Сколькими способами это можно сделать?
\кзадача

\сзадача
\пункт \лк Ч\"ертово колесо\пк \ состоит из $p$ одинаковых кабинок
($p$ --- простое число). Каждую кабинку можно покрасить в один
из $n$ цветов. Сколько есть способов раскраски?\\
\пункт[Малая теорема Ферма]
Выведите из предыдущего пункта, что $n^p-n$ делится на $p$ при любом
натуральном числе $n$ и любом простом числе $p$.
\кзадача

\сзадача
Сколькими способами можно представить число 2013 в виде суммы
нескольких натуральных слагаемых, любые два из которых равны или
различаются не больше, чем на 1?
%которые приблизительно равны?
(%Числа называются приблизительно равными,
%если они равны или отличаются на 1.
Способы, отличающиеся только
порядком слагаемых, считаются одинаковыми.)
\кзадача

\ЛичныйКондуит{0mm}{5mm}

%\СделатьКондуит{5.6mm}{6.2mm}


\end{document}
