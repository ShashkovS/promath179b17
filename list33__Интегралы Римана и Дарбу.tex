% !TeX encoding = windows-1251
\documentclass[a4paper, 11pt]{article}
\usepackage{newlistok}
%\documentstyle[11pt, russcorr, listok]{article}
\newcommand{\0}[1]{\overline{#1}}
\def\C{\mbox{$\Bbb C$}}

\УвеличитьШирину{1.5truecm}
\УвеличитьВысоту{2.5truecm}


\newcommand{\RpR}{{\cal R}([a,b])}
\newcommand{\intab}{\int\limits_a^b}

\begin{document}

\Заголовок{Интегралы Римана и Дарбу}
%\Подзаголовок{Часть 1. Определение и свойства}
\НомерЛистка{33}
\ДатаЛистка{10.2016}
\СоздатьЗаголовок

\раздел{Интеграл Римана}



%\noindent {\bf Соглашение.} Все функции этого листка предполагаются ограниченными.

\опр \выд{Разбиением  отрезка}  $[a,b]$
называется всякий конечный набор точек $\sigma=\{x_0,x_1,\ldots,x_n\}$
с условием
$a=x_0<x_1 <\ldots<x_{n-1}<x_n=b$. Разность $x_i-x_{i-1}$ обозначается
$\Delta x_i$.\\
 \выд{Диаметром разбиения} $\sigma$
называется число $\lambda(\sigma)=\max\limits_{i=1,\ldots,n}\Delta x_i$.
% Объединением двух разбиений называется их
%теоретико--множественное объединение, точки которого занумерованы в порядке
%возрастания.
\копр

\опр \выд{Отмеченным разбиением  отрезка}  $[a,b]$
называется пара $(\sigma,\xi)$, где $\sigma=\{x_0,x_1,\ldots,x_n\}$
--- разбиение отрезка $[a,b]$, а $\xi = (\xi_1,\dots,\xi_n)$ --- набор точек
из отрезков разбиения: $x_{i-1}\le \xi_i \le x_i$.
\копр

\опр Пусть $f$ --- функция на отрезке $[a,b]$, а $(\sigma,\xi)$ --- отмеченное разбиение этого отрезка. \выд{Интегральной суммой} функции $f$ при отмеченном разбиении $(\sigma,\xi)$ называется
$$
\sum_{(\sigma,\xi)}f\Delta x= \sum_{i=1}^nf(\xi_i)\Delta x_i.
$$
\копр

\опр Пусть $f$ --- функция на отрезке $[a,b]$. \выд{Интегралом Римана} функции $f$ по отрезку $[a,b]$ называется
$$
\int\limits_a^bf(x)dx=\lim_{\lambda(\sigma)\to0}\sum_{(\sigma,\xi)}f\Delta x,
$$
где предел берется по всем отмеченным разбиениям $(\sigma,\xi)$ отрезка $[a,b]$ при $\lambda(\sigma)\to0$.
Функция $f$ называется \выд{интегрируемой по Риману} на отрезке $[a,b]$, если этот предел (то есть интеграл) существует. Обозначение: $f\in \RpR$.
\копр

\задача
Дайте четкое определение (в терминах $\varepsilon$ и $\delta$) того, что число $I$ является пределом интегральных сумм $\sum_{(\sigma,\xi)}f\Delta x$ при $\lambda(\sigma)\to0$.
\кзадача

\задача Интегрируемы ли  на отрезке $[0,1]$ функции:
%\сНовойСтроки
   \пункт  $f(x)=1$;
   \пункт  $f(x)=x$;
  \пункт  $f(x)=x^{-1}$?
\кзадача

\задача
Докажите, что любая интегрируемая функция на отрезке $[a,b]$ ограничена на нем.
\кзадача

\задача
Докажите, что если изменить значение функции в конечном числе точек, то ее интегрируемость на отрезке $[a,b]$ и значение интеграла не изменятся.
\кзадача

\задача
Докажите \выд{критерий Коши} интегрируемости: функция $f$ интегрируема на отрезке $[a,b]$ по Риману тогда и только тогда, когда для любого $\varepsilon>0$ существует такое $\delta>0$, что для любых двух отмеченных разбиений $(\sigma,\xi)$ и $(\sigma',\xi')$ таких, что $\lambda(\sigma)<\delta$ и $\lambda(\sigma')<\delta$, выполняется неравенство
$$
\left|\sum_{(\sigma,\xi)}f\Delta x - \sum_{(\sigma',\xi')}f\Delta x\right|<\varepsilon.
$$
\кзадача

\раздел{Интеграл Дарбу}

\опр Пусть $\sigma$ --- некоторое разбиение отрезка $[a,b]$, $f$ --- функция,
ограниченная на этом отрезке. Положим
 $m_i=\inf\limits_{[x_{i-1}, x_i]}f(x)$,
 $M_i=\sup\limits_{[x_{i-1}, x_i]}f(x)$.
  Числа $s_{\sigma}=\sum\limits_{i=1}^n m_i\Delta x_i$ и
  $S_{\sigma}=\sum\limits_{i=1}^n M_i\Delta x_i$ называются соответственно \выд{нижней и
верхней суммами Дарбу}  функции $f$ при разбиении $\sigma$.
\копр


\задача Объясните геометрический смысл верхней и нижней сумм Дарбу
и \лк нарисуйте\пк\ их  \сНовойСтроки
   \пункт для функции $f(x)=x$ на отрезке $[0,1]$ при разбиении
  $\sigma=\{\frac i4 \mid i=0,\ldots,4 \}$;
  \пункт  для функции $f(x)=(x-1)^2$ на отрезке $[0,2]$ при разбиении
  $\sigma=\{\frac i4 \mid i=0,\ldots,8 \}$.
\кзадача

\задача
Можно ли исключить из определения 5 условие ограниченности
функции $f$ на отрезке $[a,b]$?
\кзадача

\задача Найдите нижние и верхние суммы Дарбу при разбиении
 $\sigma=\{\frac in \mid i=0,\ldots,n \}$ отрезка $[0,1]$
функций \вСтрочку
    \пункт $f(x)=x$;
    \пункт $f(x)=x^2$;
\пункт $f(x)=\sin\pi n x$.
\кзадача



\задача Что происходит с суммами Дарбу при добавлении к разбиению новых точек?
\кзадача


\задача Докажите: любая верхняя  сумма  Дарбу функции не меньше
любой е\"е нижней суммы Дарбу.
\кзадача

\задача Докажите, что для любой ограниченной  на отрезке $[a,b]$ функции $f$
существуют
$I_*=\sup s_\sigma$ и $I^*=\inf S_\sigma$, где
супремум и инфимум берутся по всем разбиениям $\sigma$ отрезка.
Сравните $I_*$ и $I^*$. Эти числа называются
\выд{нижним и верхним интегралами Дарбу} функции $f$ на отрезке $[a,b]$.
\кзадача

\задача Найдите $I_*$ и $I^*$ для\\
  \пункт $f(x)=x$ на $[0,1]$;\пункт $f(x)=x^2$ на $[0,1]$;
 \пункт функции Дирихле на отрезке
 $[a,b]$.
\кзадача

\newpage

\задача \label{first}
Докажите, что функция $f$ интегрируема (по Риману) на отрезке $[a,b]$ тогда и только тогда, когда
она ограничена на этом отрезке, и $I_* =I^*$.
При этом $\int\limits_{a}^{b} f(x)\,dx=I_* =I^*$.
\кзадача

\задача Верно ли, что всякая ограниченная на отрезке функция интегрируема
на этом отрезке?
\кзадача

\задача
Докажите, что функция, монотонная на некотором отрезке, интегрируема на
н\"ем.
\кзадача


\задача \пункт Докажите, что интеграл от неотрицательной функции
неотрицателен.\\
\пункт Верно ли, что если интеграл функции равен нулю, то и сама
функция тождественно равна нулю?\\
\пункт А если эта функция неотрицательна?
\пункт А если эта функция неотрицательна и непрерывна?\\
\спункт Верно ли, что интеграл от строго положительной функции
строго больше нуля?
\кзадача

\задача Пусть $f,g\in \RpR$. Докажите, что $f+ g\in \RpR$ и  выполнено равенство
$$\int\limits_a^b(f(x)+ g(x))\,dx =
\int\limits_a^bf(x)\,dx+ \int\limits_a^bg(x)\,dx.$$
\кзадача


\задача Пусть $f\in \RpR$, $c\in\R$. Докажите, что  $c f\in \RpR$ и выполнено равенство
$$\int\limits_a^bcf(x)\,dx  = c\int\limits_a^bf(x)\,dx.$$
\кзадача


\задача Пусть $f,g\in\RpR$ и при любом $x\in[a,b]$
выполнено $f(x)\le g(x)$. Докажите, что $$\intab f(x)\,dx \le \intab g(x)\,dx.$$
\кзадача

\задача Пусть $f\in \RpR$ и для любого $x\in[a,b]$ выполнено $m\le f(x)\le M$. Докажите,
что $$m(b-a)\le\intab f(x)\,dx\le M(b-a).$$
\кзадача

\задача
\пункт
Докажите, что $I_*=I^* \iff \inf_{\sigma} (S_\sigma-s_\sigma) = 0$.\\
\пункт
Пусть $f \in \RpR$. Докажите, что $|f|\in \RpR$ и выполнено неравенство
$$\left|\int\limits_a^b f(x)\,dx\right| \leq
\int\limits_a^b|f(x)|\, dx.$$
\кзадача

\задача Пусть  $f\in \RpR$ и $[c,d]\subset [a,b]$. Докажите, что $f\in
{\cal R}([c,d])$.
\кзадача


\задача Пусть  $a<b<c$. Докажите, что
если функция $f(x)$ интегрируема на отрезках $[a,b]$ и $[b,c]$, то она интегрируема и на
$[a,c]$, прич\"ем
$$\int\limits_a^c f(x)\,dx=\int\limits_a^b f(x)\,dx+
 \int\limits_b^c f(x)\,dx.$$
\кзадача


\задача
Докажите, что непрерывная на отрезке функция интегрируема на н\"ем.\\
{\it Указание:} непрерывная на отрезке функция равномерно непрерывна на н\"ем (см.~листок 24).
%Заметьте, что эта функция равномерно непрерывна на этом отрезке (см.~листок 24).
\кзадача

\задача
Докажите интегрируемость на отрезке
\сНовойСтроки
\пункт ограниченной функции с конечным числом точек разрыва на этом
отрезке;
\спункт ограниченной функции со сч\"етным числом точек разрыва на этом
отрезке.
\кзадача

\ЛичныйКондуит{0mm}{8mm}

% %\СделатьКондуит{5mm}{8mm}

%\ЛичныйКондуит{0mm}{6mm}

%\СделатьКондуит{5mm}{9mm}

% %\GenXMLW
\end{document}

\опр Функция $f$ называется \выд{равномерно непрерывной} на множестве $M$,
если для всякого $\epsilon>0$ найд\"ется такое
$\delta>0$, что для любых $x,y\in M$ таких, что $|x-y|<\delta$,
выполнено $|f(x)-f(y)|<\epsilon$.
\копр

\задача Если функция равномерно непрерывна, то она непрерывна.
\кзадача

\задача [Теорема Кантора]
Докажите, что функция, непрерывная на отрезке, равномерно
непрерывна на н\"ем.
\кзадача

\задача Верно ли утверждение, аналогичное теореме Кантора, для функции,
заданной на \вСтрочку \пункт интервале; \пункт прямой?
\кзадача

%\hspace*{-\parindent}{\it Указание.} В пункте б) воспользуйтесь
%компактностью отрезка.
