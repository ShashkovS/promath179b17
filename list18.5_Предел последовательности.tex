% !TeX encoding = windows-1251
\documentclass[a4paper, 12pt]{article}
\usepackage{newlistok}
%\documentstyle[11pt, russcorr, listok]{article}

%\УвеличитьШирину{1.3truecm}
%\УвеличитьВысоту{3truecm}
%\hoffset=-2.7truecm
%\voffset=-24truemm
%\pagestyle{empty}

%\documentstyle[11pt, russcorr, ll]{article}
%\def\dfrac{\displaystyle\frac}

\Подзаголовок{}
\НомерЛистка{18$\frac12$}
\ДатаЛистка{02.2015}
\Заголовок{Предел последовательности.}

\begin{document}

%\scalebox{.91}{\vbox{%
%\ncopy{1}{

\СоздатьЗаголовок

\задача В два сосуда разлили (не поровну)
1 л воды. Из 1-го сосуда перелили
половину имеющейся~в~н\"ем
воды
во 2-ой, затем из 2-го перелили половину оказавшейся
в н\"ем воды в 1-ый, %затем
снова из 1-го  пере\-ли\-ли половину
%оказавшейся в н\"ем
%воды
во 2-ой, и т.~д.
%Докажите, что независимо от того, сколько воды было сначала
%в каждом из сосудов,
%после 100 переливаний в
%них будет $2/3$~л и $1/3$~л воды с точностью до 1 миллилитра.
Сколько воды (с точностью до 1 мл) будет в 1-ом
сосуде после 50 переливаний?
\кзадача

\задача На графике функции $y=x^2$ рассмотрим точки $A_n$ и $B_n$
с абциссами $-1/n$ и $1/n$ соответственно. Пусть $M_n$ --- центр
окружности, проведенной через точки $A_n$, $B_n$ и начало координат.
Докажите, что последовательность точек $(M_n)$ имеет предел
и найдите его.
\кзадача


\задача Найдите пределы последовательностей:\\
\medskip
\пункт $\sqrt[n]{2^n+3^n}$;\\
\medskip
\пункт $\sqrt{n^2+n}-n$;\\
\smallskip
\пункт $\displaystyle{\frac12+\frac2{2^2}+\frac3{2^3}+\ldots+\frac{n}{2^n}}$;\\
\smallskip
\пункт $\displaystyle{\frac{f_n}{f_{n+1}}}$, где $f_n$ --- $n$-е число Фибоначчи.
\кзадача


\задача
При каких натуральных $k$ выполнено равенство
$
\displaystyle{
\lim\limits_{n\rightarrow\infty}\frac{n^k-(n-1)^k}{n^{2014}}=2015?}
$
\кзадача

\задача
Дано $m$ последовательностей, сумма которых
стремится к $m\alpha$, и сумма квадратов которых
стремится к $m\alpha^2$. Докажите, что каждая из этих
последовательность стремится к $\alpha$.
\кзадача

\задача Последовательность $(x_n)$ строится по следующему закону:
первый член выбирается произвольно, а каждый следующий выражается
через предыдущий по формуле $x_{n+1}=ax_n+1$. При каких $a$
последовательность $(x_n)$ обязательно будет иметь предел?
\кзадача

\задача Известно, что $\lim\limits_{n \to \infty} x_n = a$.
Найдите
%\вСтрочку
%\пункт $y_n=(2x_n-1)/(x_n+1)$;
%\пункт $y_n=(x_n^2+x_n-2)/(x_n-1)$;
%\пункт $y_n=\sqrt{x_n}$;
$\displaystyle{\lim\limits_{n \to \infty}\!\frac{x_1+\ldots+x_n}n}$.
%\пункт $y_n=(x_1+\ldots+x_n)/n$.
\кзадача






%\задача
%Первые два члена последовательности равны $0$ и $1$,
%а каждый следующий есть среднее арифметическое двух предыдущих.
%Найдите предел этой последовательности.
%\кзадача



\задача %С незапамятных врем\"ен
Издавна жители островов Чунга и Чанга раз в год
меняются~драгоцен\-ностями. Одновременно~жители Чунги привозят
половину своих драгоценностей на Чангу, а жители
Чанги %~одновре\-мен\-но привозят
треть своих драгоценностей на
Чунгу.
%Так продолжается .
Какая часть драгоценностей находится~на~каж\-дом острове?
(Общий набор драгоценностей постоянен.) %за это время %на островах не менялся.)
\кзадача

%\задача
%Последовательность $(a_n)$ состоит из
%ненулевых членов и имеет предел 0. Известно, что
%последовательность $(\frac{a_{n+1}}{a_n})$ имеет предел
%$\alpha$. Найдите все возможные значения $\alpha$.
%\кзадача


%\сзадача Петя вышел из дому и пош\"ел в школу.
%На полпути к школе он решил, что лучше пойти в кино, и
%свернул к кинотеатру. Пройдя половину пути, он
%захотел покататься на коньках и свернул к катку.
%Пройдя половину пути до катка, он подумал, что нужно
%вс\"е-таки учиться, и повернул к школе. Но на полпути
%к школе снова свернул к кинотеатру, и т.~д. Куда
%прид\"ет Петя, если будет так идти?
%\кзадача

\задача Петя шел из дома в школу.
На полпути он решил, что плохо себя чувствует, и
повернул обратно. На полпути к дому ему стало лучше,
и он повернул в школу. На полпути к школе он решил, что
всё-так нездоров, и повернул к дому. Но на полпути
к дому снова повернул к школе, и т.~д. Куда
прид\"ет Петя, если будет так идти?
\кзадача



\задача Петя шел из дома в школу.
На полпути он решил, что лучше пойти в кино, и
свернул к кинотеатру. На полпути к кинотеатру
он захотел покататься на коньках и пошел на каток.
На полпути к катку он подумал, что надо вс\"е-таки
учиться, и повернул к школе. Но на полпути
к ней снова свернул к кинотеатру, и т.~д. Куда
прид\"ет Петя, если будет так идти?
\кзадача




%\задача Найдите предел последовательности $(a_n)$, где
%%$\displaystyle{a_n=\frac12+\frac2{2^2}+\frac3{2^3}+\ldots+\frac{n}{2^n}}$.
%$a_n=1/2+2/2^2+3/2^3+\ldots+n/2^n$.
%\кзадача

%\задача
%Пусть $x_n=a_1/1+a_2/2+\dots+a_n/n$, где $a_n=1$, если
%в десятичной записи числа $n$ нет цифры 9, и $a_n=0$
%иначе. Имеет ли эта последовательность предел?
%\кзадача

\задача
%Задача Кириллова о манной каше.
По кругу сидят $n$ ребят, у каждого по тарелке каши.
Каждую минуту одновременно %происходит перераспре
%Ежеминутно
каждый из ребят берет себе
по половине каши своих соседей. Сначала в тарелках было
$1, 2, \dots, n$ поварешек каши.
%Верно ли, что спустя достаточно большое время
%каши в тарелках будет примерно поровну? Решите задачу, если
Сколько каши будет в тарелках спустя достаточно долгое время, если\\
%\вСтрочку
\пункт $n=3$;\\
\пункт $n=4$;\\
\спункт $n\in\N$?\\
\спункт А если ребята сидят в вершинах графа
и ежеминутно каждый делит свою кашу поровну между соседями?
\кзадача



\ЛичныйКондуит{0mm}{6mm}
%\GenXMLW

%\СделатьКондуит{4.1mm}{7.5mm}

\end{document}


\задача Для вычисления квадратного корня из положительного
числа $a$ можно пользоваться следующим методом
последовательных приближений. Возьмите любое положительное число
$x_0$ и постройте последовательность по такому закону:
$x_{n+1}=0,5\cdot(x_n+a/x_n).$\\
%\сНовойСтроки
%\вСтрочку
\пункт
Докажите, что $\lim\limits_{n\to\infty}x_n=\sqrt a$.\\
\спункт Сколько понадобится последовательных приближений,
чтобы найти $\sqrt{10}$ с точностью до $0,0001$,
если в качестве первого приближения взять $x_0=3$?
\кзадача


----------


\задача[\лк Теорема о двух милиционерах\пк]
Пусть $\lim\limits_{n \to \infty} x_n =\lim\limits_{n \to
\infty} y_n = а$, и последовательность $(z_n)$ такова,
что $x_n\leq z_n\leq y_n$ при любом номере $n$.  Докажите,
что $\lim\limits_{n \to \infty} z_n = a$.
\кзадача



\задача Пусть $A(x)=a_k x^k+\ldots+a_1x+a_0$ и
$B(x)=b_m x^m+\ldots+b_1x+b_0$ --- многочлены степеней $k$ и $m$
соответственно. Найдите:
\вСтрочку
\пункт $\lim\limits_{n \to \infty}A(n)/n^k$;
\пункт $\lim\limits_{n \to \infty}A(n)/B(n)$;
\пункт $\lim\limits_{n \to \infty}(S_k(n)/n^k-n/(k+1))$.
\кзадача

\сзадача Найдите %предел последовательности $(a_n)$, если\\
\вСтрочку
\пункт
$\displaystyle{
\lim\limits_{n\to\infty}
\left(\frac{1^k+2^k+\ldots+n^k}{n^k}-\frac{n}{k+1}\right)}$,
где $k\in\N$;
\пункт
$\displaystyle{\lim\limits_{n\to\infty}\frac{1^1+2^2+\ldots+n^n}{n^n}}$.
\кзадача

%\сзадача
%Дано $n$ последовательностей, сумма которых
%стремится к $n\alpha$, и сумма квадратов которых
%стремится к $n\alpha^2$. Докажите, что каждая из этих
%\кзадача

%\сзадача [Критерий Коши]
%Докажите, что последовательность $(x_n)$ сходится тогда и
%только тогда, когда выполнено условие
%$\quad\forall \varepsilon>0 \quad \exists k\in\N\quad \forall m,n\ge k
%\quad |x_m-x_n|<\varepsilon$.
%\кзадача


\end{document}


\опр Пусть $\varepsilon$ --- произвольное положительное число.
\выд{$\varepsilon$-окрестностью} точки $a$ называется интервал
$(a-\varepsilon, a+\varepsilon) = \{ x: |x-a| < \varepsilon \}$.
Обозначение: ${\cal U}_\varepsilon(a)$.
\копр

\задача
Докажите, что любые две точки на прямой имеют непересекающиеся окрестности.
\кзадача

\опр \label{limit1} Число $a$ называют \выд{пределом} последовательности
$(x_n)$, если для всякого $\varepsilon>0$ в $\varepsilon$-окрестности числа $a$
содержатся \выд{почти все} члены последовательности
(то есть все, кроме, быть может, конечного числа членов). Обозначение:
$\lim\limits_{n \to \infty} x_n = a$.
Говорят также, что $(x_n)$ стремится к $a$ при $n$,
стремящемся к бесконечности
(и пишут $x_n \to a$ при~\hbox{$n \to \infty$).}
\копр




\опр
Последовательность $(x_n)$ называется \выд{монотонно
возрастающей}, если $x_{n+1}>x_n$ при любом натуральном $n$.
\копр

\задача
Дайте определение монотонно убывающей, монотонно невозрастающей
последовательностей.
\кзадача

\задача Исследуйте на монотонность последовательности из задачи 3 листка 11.
\кзадача

\опр Говорят, что $(y_n)$ --- \выд{подпоследовательность}
последовательности
$(x_n)$, если найд\"ется такая монотонно возрастающая последовательность
$(k_n)$ натуральных чисел, что $y_n=x_{k_n}$ при всех $n\in\N$.
\копр

\задача Является ли $(y_n)$ подпоследовательностью $(x_n)$, если
\сНовойСтроки
\пункт $x_n=n$, а $y_n=2n$;
\пункт $x_n=n$, а $y_1=2;\ y_2=1; y_n=n$ при $n\geq3$;
\пункт $x_n=(-1)^n$, а $(y_n)$ есть $1,\ -1,\ 1,\ 1, -1,\ 1,\ 1,\ 1, -1,
\ldots$?
\кзадача

\задача Последовательность обладает одним из свойств:
бесконечно мала, бесконечно велика, ограниченна, неограниченна, монотонна.
В каких случаях любая е\"е подпоследовательность обязательно
обладает тем же свойством?
\кзадача

\сзадача Верно ли, что любая последовательность имеет монотонную
подпоследовательность?
\кзадача
