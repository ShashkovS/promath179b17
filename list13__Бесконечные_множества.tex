% !TeX encoding = windows-1251
\documentclass[a4paper,12pt]{article}
\usepackage{newlistok}
%\usepackage{tikz}
%\usetikzlibrary{calc}

\newcommand{\del}{\mathrel{\raisebox{-.3 ex}{${\vdots}$}}}

\УвеличитьШирину{.3truecm}
\УвеличитьВысоту{2.1truecm}
% \hoffset=-3truecm
% \voffset=-2.2truecm

%{\reflectbox{\hbox{{\tiny Подсказка к а): докажите, что любая последовательность $(f^n(x))$ фундаментальна}}}}

\begin{document}

\Заголовок{Бесконечные множества}
\Подзаголовок{}
\НомерЛистка{13}
\ДатаЛистка{05.2014}

\СоздатьЗаголовок


\задача
\пункт
Докажите, что
в любом бесконечном множестве найдется сч\"етное подмножество;\\
\пункт Докажите, что множество
$M$ бесконечно тогда и только тогда, когда
оно равномощно множеству, полученному из $M$ удалением
одного элемента.
\кзадача

\задача Равномощны ли множества точек:\\
%\вСтрочку
\пункт интервал и отрезок;\\
\пункт полуокружность~и~прямая;\\
\пункт интервал и прямая;\\
\пункт два круга;\\
\пункт окружность и треугольник;\\
\пункт квадрат %\footnote{Квадрат в этом листке --- это квадрат с внутренностью, например множество точек $(x,y)$, где $0\leq x, y\leq1$.}
и плоскость;\\
\пункт квадрат и круг;\\
%\пункт  круг и %круговое
%кольцо;
\пункт отрезок и сч\"етное объединение непересекающихся отрезков?\\
{\small ({\em Замечание:} квадрат в этом листке --- это квадрат
с внутренностью, например множество точек
$(x,y)$, где $0\leq x, y\leq1$.)}
\кзадача

%\задача Разбейте отрезок
%на сч\"етное число %объединение
%непересекающихся множеств, равномощных отрезку.
%\кзадача

\задача
Из бесконечного множества $M$ удалили некоторое сч\"етное множество
и получили бесконечное множество $M'$. Докажите, что $M$ и $M'$ равномощны.
\кзадача


\задача
Равномощно ли множество иррациональных чисел множеству
всех действительных чисел?
\кзадача

\задача
Равномощно ли множество всех лучей множеству всех
окружностей (на плоскости)?
\кзадача


% Несчетные множества

\задача
Докажите, что %следующие множества равномощны:
множество $S$ бесконечных последовательностей из 0 и
1,~\hbox{множество~всех} подмножеств множества $\N$
и множество бесконечных вправо и вниз таблиц из 0 и 1 равномощны.
\кзадача


%\раздел{Дополнительные задачи}

\задача
\пункт Дана бесконечная вправо и вниз таблица из 0 и 1.
Покажите, как по этой таблице составить бесконечную строку из 0 и 1,
которая не совпад\"ет ни с одной из строк таблицы.\\
{\footnotesize ({\em Указание:} надо, чтобы новая строка отличалась
от каждой строки таблицы хотя бы в одном месте.)}\\
\пункт
Докажите, что
%множество $S$ из задачи 6
множество бесконечных последовательностей из 0 и 1
\выд{несч\"етно:} бесконечно, но не является сч\"етным.\\
(Говорят, что множества из предыдущей задачи
имеют мощность \выд{континуум}).
\кзадача


\задача
Пусть $S$ --- множество из задачи 6. Докажите, что
множества $S$ и $S\times S$ равномощны.
\кзадача

\задача
Докажите, что множество всевозможных
прямых на %декартовой
плоскости
равномощно множеству точек этой плоскости.
\кзадача

\сзадача
%Докажите, что отрезок $[0;1]$ (множество точек $x$, где $0\leq x\leq1$)
%равномощен множеству бесконечных последовательностей из 0 и 1;
Докажите, что множество точек любого отрезка %$I$
равномощно\\
%\сНовойСтроки
%\вСтрочку
\пункт
множеству~$S$~зада\-чи 6;\\
%множеству бесконечных последовательностей из 0 и 1;
%\кзадача
%
%\сзадача
%Докажите, что множество точек любого отрезка $I$ равномощно
%\вСтрочку
\пункт
множеству точек квадрата;\\ %$I\times I$ (множеству пар $(x,y)$, где $x, y$ --- любые точки~из~ $I$);\\
\пункт
множеству точек куба.
%множеству точек плоскости.
\кзадача


\сзадача Пусть $A$ --- сч\"етное множество,
$M$ --- некоторое множество подмножеств $A$. Известно, что из
любых двух элементов $M$ один есть подмножество другого.
Обязательно ли $M$ сч\"етно?
\кзадача


\сзадача [Теорема Кантора--Бернштейна]
Если множество $A$ равномощно %некоторому
подмножеству множества $B$ и множество $B$ равномощно %некоторому
подмножеству множества $A$, то %множества
$A$ и $B$ равномощны.
\кзадача

\сзадача
%\вСтрочку
%\пункт
Отрезок представлен в виде объединения двух множеств.
%Объединение двух множеств есть квадрат.
Докажите, что одно из этих множеств
равномощно отрезку.
{\footnotesize ({\em Указание:} отрезок равномощен квадрату.)}
\кзадача



%\сзадача
%Докажите, что множества задачи 12 равномощны \
%\вСтрочку
%\пункт
%множеству взаимно однозначных отображений из $\N$ в $\N$; \
%\пункт
%множеству бесконечных последовательностей %из %целых
%\кзадача

\сзадача
Докажите, что множества задачи 6 равномощны \\
%\вСтрочку
\пункт
множеству взаимно однозначных соответствий между $\N$ и $\N$; \\
\пункт
множеству бесконечных последовательностей %из %целых
натуральных чисел.
\кзадача

%\ссзадача
\smallskip
\noindent
{\bf Интересный трудный факт.}
{\em Из любых двух множеств одно равномощно подмножеству другого.}
%\кзадача


\ЛичныйКондуит{0mm}{6mm}



% %\СделатьКондуит{8mm}{6.2mm}


\end{document}

