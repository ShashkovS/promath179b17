% !TeX encoding = windows-1251
\documentclass[a4paper, 12pt]{article}
\usepackage{newlistok}
%\documentstyle[11pt, russcorr, listok]{article}
\newcommand{\0}[1]{\overline{#1}}
\def\C{\mbox{$\Bbb C$}}

\УвеличитьШирину{1.5truecm}
\УвеличитьВысоту{3.5truecm}

\begin{document}



\Заголовок{Производная в задачах}
\НомерЛистка{12д}
\ДатаЛистка{09.2016}

\СоздатьЗаголовок

%\medskip



%\раздел{Дополнительные задачи}

\задача
%Докажите, что при любом натуральном $n$
Найдется ли $n$, при котором многочлен
$1+x+\frac{x^2}{2!}+\dots+\frac{x^n}{n!}$ имеет %не
более одного корня из $\R$?
\кзадача

\задача Может ли уравнение $x(x^2-1)(x^2-1000)=\alpha$ при некотором
$\alpha\in\R$ иметь 5 целых корней?
\кзадача


\задача Пусть $k\in\R$,
функция $f$ определена на $[a,b]$ и дифференцируема на $(a,b)$, причем $|f'(x)|\le k$
при любом $x\in(a,b)$. Докажите, что при любых $x,y\in (a,b)$ выполнено неравенство
$|f(x)-f(y)|\le k|x-y|$.
\кзадача

\задача Пусть $n\in\N$, $n$ не является точной четв\"ертой
степенью. Докажите, что тогда $\{\root{ 4} \of{ n}\}>\frac{1}{4}n^{-3/4}$.
\кзадача




\задача
Найдите суммы:
\вСтрочку
\пункт
$1+2x+3x^2+\dots+nx^{n-1}$;
\пункт
$C_n^1+2C_n^2x+\dots+nC_n^nx^{n-1}$.
\кзадача

\задача
\вСтрочку
\пункт
Точка с координатами $(x(t),y(t))$ движется в %координатной
плоскости $xOy$ так, что в каждый момент времени $t$ %выполняются равенства
выполнено $y'(t)=1/x(t)$, $x'(t)=-1/y(t)$. %Известно, что в
В некий момент
времени точка имела координаты $(12,3)$. Может ли %эта точка
она в какой-нибудь другой момент %времени
иметь координаты $(6,5)$?
Нарисуйте траекторию движения точки.\qquad
\пункт Те же вопросы для точки, движущейся по закону
$y'(t)=-x(t)$, $x'(t)=y(t)$.
%В каждом из пунктов попробуйте нарисовать траекторию движения точки.
\кзадача

%\задача
%\пункт
%Напишите уравнение, задающее множество таких точек $(p,q),$ что
%квадратный тр\"ехчлен $x^2+px+q$ имеет кратный корень, и нарисуйте
%это множество на плоскости $pOq$.\\
%\пункт Для каждого числа $x$ из множества
%$\{-3,-2,-1,-1/2,\ 0,\ 1/2,\ 1,\ 2,\ 3\}$ нарисуйте на плоскости $pOq$
%график прямой, задающейся уравнением $x^2+px+q=0.$
%Докажите, что эти прямые (и вообще все прямые вида
%$x^2+px+q=0$ на плоскости $pOq$)
%касаются некоторой кривой. Что это за кривая?\\
%\пункт Укажите на плоскости множества таких точек $(p,q),$ что квадратный
%тр\"ехчлен $x^2+px+q$ имеет два различных корня, не имеет
%действительных корней.\\
%\пункт Укажите на плоскости $pOq$ множества таких точек $(p,q),$ что
%квадратный
%тр\"ехчлен $x^2+px+q$ имеет на отрезке $[-1;1]$ два различных
%корня, кратный корень, не имеет действительных корней.
%\кзадача

\задача
\пункт
Для каждого $x$ из множества
$\{-2,-1,-1/2,-1/3,\ 0,\ 1/3,\ 1/2,\ 1,\ 2\}$ нарисуйте на плоскости
$pOq$ график прямой, задающейся уравнением $x^3+px+q=0.$
Докажите, что все прямые вида $x^3+px+q=0$ на плоскости $pOq$
касаются некоторой кривой. Что это за кривая?\\
\пункт Задайте уравнением множество таких точек $(p,q),$ что
многочлен $x^3+px+q$ имеет кратный корень.
%, и изобразите его на плоскости.
%\пункт
Нарисуйте на плоскости это множество, а также
%Укажите на плоскости
множества таких точек $(p,q),$ что %многочлен
$x^3+px+q$  имеет три разных корня, корень кратности 2,
корень кратности 3,  не имеет действительных корней.\\
\пункт
Сколько корней у многочлена $x^3-10x+12$?\\
\пункт
Исследуйте геометрически число корней уравнения $x^3+px+q=0$ на
отрезке $[-1; 1].$
\кзадача



%\задача Эллипс задан уравнением $x^2/a^2+y^2/b^2=1$.
%Напишите уравнение прямой, касающейся этого эллипса в точке $(x_0,y_0)$.
%\кзадача


\задача Пусть $f$ определена на $[0,1]$ и дифференцируема на $(0,1)$, прич\"ем $f(0)=0$, $f(1)=1$. Докажите, что тогда найдутся
такие различные $s,t\in[0,1]$, что $f'(s)\cdot f'(t)=1$.
\кзадача



\задача
%\сНовойСтроки
%\пункт [Теорема Коши]
Петя ид\"ет \пункт по плоскому полю; \пункт  по холмистой местности из пункта $A$
в пункт $B$, нигде не останавливаясь.
% Докажите, что на его пути найд\"ется точка, вектор скорости
% в которой параллелен $AB$ %$\buildrel \longrightarrow \over {AB}$
%(более точно: если функции $f$ и $g$ удовлетворяют условию $(*)$,
%то найд\"ется такая точка $t\in(a,b)$, что
%$f'(t)(g(b)-g(a))=g'(t)(f(b)-f(a))$).\\
%\пункт Пусть ещ\"е $g'(x)\ne0$ на $(a,b)$. Докажите, что найд\"ется
%такая точка $t\in(a,b)$, что
%$\frac{f'(t)}{g'(t)}=\frac{f(b)-f(a)}{g(b)-g(a)}.$\\
%\пункт Объясните геометрический смысл теоремы Коши.\\
%\пункт Верно ли утверждение пункта б), если %предполагать
%только $f$ и $g$ удовлетворяют условию $(*)$ и $g(a)\ne g(b)$?
%\кзадача
% \пункт
%\задача
% Петя ид\"ет по холмистой местности из пункта $A$
% в пункт $B$, нигде не останавливаясь.
Всегда ли на его пути найд\"ется точка, вектор скорости
в которой параллелен $AB$? %$\buildrel \longrightarrow \over {AB}$?
\кзадача





\задача
Вычислите пятьдесят седьмую производную в нуле у функции
$\arcsin(x^{13}+x^{22})$.
\кзадача


\задача
Докажите, что у многочлена
$x^{1024}+ a_1x^{512} +a_2x^{256}+\dots + a_{9}x + a_{10},$
где $a_1,\dots, a_{10}\in\R$, может быть не более 11 различных
положительных действительных корней.
\кзадача


\задача Пусть $P(x)$ --- многочлен степени $n>1$, имеющий $n$ различных
%действительных
корней $x_1, \dots ,x_n.$ Докажите, что справедливо равенство
$\frac{1}{P'(x_1)}+\frac{1}{P'(x_2)}+\dots+\frac{1}{P'(x_n)}=0$.
\кзадача


%\задача
%Напишите многочлен от $\alpha$, который при $\alpha <0{,}3$ равен
%с точностью до $0{,}1\%$ длине хорды, стягивающей в единичной
%окружности дугу $\alpha$ радиан.
%\кзадача

%\задача
%Приведите пример бесконечно дифференцируемой функции на $\R$, у которой
%множество нулей сч\"етно и не дискретно.
%\кзадача

%\задача  Пусть $f(x)$~--- дважды дифференцируемая функция на
%отрезке $[0,1]$, $x_0\in[0,1]$. Докажите, что найдутся такие
%$a$, $b$ ($0\leq a<b\leq1$), что $f'(x_0)=(f(a)-f(b))/(a-b)$.
%\кзадача

\задача
Функция $f$ дифференцируема на $\R$. Верно ли, что
%производная функция
$f'$ ограничена на любом отрезке?
\кзадача

%\break

\сзадача
Найдите все такие дифференцируемые $f:\R\rightarrow\R$,
что $f'(\frac{x+y}{2})=\frac{f(y)-f(x)}{y-x}$ при
любых $x\ne y$.
% \вСтрочку
% \пункт
% Приведите пример дифференцируемой функции $f:\R\rightarrow\R$,
% удовлетворяющей тождеству $f'(\frac{x+y}{2})=\frac{f(y)-f(x)}{y-x}$,
% где $x,y\in\R$, $x\ne y$.
% \пункт Найдите все такие функции.
\кзадача


%\end{document}



\сзадача Решите в натуральных числах уравнение $x^y=y^x$.
({\sl Указание:}
изучите функцию $f(x)=x^{1/x}$).
\кзадача

\сзадача
\выд{(Правило Лопиталя)}  Пусть функции $f$ и $g$
дифференцируемы на интервале $(a,b)$,
причем $g'$ не обращается в ноль на $(a,b)$ и
\вСтрочку
\пункт
$\lim\limits_{x\rightarrow b}f(x)=\lim\limits_{x\rightarrow b}g(x)=0$;
\пункт
$\lim\limits_{x\rightarrow b}f(x)=\lim\limits_{x\rightarrow b}g(x)=+\infty$.\\
% не обращается в ноль в некоторой
%проколотой окрестности точки $x_0$.
Предположим, что существует предел
$\lim\limits_{x\to b}\frac{f'(x)}{g'(x)}=k$. Докажите, что предел
$\lim\limits_{x\to b}\frac{f(x)}{g(x)}$ существует и равен $k$.
\label{lop}
\кзадача

\сзадача
Останется ли верным правило Лопиталя, если заменить в условии $b$ и/или
$k$ на $\pm\infty$?
\кзадача


%\задача
%Сформулируйте и докажите правило, аналогичное задаче~\ref{lop}, в случаях:
%\label{lopinf}\\
%\вСтрочку
%\пункт $x_0=\infty$;
%\пункт $f(x_0)=g(x_0)=\infty$;
%\пункт одновременно выполнены условия пунктов а) и б).
%\кзадача

\сзадача Найдите пределы:
\вСтрочку
\пункт
$\lim\limits_{x\rightarrow0} \frac{\tg x-x}{x-\sin x}$;
\пункт
$\lim\limits_{x\rightarrow+\infty}\ln x/x^\alpha$ при $\alpha>0$;
\пункт $\lim\limits_{x\rightarrow+0} x^x$.
\кзадача


%\раздел{Для зачетной работы}

\сзадача Пусть $f\colon\R\to\R$~--- дважды дифференцируемая функция
на отрезке $[0,a]$,  $f(0)=f(a)=0$ и $f''$
непрерывна на отрезке $[0,a]$.%\\
\вСтрочку
%\сНовойСтроки
\пункт
Докажите, что при
$a=\pi$ справедливо утверждение:  \лк{}существует такая точка
$\xi\in(0,a)$, что $f''(\xi)+f(\xi)=0$\пк.
\пункт
Верно ли утверждение предыдущего пункта для $a=3$?
\кзадача

%\задача  Функция $f(x)$ дважды дифференцируема при всех $x \in {\R}$.
%Известно, что $f(x) \to 0$ при $x \to +\infty$, а функция
%$f''(x)$ ограничена. Докажите, что $f'(x) \to 0$ при $x \to +\infty$.
%\кзадача


%\задача
%Пусть $f\colon \R\to\R$~--- дифференцируемая функция
%с непрерывной производной, и пусть существует
%такое число $\lambda>0$, что  для всех $x\in\R$ выполнено
%$|f'(x)|\leq\lambda|f(x)|$. Кроме того, $f(0)=0$. Докажите, что $f$
%тождественно равна~0.
%\кзадача


\ссзадача
Существует ли непрерывная на $\R$ функция, ни в одной точке не имеющая производной?
\кзадача

\задача
\пункт Функция $f$ дифференцируема $n$ раз на $\R$,
и для каждой точки $a\in\R$ одна из функций $f$, $f'$, $f''$,
\ldots, $f^{(n)}$ обращается в ноль в точке $a$.
Докажите, что $f$ --- многочлен степени не более чем $n-1$.
\сспункт
Функция $f$ определена и бесконечно дифференцируема на $\R$,
$f^{(n)}$ --- ее $n$-тая производная.
Пусть для каждого $a\in\R$ найдется такое $n\in\N$, что
$f^{(n)}(a)=0$. Докажите, что $f$ --- многочлен.
\кзадача


\ЛичныйКондуит{0mm}{8mm}

% %\GenXMLW

\end{document}

Положим
$$\Psi(x)=\cases{x,& если $0\le x\le0,5$;\cr
1-x,& если $0,5\le x\le 1$\cr}
$$
и продолжим эту функцию на всю числовую ось с периодом 1.
Эту продолженную функцию обозначим через $\varphi_0.$ Пусть, далее,
$\varphi_n(x)=\frac{1}{4^n}\varphi_0(4^nx).$
\сНовойСтроки
\пункт Нарисуйте графики функций $\varphi_0, \varphi_1, \varphi_2$.
\пункт Найдите период функции $\varphi_n$, е\"е точки дифференцируемости.
\пункт Нарисуйте графики функций $\varphi_0+\varphi_1$,
$\varphi_0+\varphi_1+\varphi_2$.
\пункт Пусть $f(x)=\sum\limits_{n=1}^{+\infty}\varphi_n(x).$

\сзадача
\пункт \выд{(Дискретное правило Лопиталя)}
Пусть $x_n$ и $y_n$ --- две
бесконечно малые последовательности, причем $y_n\ne 0$ при всех $n$.
Предположим, что существует предел
$\lim\limits_{n\to \infty}\frac{x_{n+1}-x_n}{y_{n+1}-y_n}$.
Докажите, что существует предел
$\lim\limits_{n\to \infty}\frac{x_n}{y_n}$ и найдите,
чему он равен.
\пункт Сформулируйте и докажите дискретный аналог задачи \ref{lopinf}б).
\кзадача

\задача Докажите, что
\вСтрочку
\пункт многочлен имеет кратный корень тогда и только
тогда, когда он имеет общий корень со своей производной;
\пункт %Докажите, что
при дифференцировании кратность корня многочлена
понижается на~1;
\пункт
многочлен из $\Q[x]$, неприводимый над $\Q$,
не может иметь кратный действительный корень.
\кзадача


\задача
Пусть $f(x)=(x-a_1)^{r_1}\cdot\dots\cdot(x-a_n)^{r_n}$, где
числа $a_1,\dots,a_n$ действительные и различные, а числа
$r_1,\dots,r_n$ --- целые.
Докажите, что
$
\frac{f'(x)}{f(x)}=\frac{r_1}{x-a_1}+\dots+\frac{r_n}{x-a_n}.
$
\кзадача

\задача Пусть $f$ --- дифференцируемая функция. Известно, что
уравнение $f(x)=0$ имеет $n$  решений.
Сколько  решений может иметь уравнение $f'(x)=0$?
\кзадача

\задача
\пункт Пусть $c\in\R$ и $f(x)\rightarrow c$ при $x\rightarrow+\infty$.
Верно ли, что $f'(x)\rightarrow0$ при $x\rightarrow+\infty$?
\пункт Пусть $f'(x)\rightarrow0$ при $x\rightarrow+\infty$.
Верно ли, что $f(x)\rightarrow c$ при $x\rightarrow+\infty$
для некоторого $c\in\R$?
\кзадача 