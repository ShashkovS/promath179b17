% !TeX encoding = windows-1251
\documentclass[a4paper,12pt]{article}
\usepackage{newlistok}

%\УвеличитьШирину{1.5cm}
%\УвеличитьВысоту{1.5cm}
\renewcommand{\spacer}{\vfil}
\sloppy
\begin{document}

\Заголовок{Полное исследование функций}
%\Заголовок{Зачётная работа: полное исследование функций}
\НомерЛистка{31}
\ДатаЛистка{09.2016}
\СоздатьЗаголовок






\раздел{Схема полного исследования функции}
\выдд Задача. Исследовать функцию $y=f(x)$ и построить её график.

%\vspace{-4mm}
\begin{nums}{0}
\УстановитьГраницы{-.5cm}{0cm}
\item Укажите естественную область определения функции, если область определения не задана.
\item На черновике набросайте приблизительный график функции для того, чтобы лучше понимать, о чём речь. После каждого пункта отмечайте на графике разрывы, экстремумы, нули и т.д. При необходимости обновите набросок графика.
\item Укажите множество значений (скорее всего, она будет известна после нахождения экстремумов).
\item Выясните наличие симметрий у графика (чётность, нечётность функции). Выясните, периодична ли данная функция, и, если это возможно, найдите её минимальный положительный период.
\item Найдите нули функции и точки разрыва (если они есть).
\item Найдите промежутки знакопостоянства функции (обычно, методом интервалов). Результат --- таблица ($a_1$, $a_2$ --- нули функции или точки, где $f(x)$ не определена или разрывна):

\centerline{
\begin{tabular}{|c|c|c|c|c|c|}
\hline
&$(-\infty,a_1)$&$a_1$&$(a_1,a_2)$&$a_2$&$(a_2,+\infty)$\\
\hline
\hline
$f(x)$&$+$&$0$&$+$&$0$&$-$\\
\hline
\end{tabular}}

\item Изучите поведение функции в граничных точках области определения, в окрестности точек разрыва и на бесконечности.
\item Найдите все асимптоты. Найдите точку пересечения графика с наклонной асимптотой.
\item Найдите производную функцию $f'(x)$ в тех точках, где она существует, критические точки, и исследуйте функцию на монотонность и экстремумы с помощью первой производной. Результат --- таблица ($b_1$, $b_2$ --- точки, в которых производная равна нулю или не определена):

\centerline{
\begin{tabular}{|c|c|c|c|c|c|}
\hline
&$(-\infty,b_1)$&$b_1$&$(b_1,b_2)$&$b_2$&$(b_2,+\infty)$\\
\hline
\hline
$f'(x)$&$+$&$0$&$+$&$0$&$-$\\
\hline
\hline
$f(x)$&$\nearrow$&$f(b_1)$&$\nearrow$&$f(b_2)$&$\searrow$\\
\hline
\hline
&&---&&максимум&\\
\hline
\end{tabular}}

\item Найдите вторую производную $f''(x)$ в тех точках, где она существует, и исследуйте функцию на выпуклость и точки перегиба с помощью второй производной. Результат --- таблица ($c_1$, $c_2$ --- точки, в которых вторая производная либо равна нулю, либо не определена):

\centerline{
\begin{tabular}{|c|c|c|c|c|c|}
\hline
&$(-\infty,с_1)$&$с_1$&$(с_1,с_2)$&$с_2$&$(с_2,+\infty)$\\
\hline
\hline
$f''(x)$&$+$&$0$&$+$&$0$&$-$\\
\hline
\hline
$f'(x)$&&$f'(c_1)$&&$f'(c_2)$&\\
\hline
\hline
$f(x)$&$\begin{picture}(20,10)\put(10,5){\oval(10,10)[b]}\end{picture}$&$f(c_1)$&\begin{picture}(20,10)\put(10,5){\oval(10,10)[b]}\end{picture}&$f(c_2)$&\begin{picture}(20,10)\put(10,0){\oval(10,10)[t]}\end{picture}\\
\hline
\hline
&&---&&т.перегиба&\\
\hline
\end{tabular}}

\item Сделайте крупный эскиз графика функции, отметив на нём характерные особенности графика (\hbox{см.~пп.~1--10}) и некоторые контрольные значения, в частности точку пересечения с осью ординат, и, если это возможно, точки пересечения с осью абсцисс.
\end{nums}




\задача
%Постройте (с полным исследованием) %\footnote{т.~е. как можно более
%точным указанием области определения, множества значений,
%промежутков возрастания - убывания, локальных экстремумов,
%направлений выпуклости, перегибов и асимптот})
Постройте (с полным исследованием) графики следующих функций:\\
\вСтрочку
\пункт $x+\frac1x$;
\пункт $\frac{x+3}{2-x}$;
%\пункт $\frac{x^{2}-4\,x+3}{x+1}$;
%\пункт $\frac{(x+1)(x-2)(x+3)}{x^{2}+1}$;
\пункт $\sqrt{x\,(1+x)}$;
\пункт $x\arctg x$;
\пункт $\frac{x}{(x+1)^{2}}$;
\пункт  $\root 3 \of{9-x^3}$;
\пункт $\frac{x^3}{1-x^2}$;
%\пункт $\frac{1-2\,x+4\,{x}^{2}}{1-2\,x+2\,{x}^{2}}$;
%\пункт $\frac{\arcsin x}{\sqrt{1-x^2}}$;
%\пункт $\sqrt[3]{\frac{x^2}{x+1}}$;
\спункт $\frac{\cos x}{\cos2x}$.
%\пункт $\frac{x^3+x^2-x+2}{x^2+x}$.
\кзадача

\ЛичныйКондуит{0mm}{8mm}

% %\GenXMLW

\end{document}


\setcounter{problemnum}{0}
\задача
%Постройте (с полным исследованием) %\footnote{т.~е. как можно более
%точным указанием области определения, множества значений,
%промежутков возрастания - убывания, локальных экстремумов,
%направлений выпуклости, перегибов и асимптот})
Постройте (с полным исследованием) графики следующих функций:\\
\вСтрочку
\таааа{
\пункт $x+\frac1x$;}{
\пункт $\frac{x+3}{2-x}$;}{
\пункт $\sqrt{x\,(1+x)}$;}{
\пункт $x\arctg x$;}
\таааа{
\пункт $\frac{x}{(x+1)^{2}}$;}{
\пункт  $\root 3 \of{9-x^3}$;}{
\пункт $\frac{x^3}{1-x^2}$;}{
\спункт $\frac{\cos x}{\cos2x}$.}
\кзадача









\задача
\вСтрочку\\
\пункт
$\dfrac{x^2}{2-x}$
\пункт
$\dfrac{x}{1-x^2}$
\пункт
$\dfrac{x^2+x+4}{x^2-4}$
\пункт
$\sin(\pi\cos(x)+1)$\\
\пункт
$\sqrt{x(1+x)}$
\пункт
$x^{\frac23} e^{-x}$
\пункт
$\arcsin\dfrac{2x}{1+x^2}$
\кзадача

\end{document}

\setcounter{problemnum}{0}
\задача[2]
\вСтрочку
\пункт
$\dfrac{x}{(x+2)^2}$
\пункт
$x^{\frac23} e^{-x}$
\пункт
$\dfrac{x^2+x+4}{x^2-4}$
\пункт
$\dfrac{x^2}{2-x}$
\пункт
$\sin(\pi\sin(x)+3)$
\пункт
$\arcsin\dfrac{2x}{1+x^2}$
\пункт
$\sqrt[3]{x^3+x^2}$
\кзадача


\setcounter{problemnum}{0}
\задача[3]
\вСтрочку
\пункт
$\dfrac{(x-1)^2}{x^2}$
\пункт
$x+e^{-x}$
\пункт
$\dfrac{x^2+x+4}{x^2-4}$
\пункт
$\dfrac{x^2}{2-x}$
\пункт
$\cos(\pi\sin(x)+3)$
\пункт
$\arcsin\dfrac{2x}{1+x^2}$
\пункт
$\sqrt[3]{x^3+2}$
\кзадача


\setcounter{problemnum}{0}
\задача[4]
\вСтрочку
\пункт
$\sqrt[3]{x^3+2}$
\пункт
$x+e^{-x}$
\пункт
$\cos(\pi\sin(x)+3)$
\пункт
$x + \dfrac{1}{x}$
\пункт
$\dfrac{x^2+x-1}{(x-1)^2}$
\пункт
$\arcsin\dfrac{2x}{1+x^2}$
\пункт
$\dfrac{(x-1)^2}{x^2}$
\кзадача


\setcounter{problemnum}{0}
\задача[5]
\вСтрочку
\пункт
$\dfrac{x^2+x-1}{(x-1)^2}$
\пункт
$\cos(\pi\cos(x)+3)$
\пункт
$x + \dfrac{1}{x}$
\пункт
$x^3 e^{-x}$
\пункт
$\arcsin\dfrac{2x}{1+x^2}$
\пункт
$\dfrac{(x-1)^2}{x^2}$
\пункт
$(x-5)\sqrt[3]{x^2}$
\кзадача

\setcounter{problemnum}{0}
\задача[6]
\вСтрочку
\пункт
$\arcsin\dfrac{2x}{1+x^2}$
\пункт
$\dfrac{x^2+x-1}{(x-1)^2}$
\пункт
$\dfrac{x^3}{3(x+1)^2}$
\пункт
$x^3 e^{-x}$
\пункт
$\cos(\pi\cos(x)+3)$
\пункт
$(x-5)\sqrt[3]{x^2}$
\пункт
$\sqrt{\dfrac{1-x}{x+2}}$
\кзадача




\setcounter{problemnum}{0}
\задача[7]
\вСтрочку
\пункт
$\arcsin\dfrac{2x}{1+x^2}$
\пункт
$(x-5)\sqrt[3]{x^2}$
\пункт
$\dfrac{x^3}{3(x+1)^2}$
\пункт
$\dfrac{x^2+x-1}{(x-1)^2}$
\пункт
$\sin(\sin x + \cos x)$
\пункт
$x^3 e^{-x}$
\пункт
$\sqrt{\dfrac{1-x}{x+2}}$
\кзадача




\setcounter{problemnum}{0}
\задача[8]
\вСтрочку
\пункт
$(x-5)\sqrt[3]{x^2}$
\пункт
$\dfrac{x^3}{3(x+1)^2}$
\пункт
$\dfrac{x^2(x-2)}{2(x-1)^2}$
\пункт
$\sqrt{\dfrac{1-x}{x+2}}$
\пункт
$x^3 e^{-x}$
\пункт
$\cos(\sin x + \cos x)$
\пункт
$\arcsin\dfrac{2x}{1+x^2}$
\кзадача






\setcounter{problemnum}{0}
\задача[9]
\вСтрочку
\пункт
$\sqrt{\dfrac{1-x}{x+2}}$
\пункт
$(x-5)\sqrt[3]{x^2}$
\пункт
$\arcsin\dfrac{2x}{1+x^2}$
\пункт
$\dfrac{x^3}{3(x+1)^2}$
\пункт
$\dfrac{x^2(x-2)}{2(x-1)^2}$
\пункт
$\dfrac{e^x}{1+x}$
\пункт
$\cos(\sin x + \cos x)$
\кзадача

\setcounter{problemnum}{0}
\задача[10]
\вСтрочку
\пункт
$\arcsin\dfrac{2x}{1+x^2}$
\пункт
$\sqrt{\dfrac{1-x}{x+2}}$
\пункт
$(x-5)\sqrt[3]{x^2}$
\пункт
$\cos(\sin x + \cos x + \dfrac{\pi}{6})$
\пункт
$\dfrac{6x+4}{x^3}$
\пункт
$\dfrac{x^2(x-2)}{2(x-1)^2}$
\пункт
$\dfrac{e^x}{1+x}$
\кзадача

\setcounter{problemnum}{0}
\задача[11]
\вСтрочку
\пункт
$\sqrt[3]{x^3-3x}$
\пункт
$\dfrac{e^x}{1+x}$
\пункт
$\arcsin\dfrac{2x}{1+x^2}$
\пункт
$\dfrac{x}{\ln x}$
\пункт
$\cos(\sin x + \cos x + \dfrac{2\pi}{3})$
\пункт
$\dfrac{6x+4}{x^3}$
\пункт
$x + \sqrt{x^2-1}$
\кзадача




\setcounter{problemnum}{0}
\задача[12]
\вСтрочку
\пункт
$\dfrac{x}{\ln x}$
\пункт
$\sqrt[3]{x^3-3x}$
\пункт
$\cos(\sin x + \cos x + \dfrac{2\pi}{3})$
\пункт
$\dfrac{e^x}{1+x}$
\пункт
$\arcsin\dfrac{2x}{1+x^2}$
\пункт
$\dfrac{x^2-x}{x^2-4}$
\пункт
$x + \sqrt{x^2-1}$
\кзадача

\break


\setcounter{problemnum}{0}
\задача[13]
\вСтрочку
\пункт
$x e ^{\frac{1}{2-x}}$
\пункт
$\dfrac{x}{\ln x}$
\пункт
$\sqrt[3]{x^3-3x}$
\пункт
$\dfrac{x^2-x}{x^2-4}$
\пункт
$\sin(\sin x + \cos x + \dfrac{2\pi}{3})$
\пункт
$x + \sqrt{x^2-1}$
\пункт
$\arcsin\dfrac{2x}{1+x^2}$
\кзадача



\setcounter{problemnum}{0}
\задача[14]
\вСтрочку
\пункт
$x e ^{\frac{1}{2-x}}$
\пункт
$\dfrac{x}{\ln x}$
\пункт
$\sqrt[3]{x^3-3x}$
\пункт
$\dfrac{x^2-x}{x^2-4}$
\пункт
$\sin(\sin x + \dfrac{2\pi}{3})$
\пункт
$x + \sqrt{x^2-1}$
\пункт
$\arcsin\dfrac{2x}{1+x^2}$
\кзадача


\setcounter{problemnum}{0}
\задача[15]
\вСтрочку
\пункт
$\cos(\sin x + \cos x + \dfrac{\pi}{6})$
\пункт
$\arcsin\dfrac{2x}{1+x^2}$
\пункт
$\sqrt{\dfrac{1-x}{x+2}}$
\пункт
$(x-5)\sqrt[3]{x^2}$
\пункт
$\dfrac{x^2(x-2)}{2(x-1)^2}$
\пункт
$\dfrac{6x+4}{x^3}$
\пункт
$\dfrac{e^x}{1+x}$
\кзадача


\setcounter{problemnum}{0}
\задача[16]
\вСтрочку
\пункт
$\arcsin\dfrac{2x}{1+x^2}$
\пункт
$\dfrac{x^3}{x^2+1}$
\пункт
$\dfrac{x^2+x+4}{x^2-4}$
\пункт
$\sin(\pi\cos(x)+1)$
\пункт
$\dfrac{x^2}{2-x}$
\пункт
$\sqrt{x(1+x)}$
\пункт
$x^{\frac23} e^{-x}$
\кзадача






\setcounter{problemnum}{0}
\задача[17]
\вСтрочку
\пункт
$x^3 e^{-x}$
\пункт
$\dfrac{x^3}{3(x+1)^2}$
\пункт
$\dfrac{x^2(x-2)}{2(x-1)^2}$
\пункт
$(x-5)\sqrt[3]{x^2}$
\пункт
$\sqrt{\dfrac{1-x}{x+2}}$
\пункт
$\cos(\sin x + \cos x)$
\пункт
$\arcsin\dfrac{2x}{1+x^2}$
\кзадача


\setcounter{problemnum}{0}
\задача[18]
\вСтрочку
\пункт
$\dfrac{x^2+x-1}{(x-1)^2}$
\пункт
$\arcsin\dfrac{2x}{1+x^2}$
\пункт
$\sqrt[3]{x^3+2}$
\пункт
$x+e^{-x}$
\пункт
$\cos(\pi\sin(x)+3)$
\пункт
$x + \dfrac{1}{x}$
\пункт
$\dfrac{(x-1)^2}{x^2}$
\кзадача



\setcounter{problemnum}{0}
\задача[19]
\вСтрочку
\пункт
$x + \dfrac{1}{x}$
\пункт
$\dfrac{(x-1)^2}{x^2}$
\пункт
$\dfrac{x^2+x-1}{(x-1)^2}$
\пункт
$x^3 e^{-x}$
\пункт
$\cos(\pi\cos(x)+3)$
\пункт
$\arcsin\dfrac{2x}{1+x^2}$
\пункт
$(x-5)\sqrt[3]{x^2}$
\кзадача






\setcounter{problemnum}{0}
\задача[20]
\вСтрочку
\пункт
$\dfrac{x}{\ln x}$
\пункт
$\sqrt[3]{x^3-3x}$
\пункт
$\arcsin\dfrac{2x}{1+x^2}$
\пункт
$\cos(\sin x + \cos x + \dfrac{2\pi}{3})$

\пункт
$\dfrac{6x+4}{x^3}$
\пункт
$\dfrac{e^x}{1+x}$
\пункт
$x + \sqrt{x^2-1}$
\кзадача





\setcounter{problemnum}{0}
\задача[21]
\вСтрочку
\пункт
$\sqrt{\dfrac{1-x}{x+2}}$
\пункт
$\dfrac{x^2+x-1}{(x-1)^2}$
\пункт
$\dfrac{x^3}{3(x+1)^2}$
\пункт
$(x-5)\sqrt[3]{x^2}$

\пункт
$x^3 e^{-x}$
\пункт
$\arcsin\dfrac{2x}{1+x^2}$
\пункт
$\cos(\pi\cos(x)-2)$
\кзадача


















\setcounter{problemnum}{0}
\задача[22]
\вСтрочку
\пункт
$\dfrac{x^3}{3(x+1)^2}$
\пункт
$\dfrac{e^x}{1-x}$
\пункт
$\sqrt{\dfrac{1-x}{x+2}}$
\пункт
$\arcsin\dfrac{2x}{1+x^2}$

\пункт
$(x-5)\sqrt[3]{x^2}$
\пункт
$\dfrac{x^2(x-2)}{2(x-1)^2}$
\пункт
$\cos(\sin x + \cos x)$
\кзадача



\setcounter{problemnum}{0}
\задача[23]
\вСтрочку
\пункт
$-x^3 e^{x}$
\пункт
$\cos(\sin x + \cos x)$
\пункт
$\arcsin\dfrac{2x}{1+x^2}$
\пункт
$(x-5)\sqrt[3]{x^2}$

\пункт
$\dfrac{x^3}{3(x+1)^2}$
\пункт
$\dfrac{x^2(x-2)}{2(x-1)^2}$
\пункт
$\sqrt{\dfrac{1-x}{x+2}}$
\кзадача


\setcounter{problemnum}{0}
\задача[24]
\вСтрочку
\пункт
$x + \sqrt{x^2-1}$
\пункт
$\sqrt[3]{x^3-3x}$
\пункт
$\cos(\cos x + \dfrac{\pi}{4})$
\пункт
$\dfrac{e^x}{1+x}$
\пункт
$\dfrac{x}{\ln x}$
\пункт
$\dfrac{x^2-x}{x^2-4}$
\пункт
$\arcsin\dfrac{2x}{1+x^2}$
\кзадача















\end{document}







\задача
Постройте (с полным исследованием) графики следующих функций:
\вСтрочку
\пункт $x+\frac1x$;
\пункт $\frac{x+3}{2-x}$;
%\пункт $\frac{x^{2}-4\,x+3}{x+1}$;
%\пункт $\frac{(x+1)(x-2)(x+3)}{x^{2}+1}$;
\пункт $\sqrt{x\,(1+x)}$;
\пункт $x\arctg x$;
\пункт $\frac{x}{(x+1)^{2}}$;
\пункт  $\root 3 \of{9-x^3}$;
\пункт $\frac{x^3}{1-x^2}$;
%\пункт $\frac{1-2\,x+4\,{x}^{2}}{1-2\,x+2\,{x}^{2}}$;
%\пункт $\frac{\arcsin x}{\sqrt{1-x^2}}$;
%\пункт $\sqrt[3]{\frac{x^2}{x+1}}$;
\спункт $\frac{\cos x}{\cos2x}$.
%\пункт $\frac{x^3+x^2-x+2}{x^2+x}$.
\кзадача








\раздел{Схема полного исследования функции}
\выдд Задача. Исследовать функцию $y=f(x)$ и построить её график.

%\vspace{-4mm}
\begin{nums}{0}
\УстановитьГраницы{-.5cm}{0cm}
\item Укажите естественную область определения функции, если область определения не задана.
\item Укажите множество значений (скорее всего, она будет известна после нахождения экстремумов).
\item Выясните наличие симметрий у графика (чётность, нечётность функции). Выясните, периодична ли данная функция, и, если это возможно, найдите её минимальный положительный период.
\item Найдите нули функции и точки разрыва (если они есть).
\item Найдите промежутки знакопостоянства функции (обычно, методом интервалов). Результат --- таблица ($a_1$, $a_2$ --- нули функции или точки, где $f(x)$ не определена или разрывна):

\centerline{
\begin{tabular}{|c|c|c|c|c|c|}
\hline
&$(-\infty,a_1)$&$a_1$&$(a_1,a_2)$&$a_2$&$(a_2,+\infty)$\\
\hline
\hline
$f(x)$&$+$&$0$&$+$&$0$&$-$\\
\hline
\end{tabular}}

\item Изучите поведение функции в граничных точках области определения, в окрестности точек разрыва и на бесконечности.
\item Найдите все асимптоты. Найдите точку пересечения графика с наклонной асимптотой.
\item Найдите производную функцию $f'(x)$ в тех точках, где она существует, критические точки, и исследуйте функцию на монотонность и экстремумы с помощью первой производной. Результат --- таблица ($b_1$, $b_2$ --- точки, в которых производная равна нулю или не определена):

\centerline{
\begin{tabular}{|c|c|c|c|c|c|}
\hline
&$(-\infty,b_1)$&$b_1$&$(b_1,b_2)$&$b_2$&$(b_2,+\infty)$\\
\hline
\hline
$f'(x)$&$+$&$0$&$+$&$0$&$-$\\
\hline
\hline
$f(x)$&$\nearrow$&$f(b_1)$&$\nearrow$&$f(b_2)$&$\searrow$\\
\hline
\hline
&&---&&максимум&\\
\hline
\end{tabular}}

\item Найдите вторую производную $f''(x)$ в тех точках, где она существует, и исследуйте функцию на выпуклость и точки перегиба с помощью второй производной. Результат --- таблица ($c_1$, $c_2$ --- точки, в которых вторая производная либо равна нулю, либо не определена):

\centerline{
\begin{tabular}{|c|c|c|c|c|c|}
\hline
&$(-\infty,с_1)$&$с_1$&$(с_1,с_2)$&$с_2$&$(с_2,+\infty)$\\
\hline
\hline
$f''(x)$&$+$&$0$&$+$&$0$&$-$\\
\hline
\hline
$f'(x)$&&$f'(c_1)$&&$f'(c_2)$&\\
\hline
\hline
$f(x)$&$\begin{picture}(20,10)\put(10,5){\oval(10,10)[b]}\end{picture}$&$f(c_1)$&\begin{picture}(20,10)\put(10,5){\oval(10,10)[b]}\end{picture}&$f(c_2)$&\begin{picture}(20,10)\put(10,0){\oval(10,10)[t]}\end{picture}\\
\hline
\hline
&&---&&т.перегиба&\\
\hline
\end{tabular}}

\item Сделайте крупный эскиз графика функции, отметив на нём характерные особенности графика (\hbox{см.~пп.~1--9}) и некоторые контрольные значения, в частности точку пересечения с осью ординат, и, если это возможно, точки пересечения с осью абсцисс.
\end{nums}


