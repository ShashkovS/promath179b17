% !TeX encoding = windows-1251
\documentclass[a4paper, 12pt]{article}
\usepackage{newlistok}
%\documentstyle[11pt, russcorr, listok]{article}

\УвеличитьШирину{1.3truecm}
\УвеличитьВысоту{3truecm}
%\hoffset=-2.7truecm
%\voffset=-24truemm
%\pagestyle{empty}

%\documentstyle[11pt, russcorr, ll]{article}
%\def\dfrac{\displaystyle\frac}

\Подзаголовок{}
\НомерЛистка{18}
\ДатаЛистка{01.2015}
\Заголовок{Предел последовательности.}

\begin{document}

%\scalebox{.91}{\vbox{%
%\ncopy{1}{

\СоздатьЗаголовок

\опр \label{limit3} Число $a$ называют \textit{пределом последовательности}
$(x_n)$, если %последовательность $(x_n-a)$ является бесконечно малой.
%\задача
%Докажите, что последовательность $(x_n)$ сходится к числу $a$
%тогда и только тогда, когда
$(x_n)$ можно представить в виде
$x_n=a+\alpha_n$, где $(\alpha_n)$ --- бесконечно малая
последовательность.
%\кзадача
Обозначение: $\lim\limits_{n \to \infty} x_n = a$.
Говорят также, что \textit{$(x_n)$ стремится к $a$ при $n$,
стремящемся к бесконечности}
(и пишут $x_n \to a$ при~\hbox{$n \to \infty$)}.
%или говорят, что \textit{$(x_n)$ сходится к $a$}.
\копр

\задача %Докажите, что
Может ли последовательность
%не может
иметь более одного предела?
\кзадача

\задача Найдите предел
%последовательности
$(x_n)$, если он есть:
\вСтрочку
%\пункт $x_n=1+(-1)^n$;
\пункт \hbox{$x_n=1+(-0,1)^n$;}
\пункт $x_n=\dfrac{n}{n+1}$;
%\пункт $x_n=n/(n+1)$;
\пункт $x_n=(-1)^n$;\\
%\пункт $x_n=\dfrac{2^n-1}{2^n+1}$;
\пункт $x_n=\dfrac{2^n-1}{2^n+1}$;
\пункт $x_n=1+0,1+\ldots+(0,1)^n$;
\пункт $x_n=\dfrac{1+3+\ldots+3^n}{5^{n}}$;
\пункт $x_n=\sqrt{n+1}-\sqrt{n}$.
\кзадача

\опр %Пусть $\varepsilon>0$. % --- произвольное положительное число.
\textit{Окрестность} точки $a$ %(где $\varepsilon>0$)
--- это %называют
любой интервал, содержащий точку $a$.
%$(a-\varepsilon, a+\varepsilon)$. % = \{ x: |x-a| < \varepsilon \}$.
Обозначение: ${\cal U(\it a)}$. % _\varepsilon(a)$.
\копр

\задача
Докажите, что любые две точки на прямой имеют непересекающиеся
окрестности.
\кзадача

\опр \label{limit1} Число $a$ называют \textit{пределом
последовательности}
$(x_n)$, если %для каждого $\varepsilon>0$\break
%в $\varepsilon$-окрест\-нос\-ти
в любой окрестности числа $a$
содержатся \textit{почти все} члены $(x_n)$
(то есть все, кроме конечного числа).
%Обозначение:
%$\lim\limits_{n \to \infty} x_n = a$.
%Говорят также, что $(x_n)$ стремится к $a$ при $n$,
%стремящемся к бесконечности
%(и пишут $x_n \to a$ при~\hbox{$n \to \infty$).}
\копр

\опр \label{limit2} Число $a$ называют
\textit{пределом последовательности} $(x_n)$,
если для всякого числа $\varepsilon >0$
найд\"ется такое число $N$, что при любом
натуральном $k > N$ будет выполнено
неравенство $|x_k - a| < \varepsilon$.
\копр

%\задача Докажите, что последовательность не может иметь более одного предела.
%\кзадача

\задача Докажите эквивалентность
определений~\ref{limit3},~\ref{limit1}~и~\ref{limit2}.
%(т.е. докажите, что
%число $a$ является пределом последовательности $(x_n)$ в смысле
%определения~\ref{limit1} тогда и только тогда, когда $a$ является пределом
%последовательности $(x_n)$ в смысле определения~\ref{limit2}).
\кзадача

%\задача Докажите эквивалентность
%определений~\ref{limit3}~и~\ref{limit2}.
%(т.е. докажите, что
%число $a$ является пределом последовательности $(x_n)$ в смысле
%определения~\ref{limit1} тогда и только тогда, когда $a$ является пределом
%последовательности $(x_n)$ в смысле определения~\ref{limit2}).
%\кзадача

\задача
Запишите % утверждения,
без отрицания:
\вСтрочку
\пункт
%Напишите, что значит, что
\лк число
$a$ не предел %последовательности
$(x_n)$\пк;
\пункт
%Напишите, что значит, что
\лк %последовательность
$(x_n)$ не имеет~предела\пк.
\кзадача

%\задача
%Последовательность имеет предел. Обязательно ли она ограничена?
%\кзадача

\задача
%Последовательность имеет ненулевой предел.
$\lim\limits_{n \to \infty}(x_n)>0$. %положителен.
Верно ли, что
\пункт $\!\!x_n>0$ при $n\gg0$;
%все члены $(x_n)$, начиная с некоторого, положительны;\\
\пункт $\!\!(1/x_n)$ ограничена (если определена)?
%отличны от нуля и имеют один и тот же знак?
\кзадача

%\задача Последовательность $(x_n)$ имеет предел $a$.
%\вСтрочку
%\пункт Обязательно ли $(x_n)$ ограничена?\\
%\пункт Пусть $a>0$ и все члены $(x_n)$ положительны.
%Докажите, что последовательность $(1/x_n)$ ограничена.
%\кзадача

%\задача Предел $(x_n)$ равен $a$.
%Обязательно ли ограничена
%\вСтрочку
%\пункт $\!\!(x_n)$;
%\пункт $\!\!(1/x_n)$, если $a>0$ и члены $(x_n)$~\hbox{положительны?}
%Докажите, что последовательность ограничена.
%\кзадача

\задача
Пусть
$\lim\limits_{n \to \infty} x_n = a$,
$\lim\limits_{n \to \infty} y_n = b$.
%Докажите:
Найдите
\вСтрочку
%\пункт $\lim\limits_{n \to \infty} (x_n \pm y_n)= a\pm b$;
%\пункт $\lim\limits_{n \to \infty} (x_n\cdot y_n) = ab$;\\
%\пункт если $b\ne0$ и все элементы
%последовательности $(y_n)$ отличны от нуля, то
%$\lim\limits_{n \to \infty} (x_n/y_n) = a/b$.
\пункт $\lim\limits_{n \to \infty} x_n \pm y_n$;
\пункт $\lim\limits_{n \to \infty} x_n\cdot y_n$.\\
\пункт
Что можно сказать о
$\lim\limits_{n \to \infty} x_n/y_n$?
%если $b\ne0$ и все элементы
%последовательности $(y_n)$ отличны от нуля, то
\кзадача

\задача Найти предел %последовательности
($n\! \to \!\infty$):
\вСтрочку
%\пункт $x_n=(-1)^n$;
\пункт $\!\!1+q+\ldots+q^{n}$, где $|q|<1$;
\пункт $\!\!\dfrac{n^2-n+1}{n^2}$;
%\пункт $x_n=\sqrt{n^2+n}-\sqrt{n}$;
\пункт $\!\!\sqrt[n]{2}$;
\пункт $\!\!\dfrac{n^{50}}{2^n}$;
\пункт $\!\!\root n \of n$.
\кзадача


\задача
Может ли последовательность без наименьшего и
наибольшего членов иметь предел?
\кзадача

%\задача Пусть $A(x)=a_k x^k+\ldots+a_1x+a_0$ и
%$B(x)=b_m x^m+\ldots+b_1x+b_0$ --- многочлены степеней $k$ и $m$
%соответственно. Найдите пределы:
%\вСтрочку
%\пункт $\lim\limits_{n \to \infty}A(n)/n^k$;
%\пункт $\lim\limits_{n \to \infty}A(n)/B(n)$.
%\кзадача

\задача
\вСтрочку
\пункт Последовательность
$(x_n)$ %известно, что она
имеет предел.
Докажите, что %последовательность
$(x_{n+1}-x_n)$ бесконечно малая.
\пункт Верно ли обратное?
\кзадача

\задача Последовательность $(x_n)$ положительна, а последовательность $(x_{n+1}/x_n)$ имеет пределом
некоторое число, меньшее 1. Докажите, что %последовательность
$(x_n)$ бесконечно малая.
\кзадача

\задача Найдите: % пределы: % последовательностей:
\вСтрочку
\пункт $\displaystyle{\lim\limits_{n \to \infty}\frac{4n^2}{n^2+n+1}}$
\пункт $\displaystyle{\lim\limits_{n \to \infty}\frac{n^2+2n-2}{n^3+n}}$;
\пункт $\displaystyle{\lim\limits_{n \to \infty}\frac{n^9-n^4+1}{2n^9+7n-5}}$;
\пункт $\displaystyle{\lim\limits_{n \to \infty}\frac{C^{50}_n}{n^{50}}}$.
\кзадача

\vspace*{1mm}

\задача Найдите ошибку в рассуждении:
%Рассмотрим последовательность
\лк Пусть $x_n=(n-1)/n$. Тогда
%С одной стороны,
$\lim\limits_{n \to \infty} x_n =
\lim\limits_{n \to \infty}(1-1/n)=1$.\break
С другой стороны,
$\lim\limits_{n \to \infty} x_n =
\lim\limits_{n \to \infty}(1/n)\cdot\lim\limits_{n \to \infty}(n-1)
= 0\cdot\lim\limits_{n \to \infty} (n-1)= 0$.
Отсюда $0=1$.\пк
\кзадача

%\задача Про последовательность $(x_n)$ известно, что она
%имеет предел и содержит бесконечно много положительных
%и бесконечно много отрицательных членов.
%Докажите, что %эта последовательность
%$(x_n)$ бесконечно малая.
%\кзадача


\задача
Пусть $\lim\limits_{n \to \infty} x_n =a$, $\lim\limits_{n \to
\infty} y_n = b$ и $x_n>y_n$ при $n\in\N$.
Верно ли, что
\вСтрочку
\пункт $a>b$;
\пункт $a\geq b$?
\кзадача


\задача
Обобщите теорему о двух милиционерах из листка 16
на последовательности, имеющие предел.
\кзадача

%\задача[\лк Теорема о двух милиционерах\пк]
%Пусть $\lim\limits_{n \to \infty} x_n =\lim\limits_{n \to
%\infty} y_n = а$, и последовательность $(z_n)$ такова,
%что $x_n\leq z_n\leq y_n$ при любом номере $n$.  Докажите,
%что $\lim\limits_{n \to \infty} z_n = a$.
%\кзадача

%\задача Известно, что $\lim\limits_{n \to \infty} x_n = 1$.
%Найдите предел последовательности $(y_n)$, если\\
%\вСтрочку
%\пункт $y_n=(2x_n-1)/(x_n+1)$;
%\пункт $y_n=(x_n^2+x_n-2)/(x_n-1)$;
%\пункт $y_n=\sqrt{x_n}$;
%\спункт $y_n=(x_1+\ldots+x_n)/n$.
%\кзадача

%\vspace*{-2.1truemm}

\задача Пусть $\lim\limits_{n \to \infty}\! x_n = 1$.
Найти
\вСтрочку
\пункт  $\displaystyle{\lim\limits_{n \to \infty}\!\frac{x_n^2}{7}}$;
\пункт  $\displaystyle{\lim\limits_{n \to \infty}\!\frac{x_n^2+x_n-2}{x_n-1}}$;
\пункт  $\lim\limits_{n \to \infty}\!\sqrt{x_n}$.
\кзадача

%\задача Для вычисления квадратного корня из положительного
%числа $a$ можно пользоваться следующим методом
%последовательных приближений. Возьмите любое положительное число
%$x_0$ и постройте последовательность по такому закону:
%$x_{n+1}=0,5\cdot(x_n+a/x_n).$
%\сНовойСтроки
%\вСтрочку
%\пункт
%Докажите, что $\lim\limits_{n\to\infty}x_n=\sqrt a$.\\
%\спункт Сколько понадобится последовательных приближений,
%чтобы найти $\sqrt{10}$ с точностью до $0,0001$,
%если в качестве первого приближения взять $x_0=3$?
%\кзадача


%\задача Про последовательность $(x_n)$ известно, что она
%имеет ненулевой предел и состоит из ненулевых членов.
%Докажите, что последовательность $(1/x_n)$ ограничена.
%\кзадача

%\задача Является ли бесконечно малой последовательность $(x_n)$, где
%%$x_n=\dfrac{n^{100}}{7^n}$?
%$x_n=n^{100}/7^n$?
%\кзадача


\УстановитьГраницы{0cm}{35mm}
\rightpicture{0mm}{17mm}{33mm}{pct_parabola_integral_1}
\rightpicture{0mm}{-17mm}{33mm}{pct_parabola_integral_2}
\задача
\пункт
Дана фигура, ограниченная графиком
функции $y=x^2$, осью $Ox$ и прямой $x=1$. Разобь\"ем
отрезок $[0,1]$ на $n$ равных частей и построим на каждой
части прямоугольник так, чтобы его правая верхняя вершина
лежала на графике (см.~рис. справа).
Сумму
площадей прямоугольников обозначим $S_n$.
%Найдите $\lim\limits_{n \to\infty}(S_n)$.\\
Найдите предел $(S_n)$ при $n \to \infty$.\\
\пункт Построим прямоугольники так, чтобы
их левые верхние вершины лежали на графике. % (см.~рис.).
%Соответствующую
Сумму их площадей~обозначим $s_n$. Докажите,
что %последовательность
$(s_n)$ стре\-мит\-ся к тому же числу,
что и $(S_n)$ (его естественно считать \textit{площадью}
нашей фигуры).\\
%\спункт Найдите площадь фигуры, ограниченной графиком
%функции $y=x^k$, осью $Ox$ и прямой $x=1$ ($k$ ---
%фиксированное натуральное число).
\спункт Решите ту же задачу для функции $y=x^k$, где~\hbox{$k\in\N$.}
\ВосстановитьГраницы
\кзадача





%\setbox3\vbox{
%{\hsize 3.5truecm
%\input epsf
%$$\epsfbox{list13.2}$$
%
%}}
%
%\hbox{\phantom{sdg}\hskip15.5cm\copy3}

%\задача Найдите предел последовательности $(a_n)$, где
%%$\displaystyle{a_n=\frac12+\frac2{2^2}+\frac3{2^3}+\ldots+\frac{n}{2^n}}$.
%$a_n=1/2+2/2^2+3/2^3+\ldots+n/2^n$.
%\кзадача

%\задача
%Пусть $x_n=a_1/1+a_2/2+\dots+a_n/n$, где $a_n=1$, если
%в десятичной записи числа $n$ нет цифры 9, и $a_n=0$
%иначе. Имеет ли эта последовательность предел?
%\кзадача


%\сзадача
%Из клетчатой плоскости вырезали клетки, обе координаты которых
%делятся на 10. Можно ли оставшуюся часть плоскости разрезать
%на доминошки (каждая состоит из двух соседних клеток)?
%\кзадача


% Посл и пределы


%}}}

\ЛичныйКондуит{0mm}{6mm}


%\СделатьКондуит{4.1mm}{7.5mm}

\end{document}

\задача В два сосуда разлили (не поровну)
1 л воды. Из 1-го сосуда перелили
половину имеющейся~в~н\"ем
воды
во 2-ой, затем из 2-го перелили половину оказавшейся
в н\"ем воды в 1-ый, %затем
снова из 1-го  пере\-ли\-ли половину
%оказавшейся в н\"ем
%воды
во 2-ой, и т.~д.
%Докажите, что независимо от того, сколько воды было сначала
%в каждом из сосудов,
%после 100 переливаний в
%них будет $2/3$~л и $1/3$~л воды с точностью до 1 миллилитра.
Сколько воды (с точностью до 1 мл) будет в 1-ом
сосуде после 50 переливаний?
\кзадача


В ДОПЛИСТОК:

\пункт $1/2+2/2^2+3/2^3+\ldots+n/2^n$;
\пункт $f_n/f_{n+1}$, где $f_n$ --- $n$-е число Фибоначчи.


\задача
Дано $m$ последовательностей, сумма которых
стремится к $m\alpha$, и сумма квадратов которых
стремится к $m\alpha^2$. Докажите, что каждая из этих
последовательность стремится к $\alpha$.
\кзадача

\пункт $\sqrt[n]{2^n+3^n}$;

\пункт $\!\!\root n \of n$.

\спункт $\displaystyle{\lim\limits_{n \to \infty}\!\frac{x_1+\ldots+x_n}n}$.


\vspace*{-1mm}

\задача %С незапамятных врем\"ен
Издавна жители островов Чунга и Чанга раз в год
меняются~драгоцен\-ностями. Одновременно~жители Чунги привозят
половину своих драгоценностей на Чангу, а жители
Чанги %~одновре\-мен\-но привозят
треть своих драгоценностей на
Чунгу.
%Так продолжается .
Какая часть драгоценностей находится~на~каж\-дом острове?
(Общий набор драгоценностей постоянен.) %за это время %на островах не менялся.)
\кзадача


%\сзадача Петя вышел из дому и пош\"ел в школу.
%На полпути к школе он решил, что лучше пойти в кино, и
%свернул к кинотеатру. Пройдя половину пути, он
%захотел покататься на коньках и свернул к катку.
%Пройдя половину пути до катка, он подумал, что нужно
%вс\"е-таки учиться, и повернул к школе. Но на полпути
%к школе снова свернул к кинотеатру, и т.~д. Куда
%прид\"ет Петя, если будет так идти?
%\кзадача

\задача Петя шел из дома в школу.
На полпути он решил, что лучше пойти в кино, и
свернул к кинотеатру. На полпути туда %к кинотеатру
он захотел покататься на коньках и пошел на каток.
На полпути к катку он подумал, что надо
%вс\"е-таки
учиться, и повернул к школе. Но на полпути
к ней снова свернул к кинотеатру, и т.~д. Куда
прид\"ет Петя? %, если будет так идти?
\кзадача

----------


\задача[\лк Теорема о двух милиционерах\пк]
Пусть $\lim\limits_{n \to \infty} x_n =\lim\limits_{n \to
\infty} y_n = а$, и последовательность $(z_n)$ такова,
что $x_n\leq z_n\leq y_n$ при любом номере $n$.  Докажите,
что $\lim\limits_{n \to \infty} z_n = a$.
\кзадача



\задача Пусть $A(x)=a_k x^k+\ldots+a_1x+a_0$ и
$B(x)=b_m x^m+\ldots+b_1x+b_0$ --- многочлены степеней $k$ и $m$
соответственно. Найдите:
\вСтрочку
\пункт $\lim\limits_{n \to \infty}A(n)/n^k$;
\пункт $\lim\limits_{n \to \infty}A(n)/B(n)$;
\пункт $\lim\limits_{n \to \infty}(S_k(n)/n^k-n/(k+1))$.
\кзадача

\сзадача Найдите %предел последовательности $(a_n)$, если\\
\вСтрочку
\пункт
$\displaystyle{
\lim\limits_{n\to\infty}
\left(\frac{1^k+2^k+\ldots+n^k}{n^k}-\frac{n}{k+1}\right)}$,
где $k\in\N$;
\пункт
$\displaystyle{\lim\limits_{n\to\infty}\frac{1^1+2^2+\ldots+n^n}{n^n}}$.
\кзадача

%\сзадача
%Дано $n$ последовательностей, сумма которых
%стремится к $n\alpha$, и сумма квадратов которых
%стремится к $n\alpha^2$. Докажите, что каждая из этих
%\кзадача

%\сзадача [Критерий Коши]
%Докажите, что последовательность $(x_n)$ сходится тогда и
%только тогда, когда выполнено условие
%$\quad\forall \varepsilon>0 \quad \exists k\in\N\quad \forall m,n\ge k
%\quad |x_m-x_n|<\varepsilon$.
%\кзадача


\end{document}


\опр Пусть $\varepsilon$ --- произвольное положительное число.
\textit{$\varepsilon$-окрестностью} точки $a$ называется интервал
$(a-\varepsilon, a+\varepsilon) = \{ x: |x-a| < \varepsilon \}$.
Обозначение: ${\cal U}_\varepsilon(a)$.
\копр

\задача
Докажите, что любые две точки на прямой имеют непересекающиеся окрестности.
\кзадача

\опр \label{limit1} Число $a$ называют \textit{пределом} последовательности
$(x_n)$, если для всякого $\varepsilon>0$ в $\varepsilon$-окрестности числа $a$
содержатся \textit{почти все} члены последовательности
(то есть все, кроме, быть может, конечного числа членов). Обозначение:
$\lim\limits_{n \to \infty} x_n = a$.
Говорят также, что $(x_n)$ стремится к $a$ при $n$,
стремящемся к бесконечности
(и пишут $x_n \to a$ при~\hbox{$n \to \infty$).}
\копр




\опр
Последовательность $(x_n)$ называется \textit{монотонно
возрастающей}, если $x_{n+1}>x_n$ при любом натуральном $n$.
\копр

\задача
Дайте определение монотонно убывающей, монотонно невозрастающей
последовательностей.
\кзадача

\задача Исследуйте на монотонность последовательности из задачи 3 листка 11.
\кзадача

\опр Говорят, что $(y_n)$ --- \textit{подпоследовательность}
последовательности
$(x_n)$, если найд\"ется такая монотонно возрастающая последовательность
$(k_n)$ натуральных чисел, что $y_n=x_{k_n}$ при всех $n\in\N$.
\копр

\задача Является ли $(y_n)$ подпоследовательностью $(x_n)$, если
\сНовойСтроки
\пункт $x_n=n$, а $y_n=2n$;
\пункт $x_n=n$, а $y_1=2;\ y_2=1; y_n=n$ при $n\geq3$;
\пункт $x_n=(-1)^n$, а $(y_n)$ есть $1,\ -1,\ 1,\ 1, -1,\ 1,\ 1,\ 1, -1,
\ldots$?
\кзадача

\задача Последовательность обладает одним из свойств:
бесконечно мала, бесконечно велика, ограниченна, неограниченна, монотонна.
В каких случаях любая е\"е подпоследовательность обязательно
обладает тем же свойством?
\кзадача

\сзадача Верно ли, что любая последовательность имеет монотонную
подпоследовательность?
\кзадача
