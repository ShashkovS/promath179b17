% !TeX encoding = windows-1251
\documentclass[a4paper, 11pt]{article}
\usepackage{newlistok}
%\documentstyle[11pt, russcorr, listok]{article}

\УвеличитьШирину{1truecm}
\УвеличитьВысоту{2.6truecm}
%\hoffset=-2.65truecm
%\voffset=-23truemm
%\pagestyle{empty}

%\documentstyle[11pt, russcorr, ll]{article}
%\def\dfrac{\displaystyle\frac}

\Заголовок{Аксиома полноты}
\Подзаголовок{}
\НомерЛистка{19}
\ДатаЛистка{03.2015}

\begin{document}

\СоздатьЗаголовок



\опр Пусть дано подмножество $M$ множества действительных чисел $\R$. \\
Число  $c\in\R$ называют \выд{верхней гранью} множества $M$, если $c\geq m$ для всех $m\in M$.\\
Число  $c\in\R$ % упорядоченного поля $\Bbbk$
называют \выд{точной верхней гранью} множества $M$,
если $c$ является верхней гранью %множества
$M$, но никакое меньшее число
не является верхней гранью %множества
$M$.
Обозначение: $\sup M$ (читается \лк супр\'емум\пк{}  %множества
$M$).\\
Аналогично определяется точная нижняя грань множества $М$ ($\inf M$, \лк инф\'имум\пк{} $M$).
%, т.~н.~г.~$M$).
%Множество, ограниченное и сверху, и снизу, называют
%\выд{ограниченным}.
%для любого действительного числа $C$ $$  \exists C \in \R: \qquad \forall x\in M \quad |x| < C.$$
\копр


%\задача \пункт Докажите, что конечное объединение
%ограниченных множеств ограничено.  \пункт Верно ли то же
%самое для счетного объединения ограниченных множеств?
%\пункт Верно ли то же самое для счетного пересечения ограниченных множеств?
%\кзадача

%\опр
%%Число $C\in \R$ называется \выд{верхней гранью}
%%множества $M\subset \R$, если при любом $x\in M$
%%выполнено неравенство $x\leq C$.
%%Пусть $\Bbbk$ --- упорядоченное поле.
%Элемент $c\in\Bbbk$ % упорядоченного поля $\Bbbk$
%называют \выд{точной верхней гранью} множества
%$M\subset \Bbbk$, если $c$ является
%верхней гранью %множества
%$M$, но никакой меньший элемент
%не будет верхней гранью %множества
%$M$.
%Обозначение: $\sup M$ (читается \лк супремум\пк{}  %множества
%$M$).
%%т.~в.~г.~$M$.
%\копр



\задача
Докажите, что число $c$ есть $\sup M$ тогда и только тогда, когда
выполнены два условия:\\
1) для всех $x\in M$ верно, что $x\leq c$; \quad
2) для любого числа $c_1<c$ найд\"ется такое $x\in M$, что $x>c_1$.
\кзадача

%\задача Докажите, что если у множества $M$ существует точная верхняя (нижняя)
%грань, то она единственна.
%\кзадача

\задача Может ли у множества быть несколько точных верхних
(нижних) граней?
\кзадача

\задача
%Найдите т.~в.~г.~и т.~н.~г.~множеств: %а (если они существуют):
Найдите $\sup M$ и $\inf M$, если
\вСтрочку
\пункт
$M=\{a^2+2a\ |\  -5<a\leq 5\}$;
\пункт
$\displaystyle{M=\{\pm n/(2n+1)\ |\ n\in\N\}}$.
\кзадача

%\задача \label{sqrt}
%\вСтрочку
%\пункт Докажите, что не существует такого $q \in \Q$, что $q^2 = 2$.
%Докажите, что %т.~в.~г.~множества
%$\sup\ \!\{q \in \Q\ |\ q>0 \mbox{ и } q^2 < 2\}$
%никакое
%не может быть рациональным числом.
% не является точной верхней гранью множества
%\кзадача

%\begin{quote}
%\small В действительных числах, как и в рациональных, можно выполнять
%сложение, вычитание, умножение и деление на число, отличное от нуля. Можно также
%сравнивать числа между собой. Важным отличием действительных чисел от рациональных
%является их \выд{полнота}.
%\end{quote}

\задача Пусть $A$ и $B$ --- некоторые подмножества $\R$,
и пусть известны $\sup A$ и $\sup B$.\\
\вСтрочку
\пункт Найдите $\sup (A \cup B)$. \quad
\пункт Найдите $\sup (A+B)$,
где $A+B = \{ a+b\ |\ a\in A, b\in B\}$.\\ %\stackrel{\mbox{\small def}}
\пункт Найдите $\inf(A\cdot B)$, где
$A\cdot B=\{ a\cdot b\ |\ a\in A, b\in B\}$,
если $A$ и $B$ состоят из отрицательных чисел.
\кзадача

\smallskip
\noindent {\bfseries Аксиома полноты.}
\выд{Всякое ограниченное сверху непустое
подмножество в $\R$ имеет точную верхнюю грань.}
%\end{itemize}

\smallskip

\задача
%Сформулируйте и докажите аксиому о точной нижней грани.
Каждое ли ограниченное снизу непустое подмножество в $\R$
имеет точную нижнюю грань?
\кзадача

%\задача Каждое ли ограниченное снизу непустое множество $M\subset \R$
%имеет точную нижнюю грань?
%\кзадача


%\задача [Принцип вложенных отрезков] Пусть
%$
%[a_1,b_1]\supset [a_2,b_2] \supset \dots \supset [a_n, b_n] \supset \dots
%$
%--- последовательность вложенных отрезков. Докажите, что:
%%\сНовойСтроки
%\вСтрочку
%\пункт пересечение этих отрезков непусто;\\
%%($\bigcap_{i=1}^{\infty} [a_i, b_i] \ne \emptyset;$
%\пункт если $\lim\limits_{i\to\infty} (b_i-a_i) = 0$, то эти отрезки имеют
%единственную общую точку.
%\кзадача

\задача [Принцип вложенных отрезков] Пусть
$
[a_1,b_1]\supset [a_2,b_2] \supset \dots
$
--- последовательность вложенных отрезков. Докажите, что
%\сНовойСтроки
\вСтрочку
\пункт у отрезков есть общая точка;
%($\bigcap_{i=1}^{\infty} [a_i, b_i] \ne \emptyset;$
\пункт если $\lim\limits_{i\to\infty} (b_i-a_i) = 0$,
то общая точка %ровно
одна.
\кзадача


\задача
\вСтрочку
\пункт
Любая ли последовательность $(a_1,b_1)\supseteq(a_2,b_2) \supseteq
\dots$
% \supseteq (a_n,b_n)\supseteq \dots $
вложенных интервалов имеет непустое пересечение?
\пункт
А если известно, что
и %последовательность
$(a_n)$, и
%последовательность
$(b_n)$
содержат бесконечно много различных элементов?
\кзадача

\задача На прямой дано некоторое множество отрезков.
Известно, что любые два из них имеют общую точку.
Докажите, что существует точка, принадлежащая всем отрезкам.
\кзадача



%\задача Назовем $\Q$-отрезком пересечение отрезка с множеством рациональных
%чисел. Верно ли, что любая последовательность вложенных $\Q$-отрезков имеет
%непустое пересечение?
%\кзадача

%\сзадача Докажите, что множество точек отрезка несчетно.
%\кзадача

\задача
\вСтрочку
\пункт [Компактность отрезка]
Отрезок покрыт системой интервалов.
%Докажите, что
Всегда ли можно выбрать из системы конечное число интервалов,
%которые
покрывающих отрезок?
%\кзадача
%\задача
\пункт
%Будет ли верным утверждение пункта а), если в н\"ем
А если заменить отрезок на интервал?
\кзадача



\задача [Теорема Вейерштрасса]
Докажите, что любая
неубывающая ограниченная сверху последовательность
действительных чисел имеет предел.
\кзадача

\задача Найдите пределы %следующих
последовательностей:
\вСтрочку
\пункт $x_1=2$, $x_{n+1}=(x_n+1)/2$;
\пункт $y_1=\sqrt2$, $y_2=\sqrt{2\sqrt2}$,\\
$y_3=\sqrt{2\sqrt{2\sqrt2}}$, \dots; %\quad
\пункт $z_1=\sqrt2$, $z_2=\sqrt{2+\sqrt2}$,
$z_3=\sqrt{2+\sqrt{2+\sqrt2}}$, \dots; %\quad
\спункт $t_1=1$, $t_{n+1}=1/(1+t_n)$.
\кзадача

%\задача
%Докажите, что последовательность
%$\displaystyle{x_n=1-\frac12+\frac13-...+\frac{(-1)^{n+1}}{n}}$
%имеет предел.
%\кзадача

\задача
Пусть $a_1=1$, $a_{n+1}=a_n + 1/S_n$,  где
\пункт $S_n=a_n$;
\спункт $S_n=a_1+\dots+a_n$.
Ограничена ли %последовательность
$(a_n)$?
\кзадача

\задача
Докажите, что последовательность
$\displaystyle{x_n=1-1/2+1/3-...+(-1)^{n+1}/n}$
имеет предел.
\кзадача


%\задача[Число Эйлера]
%Докажите, что
%\вСтрочку
%\пункт
% Докажите, что последовательность
%$\displaystyle{x_n=(1+1/n)^n}$
%возрастает и ограничена (а значит, имеет предел
%$\lim\limits_{n \to \infty}x_n$; его обозначают буквой $e$).
%Докажите, что
%существует
%$\displaystyle{\lim\limits_{n \to \infty}(1+1/n)^n}$
%(его обозначают буквой $e$);\\
%\пункт
%$\displaystyle{\lim\limits_{n \to
%\infty}(1+k/n)^n=e^k}$ при любом %натуральном
%$k\in\N$;
%\пункт
%$\displaystyle{\lim\limits_{n \to
%\infty}(1-1/n)^n=1/e}$;
%\пункт
%Докажите, что
%$e=\lim\limits_{n \to \infty}(1+1/1!+1/2!+\dots+1/n!)$.
%\кзадача

%\сзадача
%Докажите, что
%$\displaystyle{e=1+1/1!+1/2!+1/3!+\dots}$.
%\кзадача


%\задача Для вычисления квадратного корня из положительного
%числа $a$ можно пользоваться следующим методом
%последовательных приближений. Возьмите любое положительное число
%$x_0$ и постройте последовательность по такому закону:
%$x_{n+1}=0,5\cdot(x_n+a/x_n).$\\
%%\сНовойСтроки
%\вСтрочку
%\пункт
%Докажите, что $\lim\limits_{n\to\infty}x_n=\sqrt a$.
%\спункт Сколько понадобится последовательных приближений,
%чтобы найти значение $\sqrt{10}$ с точностью до $0,0001$,
%если в качестве первого приближения взять $x_0=3$?
%\кзадача

\задача[Вычисление квадратного корня методом последовательных
приближений] Пусть $a>0$. Возьмем любое %положительное число
$x_1>0$ и построим последовательность $(x_n)$ по закону:
$x_{n+1}=0,5\cdot(x_n+a/x_n)$ при $n\in\N$.\\
%\сНовойСтроки
\вСтрочку
\пункт
Докажите, что $\lim\limits_{n\to\infty}x_n=\sqrt a$.
\спункт Для $a=10$ найдите $n$, при котором
%Сколько понадобится последовательных приближений,
%приближение
$|x_n-\sqrt{10}|<0,0001$, если %начальное приближение
$x_0=3$.
\кзадача


\задача [Критерий Коши]
Докажите, что последовательность $(x_n)$ \выд{сходится}
(то есть имеет предел) тогда и
только тогда, когда выполнено следующее условие:
$\quad\forall \ \varepsilon>0 \quad \exists \ k\in\N\quad \forall \
m,n\ge k
\quad |x_m-x_n|<\varepsilon$.
\кзадача


%\УстановитьГраницы{0pt}{5truecm}
\сзадача На прямоугольную карту положили карту той же
местности, но меньшего масштаба
(меньшая карта целиком лежит внутри большей).
%(см.~рис.~справа).
Докажите, что можно проткнуть иголкой сразу обе карты так,
чтобы точка прокола изображала на обеих картах одну и ту же
точку местности.
\кзадача
%\ВосстановитьГраницы


\сзадача
\вСтрочку
\пункт
Пусть %$c$ --- точная верхняя грань множества
$c=\sup\{ q \in \Q\ |\ q>0 \mbox{ и } q^2 < 2\}$.
Докажите, что $c^2 = 2$.\\
%\кзадача
%
%\задача
%\пункт
%Докажите, что для любого $a\geq0$ есть %существует
%ровно одно такое %неотрицательное число
%$x\geq0$, что $x^2=a$ (это число
%обозначают $\sqrt{a}$).
%\кзадача
%\задача
%\пункт
%Является ли $\Q$ полным? % поле~$\Q$? % рациональных чисел?
\пункт
Для любого $a\geq0$ и любого $r\in\Q$ определите число
$a^r$, докажите его существование и единственность.
\кзадача

\задача Докажите, что любая последовательность имеет монотонную подпоследовательность.
\кзадача

\задача Докажите, что любая ограниченная последовательность имеет сходящуюся подпоследовательность.
\кзадача

\ЛичныйКондуит{0mm}{6mm}


%\СделатьКондуит{5mm}{7.5mm}




%\GenXMLW

\end{document}

\раздел{$***$}

\опр Число $x\in \R$ называется \выд{предельной точкой}
множества $M\subset \R$,
если в любой окрестности $x$ содержится бесконечно много чисел из $M$.
% для всякого $\varepsilon > 0$ множество $(x-\epsilon,x+\epsilon)\cap M$
%бесконечно.
\копр


\задача Найдите все предельные точки следующих числовых множеств: %\сНовойСтроки
\пункт конечное множество;\\
\пункт $\{ \frac 1n\ | \ n\in \N\}$;
\пункт $\Z$;
\пункт $(0,1)$;
\пункт $\Q$;
\пункт рациональные числа, знаменатели которых --- степени двойки.
\кзадача

%\задача Заменим в определении 3 слово \лк бесконечно\пк{}
%на слова \лк содержит не менее двух элементов\пк. Будет ли это определение
%эквивалентно старому?
%\кзадача


%\задача Верно ли, что точная верхняя грань бесконечного множества
%является его предельной точкой?
%\кзадача

%\задача Пусть $M\subset \R$ --- ограниченное множество, $A$ --- множество
%предельных точек M.
%Докажите, что для всякого $\varepsilon > 0$ множество
%$\{x\in M\ |\ x>\sup A +\varepsilon\}$ конечно.
%\кзадача

\задача \пункт Пусть $a$ --- предельная точка множества $M$. Докажите, что
существует такая последовательность $(x_n)$ элементов этого множества, что
$\lim\limits_{n \to \infty} x_n = a$.
\пункт Верно ли обратное утверждение?
\кзадача


\задача
Докажите следующие утверждения: \quad
\вСтрочку
%\сНовойСтроки
\пункт Для любого $c \in \R$
найд\"ется такое $n\in \N$, что $n>c$.\\
\пункт Для любого $\varepsilon > 0$ найд\"ется  такое $n\in \N$, что
$\frac 1n<\varepsilon$. \quad
\пункт[Аксиома Архимеда] Для любого положительного $h\in\R$ и для
любого $a\in\R$ существует и единственно такое целое число $n$, что
$nh\le a<(n+1)h$.
\кзадача

\задача Докажите, что любой отрезок из $\R$ содержит
%между любыми двумя различными действительными числами есть
бесконечно много чисел
\вСтрочку
\пункт
из $\Q$;
%рациональных;
%чисел;
\пункт
%ррациональных чисел.
из $\R\setminus\Q$.
\кзадача

\задача Выведите из принципа вложенных отрезков и аксиомы Архимеда
аксиому полноты.
%точной верхней грани.
\кзадача

\сзадача
Докажите единственность действительных чисел: для двух полных
упорядоченных полей существует биекция $f$ между ними такая,
что $f(a+b)=f(a)+f(b)$,
$f(a\cdot b)=f(a)\cdot f(b)$
и если $a\leqslant b$, то
$f(a)\leqslant f(b)$.
\кзадача

\сзадача
Докажите существование действительных чисел
(считая известными натуральные числа).
\кзадача


\опр Число $x\in \R$ называется \выд{предельной точкой}
множества $M\subset \R$,
если для всякого $\varepsilon > 0$ множество $(x-\epsilon,x+\epsilon)\cap M$
бесконечно.
\копр


\задача Найдите все предельные точки следующих множеств: \сНовойСтроки
\пункт конечное множество;
\пункт множество $\{ \frac 1n: n\in \N\}$;
\пункт множество целых чисел;
\пункт интервал $(0,1)$;
\пункт множество рациональных чисел;
\пункт множество рациональных чисел, знаменатель которых есть степень двойки.
\кзадача

\задача Заменим в определении 3 слово \лк бесконечно\пк{}
на слова \лк содержит не менее двух элементов\пк. Будет ли это определение
эквивалентно старому?
\кзадача


\задача Верно ли, что точная верхняя грань бесконечного множества
является его предельной точкой?
\кзадача

\задача Пусть $M\subset \R$ --- ограниченное множество, $A$ --- множество
предельных точек M.
Докажите, что для всякого $\varepsilon > 0$ множество
$\{x\in M\ |\ x>\sup A +\varepsilon\}$ конечно.
\кзадача

\задача \пункт Пусть $a$ --- предельная точка множества $M$. Докажите, что
существует такая последовательность $(x_n)$ элементов этого множества, что
$\lim\limits_{n \to \infty} x_n = a$.
\пункт Верно ли обратное утверждение?
\кзадача

\задача Докажите, что из любой ограниченной последовательности можно
выделить сходящуюся подпоследовательность.
\кзадача

\end{document} 