% !TeX encoding = windows-1251
\documentclass[a4paper, 12pt]{article}
\usepackage{newlistok}
%\documentstyle[11pt, russcorr, listok]{article}

\УвеличитьШирину{1.1truecm}
\УвеличитьВысоту{2.5truecm}
\hoffset=-2.5truecm
\voffset=-25truemm
%\pagestyle{empty}

%\documentstyle[11pt, russcorr, ll]{article}
%\def\dfrac{\displaystyle\frac}

%\scalebox{.89}{\vbox{%
%\ncopy{1}{

\Заголовок{Комплексные числа и преобразования плоскости}
\Подзаголовок{}
\НомерЛистка{21$\frac{1}{2}$}
\ДатаЛистка{09.2015}

\begin{document}

\СоздатьЗаголовок

%\medskip


{\hsize 13.2cm

\задача
Запишите %в виде
как функцию комплексной переменной\\
\пункт симме\-трию относительно оси $y$;\\
\пункт ортогона\-ль\-ную проекцию на ось $x$;\\
\пункт центральную симметрию с центром $A$;\\
\пункт поворот на угол $\varphi$ относительно~точки~$A$;\\
\пункт гомотетию с коэффициентом $k$ и центром $A$;\\
\пункт %скользящую
симметрию относительно прямой $y=3$ со сдвигом на 1 влево;\\
\пункт поворот, %который
переводящий ось $x$ в прямую $y=2x+1$;\\
\пункт симметрию относительно прямой $y=2x+1$.\\
\кзадача


%\задача
%Нарисуйте множества и их образы при отображениях:
%\пункт $\left\{z\left| \frac{\pi}{4} < \mathrm{arg}\, z \leq \pi, |z| \leq \frac{1}{2}\right\}\right.$,
%$\left\{z\left | \mathrm{Re}\, z = 1 \right.\right\}$,
%отображение
%$z\mapsto z^2$;
%\пункт $\left\{z\left| \mathrm{Im}\, z = 1 \right\}\right.$,
%$\left\{z\left| \frac{\pi}{4} \leq \mathrm{arg}\, z \leq \frac{\pi}{2}, \mathrm{Im}\, z < 1 \right\}\right.$, отображение
 %$z\mapsto \frac{1}{z}$;\\
%\пункт $\left\{z\left| \mathrm{arg}\, z = \frac{\pi}{4} \right\}\right.$,
%$\left\{z\left| 0 \leq \mathrm{arg}\, z \leq \pi, |z| < 1 \right\}\right.$, отображение
% $z\mapsto z + \frac{1}{z}$.
%\кзадача


%\задача
%\пункт Докажите, что $z\mapsto \frac{1}{\overline{z}}$ --- инверсия с центром в точке 0.\\
%\пункт В какие множества переводит прямые и окружности отображение $z\mapsto \frac{az+b}{cz+d}$, $a,b,c,d\in\mathbb{C}.$\\
%\пункт Найдите образ множества $\left\{z\left| \mathrm{Im}\, z >0 \right\}\right.$ при отображении $z\mapsto \frac{az+b}{cz+d}$, $a,b,c,d\in\mathbb{C}.$
%\кзадача

\задача
Куда отображение $z\longmapsto z^2$
переводит\\
\пункт
декартову координатную сетку;\\
\пункт
полярную координатную сетку;\\
\пункт
окружность $|z+i|=1$;\\
%\пункт
%$\left\{z\left| \frac{\pi}{4} < \mathrm{arg}\, z \leq \pi, |z| \leq \frac{1}{2}\right\}\right;$
\пункт
кошку (рис.~справа)?\\
\пункт
Те же вопросы для отображения
$z\longmapsto 1/z$.
\кзадача

}


%\putpict{15.8cm}{2.5cm}{pct_complex_cat}{}
%\vspace*{-5mm}


%\задача
%Куда отображение $z\longmapsto\sqrt z$ переводит
%$\{z\in\Cbb\ |\ {\rm Im}\,(z)>0\}$?
%верхнюю полуплоскость (без границы)?
%\кзадача

\задача
Куда отображение\\
\пункт
$z\longmapsto1/z$;\\
\спункт $z\longmapsto0,5(z+1/z)$\\
переводит %множество
$\{z\in\Cbb\ |\ {\rm Im}\,(z)>0,\ |z|\leq1\}$?
%полукруг радиуса 1 с центром в начале координат, лежащий в верхней
%полуплоскости?
\кзадача


\задача
Пусть карты из задачи ... листка ... лежат на комплексной плоскости.
Докажите, что найдутся такие $q,b\in\Cbb$, что
если $z\in\Cbb$ --- любая точка на первой карте, то этой же точкой местности
на второй карте будет точка $qz+b$.
Выразите с помощью  $q$ и $b$ точку,
изображающую на картах одну и ту же точку местности.
\кзадача





\vspace*{-1mm}


%\ЛичныйКондуит{0mm}{5mm}

%\СделатьКондуит{3.7mm}{7.5mm}


\end{document}





\задача
\вСтрочку
\пункт
Куда отображение $z\longmapsto1/z$ переводит
полукруг радиуса 1 с центром в начале координат, лежащий в верхней
полуплоскости?
\спункт Тот же вопрос для отображения
$z\longmapsto0,5(z+1/z)$.
\кзадача

\ЛичныйКондуит{0mm}{6mm}



%\СделатьКондуит{5mm}{7.5mm}





\end{document}

%\задача
%Нарисуйте образы следующих множеств при отображениях, задаваемых
%функциями:
%\вСтрочку
%\пункт $f(z)=\bar z$;\hfil
%\пункт $f(z)=z^n,\ n\in\N$;\\ \\ \\ \\ \\
%\пункт $f(z)=\sqrt z$;\hfil
%\пункт $f(z)=1/z$;\\ \\ \\ \\ \\
%\пункт $f(z)=e^z$;\hfil
%\пункт $f(z)=\sin z$;\\ \\ \\ \\ \\
%\пункт $f(z)=\cos z$;\hfil
%\спункт $f(z)=\tg z$;\\ \\ \\ \\ \\
%\спункт $f(z)=0,5(z+1/z)$.
%\кзадача




На прямоугольную карту положили карту той же
местности, но меньшего масштаба
(меньшая карта целиком лежит внутри большей).
Пусть масштаб первой карты в $k$ раз больше масштаба второй карты,
вторая карта сдвинута на вектор $z$ и повернута на угол $\alpha$
относительно первой карты. Найдите координаты точки, которая
изображает на обеих картах одну и ту же точку местности.

\задача
Докажите, что вещественная и мнимая части любого корня квадратного
уравнения с комплексными коэффициентами выражаются через вещественные
и мнимые части коэффициентов уравнения с помощью арифметических операций
и извлечения действительного
квадратного корня (т.~е.~\лк выражаются в радикалах\пк).
\кзадача

\задача
%\вСтрочку
%\пункт
Выразите в радикалах
$\cos\frac{2\pi}{5}$ и
$\sin\frac{4\pi}{5}$ и
%\кзадача
%\задача
%\пункт
постройте %при помощи
циркулем и линейкой правильный пятиугольник.
\кзадача 