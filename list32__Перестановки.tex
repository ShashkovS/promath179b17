% !TeX encoding = windows-1251
\documentclass[a4paper, 11pt]{article}
\usepackage{newlistok}
\usepackage[matrix,arrow]{xy}
%\newcommand{\ord}{\operatorname{ord}}

%\def\hang{\hangindent\parindent}

\УвеличитьШирину{1.2cm}
\УвеличитьВысоту{3cm}
%\hoffset=-.75truecm
%\voffset=-.3truecm
%\УвеличитьВысоту{7.5cm}
%\УвеличитьШирину{5cm}
%\pagestyle{empty}

\begin{document}

\Заголовок{Перестановки}
\НомерЛистка{32}
\ДатаЛистка{10.2016}
%\Подзаголовок{}

\СоздатьЗаголовок

%\vfill
\опр
\emph{Перестановка} чисел $1,\dots,n$ --- это взаимно
однозначное отображение множества $\{1,\dots,n\}$ на себя.
Множество перестановок чисел $1,\dots,n$ обозначается $S_n$ и
называется \emph{симметрической группой}.
\копр

\задача
Сколько элементов в симметрической группе $S_n$?
\кзадача

Перестановки записывают таблицами вида
$\displaystyle\begin{pmatrix}1&2&3&4\\2&4&1&3\end{pmatrix}$; такая
таблица означает перестановку $1\mapsto2$ (то есть $1$ переходит в
$2$), $2\mapsto4$, $3\mapsto1$, $4\mapsto3$. Вообще, если
$\sigma\in S_n$, то
$\displaystyle\sigma=\begin{pmatrix}1&2&\dots&n\\
\sigma(1)&\sigma(2)&\dots&\sigma(n)\end{pmatrix}$.
Для наглядности, ту же перестановку можно изобразить картинкой
вида
\vspace*{-2mm}
$$
\xymatrix{
1\ar[dr]&2\ar[drr]&3\ar[dll]&4\ar[dl]\\
1&2&3&4
}
$$
\vspace*{-5mm}

\опр
\emph{Произведение} перестановок $\sigma,\tau\in S_n$ определяется
так: $\sigma\tau(i)=\sigma(\tau(i))$ (для произвольных
отображений $\sigma$ и $\tau$ такое произведение обычно называется
\emph{композицией отображений}). Например, если
$$
\llap{$\tau$}=\begin{pmatrix}1&2&3&4\\3&2&1&4\end{pmatrix}
\qquad\qquad \raisebox{18pt}{$\xymatrix{
1\ar[drr]&2\ar[d]&3\ar[dll]&4\ar[d]\\
1&2&3&4 }$}
%$$
%$$
,\qquad\quad
\llap{$\sigma$}=\begin{pmatrix}1&2&3&4\\2&4&1&3\end{pmatrix}
\qquad\qquad \raisebox{18pt}{$\xymatrix{
1\ar[dr]&2\ar[drr]&3\ar[dll]&4\ar[dl]\\
1&2&3&4 }$}
,$$
\vspace*{-3mm}
то
\vspace*{-3mm}
$$
\llap{$\sigma\tau$}=\begin{pmatrix}1&2&3&4\\1&4&2&3\end{pmatrix}
\qquad\qquad \raisebox{18pt}{$\xymatrix{
1\ar[d]&2\ar[drr]&3\ar[dl]&4\ar[dl]\\
1&2&3&4 }$}.
$$
Отметим, что сначала применяется второй сомножитель, а потом
первый.
\копр

\задача
\вСтрочку
\пункт
Всегда ли $\sigma\tau=\tau\sigma$?
\пункт
Пусть
$\sigma=\displaystyle\begin{pmatrix}1&2&3&4\\2&4&1&3\end{pmatrix}$,
$\tau=\displaystyle\begin{pmatrix}1&2&3&4\\3&1&4&2\end{pmatrix}$.
Найти $\sigma\tau$ и $\tau\sigma$.
\кзадача

\задача
\пункт
Найдите перестановку $e$, удовлетворяющую условию $e\sigma=\sigma
e=\sigma$ для любой перестановки $\sigma\in S_n$.
Докажите, что такая перестановка единственна
(е\"е называют \emph{единичной} или \emph{тождественной.})
%\пункт
%%Существует ли другая перестановка $e'$ с таким свойством?
%Единственна ли тождественная перестановка?
\пункт
Докажите, что для любой перестановки $\sigma$ существует
единственная перестановка $\sigma^{-1}$ такая, что
$\sigma\sigma^{-1}=e=\sigma^{-1}\sigma$. (Эта перестановка
называется \emph{обратной} к $\sigma$. Почему?)
\пункт
Докажите, что для любых $\sigma,\tau,\eta\in S_n$ имеет место
равенство $\sigma(\tau\eta)=(\sigma\tau)\eta$.
\пункт
Докажите, что если $\sigma\tau=\sigma\eta$, то непременно
$\tau=\eta$.
\кзадача

\задача
Докажите, что для любой перестановки $\sigma\in S_n$ существует
такое натуральное число $k$, что $\sigma^k=e$. Минимальное $k$ с
этим свойством называется \emph{порядком} перестановки $\sigma$ и
обозначается $\ord\sigma$.
\кзадача

\задача\label{perm1}
Найдите порядки перестановок: \
$\displaystyle\begin{pmatrix}1&2&3&4&5\\1&3&2&4&5\end{pmatrix}$,
\
$\displaystyle\begin{pmatrix}1&2&3&4&5\\3&4&2&5&1\end{pmatrix}$,
\
$\displaystyle\begin{pmatrix}1&2&3&4&5\\2&5&4&3&1\end{pmatrix}$.
\кзадача

\опр
\emph{Циклом} $(a_1,a_2,\dots,a_k)$ называется перестановка,
циклически переставляющая элементы\break $a_1,a_2,\dots,a_k$ (то есть
$a_1\mapsto a_2$, $a_2\mapsto a_3$, \dots, $a_k\mapsto a_1$;
имеется в виду, что все
элементы~\hbox{$a_1,a_2,\dots,a_k$}~раз\-ли\-ч\-ны; все
остальные элементы множества $\{1,\dots,n\}$ переходят в себя).
Число $k$ называется \выд{длиной} цикла.
\копр

\задача
Какие из перестановок в задаче \ref{perm1} являются циклами?
\кзадача

\задача
Каков порядок цикла длины $k$?
Сколько всего циклов длины $k$ в $S_n$?
\кзадача

%\задача
%Какие перестановки задачи \ref{perm1} --- циклы?
%Каков порядок цикла длины $k$, сколько
%таких циклов~в~$S_n$?
%\кзадача


\задача
\пункт
Докажите, что если $\sigma$ и $\tau$ --- циклы, множества
элементов которых не пересекаются (такие циклы называются
\emph{независимыми}), то $\sigma\tau=\tau\sigma$ (циклы
\emph{коммутируют}).
\пункт
Верно ли, что циклы коммутируют тогда и только тогда, когда они
независимы?
\кзадача

\задача
Представьте перестановки из задачи \ref{perm1} в виде произведения
независимых циклов.
\кзадача


\задача
Докажите, что любую перестановку  можно представить в виде
произведения независимых циклов.
Сколькими способами (с точностью до перестановки множителей)?
\кзадача



\задача
\вСтрочку
\пункт
Пусть порядок перестановки равен двум. Разложим её в произведение
независимых циклов. Какими могут быть длины этих циклов?
%\кзадача
\пункт
%\задача
Пусть $\sigma$ --- это $k$-я степень цикла $(1,2,\dots,n)$.
На сколько независимых циклов раскладывается~$\sigma$?
Каковы длины этих циклов?
\кзадача


\задача
Найдите максимальный порядок перестановки
\вСтрочку
\пункт из $S_5$;
\пункт из $S_{13}$;
\спункт из $S_n$.
\кзадача


\задача
Докажите, что число $\ord\sigma$ делит $n!$ для любой перестановки
$\sigma\in S_n$.
\кзадача


\задача
Текст зашифрован программой, заменяющей взаимно однозначно каждую
букву на некоторую другую.
%\сНовойСтроки
\вСтрочку
\пункт
Докажите, что существует такое число $k$, что текст
расшифровывается применением $k$ раз шифрующей программы.
\пункт
Найдите хотя бы одно такое $k$.
\кзадача


\ЛичныйКондуит{0mm}{5mm}
\ОбнулитьКондуит
\newpage


\опр
\emph{Транспозиция} --- это цикл длины два. Транспозицию вида
$(i,i+1)$ называют \emph{элементарной}.
\копр

\задача %\вСтрочку
\пункт
Докажите, что любая перестановка % можно
представляется как произведение транспозиций.\\
\пункт
Проделайте это для перестановок из задачи \ref{perm1}.  \\
\пункт
Любая ли перестановка %можно
представляется как
произведение элементарных транспозиций?\\
\пункт
Представьте в виде произведения элементарных транспозиций
перестановки из задачи \ref{perm1}.
\кзадача

%\сзадача
%Докажите, что любую перестановку можно представить как
%произведение двух перестановок, порядок каждой из которых не
%больше двух.
%\кзадача


\задача
Несколько жителей города $N$ хотят обменяться квартирами.
У каждого есть по квартире, но каждый хочет переехать в другую
(разные люди хотят переехать в~разные квартиры).
По законам города разрешены только парные
обмены: если два человека об\-ме\-ни\-ва\-ют\-ся квартирами, то в тот же день
они не участвуют в других обменах.
Докажите, что можно устроить парные обмены так, что уже
через два дня каждый будет жить в той квартире, куда хотел переехать.
\кзадача

\задача
\пункт
%Докажите, что
Любую ли перестановку из $S_n$ можно представить как
произведение транспозиций вида $(1,k)$? % $(1,3)$, \dots, $(1,n)$.
\спункт
Пусть $T$ --- некоторое множество транспозиций из $S_n$.
Отметим на плоскости $n$ точек $A_1$, \dots, $A_n$ и соединим
некоторые из них р\"ебрами по правилу: точки $A_i$ и $A_j$
соединяются ребром, если во множестве $T$ есть транспозиция $(i,j)$.
Докажите, что получившийся граф будет связным %в том и только в том случае,
если и только если любая перестановка из $S_n$
разлагается в произведение транспозиций, %каждая из которых
принадлежащих множеству $T$.
\кзадача

\опр
\emph{Беспорядок} или \emph{инверсия} в перестановке $\sigma$
--- это такая пара $(i,j)$, что $i<j$ и $\sigma(i)>\sigma(j)$.
Перестановка называется \emph{чётной}, если число инверсий в ней
чётно, и \emph{нечётной} в противном случае. Множество всех чётных
перестановок из $S_n$ обозначается $A_n$ и называется
\emph{знакопеременной группой}.
\копр

\задача
Как увидеть (геометрически) инверсии на картинках из определения 2?
%$$
%\xymatrix{
%1\ar[dr]&2\ar[drr]&3\ar[dll]&4\ar[dl]\\
%1&2&3&4 }
%$$
%(Каким \emph{геометрическим} объектам они отвечают?)
\кзадача

\задача
\вСтрочку
\пункт
Какие перестановки в задаче \ref{perm1} чётные?
%\кзадача
%\задача
\пункт
Сколько инверсий у %перестановки
$
\displaystyle
\begin{pmatrix}
1&2&\dots&n-1&n\\n&n-1&\dots&2&1
\end{pmatrix}
$?
\кзадача

\задача
Можно ли сказать, сколько инверсий у перестановки $\sigma^{-1}$,
зная лишь число инверсий~у~$\sigma$?
\кзадача


\задача
Докажите, что любая транспозиция --- нечётная перестановка.
\кзадача

\задача
Докажите, что при умножении на транспозицию чётность перестановки
меняется.
\кзадача


\задача
Докажите, что чётность цикла зависит только от его длины. Как?
\кзадача

\задача
Докажите, что произведение двух перестановок одной чётности ---
чётная перестановка, а произведение двух перестановок разной
чётности --- нечётная перестановка.
\кзадача

\задача
Сколько элементов в $A_n$?
\кзадача

\сзадача
\пункт
Почему задача Ллойда об игре в 15 неразрешима?
\пункт
Двудольный граф правильно раскрашен в 2 цвета.
В каждой его вершине записано по числу (числа~разные).
За ход можно менять местами любые 2 числа, соедин\"енные
ребром. Может ли после нескольких ходов
%возникнуть ситуация,
оказаться, что 2 числа одного цвета поменялись местами,
а остальные числа на своих местах?
\кзадача

\сзадача
У отца было 7 дочерей. Всякий раз, когда одна выходила замуж,
каждая е\"е старшая сестра, оставшаяся в невестах,
жаловалась отцу, что нарушен обычай выходить замуж
по старшинству. После того, как вышла замуж последняя дочь,
оказалось, что отец услышал всего 7 жалоб.
%Сколькими способами
%это могло произойти?
В каком порядке дочери могли выходить замуж
(приведите пример)? Сколько всего таких порядков?
\кзадача

\сзадача Для каких $k$ в $S_n$ существует перестановка, у которой
ровно $k$ инверсий?
\кзадача


\задача
В каждой клетке таблицы $2\times n$ стоит одно из целых чисел от 1 до $n$,
прич\"ем в каждой строке стоят разные числа, и
в каждом столбце стоят разные числа. Сколько таких таблиц?
\кзадача

\сзадача
$P(x)$ и $Q(x)$ --- многочлены с целыми коэффициентами.
Пусть
%для некоторого натурального $k$
при любом целом $x$
число $P(Q(x))-x$ делится на $100$. Докажите, что тогда при любом
целом $x$ число $Q(P(x))-x$ делится на $100$.
\кзадача

\сзадача
В таблице $n$ строк и $m$ столбцов.
\выд{Горизонтальный ход} --- это любая
перестановка элементов таблицы, при которой
каждый элемент остается в той же строке, что и до перестановки.
Аналогично определяется \выд{вертикальный ход}. За какое наименьшее
число %таких
горизонтальных и вертикальных
ходов всегда удастся получить любую перестановку элементов
таблицы?
\кзадача

\сзадача
\пункт Постройте соответствие между перестановками трёх элементов и движениями плоскости, переводящими равносторонний треугольник в себя такое, что композиции перестановок будет соответствовать композиция соответствующих движений.

\пункт Аналогично постройте соответствие между чётными перестановками четырёх элементов и вращениями пространства, переводящих правильный тетраэдр в себя.

\пункт Аналогично постройте соответствие между перестановками четырёх элементов и вращениями пространства, переводящих куб в себя.
\кзадача


\ЛичныйКондуит{0mm}{5mm}

\end{document} 