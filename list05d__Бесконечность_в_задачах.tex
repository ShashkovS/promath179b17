% !TeX encoding = windows-1251
\documentclass[a4paper,11pt]{article}
\usepackage[mag=930]{newlistok}

\УвеличитьШирину{1.1truecm}
\УвеличитьВысоту{2.2truecm}
\renewcommand{\spacer}{\vfill}

\Заголовок{Бесконечность в задачах}
\Подзаголовок{}
\НомерЛистка{5д}
\ДатаЛистка{09.2014}

\begin{document}


\СоздатьЗаголовок

\задача
Число вкладчиков МегаБанка счетно, причем каждый вложил хотя бы 1 рубль.
Докажите, что деньги можно перераспределить между вкладчиками так,
чтобы у каждого стало не менее 1000000 рублей.
\кзадача


\раздел{Десятичные дроби и последовательности}

%0
%\задача Верно ли, что в натуральном ряду можно выделить\\
%\вСтрочку
%\пункт сколь угодно длинную; \пункт бесконечно длинную цепочку
%подряд идущих составных чисел? \кзадача

%1
%\задача
%Можно ли разместить внутри интервала \вСтрочку \пункт сколь угодно много,
%\пункт бесконечно
%много интервалов равной длины?
%\кзадача


\задача
Докажите, что в любой бесконечной десятичной дроби можно так переставить
цифры, что полученная дробь станет периодической
(возможно, с предпериодом).
\кзадача

\задача
Существует ли такая бесконечная последовательность натуральных чисел,
что любая другая получается из не\"е выч\"еркиванием
\вСтрочку
\пункт некоторых членов;
\пункт некоторого конечного числа членов?
\кзадача




%5
\задача
Докажите, что
\вСтрочку
\пункт
из любых одиннадцати бесконечных десятичных дробей можно
выбрать две, совпадающие в бесконечном числе позиций;
\пункт любое действительное число можно представить в виде
суммы девяти чисел, десятичные записи которых содержат только цифры 0 и 8.
\кзадача

%\задача
%Докажите, что
%\кзадача

%2
\задача Можно ли в последовательности
%$\displaystyle{\frac{1}{1},\frac{1}{2},\frac13,\frac{1}{4},\ldots}$
$1, \ 1/2, \ 1/3, \ 1/4, \ \dots$
выделить\\
\вСтрочку \пункт бесконечно длинную; \пункт сколь угодно длинную
арифметическую прогрессию?
\кзадача


%9
\задача
%черномор
Докажите, что в любой бесконечной последовательности
\вСтрочку
\пункт целых;
\пункт действительных чисел найдётся
либо неубывающая,
либо невозрастающая бесконечная подпоследовательность.
\кзадача

%11.5
\задача Два джинна по очереди выписывают цифры бесконечной
десятичной дроби. Первый своим ходом приписывает в хвост любое
конечное число цифр, второй --- одну. Если в итоге получится
периодическая дробь,  выигрывает первый, иначе --- второй. Кто
выиграет при правильной игре? %наилучшей игре сторон?
\кзадача

\раздел{Фигуры на плоскости}

\задача
Замостите плоскость квадратами, среди которых
\вСтрочку
\пункт
ровно два одинаковых;
\пункт
нет~\hbox{одинаковых.}
\кзадача

\задача
Клетки бесконечной клетчатой плоскости окрашены в 2 цвета.
%Обязательно ли
Найд\"ет\-ся ли бесконечное множество вертикалей и
бесконечное множество горизонталей, на пересечении которых
все клетки будут одного цвета?
\кзадача


%3
\задача Можно ли покрыть плоскость
\вСтрочку
%\пункт  прямую конечным числом кругов;
\пункт   конечным числом полос;
\пункт конечным числом внутренностей углов, сумма которых
меньше 360$^\circ$;
\пункт конечным числом внутренностей парабол?
\кзадача

\задача
Можно ли замостить плоскость треугольниками так, чтобы не было двух равных треугольников?
\кзадача


\раздел{Графы}


%7
\задача
Пусть человечество бессмертно, любой человек смертен,
число людей в любом поколении конечно. Докажите, что есть %найд\"ется
бесконечная цепь, % мужчин,
начинающаяся с Адама, где
каждый следующий %в цепочке
--- сын предыдущего.
\кзадача

%%8
%\задача Верна ли теорема Холла, если
%юношей счетное число, каждый знаком со счетным числом девушек?
%\кзадача



%7.4
\задача
%В неком языке (с конечным алфавитом)
Дан язык с конечным алфавитом.
Любая последовательность~букв из алфавита этого языка называется словом.
Часть слов (конечной длины) в языке --- неприличные.
Слово называется абсолютно приличным, если
оно не содержит неприличных подслов.
Известно, что существуют сколь угодно~длин\-ные абсолютно
приличные слова. Докажите,
что существует бесконечно длинное абсолютно приличное слово.
\кзадача


%10
\задача Каждое конечное слово в неком языке либо
хорошее, либо нехорошее. Докажите, что в любом бесконечном слове
можно откинуть несколько начальных букв так, что оставшееся
бесконечное слово можно будет нарезать либо только на хорошие
слова, либо только на нехорошие.
\кзадача

%11
\задача
Теорема о четырех красках утверждает, что вершины любого связного
плоского графа можно~раскрасить в %четыре
4 цвета так, что вершины,
соединенные ребром, будут разного цвета.
%Предположим, проблема решена (впрочем, так оно и есть).
Докажите с помощью этой теоремы тот же результат для графа
со сч\"етным числом вершин (степень каждой вершины конечна).
\кзадача

%6
%\задача
%\вСтрочку
%\пункт Есть три последовательности натуральных чисел:
%$(a_n)$, $(b_n)$ и  $(c_n)$.
%%$a_1,a_2,\ldots,a_n,\ldots$; \break
%%$b_1,b_2,\ldots,b_n,\ldots$; $c_1,c_2,\ldots,c_n,\ldots$.
%Докажите, что~найдутся такие номера $p$ и $q$,
%что $a_p\ge a_q$, $b_p\ge b_q$, $c_p\ge c_q$.
%\пункт
%%Верно ли аналогичное утверждение,
%%не три, а
%А если последовательностей сч\"етное~число? %количество?
%\кзадача

\раздел{Разное}

\задача
По результатам исследования британских учёных, прирост числа проблем человечества пропорционален количеству всевозможных пар проблем человечества
(т.е.~каждая новая проблема появляется через время, обратно пропорциональное числу пар). Докажите, что скоро проблем станет бесконечно много.
%(И это будет ещё одной проблемой человечества.)
\кзадача


\задача
Докажите, что существуют такие два бесконечных
подмножества $A$ и $B$ целых неотрицательных чисел,
что каждое целое неотрицательное число %единственным образом
однозначно представляется в виде $a+b$, где $a\in A$,
$b\in B$.
\кзадача

\задача
Все натуральные числа выписали в ряд в неком порядке
(каждое по разу).
Может ли сумма~любых нескольких (двух,
трех, \dots) чисел,
выписанных подряд (начиная с любого места) быть составным числом?
\кзадача

%\задача
%Докажите, что существует такое подмножество $M\subset\N\cup\{0\}$, что
%каждое натуральное число единственным образом представляется в виде
%\пункт $a-b$, где $a,b\in M$;
%\пункт $a+2b$, где $a,b\in M$.
%\кзадача

%9
%\задача В некотором царстве с целью упрощения процесса торговли выпустили неограниченное число
%монет достоинством в $n_1, n_2, \dots, n_k, \dots$ тугриков, где $n_1<n_2<\cdots<n_k<\cdots$.
%Докажите, что в некоторый момент эту процедуру можно оборвать: найдётся такое число $N$, что
%\сНовойСтроки
%\пункт любую сумму, которую можно уплатить со сдачей выпущенными монетами, на самом
%деле можно уплатить со сдачей монетами достоинством в $n_1, n_2,\dots, n_N$ тугриков; \пункт то
%же самое, но суммы уплачиваются без сдачи. \кзадача




%12
\задача Можно ли в сч\"етном множестве выделить такую несч\"етную систему
подмножеств, что для любых двух подмножеств из этой системы
их пересечение
\вСтрочку
\пункт совпадает с одним из этих двух подмножеств;
\пункт непусто и содержит не более десяти элементов; \пункт
непусто и конечно? \кзадача


\задача
Натуральный ряд представлен в виде объединения %некоторого множества
попарно непересекающихся целочисленных %бесконечных
арифметических
прогрессий с положительными разностями $d_1$, $d_2$, $d_3$, \dots.
Может ли сумма
$\frac1{d_1} + \frac1{d_2} + \frac1{d_3}+\dots$ быть меньше $0,9$, если
\вСтрочку
\пункт
число прогрессий конечно;
\пункт
прогрессий счетное число (в этом случае условие нужно понимать
так: сумма любого конечного числа слагаемых из бесконечной
суммы не превышает $0,9$).
\кзадача

\ЛичныйКондуит{0mm}{6mm}



%\СделатьКондуит{6mm}{8mm}
%\GenXMLW

\end{document}

\задача
Игра происходит на плоскости. Играют двое: первый передвигает
одну фишку-волка, второй --- $k$ фишек-овец. После хода
волка ходит одна из овец, затем после следующего хода волка
опять какая-нибудь из овец и т.~д. И волк, и овцы
передвигаются за один ход в любую сторону не
более, чем на метр. Верно ли, что при любой
первоначальной позиции волк поймает хотя бы одну
овцу (окажется с ней в одной точке)?
\кзадача


\задача
Город представляет собой бесконечную клетчатую плоскость (линии --- улицы,
клеточки --- кварталы). На одной из улиц через каждые 100 кварталов на
перекр\"естках стоит по милиционеру. Где-то в городе есть бандит
(его местонахождение неизвестно, но перемещается он только по улицам).
Цель милиции --- увидеть бандита.
Есть ли у милиции алгоритм наверняка достигнуть своей цели?
Максимальные скорости милиции и бандита
--- какие-то конечные, но неизвестные нам величины (у бандита
скорость может быть больше, чем у милиции). Милиция видит вдоль
улиц во все стороны на бесконечное расстояние.
\кзадача

\задача
В городе из  задачи 19 трое полицейских
ловят вора (местонахождение вора
неизвестно, но перемещается он только по улицам).
Максимальные скорости у полицейских и вора одинаковы.
Вор считается пойманным, если он оказался на одной улице с полицейским.
Смогут ли полицейские поймать вора?
\кзадача

