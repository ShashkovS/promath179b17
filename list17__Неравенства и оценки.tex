% !TeX encoding = windows-1251
\documentclass[a4paper,12pt]{article}
\usepackage{newlistok}
%\usepackage{tikz}
%\usetikzlibrary{calc}

%\documentstyle[11pt, russcorr, listok]{article}
\newcommand{\del}{\mathrel{\raisebox{-.3 ex}{${\vdots}$}}}
\renewcommand{\spacer}{\vfill}

\УвеличитьШирину{1.4truecm}
\УвеличитьВысоту{2.5truecm}
%\hoffset=-2.45truecm
%\voffset=-17.9truemm


\begin{document}

%\scalebox{.96}{\vbox{
%\ncopy{1}{
\Заголовок{Неравенства и оценки.}
\НомерЛистка{17}
\ДатаЛистка{12.2014}
\Подзаголовок{}
\СоздатьЗаголовок

%\noindent
%{\Bf Аксиома Архимеда.} Пусть числа $A$ и $B$ положительны.
%Тогда найд\"ется такое %натуральное число
%$n\in\N$, что $nA>B$.

\задача Что больше:
\вСтрочку
\пункт $5^{15}$ или $15^5$;
\пункт $2^{100}$ или $10^{30}$;
%\пункт $7^8$ или $8^7$;
\пункт $3^{500}$ или $7^{300}$?
\кзадача

\задача
\вСтрочку
\пункт Докажите, что %$\displaystyle{a+\frac1a\ge 2}$ при $a>0$.
$\displaystyle{a+\frac1a\ge 2}$ при $a>0$.\\
\пункт Каково наименьшее значение %$\displaystyle{a+\frac9a}$ при $a>0$?
$\displaystyle{a+\frac9a}$ при $a>0$?
\кзадача

\задача
\вСтрочку
\пункт
Число $x$ изменили не более, чем на $0,1$.
Могло ли при этом значение $x^2$ измениться более, чем на 10?
\пункт
Тот же вопрос для значения $\sqrt x$.
\кзадача

\задача Найдите первые
\вСтрочку
\пункт девять;
\пункт десять знаков после запятой у числа $\sqrt{0,999\,999\,999}$.
\кзадача

\сзадача
Можно ли уместить два точных куба между соседними точными квадратами?
%Ины\-ми словами, имеет ли решение в целых числах неравенство:
%$n^2 < a^3 < b^3 < (n+1)^2$?
\кзадача

\задача Описанный около круга квадрат разбили на $100\times100$
равных квадратиков и закрасили квадратики, не выходящие
за пределы круга. Докажите, что площадь %получившейся
закрашенной фигуры
%составляет
не меньше 94\% площади круга.
\кзадача

\задача
Докажите, что $x^{n_1}-x^{n_2}+x^{n_3}-\cdots +x^{n_{2k+1}}\ge 0$ при $x>0$, если %где
$n_1>n_2>\cdots >n_{2k+1}$ натуральные. % числа.
%Докажите что $P(x)\ge 0$ при всех $x>0$.
\кзадача

\задача
Докажите, что в любой бесконечной арифметической прогрессии с натуральными
основанием и разностью найдется число, десятичная запись которого
начинается с 1.
\кзадача



%\задача
%\пункт Найдите первые девять знаков после запятой в десятичной записи
%числа $\sqrt{0,999999999}$.
%\пункт Найдите у числа из пункта а)
%десятый знак после запятой в его десятичной записи.
%\кзадача




\задача Сколько цифр в десятичной записи %числа
\вСтрочку
\пункт $2^{40}$; \пункт $2^{100}$;
\спункт чисел $2^{2010}$ и $5^{2010}$ вместе?
\кзадача


\раздел{***}


\задача
В банк кладут 1000 рублей. В каком случае спустя 10 лет %вкладчик
получат больше денег: если банк начисляет 5\% от имеющейся суммы раз
в год или если он начисляет (5/12)\% раз в месяц?
\кзадача


\задача
Докажите, что при всех натуральных $n$ и при всех неотрицательных $x$
выполнены неравенства
\вСтрочку
\пункт[неравенство Бернулли] $(1+x)^n\ge 1+nx$;
\пункт $(1+x)^n\ge 1+nx+\frac{n(n-1)}{2}x^2.$
\кзадача

\задача Укажите такое целое $n>1$, что
\вСтрочку
\пункт $1,001^n>10^5$;
\пункт $0,999^n<10^{-5}$.
%\пункт $\sqrt[n]{n} < 1{,}001$.
\кзадача


%\раздел{Подробное изложение}

\задача
\вСтрочку
\пункт
Пусть $b>1$.
Докажите, что найд\"ется такое натуральное $k$, что при любом
натуральном $n\geq k$ будет выполнено неравенство $b^n>1000$
(то есть, $b^k>1000,$ $b^{k+1}>1000$, $b^{k+2}>1000$,
и так далее).\\
\пункт
Можно ли %в этой задаче
заменить %число
$1000$ на любое другое
конкретное число?
\кзадача

\задача
\вСтрочку
\пункт
Найд\"ется ли такое $C$,
что при любом натуральном $n\geq C$ будет выполнено неравенство
$(1,01)^n>1000n$?\\
\пункт
%Изменится ли ответ на вопрос этой задачи,
А если заменить число $1,01$
на любое конкретное число, большее 1, а число $1000$ --- на любое
конкретное положительное число?
\кзадача

\задача
При каких натуральных $n$ выполнено неравенство
\вСтрочку
\пункт
$2^n\geq n$;
\пункт
$2^n\geq n^2$?
\кзадача

\задача\пункт
Пусть $q>1$. Пусть последовательность положительных чисел $(x_n)$ такова,
что, начиная с некоторого номера, выполнено неравенство
$x_{n+1}/x_n>q$. Докажите, что тогда,
начиная с некоторого номера, выполнено неравенство $x_n>1$.\\
\пункт
Останется ли верным утверждение задачи, если $q=1$?
\кзадача



\задача Найд\"ется ли такое %число
$k$, что при всех натуральных $n\geq k$
будет выполнено %неравенство
$2^n>n^{50}$?
\кзадача


\задача
\вСтрочку
\пункт
Найд\"ется ли такое число $C$,
что при любом натуральном  $n\geq C$ будет выполнено неравенство
$n!>100^n$?
\пункт
%Изменится ли ответ,
А если заменить число $100$ на любое другое конкретное число?
\кзадача

\опр
Говорят, что неравенство верно
\лк при всех достаточно больших~$n$\пк\ или
\лк при $n$ много больше нуля\пк,
если найд\"ется такое $k$, что неравенство верно
при всех $n>k$.
Обозначение: верно при $n\gg 0$.
\копр

\задача
\вСтрочку
\пункт Докажите, что неравенство
$n^n>10^{6}\cdot n!$ выполнено
при $n\gg 0$.\\
\пункт Можно ли заменить $10^6$ на любое другое число?
\кзадача

\задача
\пункт
Докажите, что $0,001n^2>100n+179$ при $n\gg 0$.\\
\пункт
Число $C$ --- любое, $n$ и $m$ --- натуральные, прич\"ем $n>m$.
Докажите, что $x^n>Cx^m$ при $x\gg 0$.\\
\пункт
Дан многочлен $P(x)=p_k x^k+p_{k-1}x^{k-1}+\dots+p_1x+p_0$,
где $p_k>0$. Верно ли, что $P(x)>0$~при~$x\gg 0$?
\кзадача



%\задача
%Докажите, что если $a>1$ и $C$ --- любое, то
%\вСтрочку
%\пункт
%$a^n>C$ при $n\gg 0$;
%\пункт
%$a^n>n$ при $n\gg 0$.
%\кзадача

%\задача Верно ли, что
%\вСтрочку
%\пункт
%$1,01^n>100n$  при $n\gg0$;
%\пункт
%если $a>1$, $C>0$, то $a^n>Cn$ при $n\gg 0$?
%\кзадача

%\задача
%Докажите, что для любого положительного числа $C$ и любых
%натуральных чисел $n$ и $k$, где $n>k$, неравенство $x^n>Cx^k$
%выполняется при $x\gg 0$.
%\кзадача

%\задача
%Пусть $P(x)=p_n x^n+\ldots$ и $Q(x)=q_m x^m+\ldots$ --- многочлены
%степеней $n$ и~$m$, причем $n>m$ и $p_n,\ q_m>0$.
%Докажите, что $P(x)>Q(x)$ при
%$x\gg 0$.
%\кзадача




%\задача
%Докажите, что
%если $a>1$ и $C>0$, то $a^n>Cn$ при $n\gg 0$.
%\кзадача

%\раздел{***}

%\задача
%\вСтрочку
%\пункт
%Пусть $q>1$, и последовательность положительных чисел $(x_n)$ такова,
%что $x_{n+1}/x_n>q$ при $n\gg 0$. Докажите, что $x_n>1$ при $n\gg 0$.
%\пункт
%Верно ли утверждение пункта а), если $q=1$?
%%что если $x_{n+1}/x_n>1$ при $n\gg 0$, то $x_n>1$ при $n\gg 0$?
%\кзадача


%\задача Докажите, что при натуральных $n\gg 0$ %\\
%\вСтрочку
%\пункт $2^n>n^{100}$;
%\пункт если $a>1$ и $k\in\N$, %--- натуральное число,
%то $a^n>n^k$.
%\кзадача


%\задача
%Найдите такое натуральное число $C$, что при всех целых $k$
%выполняется неравенство $|k^3 -2k + 1|< C|k^4-3|$
%\кзадача

\задача Докажите, что для любого $a$ неравенство $n!>a^n$ выполнено
при $n\gg 0$.
\кзадача


%\задача
%Дано положительное число $\alpha$.
%Известно, что неравенство $1<x\alpha<2$
%имеет ровно 3 решения в целых числах~$x$.
%Сколько решений в целых числах $x$
%может иметь неравенство $2<x\alpha<3$? %Укажите все возможности.
%\кзадача



%\сзадача
%Коэффициенты $p$ и $q$ квадратного уравнения $x^2+px+q=0$ изменили
%не больше, чем на 0,01. Мог ли больший корень уравнения измениться
%больше, чем на 100?
%\кзадача


%\сзадача Есть ли такое целое $n$, что
%$[10^9\cdot\{\sqrt n\}]=987654321$? ($\{x\}=x-[x]$ ---
%дробная часть $x$.)
%\кзадача

%}}}

\ЛичныйКондуитФ{.1mm}{6mm}{8}
%\ЛичныйКондуит{0mm}{6mm}

%\GenXMLW


\end{document}

\раздел{***}

\задача
Десятичная запись числа $a\in\N$ состоит из $n$ цифр.
\вСтрочку
\пункт Сколько цифр может быть~в~десяти\-чной записи числа $a^3$?
\пункт Десятичная запись $a^3$ состоит из $k$ цифр.
Возможно ли, что $n+k=2013$?
\кзадача




\задача Найдите наименьшее значение
выражения $\sqrt{(x-1)^2+y^2}+\sqrt{x^2+(y-1)^2}$.
%При каких $x$ и $y$ оно достигается?
\кзадача


\задача
\вСтрочку
\пункт Докажите, что $\frac1{n+1}+\frac1{n+2}+\cdots +\frac1{2n}\ge\frac12$
при любом $n\in\N$.\\
\пункт \выд{(Гармонический ряд)} Для любого ли числа $C$ найдется
такое $n\in\N$, что будет выполнено
неравенство $1+\frac12+\cdots+\frac1n\ge C$?
%\кзадача
%\сзадача
\пункт Тот же вопрос для неравенства $\frac1{1^2}+
\frac1{2^2}+\cdots +\frac1{n^2}\ge C$.
\кзадача

%\УстановитьГраницы{0cm}{3.2cm}
\сзадача
Есть неограниченное число одинаковых
кирпичей (прямоугольных параллелепипедов). %Кирпичи
Их~кладут друг на друга со сдвигом (так, чтобы не падали). % (см.~рис.).
\лк Крышу\пк\ какой наибольшей длины можно получить?
%\ВосстановитьГраницы
\кзадача


%\задача Докажите для всех натуральных $n$ неравенства
%\вСтрочку
%\пункт
%$\displaystyle{\frac{a^{n+1}}{b^n}\ge(n+1)a-nb}$, если $a,\ b>0$;\\
%\пункт $\displaystyle{\left(1+\frac1{n+1}\right)^{n+1}
%\ge\left(1+\frac1n\right)^n}$;
%\пункт $\displaystyle{\left(1+\frac1{n-1}\right)^{n}
%\ge\left(1+\frac1n\right)^{n+1}}$;
%\пункт $\displaystyle{2\le\left(1+\frac1n\right)^n\le4}$;
%%\спункт $\displaystyle{\left(1+\frac1n\right)^n\leq3}$;
%\спункт[неравенство Коши]
%$\displaystyle{\frac{a_1+\dots+a_n}{n}\geq\root n\of{a_1\dots a_n}}$,
%если $a_1,\dots,a_n$ положительны;
%\спункт $\displaystyle{\left(\frac{n}4\right)^n\le n!
%\le\left(\frac{n+1}2\right)^n}$.
%\кзадача

\задача Докажите для $n\in\N$: % неравенства
\вСтрочку
\пункт
$a^{n+1}/b^n\ge(n+1)a-nb$, если $a,b>0$;
\пункт $(1+\frac1{n+1})^{n+1}\ge(1+\frac1n)^n$;\break
\пункт $(1+\frac1{n-1})^{n}\ge(1+\frac1n)^{n+1}$;
\пункт $2\le(1+\frac1n)^n\le4$;
\пункт $(1+\frac1n)^n\leq1+\frac1{1!}+\frac1{2!}+\ldots+\frac1{n!}$.
%\спункт[неравенство Коши]
%$(a_1+\dots+a_n)/n\geq \root n\of{a_1\dots a_n}$,
%если числа $a_1,\dots,a_n$ положительны;
%\спункт $(n/4)^n\le n!\le((n+1)/2)^n$;
%\спункт $(1+\frac1n)^n\leq3$.
\кзадача

\vspace*{-1truemm}

\раздел{***}

\vspace*{-2truemm}



\задача
Решите в натуральных числах уравнения:
\вСтрочку
\пункт $a!+b!+c!=d!$;
\пункт $n(n+1)=m(m+2)$;\\
\пункт $(1/a)+(1/b)+(1/c)=1$;
\пункт $x+y+z=xyz$;
%\пункт $x^2+3x=y^2$;
\пункт $x^3+4x^2+1=y^3$;
\пункт $xy+yz+xz=xyz+2$.
\кзадача



\сзадача
Из клетчатой плоскости вырезали клетки, обе координаты которых
делятся на 10. Можно ли оставшуюся часть плоскости разрезать
на доминошки (каждая состоит из двух соседних клеток)?
\кзадача

\сзадача Найд\"ется ли такое целое $n$, что
первые 9 знаков после запятой у числа %в записи  %числа
$\{\sqrt{n}\}$ будут $987654321$?
\кзадача


\раздел{Запас}

\задача Верно ли, что при натуральном $n$ и $0\leq a\leq1$
выполнены неравенства
\вСтрочку
\пункт $(1-a)^n\ge 1-na$;
\пункт $(1-a)^n\le 1-an+a^2n(n-1)/2.$
\кзадача

\задача
Верно ли, что существует такое натуральное число $n$, для которого\\
\вСтрочку
\пункт $\sqrt{n+1} - \sqrt n < 0{,}1$?
\пункт $\sqrt{n^2 + n} -n < 0{,}1$?
\кзадача

\задача
Найд\"ется ли такое число $C$,
что при любом числе $n\geq C$ будет выполнено неравенство
$0,001n^2>100n+57$?
\кзадача

\задача Докажите неравенства: \вСтрочку\\
\пункт $|x+y| \leq |x| + |y|$;
\пункт $|x-y| \geq |x| - |y|$;
\пункт $|x-y| \geq ||x| - |y||$.

\noindent В каждом из случаев выясните, когда неравенство превращается в
равенство.
\кзадача

\раздел{Подробное изложение}

\задача
\вСтрочку
\пункт
Пусть $b>1$.
Докажите, что найд\"ется такое натуральное $k$, что при любом
натуральном $n\geq k$ будет выполнено неравенство $b^n>1000$
(то есть, $b^k>1000,$ $b^{k+1}>1000$, $b^{k+2}>1000$,
и так далее).
\пункт
Можно ли %в этой задаче
заменить %число
$1000$ на любое другое
конкретное число?
\кзадача

\задача
\вСтрочку
\пункт
Найд\"ется ли такое $C$,
что при любом натуральном $n\geq C$ будет выполнено неравенство
$(1,01)^n>1000n$?
\пункт
%Изменится ли ответ на вопрос этой задачи,
А если заменить число $1,01$
на любое конкретное число, большее 1, а число $1000$ на любое
конкретное положительное число?
\кзадача

\задача
При каких натуральных $n$ выполнено неравенство
\вСтрочку
\пункт
$2^n\geq n$;
\пункт
$2^n\geq n^2$?
\кзадача

\задача\пункт
Пусть $q>1$. Пусть последовательность положительных чисел $(x_n)$ такова,
что, начиная с некоторого номера $m$, выполнено неравенство
$x_{m+1}/x_m>q$. Докажите, что тогда,
начиная с некоторого номера $k$, выполнено неравенство $x_k>1$.
\пункт
Останется ли верным утверждение задачи, если $q=1$?
\кзадача



\задача Найд\"ется ли такое %число
$k$, что при всех натуральных $n\geq k$
будет выполнено %неравенство
$2^n>n^{50}$?
\кзадача


\задача
\вСтрочку
\пункт
Найд\"ется ли такое число $C$,
что при любом натуральном  $n\geq C$ будет выполнено неравенство
$n!>100^n$?
\пункт
%Изменится ли ответ,
А если заменить число $100$ на любое другое конкретное число?
\кзадача



\end{document} 