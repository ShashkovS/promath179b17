% !TeX encoding = windows-1251
\documentclass[a4paper,12pt]{article}
\usepackage{newlistok}

\УвеличитьВысоту{2.5cm}
\УвеличитьШирину{1.0cm}

\ВключитьКолонитул
\Заголовок{Асимптотические обозначения}
\НомерЛистка{17$\frac12$}
\ДатаЛистка{\ммгг}

\renewcommand{\spacer}{\vfill}
\begin{document}
\СоздатьЗаголовок

\опр
Пусть $(x_n)$ и $(y_n)$ --- две последовательности.
Говорят, что $x_n = O(y_n)$ (читается как \лк икс-эн есть о большое от игрек-эн\пк),
если существуют константа $C$ такая, что $|x_n|\le C\cdot|y_n|$ при $n \gg 0$.
Говорят, что $x_n = o(y_n)$ (читается как \лк икс-эн есть о малое от игрек-эн\пк),
если для любого числа $\ep>0$ неравенство $|x_n|\le \ep\cdot|y_n|$ выполнено при $n\gg0$.

\small
Используя асимптотические обозначения очень удобно выделять самую \лк весомую\пк часть последовательности.
Например, пишут $(n+1)^2 = n^2 + o(n^2)$ или $(n+1)^2 = n^2 + O(n)$, имея в виду,
что заменив каждое асимптотическое выражение на подходящую последовательность, удовлетворяющую этой асимптотике, можно получить тождество.
В нашем примере в качестве такой последовательности выступает $(2n+1)$, ведь $2n+1 = o(n^2)$ и $2n+1 = O(n)$.

\копр

\задача
Докажите, что
\пункт
$x_n = O(1)$ тогда и только тогда, когда последовательность $(x_n)$ ограничена;
\пункт
$x_n = o(1)$ тогда и только тогда, когда последовательность $(x_n)$ бесконечно малая;
\пункт
если в последовательности $(y_n)$ нет нулевых членов, то $x_n=O(y_n)$ тогда и только тогда, когда $(x_n/y_n)$ ограничена,
а $x_n=o(y_n)$ равносильно тому, что $(x_n/y_n)$ --- бесконечно малая.

\кзадача

\задача
Какой смысл у тождеств: \quad $o(1) + o(1) = o(1)$, \quad $o(1)\cdot O(1) = o(1)$, \quad $o(1) + O(1) = O(1)$?
\кзадача

\задача
Какие из следующих утверждений верны:
\пункт
$\sin n = O(1)$;\quad
%\пункт
$\sin n = o(1)$;
\smallskip%
\\
\пункт
$n^2 = O(n^3)$;\quad
%\пункт
$n^2 = o(n^3)$;\quad
%\пункт
$n^2 = O(n)$;\quad
%\пункт
$n^2 = o(n)$;\quad
%\пункт
$1/n^2 = O(1/n^3)$;\quad
%\пункт
$1/n^2 = o(1/n)$.
\кзадача


\задача
Докажите, что
\пункт
$(n+1)^3 = n^3 + o(n^3)$;
\пункт
$(n+1)^4 = n^4 + 4 n^3 + O(n^2)$;
\\\пункт
$1 + 2 + \ldots + n = n^2/2 + O(n)$;
\пункт
$1^2 + 2^2 + \ldots + n^2 = n^3/6 + O(n^2)$;
\кзадача



\vspace*{-2mm}

\ввзадача[основные асимптотики]
Докажите, что
\пункт
$n^k = o(n^l)$ при $k < l$, где $k,l\in\Z$;
\\\пункт
$n^k = o(a^n)$ при $a > 1$;
\пункт
$a^n = o(n^k)$ при $0 < a < 1$;
\пункт
$a^n = o(n!)$;
\пункт
$n! = o(n^n)$;
\кзадача
\vspace*{-2mm}

\задача
Можно ли утверждать, что $x_n = o(z_n)$, если
\пункт
$x_n = o(y_n)$ и $y_n = o(z_n)$;
\\\пункт
$x_n = O(y_n)$ и $y_n = O(z_n)$;
\пункт
$x_n = o(y_n)$ и $y_n = O(z_n)$;
\пункт
$x_n = O(y_n)$ и $y_n = o(z_n)$.
\кзадача

\задача
Известно, что $x_n=O(n^4)$ и $y_n = o(n^3)$.
Что можно сказать про $x_n + y_n$ и $x_n \cdot y_n$?
\кзадача

\опр
\small
Используют также обозначения вида $n\cdot(1 + O(1/n)) + 2^{o(1)} = n + O(1)$, где асимптотические обозначения есть с обеих сторон равенства.
В этом случае имеют в виду следующее:
%что множества последовательностей, удовлетворяющих данным асимптотикам, совпадают.
если заменить каждое асимптотическое выражение в левой части на любую последовательность, удовлетворяющую этой асимптотике,
то в правой части можно заменить каждое асимптотическое выражение на подходящую последовательность так, чтобы получить тождество.
\копр

\взадача
Докажите, что
\пункт
$x_n\cdot O(y_n) = O(x_n y_n)$;\quad
$O(x_n) \cdot O(y_n) = O(x_n y_n)$;
\\\пункт
$x_n\cdot o(y_n) = o(x_n) \cdot o(y_n) = o(x_n) \cdot O(y_n) = O(x_n) \cdot o(y_n) = o(x_n y_n)$;
\\\пункт
если $x_n=O(y_n)$, то $O(x_n) + O(y_n) = O(y_n)$ и $o(x_n) + o(y_n) = o(y_n)$;
\кзадача

\раздел{$***$}

\задача
\пункт
Предполагая, что формула $\sqrt{1+1/n} = 1 + 1/2n + a/n^2 + O(1/n^3)$ верна для некоторой константы $a$, найдите значение $a$.
\спункт Докажите, что при этом $a$ формула действительно верна.
\кзадача


\задача
Укажите такие числа $a$ и $b$, что $\sqrt[3]{1+1/n} = 1 + a/n + b/n^2 + O(1/n^3)$, считая что для некоторых $a$ и $b$ эта формула действительно верна.
\кзадача

\задача
Пусть $k\in\N$. Укажите такое число $a$, что $\sqrt[k]{1+1/n} = 1 + a/n + O(1/n^2)$.
\кзадача

\задача
\пункт
При анализе алгоритма выяснилось, что время его работы $T(n)$ на входе длины $n$ удовлетворяет соотношению $T(n) = T(\hs{n/2}) + T(\hs{n/3}) + O(n)$. Докажите, что $T(n) = O(n)$.
\\\спункт
Что можно сказать о $T(n)$, если $T(n) = 2T(\hs{n/2}) + O(n)$?
\кзадача

\сзадача
Считая, что при неких $a$ и $b$ верна формула
$1 + 1/2^2 + 1/3^2 + \ldots + 1/n^2 = a + b/n + O(1/n^2)$, найдите $b$.
(Найти $a$ гораздо сложнее, $a=\pi^2/6$.)
\кзадача


\взадача[асимптотика факториала]
\невСтрочку
\vspace*{-2mm}
\пункт
Докажите, что для любого натурального числа~$n$ выполнены неравенства $\left(\dfrac n4\right)^n\leqslant n!\leqslant\left(\dfrac{n+1}2\right)^n$;
\vspace*{-3mm}
\сспункт[формула Стирлинга]
Докажите, что $n! = \sqrt{2\pi n}\,\hr{\dfrac{n}{e}}^n \cdot \bbbr{1 + O\hr{\dfrac{1}{n}}}$.
\кзадача
\vspace*{-4mm}


\ЛичныйКондуит{0mm}{6mm}

\end{document}




