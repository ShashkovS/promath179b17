% !TeX encoding = windows-1251
\documentclass[a4paper,12pt]{article}
\usepackage{newlistok}
%\usepackage{tikz}
%\usetikzlibrary{calc}

%\documentstyle[11pt, russcorr, listok]{article}
\newcommand{\del}{\mathrel{\raisebox{-.3 ex}{${\vdots}$}}}
%\renewcommand{\spacer}{\vfill}

%\УвеличитьШирину{1.4truecm}
%\УвеличитьВысоту{2.5truecm}

\begin{document}
\Заголовок{Асимптотика}
\НомерЛистка{7д}
\ДатаЛистка{01.2015}
\Подзаголовок{}
\СоздатьЗаголовок

\begin{center}
{\bf Комбинаторика}
\end{center}

\задача Из клетчатой плоскости выбросили все клетки, обе
координаты которых делятся на 10. Можно ли оставшееся \пункт разрезать на доминошки; \пункт обойти ходом коня?
\кзадача

\задача Можно ли расставить во всех точках плоскости с целыми координатами
натуральные числа так, чтобы каждое натуральное число присутствовало, %стояло в какой-нибудь точке,
и чтобы на любой прямой, проходящей через две точки с целыми координатами,
но не проходящей через начало координат, расстановка чисел была периодической?
\кзадача

\задача На плоскости живут мальчики и девочки, расстояние между любыми двумя
не меньше 1~м. Каждый ребенок общается только с теми, которые живут от
него не дальше 10~км. Девочка счастлива, если число мальчиков, с которыми
она целуется, больше числа девочек, с которыми целуется любой ее знакомый
мальчик.  Мальчик счастлив, если число девочек, с которыми
он танцует, больше числа мальчиков, с которыми танцует
любая его знакомая девочка. Докажите, что полное счастье невозможно.
\кзадача

\задача На клетчатой плоскости стоит квадрат $n\times n$, составленный из
фишек.  Разрешается прыгать фишкой через соседнюю по стороне на свободную
клетку, при этом фишка, через которую перепрыгнули, снимается. Докажите,
что для любого $\varepsilon>0$ ситуация, в которой нельзя больше сделать ни
одного хода, может возникнуть не раньше, чем через
$n^2\bigl(\frac12-\varepsilon\bigr)$ ходов для всех достаточно больших $n$.
\кзадача

\begin{center}
{\bf Комбинаторная геометрия}
\end{center}

%\задача \пункт Плоскость покрыта конечным числом углов. Докажите, что сумма
%градусных мер не меньше $360^\circ$.
%\пункт Докажите, что плоскость нельзя покрыть конечным числом
%внутренностей парабол.
%\кзадача

\задача Существует ли функция из какого-то шара в какой-то круг, не
уменьшающая расстояния?
\кзадача

%\задача (Лемма Минковского) На плоскости дана выпуклая фигура площади $>4$ с
%центром симметрии в целой точке. Докажите, что внутри нее есть еще хотя бы
%одна целая точка.
%\кзадача

\задача Можно ли разбить плоскость на \пункт равные выпуклые семиугольники; \пункт
выпуклые семиугольники 100 типов?
\кзадача

\задача[Зоны Берлюэна] Тысячной зоной Берлюэна с центром в
целочисленной точке $O$ называется множество всех точек $M$ плоскости, для
которых существует ровно 999 точек целочисленной решетки, удаленных от $M$
на расстояние меньшее, чем $OM$. Найдите площадь тысячной зоны Берлюэна.
\кзадача

%\задача (Формула Пика) Докажите, что площадь многоугольника с вершинами в
%узлах целочисленной решетки равна $n+\displaystyle\frac k2-1$, где
%$n$ --- количество узлов внутри, а $k$ --- на границе многоугольника.
%\кзадача

\задача[О кляксе] На плоскости есть конечная клякса. Она эволюционирует
по следующему закону:  если в круге радиуса 1 с центром в некоторой точке
больше половины площади белая, то точка становится белой, иначе ---
становится черной. Можно ли нарисовать кляксу, которая в некоторый момент
увеличит свою площадь более, чем в 1000 раз?
\кзадача

\begin{center}
{\bf Алгебра и теория чисел}
\end{center}

\задача Докажите, что для подходящего $N$ уравнение $x^3+y^3+z^3+t^3=N$
имеет не менее 1000 решений в натуральных числах.
\кзадача

%\задача Последовательность натуральных чисел $(a_n)$ удовлетворяет
%следующим условиям: 1) при любых различных $n$ и $k$ число $a_n-a_k$
%делится на $n-k$; 2) существует такой многочлен $P(x)$ степени $m$, что
%$a_n<P(n)$ при любом $n$.  Докажите, что $a_n$ есть многочлен.
%\кзадача

%\задача Если значения унитарного многочлена в целых точках
%есть квадраты целых чисел, то он сам есть квадрат другого многочлена.
%\кзадача

\задача Пусть $\sigma(n)$ --- сумма цифр числа $n$.
\пункт Докажите, что найдется бесконечно много $n$ таких, что
$\sigma(2^{n+1})<\sigma(2^n)$.
\пункт Докажите, что последовательность $\displaystyle\frac1{\sigma(2^n)}$ бесконечно малая.
\кзадача

%\задача Докажите, что не существует таких многочленов
%$$
%  P_1(x_1,\ldots,x_{1000}),P_2(x_1,\ldots,x_{1000}),\dots,
%  P_{1001}(x_1,\ldots,x_{1000}),
%$$
%что для любого ненулевого многочлена от 1001 переменной при подстановки вместо
%переменных многочленов $P_1$, $P_2$,\dots,$P_{1001}$ получится ненулевой
%многочлен.
%\кзадача

\ЛичныйКондуит{0mm}{6mm}
%\GenXMLW

\end{document}

\begin{center}
{\bf Анализ}
\end{center}

\задача В клетках квадранта $i,j\geq0$ расставлены действительные числа так,
что число в любой клетке равно полусумме чисел над ним и справа от него.
Докажите, что все числа равны, если а) они ограничены по модулю; б) они
неотрицательны.
\кзадача

\задача Докажите, что
$$
  \sum_{i=1}^n \frac1i-\ln n\to\const.
$$
\кзадача

\задача Докажите, что
$$
  \sum_{p_i<n} \frac1{p_i}-\ln\ln n\to\const,
$$
где $p_i$ --- все простые числа.
\кзадача

\задача Пусть $A_n$ --- последовательность, сходящаяся к бесконечности.
Докажите, что существует такое $x$, что для любого $\varepsilon>0$ неравенство
$|kx-A_n|<\varepsilon$ справедливо для бесконечного количества натуральных $n$ и
$k$.
\кзадача

\задача Докажите, что
$$
  \frac{\displaystyle
  \sum_{i\leq n, \exists a,b: i=a^2+b^2} \frac1i}{\sqrt{\ln n}}
  \to\const.
$$
\кзадача
