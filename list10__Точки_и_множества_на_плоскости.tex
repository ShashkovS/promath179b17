% !TeX encoding = windows-1251
\documentclass[a4paper,11pt]{article}
\usepackage{newlistok}
%\documentstyle[11pt, russcorr, listok]{article}
\newcommand{\del}{\mathrel{\raisebox{-.3 ex}{${\vdots}$}}}

\УвеличитьШирину{1.3truecm}
\УвеличитьВысоту{3.1truecm}
\hoffset=-2.5truecm
\voffset=-27.2truemm

\Заголовок{Точки и множества на плоскости}
%и цепные дроби}
\НомерЛистка{10}
\ДатаЛистка{04.2014}

\renewcommand{\spacer}{\vfil}

\begin{document}

\СоздатьЗаголовок


\задача
На плоскости дана точка с коодинатами $(x,y)$. Как найти точку с координатами
\пункт $(x+1,y)$;
\пункт $(x+2,y-1)$;
\пункт $(y,x)$;
\пункт $(2x,2y)$;
\пункт $(\frac{1}{2}x,\frac{1}{2}y)$;
\пункт $(-x,y)$;
\пункт $(-x,-y)$?
\кзадача

\задача
Концы отрезка $PQ$ имеют координаты $P(a,b)$ и $Q(c,d)$. Найдите координаты
\пункт середины этого отрезка;
\пункт точки $R$, которая делит этот отрезок в отношении $PR:RQ=3:2$.
\кзадача


\задача
Три вершины параллелограмма имеют координаты $(0,0)$, $(a,b)$, $(c,d)$.  Найдите координаты четвёртой вершины. Сколько решений (вариантов положения четвёртой вершины) имеет эта задача?
\кзадача

\задача
Две вершины квадрата имеют координаты $(0,0)$ и $(x,y)$. Найдите координаты двух других его вершин. Сколько решений (возможных вариантов) есть в этой задаче?
\кзадача

\задача
Найдите координаты точки пересечения медиан треугольника, вершины которого имеют координаты $(x_1,y_1)$, $(x_2,y_2)$, $(x_2,y_3)$.
\кзадача


В задачах 6 -- 19 надо нарисовать множества точек $(x,y)$, координаты которых удовлетворяют некоторым условиям (в одной задаче может быть несколько множеств).

\задача
$x=y$; $x>y$; $x<y$.
\кзадача

\задача
$|x|=|y|$; $|x|<|y|$; $|x|>|y|$.
\кзадача

\задача
$x+y>1$; $x+y=1$; $x+y<1$.
\кзадача

\задача
$2y>x+1$; $2y=x+1$; $2y<x+1$.
\кзадача

\задача
$x^2+y^2>1$; $x^2+y^2=1$; $x^2+y^2<1$.
\кзадача

\задача
$xy>1$; $xy=1$; $xy<1$.
\кзадача

\задача
$|y|=|x|+1$.
\кзадача

\задача
$y^2>x^2$; $y^2=x^2$; $y^2<x^2$.
\кзадача

\задача
$y>x^2$; $x>y^2$.
\кзадача

\задача
$x^2+xy+y^2>0$; $x^2+2xy+y^2>0$.
\кзадача

\задача
одно из чисел $x$ и $y$ целое.
\кзадача

\задача
оба числа $x$ и $y$ целые.
\кзадача


\emph{Целой частью} числа $x$ называют наибольшее целое число, не превосходящее $x$ (обозначение $[x]$ или $\lfloor x\rfloor$); например, $\lfloor 2{,}3\rfloor =2$, $\lfloor -2{,}3\rfloor=-3$, и $\lfloor 99\rfloor=99$.

\задача
$\lfloor x\rfloor =\lfloor y\rfloor$.
\кзадача

\emph{Дробной частью} числа $x$ называется разница между числом и его целой частью (будем обозначать её $\{x\}$), так что $x=\lfloor x\rfloor +\{x\}$.

\задача
$\{x\}=\{y\}$.
\кзадача

\задача
Найдите количество целых решений (оба числа целые) уравнения $x^2+y^2<10000$ с ошибкой не более чем на $20\%$.
\кзадача

\задача
Нарисуйте примерный вид графика $y=x+1/x$. Имеет ли этот график ось симметрии?
\кзадача

\задача
Верно ли такое утверждение: \emph{$\lfloor\lfloor a/b\rfloor/c\rfloor=\lfloor a/bc\rfloor$ для любых положительных целых чисел $a,b,c$}?
\кзадача

\ЛичныйКондуит{0mm}{6mm}

%\СделатьКондуит{7mm}{7mm}

\end{document}

\кзадача 