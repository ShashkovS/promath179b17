% !TeX encoding = windows-1251
\documentclass[a4paper, 12pt]{article}
\usepackage{newlistok}
%\usepackage{tikz}
%\usetikzlibrary{calc}

%\documentstyle[11pt, russcorr, listok]{article}
\newcommand{\del}{\mathrel{\raisebox{-.3 ex}{${\vdots}$}}}

\УвеличитьШирину{.5truecm}
\УвеличитьВысоту{2truecm}
\hoffset=-2.5truecm
\voffset=-25truemm


\Заголовок{Арифметика вычетов: приложения}
\НомерЛистка{6д}
\ДатаЛистка{12.2014}
\begin{document}

%\scalebox{.93}{\vbox{
%\ncopy{1}{
\СоздатьЗаголовок


%\раздел{Многочлены}

%\раздел{Теорема Эйлера}


\раздел{Представимость чисел в виде суммы двух квадратов}

%\задача
%Пусть $a\in\Z_p$ и $a\neq0$. Известно, что среди остатков $a$, $a^{-1}$, $-a$, $(-a)^{-1}$ есть совпадающие.
%Докажите, что $a^2+1=0$ или $a+1=0$ или $a-1=0$.
%\кзадача

%\задача
%Пусть $p > 2$~--- простое. Сколько из чисел $1,
%2,\ldots, p - 1$ удовлетворяют сравнению $x^{\frac{p - 1}2} - 1
%\equiv 0 \pmod{p}$, а сколько~--- сравнению $x^{\frac{p - 1}2} + 1
%\equiv 0 \pmod{p}$?
%\кзадача

\задача Пусть $p$~--- простое вида $4k + 1$, и пусть $x=(2k)!$. Докажите, что
%найдется такое целое число $x$, что
$x^2 \equiv -1 \pmod{p}$.
\кзадача

\задача Пусть $p$~--- простое вида $4k + 1$, и пусть $x$ удовлетворяет сравнению $x^2 \equiv -1 \pmod{p}$. Докажите, что
\сНовойСтроки
\пункт $(a + xb)(a - xb)\equiv a^2 + b^2 \pmod{p}$ при $a,b\in\Z$;
\пункт среди чисел вида $m + xn$, где $m,n\in\Z$, $0 \leq m,n
\leq [\sqrt p]$, найдутся два с равными остатками от деления на $p$;
\пункт найдётся ненулевое число вида $a + bx$, делящееся на
$p$, где $a,b\in\Z$, причем $|a|<\sqrt p$ и $|b|<\sqrt p$;
% не равны одновременно нулю и по абсолютной величине оба меньше $\sqrt p$;
\пункт $p$ представимо в виде суммы двух квадратов целых чисел.
\кзадача

\задача Пусть $p$~--- простое число вида $4k+3$, числа $a$ и $b$ целые и $a^2 + b^2$ делится на $p$.
Докажите, что $a$ делится на $p$ и $b$ делится на $p$. %{\it Указание:} воспользуйтесь задачей 11, а).
\кзадача

\задача Докажите, что произведение чисел, представимых в виде суммы
двух квадратов целых чисел, само представимо в виде суммы двух
квадратов целых чисел. \кзадача

\задача Сформулируйте и докажите теорему о том,
как по разложению числа на простые множители узнать, представимо ли это число
в виде суммы двух квадратов целых чисел.
\кзадача

\раздел{Функция Эйлера и китайская теорема об остатках}


%\задача \пункт  Найдите $\varphi({p}^{\alpha})$, где $p$ простое, $\alpha\in\N$.\\
%\пункт Докажите, что $\varphi(ab)=\varphi(a)\varphi(b)$, если $(a,b)=1$.
%; \пункт $\varphi({p_1}^{\alpha_1})\dots\varphi({p_k}^{\alpha_k})$.
%\кзадача

\опр
Определим функцию Эйлера $\varphi(m)$ как количество обратимых элементов в $Z_m$.
\копр

\задача
Докажите, что это определение согласуется с данным в задаче 15 листка $15\frac12$.
\кзадача

\опр
Определим множество $\Z_k\times\Z_l$ как множество всех пар, в которых первый элемент принадлежит $\Z_k$, а второй принадлежит $\Z_l$).\\
Суммой и произведением пар $(\alpha,\beta)$ и $(\gamma,\delta)$ из $\Z_k\times\Z_l$ будем считать пары $(\alpha+\gamma, \beta+\delta)$ и $(\alpha\gamma,\beta\delta)$ соответственно.\\ Нулем в $\Z_k\times\Z_l$ будем называть пару $([0],[0])$, а единицей --- пару $([1],[1])$.\\ Тогда в $\Z_k\times\Z_l$ можно (аналогично листку $15\frac12$) определить делители нуля, обратимые элементы.
\копр


\задача Пусть $k$ и $l$ --- взаимно простые натуральные числа. Сопоставим элементу $[n]_{kl}$ пару элементов $([n]_k,[n]_l)$. Докажите, что
\сНовойСтроки
\пункт в $([0],[0])$ переходит только $[0]$;
\пункт данное сопоставление является биекцией между $\Z_{kl}$ и $\Z_k\times\Z_l$;
\пункт при данном сопоставлении делители нуля переходят в делители нуля, а обратимые элементы --- в обратимые элементы;
\пункт $\varphi(kl) = \varphi(k)\varphi(l)$.
\кзадача

\задача Найдите \пункт $\varphi(1)$, \пункт $\varphi(p)$, \пункт $\varphi(p^k)$, \пункт $\varphi(m)$. где $p$ --- простое, $k,m$ --- произвольные натуральные числа.
\кзадача

\задача [Китайская теорема об остатках]\\
\пункт Пусть натуральные $m_1, \dots, m_k$ попарно взаимно просты.
Докажите, что для любых целых $b_1,\dots,b_k$ существует такое
целое $x$, что
$x\equiv b_1\!\pmod{m_1}$, \dots,
$x\equiv b_k\!\pmod{m_k}$,
и это $x$ можно выбрать так, что
%такое $x$ найд\"ется на отрезке
$0\leq x< m_1\cdot m_2\cdot\ldots\cdot m_k$.\\
\пункт Используя функцию Эйлера, явно укажите такое $x$.
\кзадача

\задача
Найдите такое целое $a>0$, что $a/2$ --- точный квадрат, $a/3$ --- точный куб, $a/5$ --- точная 5-я степень.
\кзадача


\сзадача
Существует ли
\вСтрочку
\пункт
сколь угодно длинная;
\пункт
бесконечная арифметическая прогрессия, каждый член которой --- степень
натурального числа с целым показателем, большим~1?
%\пункт бесконечной арифметической прогрессии с таким
%свойством не существует.
\кзадача

\ЛичныйКондуит{0mm}{6mm}
%\GenXMLW
%\СделатьКондуит{5.4mm}{7mm}

%}}}

\end{document}

\задача Пусть $p$ --- простое число,
$f(x)=a_nx^n+\dots+a_1x+a_0$ --- многочлен с коэффициентами в $\Z_p.$
Докажите, что если уравнение $f(x)=0$ имеет в $\Z_p$ более $n$
решений, то $a_n=\dots=a_1=a_0=0$.
\кзадача





\Заголовок{Арифметика остатков} \ДатаЛистка{09.2005}

\СоздатьЗаголовок


\опр Пусть $m\in\N$. Говорят, что \выд{$a$ сравнимо с $b$ по модулю
$m$}, если $a - b$ делится на~$m$. Обозначение: $a\equiv b
\pmod{m}$. Для  каждого целого $r$ множество целых чисел, сравнимых
с $r$ по модулю $m$, называется \выд{классом вычетов по модулю $m$}
и обозначается через $[r]_m$. Если известно, о каком числе $m$ идёт
речь, мы будем иногда писать $[r]$ вместо $[r]_m$. Множество всех
классов вычетов по  модулю $m$ обозначается $\Z_m$. Класс $[0]_m$
называется \выд{нулевым} классом. \копр

\задача Докажите, что $[r]_m = \{mq + r\ |\ q\in\Z\}$. Сколько
элементов в множестве $\Z_m$? \кзадача

\опр Для любых классов вычетов $[r]$ и $[s]$ по модулю $m$ определим
их сумму  и произведение, положив $[r] + [s] = [r + s]$ и
$[r]\cdot[s] = [r\cdot s]$. \копр

\задача Докажите, что сложение и умножение в $\Z_m$ определены
корректно. \кзадача

\noindent Замечание. Можно представлять себе $\Z_m$ как множество
чисел $0$, $1$, $2$, \ldots, $m - 1$, которые складываются и
умножаются \лк по модулю $m$\пк\ (как остатки от деления на $m$).

\задача Составьте таблицы сложения и умножения в $\Z_2$, $\Z_3$ и
$\Z_4$. \кзадача

\задача Вычислите сумму всех элементов $\Z_m$. \кзадача

\задача Пусть $p$~--- простое число. Докажите, что $С_p^k$ делится
на $p$ при всех таких целых $k$, что $1 < k < p$. Докажите, что в
$\Z_p$ выполнено тождество $([a] + [b])^p = [a]^p + [b]^p$. \кзадача

\задача Приведите пример, когда произведение двух ненулевых классов
вычетов по модулю $m$ является нулевым классом. Такие классы
называют \выд{делителями нуля} в $\Z_m$. \кзадача

\задача Докажите, что целое число $m > 1$ простое если и только если
в $\Z_m$ нет делителей нуля. \кзадача

\задача Докажите, что целое число $m > 1$ является простым тогда и
только тогда,  когда для любого ненулевого класса вычетов $[a]_m$
найдётся такой $[b]_m$, что $[a]_m\cdot[b]_m = [1]_m$ (такой класс
$[b]$ называется \выд{обратным} (по умножению) к классу $[a]$).
\кзадача

\задача Пусть $p$~--- простое число. \вСтрочку \пункт Найдите все
такие $[a]$ из $\Z_p$, что $[a]^2 = [1]$ (то есть, $[a]$ обратен (по
умножению) сам себе). \пункт Чему равно произведение всех ненулевых
элементов $\Z_p$? \кзадача

\задача[Теорема Вильсона\footnote{Александр Вильсон (1714 --
1786)~--- шотландский астроном и математик-любитель.}] Докажите, что
натуральное число $m > 1$ является простым тогда и только тогда,
когда $(m - 1)! + 1 \equiv 0 \pmod{m}$. \кзадача

\задача[Малая теорема Ферма\footnote{Пьер Ферма (1602 -- 1665)
--- великий французский математик, один из основоположников теории
чисел.}] Пусть $p$~--- простое число, $a$~--- целое, и пусть $(a,p)
= 1$.\\ \вСтрочку \пункт Домножим все элементы $\Z_p$ на $[a]$.
Верно ли, что снова получатся все элементы $\Z_p$?\\
\пункт Докажите, что $a^{p-1} \equiv 1 \pmod{p}$. \пункт Докажите,
что $b^{p} \equiv b \pmod{p}$ при любом целом $b$. \кзадача

\сзадача \вСтрочку \пункт Докажите, что если $x^2 + 1$ делится на
нечётное простое число $p$, то $p$ имеет вид $4k + 1$. \пункт
Докажите, что простых чисел вида $4k + 1$ бесконечно много. \кзадача

\сзадача \вСтрочку \пункт Числа $p$ и $q$ простые, $2^{p} - 1\del
q$. Докажите, что $q - 1\del p$. \пункт Простое ли $2^{13} - 1$?
\кзадача

\сзадача Изобразим элементы $\Z_n$ точками, зафиксируем какой-либо
класс $[a]\in\Z_n$, и из каждой точки $[x]\in\Z_n$ проведём стрелку
в точку $[ax]$. Докажите, что если $[a]$ обратим (по умножению), то
на этой картинке движение по стрелкам распадается на
непересекающиеся циклы, причём каждый цикл, содержащий хоть одно
обратимое число, весь состоит из обратимых чисел, и все циклы,
состоящие из обратимых чисел, имеют одинаковую длину. \кзадача

\сзадача[Теорема Эйлера\footnote{Леонард Эйлер (1707 -- 1783)~---
швейцарец, работавший главным образом в России и в Германии.
Крупнейший математик XVIII в., внёсший значительный вклад во все
разделы математики.}] Пусть $m\in\N$, $\varphi(m)$~--- количество
натуральных чисел, меньших $m$ и взаимно простых с $m$.  Докажите,
что $a^{\varphi(m)} \equiv 1 \pmod{m}$, если $a\in\Z$ и $(a,m) = 1$.
\кзадача

\сзадача \пункт Пусть $p$~--- простое число, $\alpha\in\N$. Найдите
$\varphi(p^\alpha)$. \пункт Как найти $\varphi(n)$ по каноническому
разложению $n$ на простые множители? \кзадача

\сзадача Пусть $a_n$ есть число, образованное последними десятью
цифрами числа $2^n$. Докажите, что последовательность $(a_n)$
периодична (с некоторого момента), и найдите длину периода. \кзадача

\задача[Китайская теорема об остатках] Пусть натуральные числа $m_1,
\dots, m_k$ попарно взаимно просты. Докажите, что для любых целых
чисел $b_1,\dots,b_k$ существует такое число $x$, что $x\equiv b_1
\pmod{m_1}$, \dots, $x \equiv b_k \pmod{m_k}$, и это $x$ можно
выбрать так, что $0 \leq x < m_1 \cdot m_2 \cdot \dots \cdot m_k$.
\кзадача

\сзадача Существует ли \вСтрочку \пункт сколь угодно длинная; \пункт
бесконечная арифметическая прогрессия, каждый член которой~---
степень натурального числа с целым показателем, большим~1? \кзадача


