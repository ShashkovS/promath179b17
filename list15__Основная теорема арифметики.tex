% !TeX encoding = windows-1251
\documentclass[a4paper,11pt]{article}
\usepackage[mag=980]{newlistok}
%\usepackage{tikz}
%\usetikzlibrary{calc}

\УвеличитьШирину{1.5truecm}
\УвеличитьВысоту{2.2truecm}
%\hoffset=-2.5truecm
%\voffset=-25truemm


\begin{document}

\Заголовок{Простые числа. Основная теорема арифметики}
\НомерЛистка{15}
\ДатаЛистка{10.2014}

\СоздатьЗаголовок

%\vspace*{-1.5truemm}

%\опр
%Натуральное число $p>1$ называется \выд{простым}, если оно имеет ровно два
%натуральных делителя: 1 и $p$, в противном случае оно
%называется \выд{составным}.
%\копр

\теорема[Основная теорема арифметики] Для каждого натурального числа $n > 1$ существует и единственно (с точностью до порядка сомножителей) его представление в виде $n=p_1^{\al_1}\cdot \ldots \cdot p_k^{\al_k}$, где $p_1,\dots,p_k$ --- различные простые. Такое представление называется каноническим разложением $n$ на простые множители.%
\ктеорема
%Указанное разложение числа в произведение простых называется \textit{каноническим}.

\задача[Существование]  Докажите что для каждого натурального числа $n>1$ найдутся такие простые числа
$p_1,\dots,p_k$, что $n=p_1\cdot \ldots \cdot  p_k$.
\кзадача

%Вся сложность теоремы заключается в единственности. В этом листке будут приведено три различных доказательства.

\задача[Единственность: доказательство <<от противного>>] 
Предположим, что $n$ --- минимальное натуральное число, большее 1, 
у которого есть два различных разложения в произведение простых: 
$n = p_1\cdot \ldots \cdot p_k$ и $n = q_1 \cdot \ldots \cdot q_s$, 
где $p_1\le \dots \le p_k$, $q_1\le \dots \le q_s$ --- простые числа. 
Докажите, что
\\
\пункт $n \ge p_1^2$, $n \ge q_1^2$;
\пункт $n > p_1 q_1$;
\\\пункт $p_2\cdot \ldots \cdot p_k$ делится на $q_1$
({\it Указание:} рассмотрите число $n-p_1q_1$);
\\\пункт пункт в) противоречит нашему предположению.
\кзадача

\bigskip
Есть два других доказательства теоремы, опирающиеся на такое свойство простых чисел:

\mbox{\it Для любого простого числа $p$ верно следующее утверждение: если $mn\, \vdots\, p$, то либо $m \,\vdots \, p$, либо $n\, \vdots\, p$.} \hfill \mbox{\textbf{(*)}}
\bigskip

\задача Докажите единственность разложения на простые, используя \textbf{(*)}.
\кзадача

\задача[Единственность: доказательство, использующее представление (a,b) в виде $ax+by$]
\\
Пусть $a,b$ --- целые числа, причем $(a,b)=1$. Тогда %\пункт найдутся целые числа $x$ и $y$ такие, что $ax+by=1$;
\\
\пункт найдется такое целое $x$, что $ax\equiv 1\ (\bmod\ b)$;
\пункт если $ca\equiv 0\ (\bmod\ b)$, где $c$ --- целое, то $с\equiv 0\ (\bmod\ b)$.
\\
\пункт Из пункта б) выведите свойство \textbf{(*)}.
\кзадача

%\задача\сНовойСтроки(\textit{Доказательство, использующее <<идеал>>.}) Пусть $a,p$ --- целые числа, причем $p$ --- простое число, $a$ не делится на $p$. Обозначим %через $M(a,p)$ множество таких целых чисел $k$, что $ak\, \vdots\, p$.
%Докажите, что
%\пункт  сумма элементов $M(a,p)$ лежит в $M(a,p)$;
%\пункт  если умножить элемент $M(a,p)$ на целое число, то снова получится элемент $M(a,p)$;
%\пункт в $M(a,p)$ найдётся минимальный положительный элемент $m$;
%\пункт все элементы $M(a,p)$ делятся на $m$;
%\пункт $m=p$;
%\пункт из предыдущего пункта следует свойство \textbf{(*)}.
%\кзадача


\опр
Назовём \выд идеалом в множестве целых чисел $\Z$ любое подмножество $I$ с такими свойствами:\\
1) если $i\in I$ и $j\in I$, то и $i+j\in I$ (сумма любых двух чисел из идеала также принадлежит этому идеалу);\\
2) если $i\in I$, $n\in \Z$, то $ni\in I$ (умножая число из идеала на {\it любое целое}, мы получаем число из этого идеала).
\копр

\задача
Верно ли, что разность любых двух чисел из идеала также принадлежит этому идеалу?
\кзадача

\задача
Какие из следующих множеств являются идеалами в $\Z$:
\пункт $\Z$;
\пункт $\N$;
\пункт множество чётных целых чисел;
\пункт множество нечётных целых чисел;
\пункт $\{0\}$ (нулевой идеал);
\пункт множество чисел, делящихся на целое число $m$ (обозначение: $m\Z$);
%\пункт множество чисел вида $ax+by$, где $a$ и $b$ --- данные целые, $x$ и $y$ --- любые целые.
\пункт \textit{сумма идеалов $I_1$ и $I_2$} --- множество всевозможных сумм вида $x_1+ x_2$, 
где $x_1 \in I_1$, $x_2 \in I_2$;
\пункт \textit{пересечение идеалов $I_1$ и $I_2$} --- множество $\{x \mid x\in I_1, x \in I_2\}$.
\кзадача


\задача[Теорема об идеалах в $\Z$]
Пусть $r$ --- наименьшее положительное число, принадлежащее ненулевому идеалу $I$ (оно называется {\it порождающим элементом} идеала $I$). Докажите, что
\\
\пункт любое число из $I$ делится на $r$;
\пункт $I$ состоит из всех целых чисел, делящихся на $r$.
\кзадача

\задача[Единственность: доказательство с идеалами]
Пусть $m$ и $n$ --- ненулевые целые числа, $mn\divs p$ и $(m,p)=1$. Рассмотрим множество $J$ всех таких целых чисел $j$, что $mj\divs p$. Докажите, что
\пункт $J$ --- идеал;
\пункт $1\not \in J$; $p\in J$;
\пункт наименьшее положительное число в $J$ равно $p$;
\пункт $n\divs p$, откуда следует свойство \textbf{(*)}.
\кзадача




\задача  Числа $a$, $b$, $c$, $n$ натуральные, $(a,b)=1$, $ab=c^n$.
Найдутся ли такие целые $x$ и $y$, что~$a=x^n$, $b=y^n$?
%Верно ли, что $a=x^n$ %и $b=y^n$
%для некоторого целого $x$?
%натуральных чисел $x$ и $y$?
\кзадача

\задача
Решите в натуральных числах уравнение $x^{42}=y^{55}$.
\кзадача

\задача
Найдите каноническое разложение числа \вСтрочку
\пункт 2014; \пункт 1002001; \пункт 17!. %; \пункт $C_{20}^{10}$.
\кзадача

\опр \выд{Наименьшим общим кратным} ненулевых целых чисел $a$ и $b$
называется наименьшее натуральное число, которое делится на $a$ и на $b$.
Обозначение: $[a,b]$ или НОК$(a,b)$.
\копр

%\задача
%Докажите, что $[a,b]$ существует и единственно
%для любых ненулевых целых $a$ и $b$.
%\кзадача

\задача
\вСтрочку
\пункт
Как, зная канонические разложения %на множители
%натуральных
чисел $a$ и $b$, найти
$(a,b)$ и $[a,b]$?
\пункт
Найдите $[192,270]$.
\пункт
Докажите, что $ab=(a,b) \cdot [a,b]$.
\пункт
Верно ли, что числа $[a,b]/a$ и $[a,b]/b$ взаимно~просты?
\кзадача

%\задача Верно ли, что \вСтрочку
%\пункт $[ca,cb]=c[a,b]$ при $c>0$;
%\пункт $[a,b]/a$ и $[a,b]/b$ взаимно~просты?
%\кзадача

\задача Докажите, что любое общее кратное
целых чисел $a$ и $b$ делится на $[a,b]$.
\кзадача

%\задача
%Докажите, что $ab=(a,b) \cdot [a,b]$ для любых натуральных
%чисел $a$ и $b$.
%\кзадача

\задача
Про натуральные числа $a$ и $b$
известно, что $(a,b)=15$, $[a,b]=840$. Найдите $a$ и $b$.
\кзадача

\опр
Пусть $a_1,\dots,a_k$ --- натуральные. Назовём их \textit{наибольшим общим делителем} порождающий элемент идеала $a_1\Z + \dots +a_k\Z$, \textit{наименьшим общим кратным} --- порождающий элемент идеала $a_1\Z \cap \dots \cap a_k\Z$.
\копр

\задача\сНовойСтроки Докажите, что для положительных чисел $a_1,\dots,a_k$
\пункт наибольший общий делитель $d$ является наибольшим числом, которое делит данные числа; \пункт наибольший общий делитель делится на все остальные общие делители; \пункт найдутся такие целые числа $x_1,\dots,x_k$, что $d = a_1x_1+\dots+a_kx_k$;
\пункт наименьшее общее кратное является наименьшим натуральным числом, которое делится на данные числа; \пункт наименьшее общее кратное делит остальные \textit{общие кратные} данных чисел. %, то есть такие $x$, что $x\, \vdots \, a_1, x\, \vdots\, a_2, \ldots, x\, \vdots\, a_k$.
\кзадача



%\задача
%Может ли %наименьшее общее кратное чисел
%НОК$(1, 2, \dots, n)$
%быть в $2008$ раз больше, чем %наименьшее общее кратное чисел
%НОК$(1, 2, \dots, m)$?
%\кзадача

%\задача
%Найдите
%$\displaystyle{\frac{{\rm НОК}(1,\,2,\,3,\,\dots\,,\,99)}{{\rm НОК}(2,\,4,\,6,\,\dots\,,\,200)}}$.
%\кзадача

%\сзадача Найдутся ли 100 таких различных натуральных чисел, что для
%любых двух чисел $a$ и $b$ из них
%\пункт $ab$~делится на сумму всех 100 чисел;
%\пункт $a+b$ делится на $a-b$;
%\пункт $(a,b)=|a-b|$?
%\пункт Тридцать~три~богаты\-ря едут верхом по кольцевой дороге против часовой стрелки.
%Могут ли они ехать неограниченно %различными
%долго~с~\hbox{разными} посто\-янными скоростями,
%если на дороге есть только одна точка, %в которой
%где богатыри %имеют возможность
%могут обгонять друг друга?
%\кзадача


\vspace{-1mm}
\ЛичныйКондуит{0mm}{6mm}

%\СделатьКондуит{4.3mm}{6.5mm}

%\GenXMLW
\end{document}

\задача
Назов\"ем ч\"етное число $n$ \выд{ч\"етнопростым}, если $n$
не раскладывается
в произведение двух ч\"етных чисел. (Например, 6 --- ч\"етнопростое,
а 12 --- нет.)
%Верно ли, что любое ч\"етное
%число единственным образом раскладывается в произведение ч\"етнопростых
%чисел (с точностью до порядка сомножителей)?
Будет ли верна <<основная теорема арифметики>> для чётнопростых чисел?
\кзадача

\задача При каких натуральных $k$ число $(k-1)!$ не делится на $k$?
\кзадача

\задача
\вСтрочку
\пункт [Теорема Лежандра] Докажите, что простое число
$p$ входит в каноническое разложение числа $n!$
в степени $[n/p]+[n/p^2]+[n/{p^3}]+\dots$
(где $[x]$ --- это \выд{целая часть} числа $x$).\\
С какого момента слагаемые в этой сумме станут равными нулю?\\
\пункт Сколько у $2000!$ нулей в конце его десятичной записи?
% числа $2000!$?
\пункт Может ли $n!$ делиться на $2^n$ ($n\geq1$)?
\кзадача


\задача
Число $p$ простое. Докажите, что $C_p^k$ делится на
$p$, если $0<k<p$.
\кзадача

%\задача[Малая теорема Ферма]
%Докажите: $n^p-n$ делится на $p$, если $p$ ---
%простое,~\hbox{$n$ --- целое.}
%%Пусть $p$ простое.
%\кзадача

\задача[Малая теорема Ферма]
Пусть $p$ --- простое число, $n$ --- целое число.
Докажите, что\\
\вСтрочку
\пункт  $n^p-n$ делится на $p$;
\пункт  если $(n,p)=1$, то $n^{p-1}-1$ делится на $p$.
\кзадача


%\задача
%Для каких $k\in\N$ есть $k$ последовательных
%целых чисел, являющихся составными?
%\кзадача


\сзадача
\вСтрочку
\пункт
Числа $p$ и $q$ простые, $2^{p}-1\divs q$. Докажите,
что $q-1\divs p$.
\пункт
Простое ли $2^{13}-1$?
\кзадача


\сзадача Может ли быть целым число
\вСтрочку
\пункт
$\displaystyle{\frac{1}{2}+\frac{1}{3}+\frac{1}{4}+\ldots+\frac{1}{n}}$;
\пункт
$\displaystyle{\frac{1}{3}+\frac{1}{5}+\frac{1}{7}+\ldots+\frac{1}{2n+1}}$?
\кзадача


\vspace*{-1mm}
\раздел{***}

\vspace*{-2mm}


\задача
Пусть $a$ и $b$ --- целые числа, не равные одновременно нулю, $J$ --- множество чисел вида $ax+by$, где $x$ и $y$ целые, и $r$ --- наименьшее положительное число в $J$.
Докажите, что\\
\пункт $J$ --- идеал;
\пункт $r$ делится на $(a,b)$;
\пункт $a$ и $b$ делятся на $r$;
\пункт $r=(a,b)$, то есть $J$ состоит из всех чисел, делящихся на $(a,b)$.
\кзадача

\задача
Пусть $a$ и $b$ --- целые числа, причем $(a,b)=1$. Докажите, что\\
\пункт найдутся такие целые $x$ и $y$, что $ax+by=1$;
\пункт если $ca$ делится на $b$, где $c$ --- целое, то $c$ делится на $b$.
\кзадача


\задача[Основная теорема арифметики] Докажите следующие утверждения:
\сНовойСтроки
\vspace*{-3pt}
\пункт если $p$ --- простое число, $m$ и $n$ --- целые, и $mn\divs p$,
то либо $m\divs p$, либо $n\divs p$;
\пункт для каждого целого $n>1$ найдутся такие простые
$p_1,\dots,p_k$, что $n=p_1\cdot \dots \cdot  p_k$; %($k$ --- натуральное);
\пункт[каноническое разложение] Для каждого целого $n>1$ найдутся такие
различные простые $p_1,\dots,p_k$ и натуральные $\al_1,\dots,\al_k$, что
$n=p_1^{\al_1}\cdot \dots \cdot p_k^{\al_k}$; %($k$ --- натуральное);
\пункт разложения из пунктов б) и в) единственны с точностью до порядка
сомножителей.
\кзадача

\раздел{Запас}

\задача
$a$ и $b$ --- натуральные числа.
Докажите, что если $[a,a+5]=[b,b+5]$, то $a=b$.
\кзадача

\задача
Число $n$ натуральное. Пусть $k$ --- наименьшее
целое число, большее 1 и
взаимно простое с каждым из чисел $1$, $2$, \dots, $n$.
Докажите, что $k$ существует и является простым.
\кзадача

\задача %Пусть $n\in\N$.
Сколько двоек в разложении числа
$1001\cdot1002\cdot\ldots\cdot2000$ на простые множители?
\кзадача

\задача Докажите, что при любом целом $n>1$ между $n$ и $n!$ есть простое
число.
\кзадача

\задача
Решите в натуральных числах уравнения:
\вСтрочку
\пункт
$x^y=y^x$;
\пункт
$u^x+u^y=u^z$.
\кзадача

\сзадача Найдите все натуральные $n$,
у которых не меньше чем $\sqrt n$ натуральных делителей.
\кзадача

\сзадача
Для целых $a,b$ и натурального $n$ докажите:  %, что
$b^n(a+b)(a+2b)\cdot\ldots\cdot(a+nb)$ кратно $n!$.
\кзадача


\сзадача Числа $a$, $b$, $c$ и $d$ натуральные, %прич\"ем
$ab=cd$. Может ли число $a+b+c+d$ быть простым?
\кзадача


\end{document} 