% !TeX encoding = windows-1251
\documentclass[a4paper, 12pt]{article}
\usepackage{newlistok}

\УвеличитьШирину{1truecm}
\УвеличитьВысоту{2.5truecm}

\renewcommand{\spacer}{\vfill}


\begin{document}

\НомерЛистка{8д}
\ДатаЛистка{04.2015}
\Заголовок{Приближение действительных чисел рациональными}
\СоздатьЗаголовок
\bigskip


\задача В каждой клетке бесконечной шахматной доски сидит по зайцу
(все зайцы одинаковы~и одинаково расположены). \сНовойСтроки \пункт
Охотник стреляет по направлению с иррациональным тангенсом угла
наклона к линиям доски. Докажите, что он попадет хотя бы в одного
зайца. \пункт Докажите, что если тангенс угла наклона рационален, то
достаточно малых зайцев можно расположить так,
что~охотник~промахнется.
\кзадача

\задача Конь прыгает скачками $(\sqrt 2, \sqrt 3)$ по полю, где
квадратно-гнездовым способом посеяна кукуруза. Докажите, что он
обязательно сшибет хотя бы один росток (конь сшибает росток только в
том случае, если приземляется на него; в прыжках конь ростки не
задевает).
\кзадача

\опр \выд{Коэффициентом качества} приближения $p/q$ иррационального
числа $\alpha$ (где $p,q\in \Z$, $q > 0$) называется число
\vspace*{-4mm}
  $$
  q\left|\alpha - {\frac pq}\right|.
  $$
Из двух приближений лучшим считается то, у которого меньший
коэффициент качества.
\копр

\задача Какое из приближений числа $\sqrt 2$ лучше: $3/2$; $7/5$ или
$1{,}41$?
\кзадача

\задача Пусть $\alpha$~--- некоторое иррациональное число. Докажите,
что для любого $q \in \N$ существует приближение $p/q\in\Q$ числа
$\alpha$ с коэффициентом качества, меньшим $1/2$.
\кзадача

\задача Докажите, что  для любых натуральных чисел $N$, $k$ и любого
иррационального числа $\alpha$ существует по крайней мере $k$ таких
различных дробей $p/q \in \Q$, что $q \leq Nk$ и
$\displaystyle{q\left|\alpha - {\frac pq}\right| < {\frac1N}}$.
\кзадача

\задача Докажите, что  для любого $\alpha\notin\Q$ и для сколь
угодно большого $N$ существует бесконечно много различных
приближений $p/q \in \Q$ с коэффициентом качества, меньшим~$1/N$.
\кзадача

\задача Пусть число $\alpha$ иррационально. Докажите, что существует
бесконечно много таких рациональных чисел $p/q$, что
\vspace*{-4mm}
  $$
  \left|\alpha - {\frac pq}\right| < \frac1{q^2}.
  $$
\vspace*{-3mm}
\кзадача

\опр Число $\alpha\in \R$ называется {\it $t$-неприближаемым}, если
найдется такое положительное число  $c\in\R$, что при любых $m \in
\Z$, $n\in\N$ выполнено одно из двух условий:
\vspace*{-2mm}
  $$
  \alpha = \frac{m}n \qquad \hbox{или} \qquad
  \left|\alpha - \frac{m}n\right| > \frac{c}{n^t}.
  $$
\vspace*{-3mm}
\копр

\задача Докажите, что рациональные числа 1-неприближаемы.
\кзадача

\задача Докажите, что число $\displaystyle{e =
\sum\limits^{+\infty}_{i = 0}{\frac1{i!}}}$ иррационально.
\кзадача

\задача Пусть $\alpha\notin\Q$~--- корень многочлена $A(x)$ степени
$n\in\N$ с целыми коэффициентами. \пункт Докажите, что  для любого
рационального числа $p/q$, не являющегося корнем $A(x)$, справедливо
неравенство $|q^nA(p/q)| \geq 1$. \пункт Докажите, что $\alpha$
является $n$-неприближаемым.
\кзадача

\задача Докажите, что ряд $\displaystyle{\sum\limits^{+\infty}_{i =
1} {\frac1{2^{i!}}}}$ сходится к \выд {трансцендентному числу},
т.~е.~к такому числу, которое не может быть корнем ненулевого
многочлена с целыми коэффициентами.
\кзадача

\сзадача Докажите, что для любого $\varepsilon > 0$ множество
действительных чисел, не являющихся $(2 +
\varepsilon)$-неприближаемыми, имеет меру 0 (т.~е.~для каждого
$\delta > 0$ это множество можно покрыть счётной системой
интервалов, сумма длин которой меньше $\delta$).
\кзадача

\ссзадача \вСтрочку \пункт[Теорема Гурвица--Бореля] Докажите, что
для любого иррационального числа $\alpha$ существует бесконечно
много таких его приближений $p/q\in\Q$, что
$\displaystyle{\left|\alpha - {\frac pq}\right| < {\frac1{q^2\sqrt 5}}}$.\\
\пункт Число $\sqrt 5$ в условии теоремы пункта~а) нельзя увеличить:
найдутся иррациональные числа, имеющие лишь конечное число
приближений, удовлетворяющих неравенству измененной теоремы.
\кзадача



\ЛичныйКондуит{0mm}{6mm}
%\GenXMLW


\end{document}

